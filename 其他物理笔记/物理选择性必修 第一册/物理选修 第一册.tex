=\documentclass[12pt,a4paper]{ctexart}

\title{物理选修\ \ 第一册}
\author{啊波呲}

\setlength{\parskip}{0em}
\usepackage{amsmath,mathtools,amssymb,geometry,wrapfig,graphicx,empheq,pifont,enumitem,upgreek}
\renewcommand{\baselinestretch}{1.65}
\geometry{left=1.0cm,right=0.9cm,bottom=1.9cm,top=2.0cm}
\setenumerate[1]{itemsep=1pt,partopsep=0pt,parsep=\parskip,topsep=3pt}
\setitemize[1]{itemsep=1pt,partopsep=0pt,parsep=\parskip,topsep=3pt}

\usepackage{tikz}
\usepackage{xcolor}
\newcounter{exam}[section]
\setcounter{exam}{0}
\newcommand{\bre}{\ \ \ }
\newcommand{\examlabel}{\textbf{例\theexam}}
\newcommand{\soln}{\textbf{解}\bre}
\newcommand{\notes}{\textbf{注意}\bre}
\newcommand{\unit}[1]{\ \mathrm{#1}}

\newenvironment{example}{\bigskip\par\refstepcounter{exam}\examlabel\bre}{\par}
\newenvironment{solution}{\par\soln}{\par\bigskip}

\begin{document}
\maketitle
\pagenumbering{roman}
\tableofcontents

\newpage
\pagenumbering{arabic}

\setlength{\abovedisplayskip}{3pt}
\setlength{\belowdisplayskip}{3pt}

\section{动量守恒定律}
\subsection{动量}

用两根长度相同的细线, 分别悬挂两个完全
相同的钢球 A, B, 且两球并排放置.拉起 A 球, 
然后放开, 该球与静止的 B 球发生碰撞.可以看
到, 碰撞后 A 球停止运动而静止, B 球开始运动, 
最终摆到和 A 球被拉起时同样的高度.

碰撞后, A 球的速度大
小不变地传给了 B 球.所有的碰撞都有这样的特点吗?

将上面实验中的 A 球换成大小相同的 C 球, 
使 C 球质量大于 B 球质量, 用手拉起 C 球至某一高度后放
开, 撞击静止的 B 球.我们可以看到, 碰撞后 B 球获得较
大的速度, 摆起的最大高度大于 C 球被拉起时的高度.

可以看出, 质量大的 C 球与质量小的 B 球碰撞
后, B 球得到的速度比 C 球碰撞前的速度大, 两球碰撞前
后的速度之和并不相等.

经过多次实验验证, 我们发现:碰撞前后, 
两个小球的动能之和并不相等, 但是质量与速度的乘积之和却基本不变.

上面的实验提示我们, 质量与速度的乘积
$mv$ 这个物理量具有特别的意义.

\bigskip

物理学中把质量和速度的乘积$mv$定义为物体的\textbf{动量}, 用字母$p$表示, 即
\begin{empheq}[box=\fbox]{equation*}
    p = mv.
\end{empheq}
在国际单位制中, 动量的单位是\textbf{千克米每秒}($\mathrm{kg\cdot m/s}$).

动量是矢量,动量的方向与速度的方向相同.

动量是状态量. 我们说物体在某一时刻的动量, 
是指物体的质量$m$与它在这一时刻速度$v$的乘积.
这里$v$是物体在这一时刻的瞬时速度, 而不是某段时间内的平均速度.

物体在一个过程中\textbf{动量的变化量}是
它的末动量与初动量的矢量差, 也叫做\textbf{动量的增量}. 如果初末动量在一条直线上, 即运动是一维的, 那么选定正方向后,
动量的方向可以用正负号表示, 从而将矢量运算转化为代数运算, 即
$$\Delta p = p^{\prime}-p = m\Delta v = mv^{\prime} - mv.$$
此时动量的增量$\Delta p$为负表示动量向负方向增加.

\subsection{动量定理}

\subsubsection{冲量}
假定一个质量为$m$的物体在光滑的水平面上受
到恒力 $F$ 的作用,做匀变速直线运动.在初始时刻,物体
的速度为 $v$,经过一段时间$\Delta t$,它的速度为 $v^{\prime}$,那么,这
个物体在这段时间的加速度
$$a = \frac{v^{\prime}-v}{\Delta t}.$$
根据牛顿第二定律, 有
\begin{equation*}
    F = ma = m\frac{v^{\prime}-v}{\Delta t} = \frac{mv^{\prime}-mv}{\Delta t}.
    \label{动量的变化率}
\end{equation*}
即
\begin{equation}
    F\Delta t = p^{\prime}-p.
    \label{动量定理1}
\end{equation}
这就是说, \textbf{力与力的作用时间的乘积, 等于在这段时间内动量的变化量}.

\eqref{动量定理1} 式的左边是力与力作用时间的乘积, 它反映了力的作用对时间的累积效应. 物理学中把
$F\Delta t$这个量称为力的\textbf{冲量}, 用$I$表示, 即
\begin{empheq}[box=\fbox]{equation*}
    I = F\Delta t.
\end{empheq}
冲量的单位是\textbf{牛秒}($\mathrm{N\cdot s}$).事实上, 它与动量的量纲相同, 并且
$$1\ \mathrm{N\cdot s} = 1\ \mathrm{kg\cdot m/s}.$$

冲量是矢量,冲量的方向与力的方向相同.

冲量是过程量. 我们说力$F$在一段时间$\Delta t$内的冲量为$F\Delta t$, 
是指这个力$F$对物体作用了$\Delta t$时间产生的冲量.

\subsubsection{动量定理}
有了冲量的概念, \eqref{动量定理1} 式就可以写成
\begin{empheq}[box=\fbox]{equation*}
    I = \Delta p.
\end{empheq}
也可以展开写为
\begin{empheq}[box=\fbox]{equation*}
    F\Delta t = mv^{\prime}-mv.
\end{empheq}
它们表示, \textbf{物体在一个过程中所受力的冲量等于它在这个过程始末的动量变化量}, 这就是\textbf{动量定理}.

值得注意的是, 动量定理中所提到的``力的冲量''指的是物体所受合力的冲量.除此之外, 物体受到的任何一个分力
都有冲量, 比如质量为$m$的物体重力的冲量为$$I_\mathrm{G} = mg\Delta t.$$
其中$g$是重力加速度, $\Delta t$是重力的作用时间.然而重力无时无刻不作用在物体上, 所以我们可以
取任意的$\Delta t$, 并说$mg\Delta t$是物体的重力在时间$\Delta t$内的冲量.

事实上, 动量定理不仅适用于单体, 也适用于系统, 即我们有:

\textbf{物体系统在一个过程中所受合外力的冲量等于该系统内各个物体在这个过程始末的动量变化量的矢量和.}
\bigskip

\eqref{动量定理1} 式还可以写成$$F = \frac{\Delta p}{\Delta t},$$
由此可知, \textbf{物体动量的变化率等于它所受合外力的大小}.

一定质量的物体, 改变一定的速度, 动量的改变量是一定的. 此时$$F\propto \displaystyle\frac{1}{\Delta t},$$
即力的作用时间越长, 力越小. 物体相互碰撞时出现的力称为\textbf{冲力}. 为了安全, 我们通常延长冲力的作用时间
从而减小冲力的大小, 这就是\textbf{``缓冲''}.易碎物品运输时要用柔软材料包装, 跳高时运动员
要落在软垫上, 就是这个道理.

% \begin{wrapfigure}{r}{4cm}
%     \flushright %右侧对齐
%     \includegraphics[width=0.25\textwidth]{路径}
%     \label{fig}
% \end{wrapfigure}
\setlength{\abovedisplayskip}{3pt}
\setlength{\belowdisplayskip}{3pt}

\begin{example}
    一个质量为$m = 2\ \mathrm{kg}$的物体,在$F_1 = 8\ \rm{N}$的水平拉力作用下,
    从静止开始沿水平面运动了$t_1 = 5\ \rm{s}$,然后拉力减小为$F_2=5\ \rm{N}$,方向不变,物体又运动了
    $t_2=4\ \rm{s}$后撤去外力,物体再经过$t_3=6\ \rm{s}$停下来. 试求物体在水平面上所受的摩擦力.
\end{example} 

\textbf{解}\bre 设拉力的方向为正方向, 则摩擦力沿负方向.
注意到全程中摩擦力的大小$F_\mathrm{f}$是恒定不变的, 对全过程应用动量定理得
$$F_1t_1 + F_2t_2 - F_\mathrm{f}(t_1 + t_2 + t_3) = 0-0.$$
由此解得$F_\mathrm{f} = 4\ \mathrm{N}$.

\textbf{注意}\bre 本题中的拉力都是水平的, 如果拉力是倾斜的, 并且在整个过程中
大小发生了变化, 要注意摩擦力也会随之变化.


\begin{example}
    某人所受重力为$G$, 穿着平底鞋起跳, 竖直着地过程中,
    双脚与地面间的作用时间为$t$, 地面对他的平均冲击力大小为$4G$. 若他穿上带有减震气垫
    的鞋起跳, 以与第一次相同的速度着地时, 双脚与地面间的作用时间变为$2.5t$,
    求地面对他的平均冲击力$\bar{F}$变为多少.
\end{example}

\textbf{解}\bre 由两过程的末速度相同, 知两过程的动量变化量相同. 由动量定理知
两过程的合力冲量相同, 即
$$(4G-G)t = (\bar{F} - G)2.5t.$$
由此解得$\bar{F} = 2.2G.$

\textbf{注意}\bre 动量定理$I = \Delta p$中, $I$为合力的冲量.在本题中, 要注意不要把重力当作合力计算.

\subsubsection{流体的冲力}
\setlength{\abovedisplayskip}{3pt}
\setlength{\belowdisplayskip}{3pt}
流体以速度$v$冲击到某一点上, 根据动量定理, 流体的冲力$F$满足$$Ft = mv.$$
其中$m$为流体质量, $t$为流体某一点的运动时间.

因为流体的体积$V = vt\cdot S$($S$为流体截面), 所以
流体的质量$m = \rho V = \rho\cdot vt \cdot S$($\rho$为流体的密度).因此
$$Ft = \rho\cdot vt \cdot S \cdot v.$$

由此得到流体的冲力公式
\begin{empheq}[box=\fbox]{equation*}
    F = \rho Sv^2.
\end{empheq}
式中$F$为流体的冲力, $\rho$为流体密度, $S$为流体截面, $v$为流体速度.
\subsubsection{变力的冲量}

当物体所受的合力变化时, 我们通常用动量定理求解合力的
平均冲量, 进而求解某个分力的冲量.

特别地, 当物体所受的某个力$F$与物体的速度$v$成正比时(比如阻力),
设这个力与速度的关系为$F = kv$, 那么这个力的冲量
$$I_F = \sum F_i\Delta t_i =\sum kv_i\Delta t_i= ks.$$
其中$F_i, v_i$分别是物体在每个极短时间$\Delta t_i$内的力和速度, $s$是物体的位移.

\subsection{动量守恒定律}

前面我们学习了动量定理. 力$F$作用在质量为$m$的物体上, 作用时间为$t$,
物体在这个过程始末的速度分别为$v$, $v^{\prime}$, 则有
$$Ft = mv^{\prime} - mv.$$

当外力$F$为0时, 有$$mv^{\prime} = mv.$$
即物体的动量保持不变, 我们可以说这个物体的动量是守恒的.这个结论我们很好理解, 应用牛顿第一定律就可以解释.

上面的讨论是对单个物体而言的, 那么对多个物体有没有``动量守恒''的规律呢?
在研究这个问题之前, 我们先介绍一个概念.

当我们对两个(或多个)物体进行研究时, 它们组成一个\textbf{力学系统}.
系统内两个(或多个)物体的相互作用力称为\textbf{内力}, 系统以外的物体对系统的作用力称为\textbf{外力}.

下面我们结合一个具体情境来研究, 系统所受合外力为0时, 系统的总动量如何变化.

质量为$m_1$和$m_2$的两个小球A和B, 在光滑水平面上沿同一方向
做匀速直线运动, 速度分别是$v_1,\ v_2$, 且$v_2>v_1$. 经过一段时间后,
B追上了A, 两球发生碰撞, 碰撞后A, B的速度分别变为$v_1^{\prime},\ v_2^{\prime}$.

设碰撞过程中, B对A的作用力为$F_1$, A对B的作用力为$F_2$, 由牛顿第三定律可知,
$F_1$与$F_2$大小相等, 方向相反, 即$F_1 = -F_2$.

对A应用动量定理得$$F_1t = m_1v_1^{\prime} - m_1v_1.$$

对B应用动能定理得$$F_2t = m_2v_2^{\prime} - m_2v_2.$$

联立以上三个式子, 得
\begin{equation}
    \label{动量守恒1}
    m_1v_1^{\prime}-m_1v_1 = -(m_2v_2^{\prime} - m_2v_2).
\end{equation}
整理得
\begin{equation}
    \label{动量守恒2}
    m_1v_1 + m_2v_2 = m_1v_1^{\prime} + m_2v_2^{\prime}.
\end{equation}
我们发现, 两个球碰撞前后, 系统的总动量是不变的, 也可以说系统的动量是守恒的.
由此, 我们得到动量守恒定律.

\subparagraph{动量守恒定律} \textbf{如果一个系统不受外力或所受外力的矢量和为0, 那么这个系统的总动量保持不变,
    即系统的动量守恒.}
\bigskip

\eqref{动量守恒1} 和 \eqref{动量守恒2} 分别是动量守恒定律的两种表达形式, 即
\begin{empheq}[box=\fbox]{equation*}
    \Delta p_1 = -\Delta p_2.
\end{empheq}
其中$\Delta p_1$, $\Delta p_2$分别为两个物体在一个过程中的动量变化量,
这种形式适用于系统中有且只有两个物体时.

或者
\begin{empheq}[box=\fbox]{equation*}
    p = p^{\prime}.
\end{empheq}
其中$p$, $p^{\prime}$分别为系统的初动量, 末动量. 这种形式不限制系统内的物体数量.
当系统中有两个物体时, 上式可以写成
\begin{empheq}[box=\fbox]{equation*}
    m_1v_1 + m_2v_2 = m_1v_1^{\prime} + m_2v_2^{\prime}.
\end{empheq}
这是动量守恒最常见的表达形式.

对于整个系统来说, 因为系统的总动量$p = m_\text{系统}v_\text{质心}$,
所以
$$m_\text{系统}v_\text{质心} = m_\text{系统}v_\text{质心}^{\prime},$$
由此可得$$v_\text{质心} = v_\text{质心}^{\prime},$$ 即\textbf{系统质心速度不变}.
因此系统动量守恒等价于系统质心的速度守恒.

应用动量守恒定律时, 有几点注意事项:
\begin{enumerate}
    \item 动量守恒指的是总动量在相互作用的过程中时刻守恒, 而不是只有始末状态才守恒. 因此在实际使用时, 可以任选两个状态来列方程.
    \item 系统的动量守恒, 不能说明单体的动量守恒. 在应用动量守恒定律时, 一定要明确是哪些物体构成的系统.
    \item 动量守恒定律的表达式是矢量式, 对于一维的问题可以先规定正方向, 再进行计算.
    \item 动量与参考系的选择有关, 一般以地面为参考系.
    \item 动量守恒不能说明机械能守恒, 机械能守恒也不能说明动量守恒, 它们的判定条件不同, 没有必然联系.
    \item 动量守恒定律不但适用于宏观低速运动的物体, 而且还适用于微观高速运动的粒子. 它与牛顿运动定律相比, 适用范围要广泛得多. 并且动量守恒定律不需要考虑物体间的作用细节, 在解决问题上比牛顿运动定律更简捷.
\end{enumerate}

\subsubsection*{动量守恒的条件}

虽然动量守恒定律要求系统所受的合外力为0, 但是在以下两种情况下, 我们也可以用
动量守恒定律解决问题:
\begin{enumerate}
    \item 系统所受合外力不为0, 但在系统各部分相互作用的瞬时过程中, 系统内力
          远远大于外力, 外力相对来说可以忽略不计, 这时系统的动量近似守恒. 例如爆炸, 反冲等过程.
    \item 系统所受合外力不为0, 但系统在某一方向上不受外力或该方向上外力之和为0.
          则系统在该方向上的动量守恒.例如物块在斜面上下滑的问题. 在分析这类问题时, 要注意先把速度分解成沿该方向的速度再代入计算.
\end{enumerate}

\begin{example}
    两个木块A, B置于光滑水平面上, 它们的质量$m_\mathrm{A} = m_\mathrm{B} = 2$\ kg.
B与一轻质弹簧的一段相连,
弹簧的另一端固定在墙上. 当A以$v = 4$\ m/s的速度向B撞击时, 由于有橡皮泥(质量不计)而粘在
一起运动. 求弹簧被压缩到最短时, 弹簧的弹性势能$E_\mathrm{p}$.
\end{example}

\textbf{解}\bre 在弹簧压缩前, 两木块的动量守恒, 并且最终具有相同的速度$v_\text{共}$,
根据动量守恒定律有$$m_\mathrm{A}v + 0 = (m_\mathrm{A}+m_\mathrm{B})v_\text{共}.$$

弹簧压缩的过程中, 根据能量守恒定律, 两物体的动能转化为弹簧的弹性势能, 即
$$\frac12 (m_\mathrm{A}+m_\mathrm{B}) v_\text{共}^2 = E_\mathrm{p}.$$

联立以上两式, 代入数据可得$E_\mathrm{p} = 8$\ J.

\subsection{反冲}
系统在内力的相互作用下, 当一部分向某一方向运动时, 剩余部分
将向相反方向运动, 这种现象叫做\textbf{反冲}.

发射炮弹时, 炮弹从炮筒中飞出, 炮身则向后退, 
这就是反冲现象. 射击前, 炮弹静止在炮筒中, 它们的总动量为
0. 炮弹射出后以很大的速度向前运动, 根据动量守恒定
律, 炮身必将向后运动. 只是由于炮身的质量远大于炮弹
的质量, 所以炮身向后的速度很小.

用枪射击时, 子弹向前飞去, 枪身发生反冲向后运动. 枪身的反冲会影
响射击的准确性, 所以用步枪射击时要把枪身抵在肩部, 
以减少反冲的影响.

以发生反冲的系统为研究对象, 系统两部分间的相互作用力是内力,
在系统外力(如重力, 空气阻力等)可以忽略的情况下, 我们知道,
系统的动量守恒.

爆炸也是反冲运动的一种. 爆炸过程中,
系统内部在极短时间内释放出大量的能量, 内力远远大于外力,
因此, 对整个系统而言, 我们可以利用动量守恒来解决问题.

在光滑水平面上, 质量为$m_1$, $m_2$的两个物体紧靠在一起, 它们之间有少许炸药(质量不计), 炸药爆炸后,
两个物体分开而向相反的方向运动, 速度大小分别为$v_1$, $v_2$. 以
$v_1$的方向为正方向, 根据动量守恒, 有
\begin{equation}
    0 = m_1v_1 - m_2v_2.
    \label{爆炸1}
\end{equation}
由此可得$$\frac{v_1}{v_2} = \frac{m_2}{m_1}.$$
这就是说, \textbf{原本静止的系统因爆炸而分成两部分, 这两部分的速度大小之比是它们质量比的反比.}

研究完速度, 我们再来研究动能和动量. 根据 \eqref{爆炸1}, 显然这两个物体的动量相等, 那它们的动能又有什么关系呢?

设这两个物体的动能分别为$E_\mathrm{k1}$, $E_\mathrm{k2}$, 则
$$\frac{E_\mathrm{k1}}{E_\mathrm{k2}} = \frac{\frac12 m_1v_1^2}{\frac12m_2v_2^2} = \frac{m_1v_1^2}{m_2v_2^2} = \frac{v_1}{v_2}.$$
即它们的动能之比等于速度之比.

当爆炸产生的化学能全部转化为动能时, 则
$$E_\text{化学} = \Delta E_\mathrm{k} = \left(\frac12m_1v_1^2 + \frac12m_2v_2^2\right)-0.$$
当转化比为$\eta$时, 则
$$\eta E_\text{化学} = \Delta E_\mathrm{k} = \left(\frac12m_1v_1^2 + \frac12m_2v_2^2\right)-0.$$

\begin{example}
    在粗糙水平面上, 质量为$m_1$, $m_2$的两个物体紧靠在一起,
它们之间有少许炸药(质量不计), 炸药爆炸后,
两个物体分别滑动$x_1$, $x_2$的距离而停止,
两物体与水平面间的动摩擦因数均为$\mu$.
求两物体的质量比, 以及爆炸后瞬间的速度大小之比,动能大小之比.
\end{example}

\textbf{分析}\bre 爆炸过程中产生的内力极大, 时间极短, 摩擦力相对来说可以忽略不计,
因此系统的动量可以看作是守恒的.

\textbf{解}\bre 设爆炸后瞬间两物体的速度大小分别为$v_1$, $v_2$.
炸药爆炸后, 分析两物体滑行的过程, 易知两个物体的
加速度恒定且均为$a = \mu g$. 由匀变速直线运动位移与速度的关系可知
$$-2ax_1 = 0 - v_1^2,$$ $$-2ax_2 = 0 - v_2^2.$$
可得两物体爆炸后瞬间的速度大小之比$v_1:v_2 = \sqrt{x_1}:\sqrt{x_2}$.

炸药爆炸过程中, 取$v_1$的方向为正方向, 应用动量守恒定律得
$$0 = m_1v_1 - m_2v_2.$$
解得$m_1:m_2 = v_2:v_1 = \sqrt{x_2}:\sqrt{x_1}$.

爆炸后两物体的动能之比为
$$\frac{E_\mathrm{k1}}{E_\mathrm{k2}} = \frac{\frac12 m_1v_1^2}{\frac12m_2v_2^2}=\frac{m_1v_1\cdot v_1}{m_2v_2\cdot v_2}.$$
由上可知$m_1v_1 = m_2v_2$, 所以$$\frac{E_\mathrm{k1}}{E_\mathrm{k2}} =\frac{v_1}{v_2} = \frac{\sqrt{x_2}}{\sqrt{x_1}}.$$

\subsection{碰撞}

碰撞是两物体间极短的相互作用. 发生碰撞的系统所受合外力为0, 因此系统的
动量守恒, 也就是\textbf{系统质心的速度守恒}.

\subsubsection{碰撞的类型}

\subparagraph{完全非弹性碰撞} 在非弹性碰撞过程中, 物体往往会发生形变, 还会发声发热.
因此, 在非弹性碰撞过程中会有动能损失, 转化为其它形式的能, 即动能不守恒.

系统碰撞后, 物体结合在一起 (即共速), 此时动能损失最大. 这种碰撞叫做\textbf{完全非弹性碰撞}.

质量为$m_1$, $m_2$, 初速度为$v_1$, $v_2$的两个物体发生完全非弹性碰撞,
系统亏损的动能为$$\Delta E_\mathrm{k} = \left(\frac12 m_1 v_1^{\prime 2} + \frac12 m_2 v_2^{\prime 2}\right) - \left(\frac12 m_1 v_1^2 + \frac12 m_2 v_2^2\right).$$
系统的动量守恒, 有$$m_1v_1 + m_2v_2 = (m_1+m_2)v_\text{共}.$$

由上式可知, 它们的末速度$v_\text{共}$可以用加权平均数表示
\begin{equation}
    v_\text{共} = \frac{m_1v_1 + m_2v_2}{m_1+m_2}.
\end{equation}
事实上, 这也是系统质心速度的公式.

\subparagraph{完全弹性碰撞} 在理想情况下, 物体碰撞后, 形变能够完全恢复, 并且不发声发热.
因此系统没有动能损失, 这种碰撞称为\textbf{完全弹性碰撞}.

质量为$m_1$, $m_2$, 初速度为$v_1$, $v_2$的两个物体发生完全非弹性碰撞,
根据动量守恒, 动能守恒分别有
$$m_1v_1 + m_2v_2 = m_1v_1^{\prime} + m_2v_2^{\prime}, $$
$$\displaystyle\frac12 m_1 v_1^2 + \displaystyle\frac12 m_2 v_2^2 = \displaystyle\frac12 m_1 v_1^{\prime 2} + \displaystyle\frac12 m_2 v_2^{\prime 2}.$$


特别地, 当两物体质量相同时, 即$m_1 = m_2$时, 容易得到$$v_1 + v_2 = v_1^{\prime}+ v_2^{\prime},$$
$$v_1^2 + v_2^2 = v_1^{\prime 2}+ v_2^{\prime 2}.$$
解得$v_1^{\prime} = v_2$, $v_2^{\prime} = v_1$, 即\textbf{两物体的速度互换}.

\section{简谐运动}

\subsection{机械振动}
我们把物体或物体的一部分在一个位置附近的往复运动, 
称为\textbf{机械振动}, 简称\textbf{震动}. 

把一个有孔的小球连接在弹簧的一端, 弹簧的另一端固定, 小球和弹簧套在光滑的杆上, 能够自由滑动. 弹簧的质量可以忽略, 小球运动时的空气阻力也可以忽略. 


小球静止时所受合力为0, 处于\textbf{平衡位置}. 向一侧拉动小球, 然后放开, 它就在平衡位置附近运动起来. 这样的理想化模型称为\textbf{弹簧振子}. 

由于没有阻力, 小球在运动过程中没有机械能的损失, 因而能够在平衡位置附近做周期性的往复运动, 即周期性的振动. 

\subsubsection{简谐运动}

为了研究弹簧振子的运动规律, 我们利用频闪照相机得到它的位移—时间图像, 即$x-t$图像, 也称为\textbf{振动图像}. 

我们发现, 小球位移与时间的关系可以用正弦函数来表示, 并可通过实验进行证明. 弹簧振子的这种运动是最简单的震动, 我们把这种运动叫做\textbf{简谐运动}. 

\textbf{如果物体的位移与时间的关系遵从正弦函数的规律, 即它的振动图像($x-t$图像)是一条正弦曲线, 这样的振动就是简谐运动. }

因此, 位移$x$关于时间$t$的函数表达式可以写为
\begin{equation}
    x = x(t) = A\sin(\omega t+\varphi_0),
    \label{位移函数}
\end{equation}
其中$A$, $\omega$, $\varphi_0$是参量, 下面我们就来介绍它们的物理意义.

\subsubsection{振幅}

因为$\sin(\omega t+\varphi_0)\in [-1,1]$, 所以$x\in [-A,A]$.
即$$0 \leqslant |x|\leqslant 1.$$
可以看出, $A$是小球离开平衡位置的最大距离, 称为振幅. 振幅是表示物体振动幅度的物理量. 如果记平衡位置为$O$点, $M$和$N$分别是右端和左端的最远位置,  那么$$|OM|=|ON|=\frac12|MN| = A.$$振幅是小球的运动范围的一半. 

\subsubsection{周期\bre 频率}

做简谐运动的小球, 如果在经过$O$点时开始计时, 那么它将向右经过$M$点, 然后向左回到$O$点, 又继续运动到$N$点, 之后又向右回到$O$. 这样一个完整的震动过程称为一次\textbf{全振动}. 事实上, 不管从哪里开始计时, \textbf{做简谐运动的物体完成一次全振动的时间总是相同的}. 

做简谐运动的物体完成一次全振动所需要的时间, 叫做简谐运动的\textbf{周期}. 
在一个周期内, 小球完成一次全震动, \textbf{经过的路程是振幅的四倍}. 

物体完成全振动的次数与所用时间之比, 叫做简谐运动的\textbf{频率}, 数值上等于单位时间内完成全振动的次数. 
用$T$表示周期, $f$表示频率, 则有
$$f = \frac{1}{T}.$$

在国际单位制中, 周期的单位是秒, 频率的单位是\textbf{赫兹}(Hz). $1\ \text{Hz} = 1\ \text{s}^{-1}.$
周期和频率都是描述振动快慢的物理量. 周期越小, 频率越大, 表示震动越快. 

对于函数$x(t)$, 我们知道, $(\omega t+\varphi_0)$在每增加$2\uppi$的过程中, 函数值$x(t)$(即位移$x$)周期性变化一次, 函数的周期$T$就是简谐运动的周期$T$. 于是有
$$\omega(t+T)+\varphi_0 = (\omega t+\varphi_0) +2\uppi.$$
由上式解得$$\omega = \frac{2\uppi}{T}.$$
根据周期与频率的关系, 上式还可以写为
\begin{empheq}[box=\fbox]{equation*}
    \omega = 2\uppi f.
\end{empheq}

可见, $\omega$是一个与频率成正比, 与周期成反比的物理量, 它也可以描述
简谐运动的快慢, 叫做\textbf{圆频率}. 

实验证明, \textbf{简谐运动的周期与其振幅无关}.

\subsubsection{相位}

对于函数$x(t)$, 在一个周期内, 每一个$(\omega t+\varphi_0)$对应着一个函数值. 也就是说, 
当$(\omega t+\varphi_0)$确定时, 位移$x$也就确定了. 因此, $(\omega t+\varphi_0)$可以描述
物体正处于运动周期中的哪个状态, 物理学中把它叫做\textbf{相位}.

$\varphi_0$是$t = 0$时的相位, 称为\textbf{初相位}, 简称\textbf{初相},用$\varphi$表示.

两个具有相同频率的简谐运动, 我们通常研究它们的\textbf{相位差}. 
如果两个简谐运动的频率相同, 它们的相位差就是初相之差. 即
$$\Delta\varphi = \varphi_1 - \varphi_2.$$
此时, 我们说1的相位比2超前$\Delta\varphi$.

我们把两个相同的弹簧振子并列悬挂. 把它们拉到同一位置, 然后放开.
可以发现, 两个小球同时释放时, 除了振
幅和周期都相同外, 还总是向同一方向运动, 同时经过平
衡位置, 并同时到达同一侧的最大位移处.
在一个周期内, 如果不同时释放小球,它们的步调就不一致.

同时放开的两个小球振动步调总是一致, 我们
说它们的相位是相同的; 而对于不同时放开的两个小球, 
我们说第二个小球的相位落后于第一个小球的相位.

\subsubsection{位移}
做简谐运动的物体, 偏离平衡位置的距离, 叫做这个物体做简谐运动的位移.

通过前面的分析知道, 根据一个简谐运动的振幅$A$, 周期$T$, 初相位$\varphi_0$, 可以
确定物体在任意时间$t$的位移
\begin{empheq}[box=\fbox]{equation}
    x = A\sin\left(\frac{2\uppi}{T}t+\varphi_0\right).
    \label{位移}
\end{empheq}

特别地, 如果规定物体处于平衡位置时$t = 0$, 即初相位$\varphi_0 = 0$, 那么
物体在时间$t$的位移$$x = A\sin \frac{2\uppi}{T}t.$$
如果规定物体处于正方向最大位移处时$t = 0$, 则有$\sin\varphi_0 = 1, $
取$\varphi_0 = \displaystyle\frac{\uppi}{2}$, 由三角函数的知识可知, 物体在时间$t$的位移
$$x = A\cos \frac{2\uppi}{T}t.$$

\subsubsection{速度}

我们知道, 速度是衡量位移变化快慢的物理量.
做机械运动的物体, 其在某一时刻的速度就是它在这一时刻位移的瞬时变化率.
由数学知识可知, 速度就是位移关于时间的导函数.

如果物体做简谐运动, \eqref{位移函数} 式是物体的位移关于时间的函数, 那么该物体在时间$t$的速度
$$v = x^{\prime}(t) = A\omega\cos(\omega t+\varphi_0).$$

\subsubsection{回复力\bre 加速度}
根据牛顿运动定律, 在简谐运动中, 一定存在一个力, 在物体远离平衡位置时, 它迫使物
体的运动速度逐渐减小直到减为0, 然后, 物体在这个力
的作用下, 运动速度又由 0 逐渐增大并回到平衡位置; 物
体由于具有惯性, 到达平衡位置后会继续向另一侧运动, 
这个力使它再一次回到平衡位置.

正是在这个力的作用下, 物体才能在平衡位置附近做往复运动. 我们把这样的力叫做\textbf{回复力}.

对于水平方向的弹簧振子来说, 小球做简谐运动的回复力是弹簧对小球的弹力.大小是$F = kx$, 其中
$k$是弹簧的劲度系数.

理论上可以证明, 做简谐运动的物体都受到这样的回复力的作用, 它的大小与物体相对平衡位置的
位移成正比, 方向与位移方向相反, 这个力可以用$$F = -kx$$表示.其中符号表示$F$与$x$反向.
反过来说, \textbf{如果物体在运动方
向上所受的力与它偏离平衡位置位移的大小成正比, 并且
总是指向平衡位置, 物体的运动就是简谐运动.}

应当指出的是, 类似于向心力的概念, 回复力是一种效果力, 而不是拉力, 弹力这些性质力.
在分析物体所受的力时, 我们只需考虑物体实际所受的性质力, 不用再外加一个``回复力''.
这一点我们应该在学习向心力的时候就已经清楚了.

我们知道, 做简谐运动的物体的周期与其振幅无关. 事实上, 更进一步的研究表明, 
力学中一切做简谐运动的物体, 其往复运动的周期$T$均符合
\begin{equation}
    T = 2\uppi\sqrt{\frac{m}{k}},\label{简谐运动周期公式}
\end{equation}
其中$m$是做简谐运动的物体质量, $k$为回复力表达式中的比例系数.

有了回复力的概念, 我们重新来定义简谐运动中的平衡位置. 做简谐运动的物体, 所受回复力为0时
所在的位置, 叫做简谐运动的\textbf{平衡位置}.
事实上, 当物体处于平衡位置时, 所受的回复力一定为0, 但合力却不一定为0. 接下来要学习的
单摆就给出了一个例子.

由牛顿第二定律可知, 做简谐运动的物体, 其加速度大小也与位移成正比,
方向与位移方向相反. 另一方面, 加速度是速度的瞬时变化率, 也就是位移关于时间的二阶导数. 
如果物体做简谐运动, \eqref{位移函数} 式是物体的位移关于时间的函数, 那么
该物体在时间$t$的加速度
$$a = x^{\prime}(t) = -A\omega^2\sin(\omega t+\varphi_0).$$

\setlength{\abovedisplayskip}{0pt}
\setlength{\belowdisplayskip}{0pt}

\subsection{单摆}

生活中经常可以看到悬挂起来的物体在竖直平面内摆动.将一个小球用细线悬挂起来,把它拉离
最低点一段距离,然后放开, 小球就会来回摆动. 如果细线的长度不可改变, 细线的质量与小球相比可
以忽略, 球的直径与线的长度相比也可以忽略; 与小球受到的重力及
线的拉力相比,空气等对它的阻力可以忽略, 这样的装
置就叫做\textbf{单摆}.单摆是一种理想化模型.显然,
单摆摆动时摆球在做振动, 那么单摆的运动是否是简谐运动呢?

\subsubsection{单摆的运动}

\begin{wrapfigure}{r}{5cm}
    \flushright
    \includegraphics[width=0.2\textwidth]{pic/2.2-1.pdf}
    \label{2.2-1}
\end{wrapfigure}

如图, 单摆摆长为 $l$,摆球质量为 $m$. 将摆球拉离
平衡位置 $O$ 后释放, 摆球沿圆弧做往复运动. 当摆球沿圆
弧运动到某一位置 $P$ 时, 摆线与竖直方向的夹角为$\theta$. 此时
摆球受到重力 $G$ 和摆线拉力 $F_\text{T}$ 的作用. 重力 $G$ 沿圆弧切线
方向的分力 $$F = mg\sin\theta.$$ 正是这个力充当回复力, 迫使摆
球回到平衡位置 $O$. 而重力沿绳方向的分力$F_1$与绳子的拉力$F_\text{T}$
为摆球做圆周运动提供向心力.
需要注意的是, 与弹簧振子不同, 单摆在平衡位置的合力并不为0.因为摆球在
竖直平面内做圆周运动, 所以在$O$点处绳子的拉力略大于小球的重力.这也说明,
做简谐运动的物体, 其平衡位置是回复力为0的位置,而非合力为0的位置.

回复力 $F$ 与摆球从 $O$ 点到 $P$ 点的位移 $x$ 并不成正比也不
反向.这样来看, 单摆似乎并不做简谐运动. 
但是, 当摆角 $\theta$ 很小时, 摆球运动的圆弧可以看成
直线, 可认为 $F$ 指向平衡位置 $O$, 与位移 $x$ 反向. 圆弧$\overset{\LARGE{\frown}}{OP}$
的长度可认为与摆球的位移 $x$ 大小相等, 即
$$\sin\theta \approx \theta = \frac{\overset{\LARGE{\frown}}{OP}}{l} \approx \frac{x}{l}.$$
因此, 回复力$F$就可以表示为$$F = -\frac{mg}{l}x.$$
式中符号表示回复力方向与位移$x$方向相反.对于一个确定的单摆来说, 
$\displaystyle\frac{mg}{l}$是一个确定的值.于是上式符合$F = -kx$的形式. 所以,单摆在摆角很小的情况下做简谐运动.

\setlength{\abovedisplayskip}{5pt}
\setlength{\belowdisplayskip}{5pt}
\subsubsection{单摆的周期}
一条短绳系一个小球, 它的振动周期较短. 悬绳较长的
秋千, 周期较长. 实验表明: 单摆做简谐运动的周期与摆长有关, 摆长
越长, 周期越大; 单摆的周期与摆球质量和振幅无关.

那么, 单摆的周期与摆长有什么定量关系呢?
荷兰物理学家惠更斯进行了详尽的研究, 发现单摆做简谐运动的周
期 $T$ 与摆长 $l$ 的二次方根成正比, 与重力加速度 $g$ 的二次方
根成反比, 而与振幅和摆球质量无关. 惠更斯确定了计算
单摆周期的公式
\begin{eqnarray}
    T = 2\uppi\sqrt{\frac{l}{g}}.\label{单摆周期公式}
\end{eqnarray}

这一周期公式同样符合简谐运动的周期公式 \eqref{简谐运动周期公式}. 
事实上, 把$k = \displaystyle\frac{mg}{l}$代入 \eqref{简谐运动周期公式},
就得到了上式.

\subsubsection{实验:用单摆测量重力加速度}

惠更斯在推导出单摆的周期公式后, 用一个单摆测出
了巴黎的重力加速度.我们也可以采用同样的办法, 测量
所在地区的重力加速度数值. 

当摆角较小时, 单摆做简谐运动, 根据其周期公式 \eqref{单摆周期公式}
可得
$$g = \frac{4\uppi^2l}{T^2}.$$
为了提高实验精度, 我们一般要求摆角$\theta<5^{\circ}$.

将细线穿过球上的小孔并打结固定, 然后把细线上端固定在铁
架台上, 就制成一个单摆. 在实验过程中,需要注意以下问题.
\begin{enumerate}
    \item 应选择密度较大的摆球, 最好是金属小球;
    \item 摆球要在同一竖直平面内摆动, 不能形成圆锥摆;
    \item 要从摆球经过平衡位置时开始计时,因为此处摆球速度最小, 计时误差最小;
    \item 通常测量多次全振动的时间来计算周期, 并且在数``0''的同时按下秒表, 此后
    每当摆球从同一方向经过平衡位置时计数1次.
    \item 摆长为绳长加摆球半径, 不要加摆球直径或只算绳长.
\end{enumerate}

本实验可以采用图像法来处理数据, 即用横轴表示摆长$l$,用纵轴表示$T^2$.
改变摆长, 多次测量周期$T$, 进行描点连线.这样得到的数据理论上为一条过原点的直线. 如果错把绳长作为摆长计算, 则直线
在纵轴上的截距为正; 如果把摆长算成绳长加摆球直径, 则直线在纵轴上的截距为负.
但无论如何, 直线的斜率均为$$k = \frac{4\uppi^2}{g}.$$即测量的重力加速度为准确值.

\subsection{外力作用下的振动}

\subsubsection{固有振动}

通过对弹簧振子及单摆的研究,我们知道弹簧振子
与单摆在没有外力干预的情况下做简谐运动,周期或频
率与振幅无关,仅由系统自身的性质决定. 我们把这种
振动称为\textbf{固有振动}, 其振动频率称为\textbf{固有频率}.

如果系统受到外力作用, 它将如何运动?

\subsubsection{阻尼振动}

由于实际的振动系统都会受到摩擦力, 黏滞力等阻
碍作用, 在振动过程中要不断克服外界阻力做功, 消耗能量, 
振幅必然逐渐减小. 这种振幅随时间逐渐减小的
振动称为\textbf{阻尼振动}.当阻力很小时, 在不长的时间内看不出
明显的振幅减小, 这样我们就可以把它当作简谐运动处理.

细化来说, 振动系统能量衰减的方式可以分为两种.一种是由于振
动系统受到摩擦阻力的作用,使振动系统的机械能逐渐转
化为内能. 例如单摆运动时受到空气的阻力. 

另一种是由于振动系统引起邻近介质中各质点的振动,使能量向四周
辐射出去, 从而自身机械能减少. 例如音叉发声时, 一部
分机械能随声波辐射到周围空间, 导致音叉振幅减小.

\subsubsection{受迫振动}
阻尼运动最终要停下来, 那么怎样才能产生持续的振动呢?
最简单的办法是使周期性的外力作用于振动系统, 
外力对系统做功, 补偿系统的能量损耗, 使系统的振动维
持下去. 这种周期性的外力. 叫做\textbf{驱动力}, 系统在驱动力作
用下的振动. 叫做\textbf{受迫振动}.机器运转时
底座发生的振动, 扬声器纸盆的振动, 都是受迫振动.

大量实验证明: \textbf{物体做受迫振动时, 振动稳定后的频率等于驱动力的频率, 与物体的
固有频率没有关系.}

\subsubsection{共振}

\begin{wrapfigure}{r}{5cm}
    \flushright
    \includegraphics[width=0.22\textwidth]{pic/2.3-1.pdf}
    \label{2.3-1}
\end{wrapfigure}

我们知道, 做简谐运动的物体, 其运动周期与振幅无关. 
如果物体在周期性变化的驱动力作用下振动, 物体的振幅是否也与其固有频率无关呢? 

实验结果告诉我们, 物体在做受迫振动时, 驱动力的频率与物体的
固有频率相差越小, 受迫振动的振幅越大; 当驱动力的频
率与物体的固有频率相等时, 受迫振动的振幅达到最大.

右图反映了受迫振动振幅 $A$ 与驱动力频率 $f$ 之间的
关系. 图中 $f_0$ 等于物体的固有频率, 可以看出, 当驱动力
的频率等于固有频率时, 物体做受迫振动的振幅达到最大
值, 这种现象称为\textbf{共振}.

\section{机械波}

\subsection{波的产生和传播}

\subsubsection{波的形成}

取一条较长的软绳, 用手握住一端拉平后
向上抖动一次, 可以看到绳上形成一个凸起部分, 这个凸
起部分向另一端传去. 向下抖动一次, 可以看到绳上形成
一个凹下部分, 这个凹下部分也向另一端传去. 连续向上, 
向下抖动长绳, 可以看到一列波产生和传播的情形.

在绳上做个标记, 在波传播的过程中, 这个标记
怎样运动? 它是否随着波向绳的另一端移动? 

仔细观察会发现, 这个标记只是在上下
振动, 没有向前运动. 相应地, 绳子上的各个点都只是在
上下运动, 但振动却传播出去了. 
振动的传播称为\textbf{波动}, 简称\textbf{波}.

一条绳子可以分成一个个小段, 一个个小段可以看作一个个相连的质点.
这些质点之间存在着相互作用. 当手握住绳子上下振动时, 绳端带动相邻的质点,
使它也上下振动; 这个质点又带动更远一些的质点, 绳子上的质点逐渐都跟着振动起来,
只是后面的质点总比前面的质点迟一些开始振动,也就是说, 
\textbf{后一个质点振动的相位总比前一个质点落后一些}.

这样,绳端这种上下振动的状态就沿绳传播出去了, 从整体上看, 就是一些凹凸相间的波形.

\subsubsection{横波\bre 纵波}

前面提到的, 在绳子上传播的波中, 质点上
下振动, 波向右传播, 二者的方向相互垂直. 像这样, 
质点的振动方向与波的传播方向相互垂直的波, 叫做\textbf{横波}.
在横波中, 凸起的最高处叫做\textbf{波峰},
凹下的最低处叫做\textbf{波谷}.

下面我们来看另外一种波. 

将一根长而软的弹簧水平放置在光滑平
面上, 在左端沿弹簧轴线方向不断推拉弹簧, 观察到, 
弹簧圈密集的部分和稀疏的部分交替
向右传播, 在弹簧上形成一种与横波不一样的波.

这种波又是如何形成的呢? 我们把一系列弹簧圈看成
一系列质点, 它们之间由弹力联系着. 手执弹簧一端左右
振动起来以后, 近端的质点依次带动远端的质点左右振动, 
但后一个质点总比前一个质点迟一些开始振动. 这样, 
弹簧一端左右振动的状态就沿弹簧传播开来. 从整体上看, 
就形成了疏密相间的波.

在这个例子中, 弹簧上的质点左右振动, 波向右传播, 
二者的方向在同一直线上. 质点的振动方向与波的传播方
向在同一直线上的波, 叫做\textbf{纵波}.
在纵波中, 质点分布最密的位置叫做\textbf{密部},质点分布最疏的
位置叫做\textbf{疏部}. 

声波是一种纵波. 发声体振动时也带动空气振动, 在空气中产生纵波.
例如振动的音叉, 它的叉股向一侧振动时, 压缩邻近的空气, 使
这部分空气变密, 叉股向另一侧振动时, 又使这部分空
气变得稀疏. 这种疏密相间的状态向外传播就形成声波. 
声波传入人耳, 使鼓膜振动, 就引起声音的
感觉. 声波不仅能在空气中传播, 也能在液体或固体中
传播, 但不能在真空中传播. 

\subsubsection{机械波}
绳上和弹簧上的波是在绳和弹簧上传播的, 水波是在
水面传播的, 声波通常是在空气中传播的. 绳, 弹簧, 水,
空气等是波借以传播的物质, 叫做\textbf{介质}. 

组成介质的质点之间有相互作用, 一个质点的振动会引起相
邻质点的振动. 机械振动在介质中传播, 形成了\textbf{机械波}.
声音不能在真空中传播, 是由于真空中没有能使声波传播的介质, 
因此我们说, 声音的传播需要介质, 实质上是振动的传播需要介质.

\textbf{介质中有机械波传播时, 介质本身并不随波一起传播.}
例如绳上或弹簧上有波传播时, 它们的质点在各自的平衡位置附近发生振动, 但
并不随波迁移, 传播的只是振动这种运动形式.

介质中本来静止的质点, 随着波的传来而发生振动, 
这表示它获得了能量. 这个能量是从波源通过前面的质
点依次传来的, 所以波在传播``振动''这种运动形式的
同时, 也将波源的能量传递出去. 因此, 
\textbf{波是传递能量的一种方式}.

波不但传递能量, 而且可以传递信息. 我们用语言进
行交流, 就是利用声波传递信息.

\subsection{波的描述}
\subsubsection{波长\bre 频率 \bre 波速}

在波动中, 各个质点的振动周期或频率是相同的, 它
们都等于波源的振动周期或频率, 这个周期或频率也叫做
波的周期或频率. 所以, \textbf{波的频率由波源决定, 与介质无关.}

在波的一个周期中, 介质上一个质点$P$的振动传到更远一些
的另一个质点$Q$, 使$Q$开始振动. 这时质点$P$恰好结束了一次全振动
而开始下一次全振动, 此后, 这两个质点的振动步调总是一致, 
也就是说, 质点$P$和$Q$的相位相同, 它们在任何时刻相对平衡位置的
位移的大小和方向总是相同的.

在波的传播方向上, 振动相位总是相同的两个相邻质
点间的距离, 叫做\textbf{波长}, 通常用 $\lambda$ 表示.
在横波中, 两个相邻波峰或两个相邻波谷之间的距离
等于波长; 在纵波中,两个相邻密部或两个相邻疏部之间
的距离等于波长.

由质点 $P$ 发出的振动, 经
过一个周期传到质点 $Q$, 也就是说, 经过一个周期 $T$, 振
动在介质中传播的距离等于一个波长 $\lambda$, 所以机械波在介质
中传播的速度为$$v = \frac{\lambda}{T}.$$
由于$f = T^{-1}$, 上式也可以写成
\begin{empheq}[box=\fbox]{equation*}
    v = \lambda f.
\end{empheq}

\textbf{机械波在介质中的传播速度由介质本身的性质决定, 
与波源的振动无关.} 在不同的介质中, 波速是不同的. 
以声速为例, 声波在空气中传播的速度为$332\unit{m/s}$, 
在水中传播的速度为$1450\unit{m/s}$, 
在玻璃中传播的速度为$5000$至$6000\unit{m/s}$, 在橡胶中
传播的速度为$30$到$50\unit{m/s}$.

\subsubsection{波的图像}

\begin{wrapfigure}{r}{9cm}
    \flushright
    \includegraphics[width=0.47\textwidth]{pic/2.4-1.pdf}
    \label{2.4-1}
\end{wrapfigure}

过去我们研究的是单个质点的运动情况, 用 $x-t$ 图像可
以很方便地描述质点在任意时刻的位移. 而波却是很多质
点的运动, 在同一时刻各个质点的位移都不尽相同, 不方
便用 $x-t$ 图像来描述. 能否用其他的图像来描述波呢? 我们
以横波为例研究波的图像.

一条波动的绳子, 它上面的质点呈现凹凸相间的波形.
在某一时刻拍一张照片, 照片记录了绳上各质点在该时刻的具体位置.
如果建立直角坐标系, 把该时刻绳上各质点的具体位置反映在坐标系中, 
就可以得到这一时刻绳子上\textbf{波的图像}, 也叫\textbf{波形图}.

在上图中, 用横坐标 $x$ 表示在波的传播方向上绳中
各质点的平衡位置, 纵坐标 $y$ 表示某一时刻绳中各质点偏
离平衡位置的位移. 我们规定, 位移向上时 $y$ 取正值, 向
下时 $y$ 取负值. 这样, 该图像看起来跟``照片''中的波形一致.

如果波的图像是正弦曲线, 这样的波叫做正弦波, 也
叫\textbf{简谐波}.可以证明, 介质中有简谐波传播时, 
介质上的质点做简谐运动.

简谐波的波形图与质点的振动图像都是正弦曲线, 但
它们的意义是不同的. 波形图表示介质中的``各个质点''
在``某一时刻''的位移, 振动图像则表示介质中``某一质
点''在``各个时刻''的位移.
形象地来比喻, 波形图是记录着许多人在某一时刻的集体照片, 
而振动图像是一个人在一段时间内的录像带.
可以说, \textbf{波形图研究某时刻所有质点的空间分布规律, 
振动图像研究一个质点的位移随时间的变化规律.}

\subsubsection*{判断波形图上任意点的振动方向}

要确定质点$P$的振动方向, 首先要明确波的传播方向, 确定波源在哪个
位置. 如上图所示, 假设波向右传播, 那么我们逆着波的传播方向, 
在波形图上往前找一点(即向左找一点, 这点与$P$点的距离不超过四分之一个波长).
分析波的传播过程可知, 如果这一点在$P$点的上方, 那么$P$点在下一时刻
将向上振动; 如果这一点在$P$点的下方, 那么$P$点在下一时刻
将向下振动. 在上图中, $P$点左边的点在$P$点上方, 所以$P$点
向上振动; 如果波向左传播, 那么$P$点向下振动.

\begin{wrapfigure}{r}{7cm}
    \flushright
    \includegraphics[width=0.35\textwidth]{pic/2.4-2.pdf}
    \label{2.4-2}
\end{wrapfigure}

下面再介绍另外一种判断方法.我们可以作出微小时间$\Delta t$后的波形,
这样就知道了各质点在$\Delta t$时间后的位置, 振动的方向也就知道了.
右图虚线所示波形是实线在$\Delta t$后的波形, $P^{\prime}$是$P$点
在$\Delta t$时间后的位置, 由此可知$P$点向上振动.

\subsubsection*{关于波动与振动的几个推论}

(1) 在波的传播方向上, 当两质点平衡位置的距离为波长的整数倍(即半波长的偶数倍)时, 
这两个质点的速度和位移总是相同的, 即振动步调相同(或相位相同). 反之, 当两个质点
的相位相同时, 它们平衡位置之间的距离一定是整数倍的波长, 
即$$\Delta x = n\lambda, n = 1,2,3,\cdots, $$
或者$$\Delta x = \frac{n\lambda}{2}, n = 2,4,6,\cdots.$$

(2)在波的传播方向上, 当两质点平衡位置的距离为半波长的奇数倍时, 
这两个质点的速度和位移总是大小相等, 方向相反的, 即振动步调相反. 
反之, 当两个质点
的振动步调相反时, 它们平衡位置之间的距离一定是半波长的奇数倍,即 
$$\Delta x = \frac{n\lambda}{2}, n = 1,3,5,\cdots.$$

(3)介质中任一质点起振的方向必然与波源起振的方向一致. 因此可以说,
波最前沿的质点的振动方向就是波源的起振方向.

(4)对于简谐振动与简谐波而言, 无论计时起点如何, 经过一个周期$T$, 
质点都完成了一次完整振动, 其路程为$4A$($A$为振幅或波峰的纵坐标);
相应地, 无论计时起点如何, 经过$\displaystyle\frac{T}{2}$后, 
质点相对于平衡位置的位移与速度都与计时开始时等大反向, 其路程恰好是$2A$.
但是, 经过$\displaystyle\frac{T}{4}$后, 质点所经过的路程却不是定值, 
并不必然等于$A$.

\subsection{波的性质}

\subsubsection{波的反射}
我们以水波为例来研究波的反射.

在水槽中一端有一振动发生器, 振动发生器在水槽中能够产
生水波. 在水槽中斜向放置一块挡板,
当水波遇到挡板时会发生\textbf{反射}. 
如果用一条射线代表水波的入射方向(入射
线), 用另一条射线代表水波的反射方向(反射线), 我们
发现水波的反射与初中学过的光的反射遵循同样的规律:
\textbf{反射线,法线与入射线在同一平面内, 反射线与入射线分
居法线两侧, 反射角等于入射角.}
即:\textbf{三线共面, 两线分居,两角相等}.

\subsubsection{波的折射}

我们知道光从一种介质进入另一种介质时会发生折射,
机械波会发生折射吗? 理论和实验证明, 一切波都会发生
折射现象. 一列水波在深度不同的水域传播时, 在交界面
处将发生\textbf{折射}.

水波发生折射的根本原因是, 深度不同的水域, 相当于不同的介质;
而不同介质的波速$v$不同. 深水域的水波波速较小, 浅水域的水波波速较大.

波的频率$f$只由波源决定, 在
折射过程中不会变化; 由$v = \lambda f$可知, 波长$\lambda$在折射过程中发生变化.

波的折射定律可以表述如下:

\begin{enumerate}
    \item \textbf{反射线,法线与入射线在同一平面内, 反射线与入射线分
    居法线两侧;}
    \item \textbf{波从波速较大的介质射入波速较小的介质中, 折射角小于入射角; 
    波从波速较小的介质射入波速较大的介质中, 折射角大于入射角.}
\end{enumerate}

事实上, 折射角和入射角存在一个定量关系. 在学习光的折射时, 我们再详细给出这一关系.

\subsubsection{惠更斯原理}

我们知道, 波是波源的振动在介质中的传播.从宏观上看, 波是往前不断推进的一个过程:
后面的波推前面的波, 前面的波再往前走, 对于水波来说, 就是``长江后浪推前浪''.
从微观上看, 一个波源形成一个波面, 旧的波面再产生新的波面, 这又是怎样的一个过程呢?
如果前面的波面遇到了障碍物, 它以后的运动形式是怎么样的? 对于这个问题, 我们可以用惠更斯原理来解释.

假设水面上的某一点有一个波源, 水波向四周传开.由于向各个方向的波速都一样, 
因此向四面八方传播的波峰组成了一个个圆, 波谷也组成了一个个圆.
实际上, 任何振动状态相同的点都组成了一个个圆.
我们把这些圆叫做一个个\textbf{波面}, 而与波面垂直的那些线, 代表波的传播方向, 叫做\textbf{波线}.

\subparagraph{惠更斯原理} 介质中任意波面上的各点, 都可以看作发射子波的波源.
而在其后的任意时刻, 这些子波在波的传播方向的包络面, 就是新的波面.

惠更斯原理不是从某个定律推导出来的, 也不是直接由实验总结出来的. 它之所以正确, 是因为由它得出来的结论
都与事实相符.

\subsubsection{波的衍射}

波在传播过程中如果被较大的障碍物挡住, 就会在障碍物的后方形成一个没有波的阴影区.
但是当障碍物较小时, 波能绕过障碍物的边缘, 在障碍物的阴影区内继续传播.
这种波绕过障碍物继续传播的现象叫做波的\textbf{衍射}.

在波的传播方向上放上两块挡板,挡板中间留有适当的空隙. 
如果波只是沿着直线传播, 那么应该只有空隙后面才有直线传播.
而实际上,在挡板后面本来不应该有波的区域却能看到波的传播, 这也是波的衍射现象.

在不改变波源的情况下, 将挡板的缝隙调整得很大, 我们会发现波基本上是沿着直线传播的.
在挡板后方, 波的传播被挡板挡住了, 衍射现象不明显.

接下来,缩小挡板的缝隙, 此时缝隙的大小与波长近似, 在挡板的后方就出现了明显的衍射现象. 

我们继续减小挡板的缝隙, 将缝隙减小到比波长还要小. 挡板后面波的传播范围不但没有减小, 反而扩大了, 
也就是说波的衍射现象变得更加明显了.

实验证明: \textbf{当障碍物或孔的尺寸比波长小, 或者与波长相差不大的时候, 
波会产生明显的衍射现象.}

不只是水波, 声波也能发生衍射. 通常的声波, 波长为 1.7\ cm到
17\ m, 跟一般障碍物的尺寸相当, 所以声波能绕过一般的
障碍物, 使我们能听到障碍物另一侧的声音.

一切波都能发生衍射.衍射是波特有的现象.

\subsubsection{波的干涉}

在介质中常常有几列波同时传播.两列波相遇时,会
不会像两个小球相碰时那样,改变各自的运动特征呢?

事实表明, \textbf{几列波相遇时能够保持各自的运动特征
继续传播而互不干扰. 在它们重叠的区域里, 介质的质点同时参与这
几列波引起的振动, 质点的位移等于这几列波单独传播时引起的位移的矢量和.}
这就是\textbf{波的叠加原理}.

如果相遇的两列波周期相同, 在它们重叠的区域里会发
生什么特别现象吗?

在水槽外放置一个振动片, 振动片上固定两根相同的细杆. 振动片振动时, 两根细杆周期性地击打水面,
形成两个波源. 这两列波的频率相同, 振动方向也相同. 由于两根细杆是同步振动的, 所以它们振动的相位
差恒为0.

这两列波相遇后, 水面上出现了一条条相对平静的区域和激烈振
动的区域, 并且这两类区域在水面上的位置是稳定的.

\begin{figure}[htbp]
    \centering
    \includegraphics[width=13cm]{pic/3.3-1.jpg}
    \label{3.3-1}
\end{figure}

怎样解释上面的现象呢?如上图所示, 
用两组同心圆表示从波源发出的两列波, 
蓝线圆表示波峰, 黑线圆表示波谷. 在一组同心圆中, 相
邻的蓝黑线圆间的距离等于半个波长, 
同色线圆之间的距离等于一个波长.

\subparagraph{加强点} 如果在某一时刻, 两列波的波峰与波峰相遇(比如上图中的$M$点), 也就是说, 两列波
在这一点引起的振动具有相同的相位. 那么经过半个周期后, 一定是
波谷和波谷在这一点相遇.

根据波的叠加原理, 波峰与波峰相遇时, 两列波引起的振动在$M$点叠加, 
引起的位移是两列波的振幅之和, 并且在这一点向上的位移最大;
波谷与波谷相遇时, 引起的位移也是两列波的振幅之和, 
在这一点向下的位移最大.由于两列波在这一点
的相位相同, 所以他们引起的振动总是相互增强而非抵消, 我们把这样的
点叫做振动的\textbf{加强点}.

\subparagraph{减弱点} 如果在某一时刻, 两列波的波峰与波谷相遇
(比如上图中的$N$点), 也就是说, 两列波
在这一点引起的振动相位相反. 那么经过半个周期后, 仍然是
波谷和波峰在这一点相遇.两列波的振动在这一点相互抵消, 
引起的位移相互削弱.如果两列波的振幅相同, 那么质点
的位移之和就总等于0.由于两列波在这一点
的相位相反, 所以他们引起的振动总是相互削弱, 我们把这样的
点叫做振动的\textbf{减弱点}.

\subparagraph{加强区与减弱区} 在上图中标出波峰与波峰, 波谷与波谷相遇的点, 
并将它们连起来(如上图中实线), 根据前面的分析可知, 这些点的振动总是相互增强, 是振动的加强点.
事实上, 在这些线上的点都是振动的加强点, 我们称这些线为振动的加强区.

类似地, 标出波峰与波谷相遇的点, 并将它们连起来(如上图中虚线), 这些线
上的点振动相互削弱, 是振动的减弱点, 我们称这些线为振动的减弱区.

\subparagraph{波的干涉与干涉条件} 通过理论分析可知, 
频率相同,相位差恒定,振动方向相同的两列
波叠加时, 某些区域的振动总是加强, 某些区域的振动总
是减弱, 这种现象叫做波的\textbf{干涉}. 形成的这
种稳定图样叫做干涉图样.

需要注意的是, 我们说的波的增强与减弱, 是指能量的增强与减弱, 
由叠加波的振幅$A$体现, 但位移可以是$-A$到$A$的任意值.

不仅水波会发生干涉现象, 声波, 电磁波等一切波, 
只要满足上述条件都能发生干涉. 跟衍射一样, 干涉也是
波特有的现象.主动降噪耳机就是利用了波的干涉原理, 
在耳机内产生与噪声波相互削弱的声波, 使我们听不到噪音.

\subsubsection{多普勒效应}

仔细听急救车的鸣笛声, 会发现一个现
象: 当车从你身边疾驰而过的时候, 鸣笛的音
调会由高变低. 这是怎么回事呢?

1842 年, 奥地利物理学家多普勒带着女儿在铁道旁散
步时就注意到了类似上面描述的现象. 他经过认真的研究, 
发现波源与观察者相互靠近或者相互远离时, 接收到的波
的频率都会发生变化. 人们把这种现象叫做\textbf{多普勒效应}.

为了理解多普勒效应, 我们做如下模拟实验: 让一列人沿街行走, 
观察者站在侧面不动, 假设每分钟有30人从他身体通过, 这种情况下
的``过人频率''是30人每分钟. 
如果现在观察者逆着队伍行走, 每分钟与观察者相遇的人数增加, 也就是频率
增加; 反之, 如果观察者顺着队伍走, 频率降低.

对于机械波也有类似的结论:
\begin{enumerate}
    \item 波源与观察者相对静止时, 接收频率等于波源频率;
    \item 波源与观察者相互靠近时, 接收频率大于波源频率;
    \item 波源与观察者相互远离时, 接收频率小于波源频率.
\end{enumerate}

多普勒效应在科学技术中有广泛的应用.
交通警察向行进中的车辆发射频率已知的超声波, 同时测量反射波的
频率, 根据反射波频率变化的多少就能知道车辆的速度.这种
测速仪叫做多普勒测速仪.

医生向人体内发射频率已知的超声波, 超声波被血管
中的血流反射后又被仪器接收. 测出反射波的频率变化, 
就能知道血流的速度.这种方法俗称``彩超''.

理论和实验都证明, 光波或电磁波都有多普勒效应, 
宇宙中的星球都在不停地运动.测量星球上某些元素发出的光波的频
率, 然后与地球上这些元素静止时发光的频率对照, 就可
以算出星球靠近或远离我们的速度.

\section{光}

\subsection{光的折射}

\begin{wrapfigure}{r}{7cm}
    \flushright
    \includegraphics[width=0.35\textwidth]{pic/4.1-1.pdf}
    \label{4.1-1}
\end{wrapfigure}

阳光照射水面时,我们能够看到水中的鱼和草,同时也
能看到太阳的倒影,这说明: 光从空气射到水面时, 一部分
光射进水中, 另一部分光返回到空气中. 一般来说, 光从第
1 种介质射到该介质与第 2 种介质的分界面时, 一部分光会
返回到第 1 种介质, 这个现象叫做\textbf{光的反射}; 另一部分光会
进入第 2 种介质, 这个现象叫做\textbf{光的折射}.

初中时我们已经学过光的反射定律, 下面我们研究光的折射.

\subsubsection{折射定律}

让一束光由一种介质斜着射向另一种介
质, 例如, 从空气射向水中, 入射光线与法线的夹角 $\theta_1$ 称
为入射角, 折射光线与法线的夹角 $\theta_2$ 称为折射角.

1621 年, 荷兰
数学家斯涅耳找到了两者之间
的关系, 并把它总结为光的\textbf{折射定律}:

(1) \textbf{折射光线与入射光线, 法线处在同一平面内;}

(2) \textbf{折射光线与入射光线分别位于法线的两侧; }

(3) \textbf{入射角的正弦与折射角的正
弦成正比, 即}
\begin{empheq}[box=\fbox]{equation}
    \frac{\sin\theta_1}{\sin\theta_2} = n_{12}.
    \label{折射定律}
\end{empheq}

其中$n_{12}$是比例常数, 与两角的大小无关, 只与两种介质的性质有关.
我们在初中学过的透镜就是根据光的折射原理制成的.

事实表明, 与光的反射现象一样, 在
光的折射现象中, 光路也是可逆的.

\subsubsection{折射率}
下面我们主要讨论光从真空射入介质的情形, 这时, 我们把 \eqref{折射定律}
中的$n_{12}$简单记为$n$.

对于不同的介质来说, $n$的值是不同的. 可见$n$与介质有关系, $n$的值越大,
光从真空斜射入这种介质时, 偏折的角度越大.

光从真空射入某种介质发生折射时, 入射角的正弦与
折射角的正弦之比, 叫做这种介质的绝对折射率, 简称\textbf{折
射率}, 用符号 $n$ 表示.

我们规定真空的折射率为
1, 空气的折射率近似为 1(为1.00028), 水的折射率为 1.33, 
玻璃的折射率为1.5到1.8.

由折射率的定义以及光的折射定律 \eqref{折射定律} 可以得到, 
当光从折射率为$n_1$的介质射入折射率为$n_2$的介质时, 
折射角$\theta_2$与入射角$\theta_1$的关系为
$$n_1\sin\theta_1 = n_2\sin\theta_2.$$

研究表明, 光在不同介质中的传播速度不同; \textbf{某种介
质的折射率, 等于光在真空中的传播速度 $c$ 与光在这种介
质中的传播速度 $v$ 之比}, 即
\begin{empheq}[box=\fbox]{equation}
    n = \frac{c}{v}.
    \label{速率关系}
\end{empheq}

由于光在真空中的传播速度 $c$ 大于光在任何介质中的
传播速度 $v$, 因而任何介质的折射率 $n$ 都大于 1. 所以, 光
从真空射入任何介质时, $\sin\theta_1$ 都大于 $\sin\theta_2$, 
即入射角总是大于折射角.

因为光在空气中的传播速度与在真空中的传播速度相近, 
所以空气的折射率近似为1. 在实际应用中, 我们遇到最多的情况是光从空气射入某种介质, 或光从某种介质
射入空气.而空气对光的传播影响很小, 可以作为真空处理.

\subsection{光的全反射}
对于折射率不同的两种介质, 我们把折射率较小的介质叫做\textbf{光疏介质}, 
折射率较大的介质称为\textbf{光密介质}.光密介质与光疏介质是相对的.
光从光疏介质射入光密介质时, 折射角小于入射角;光从光密介质射入光疏介质时, 
折射角大于入射角.

光从光密介质射入光疏介质时, 在界面处有一部分光\textbf{反射回}原介质中, 另一部分光
\textbf{折射到}光疏介质中. 实验表明, 随着入射角的增大, 折射角也会逐渐增大, 折射光线与
法线越来越远, \textbf{且折射光线越来越弱, 反射光线越来越强}. 

当入射光增大到某一角度$C$, 使折射角达到$90^{\circ}$时, 折射光线会完全消失, 
只剩下反射光线.这种现象叫做\textbf{全反射}, 这时的入射角叫做\textbf{临界角}.

当光从光密介质射入光疏介质时, 如果入射角等于或
大于临界角, 就会发生全反射现象.需要注意的是, 当光从
光疏介质射入光密介质时, 由于折射角小于入射角, 无论入射角如何增大(直至与界面重合),
折射角也不会到达$90^{\circ}$, 因此不会发生全反射.

由光的折射定律 \eqref{折射定律} 可知, 
光从介质射入空气或真空时, 发生全反射的临界角
$C$ 与介质的折射率 $n$ 的关系是
$$\sin C = \frac{1}{n}.$$
从这个式子可以看出, 介质的折射率越大, 发生全反射的临界角越小.
科研和技术中常常通过测量临界角来测定材料的折
射率. 

全反射是自然界中常见的现象. 例如, 水中或玻璃中
的气泡, 看起来特别明亮, 就是因为光从水或玻璃射向气
泡时, 一部分光在界面上发生了全反射的缘故.

生活中利用光的全反射, 我们制作了全反射棱镜, 可用于改变光的方向; 
光导纤维可用于导光. 当光在有机玻璃棒内传播时, 如果从有机玻璃射向
空气的入射角大于临界角, 光会发生全反射, 于是光在有机玻璃棒内
沿锯齿形路线传播. 这就是光导纤维导光的原理.它由内芯和外套两层组成, \textbf{内
芯的折射率比外套的大}, 光传播时在内芯与外套的界面上
发生全反射.

医学上还用这个原理制作了内窥镜. 如果把许多光导纤维聚集成束, 并使两端的排列顺序相同, 
图像就可以从一端传播到另一端. 实际的内窥镜装有两组光纤, 一
组把光传送到人体内部进行照明,另一组把体内的图像传
出供医生观察.

光是一种电磁波. 像无线电波那样, 
光可以作为载体来传播信息. 光纤传输有传输容量大, 衰减小, 抗干扰性及保密性强等多方面的
优点.

\setlength{\abovedisplayskip}{0pt}
\setlength{\belowdisplayskip}{0pt}

\subsection{光的色散与光的电磁理论}

初中时我们学过, 当白光通过三棱镜时, 在光屏上会形成一条彩色光带, 这种现象叫做\textbf{光的色散}.光的色散现象说明:
\begin{enumerate}
    \item 白光是复色光, 由红,橙,黄,绿,蓝,靛,紫这几种色光组成;
    \item 同一种介质对不同颜色的光的折射率$n$不同, 对红光的折射率最小, 
    对紫光的折射率最大.由 \eqref{速率关系} 可知, 在同一种介质中, 红光的传播速度最大, 
    紫光的最小.
\end{enumerate}

前面的学习中, 我们讨论了很多光的现象, 那么光的本质是什么呢?
关于这个问题, 历史上存在不同的观点.19世纪60年代, 麦克斯韦预言了电磁波
的存在, 并认为光也是一种电磁波. 此后, 赫兹在实验中证实了这种假说.
事实上, 光除了具有电磁波的性质, 还具有粒子的性质, 我们将在之后的内容中介绍.

麦克斯韦的电磁场理论在《电与磁》中有系统的介绍, 这里我们不再重复叙述.
下面我们介绍电磁波的特点.

\begin{enumerate}
    \item 电磁波是横波;
    \item 与机械波不同, 电磁波的传播不需要介质, 在真空中也能传播;
    \item 电磁波具有反射, 折射, 干涉, 衍射等一切波的特性;
    \item 电磁波在真空中的传播速度$c = 3\times 10^8\unit{m/s}$, 它与电磁波
    的频率$\nu$, 在真空中的波长$\lambda$的关系是$$c = \lambda\nu.$$其中$\nu$由波源决定. 
\end{enumerate}

按照麦克斯韦的理论, 光也是一种电磁波.光的颜色是由光的频率决定的, 而频率
是由光源决定的, 不受介质影响. \textbf{从红光到紫光, 光的频率依次增大.}

我们知道, 机械波的波速只与介质有关, 而与波的频率无关. 但电磁波并非如此.
实验事实表明, \textbf{光(或电磁波)的频率越大(波长越小), 在同种介质中的传播速率越小, 折射率越大.}

\subsection{光的干涉}

我们知道, 如果两列机械波的频率相同, 相位差恒定, 
振动方向相同, 就会发生干涉. 光是一种电磁波, 那么光也
应该会发生干涉现象.怎样才能观察到光的干涉现象呢?

\subsubsection{光的双缝干涉}
使白光源通过红色滤光片, 分离出红色光. 单色光先通过横向的单缝, 
然后照射在有两个纵向狭缝$S_1$, $S_2$的挡板上. 
我们认为从单缝发出的光是彼此平行的激光束.平行激光束垂直于挡板
通过两个狭缝, 我们在挡板后面的光屏上观察现象.
这个实验称为\textbf{杨氏双缝干涉实验}.

我们发现, 光屏上呈现明暗(红黑)相间的条纹.
如何解释这个现象呢?事实上, 
由于狭缝$S_1$, $S_2$很小, 狭缝就成了两个波源, 
它们的频率, 相位和振动方向总是相同的.
这两个波源发出的光在挡板后面的空间互相
叠加, 发生干涉现象: 来自两个光源的光在一些位置相互
加强, 在另一些位置相互削弱, 因此在挡板后面的屏上得
到明暗相间的条纹.

\begin{wrapfigure}{r}{6cm}
    \flushright
    \includegraphics[width=0.3\textwidth]{pic/4.4-1.png}
    \label{4.4-1}
\end{wrapfigure}

具体来说, 狭缝 $S_1$ 和 $S_2$ 发出的波源, 到屏上 $P_0$ 点的距离相
同. 由于两列波到达 $P_0$ 点的路程一样, 所
以这两列波的波峰或波谷同时到达 $P_0$ 点, 也就是相位仍然
相同. 在这点, 两列波叠加后相互加强, 因此这里出现亮
条纹. 

再考察 $P_0$ 点上方的另外一点 $P_1$ , 它距 $S_2$ 比距 $S_1$ 远一
些, 两列波到达 $P_1$ 点的路程不相同, 两列波的波峰或波谷
不一定同时到达$P_1$. 如果路程差正好是半个波长$\displaystyle\frac12\lambda$, 那
么两列波在这里恰好互相抵消, 于是这里出现暗条纹. 

对于更远一些的 $P_2$ 点, 来自两个狭缝的光波的路程差
更大. 如果路程差正好等于波长 $\lambda$, 那么这里也出现亮条
纹. 距离屏的中心越远, 路程差越大. 每当路程差等于波长的整数倍
$$k\lambda, \ k = 1, 2, 3,\cdots$$
时, 两列光波得到加强, 屏上出现亮条纹; 每当路程差等
于半波长的奇数倍$$\frac{k}{2}\lambda, \ k = 1, 3, 5, \cdots$$时, 
两列光波相互削弱, 屏上出现暗条纹. 

通过几何分析可知, 如果狭缝$S_1$, $S_2$的间距为$d$, 挡板与屏的距离为$l$, 
光的波长为$\lambda$, 那么相邻两条亮条纹或暗条纹的中心间距是
$$\Delta x = \frac{l}{d}\lambda.$$
这就是说, 对于同一套实验装置, 光的波长越大, 干涉条纹的间距越大.例如红光的
干涉条纹间距大于蓝光的干涉条纹间距, 蓝光的干涉条纹更密.

通过这个实验, 我们可以测量光的波长.

\subsubsection{光的薄膜干涉}

把铁丝圈在肥皂水中蘸一下, 让它挂上一层薄薄的液膜. 把这层液膜当作一
个平面镜, 用它观察酒精灯的火焰, 发现火焰呈现彩色条纹, 这是什么原因?

液膜具有厚度, 火焰的像是由液膜的前后两个表面反射的光共同形成的.
来自两个面的反射光相互叠加, 发生干涉, 也称\textbf{薄膜干涉}.
实际上, 由于重力作用, 竖直放置的液膜上薄下厚.从上到下, 两个表面
的反射光有的相互增强, 有的相互减弱, 形成了明暗相间的条纹.
由于火焰的光是复色光, 薄膜上不同颜色的光的条纹的明暗位置不同, 相互交错, 
所以, 看上去会有彩色条纹.

薄膜干涉在技术中有很多应用.比如, 可以用该原理来检测平面的平整度.
把平整的样板放在待检测平面上, 并在一侧垫上一块薄片, 使之形成一定的角度.
用单色光照射, 入射光从两个平面反射出两束光波, 形成干涉条纹. 如果待检测平面平整, 
那么两平面的间距均匀变化, 产生的干涉条纹互相平行; 如果待检测平面不平整, 那里的干涉
条纹就会发生弯曲.

常见光学镜头呈淡紫色, 是因为镜片上涂了增透膜(氟化镁), 膜的厚度经过特别设计.
人眼对绿光最敏感, 各色光射到镜头上时, 膜前后表面反射的绿色光相消, 使得绿色的透射增强, 
以此提高成像质量.由于反射光中缺少绿色, 所以光学镜头看起来呈淡紫色.这种膜称为\textbf{增透膜}.

为了减少紫外线对眼睛的伤害, 生产厂家在眼睛上涂了一层膜. 它的厚度经特别设计, 
使膜前后表面反射的紫外线增强, 从而减弱了紫外线的透射, 这种膜叫做\textbf{增反膜}.

\subsection{光的衍射}

在挡板上安装一个宽度可调的狭缝, 缝后放一个光屏. 
用单色平行光照射狭缝, 我们看到, 当缝比较宽时, 光沿
着直线通过狭缝, 在屏上产生一条与缝宽相当的亮条纹. 
但是, 当缝调到很窄时, 尽管亮条纹的亮度有所降低, \textbf{但
是宽度反而增大了}, 而且两侧出现了对称分布且不等距的明暗相间的条纹.

这表明, 光没有沿直线传播, 它绕过了缝的边缘, 传
播到了相当宽的地方. 这就是光的衍射现象.

从缝不同位置射入的光到达光屏侧边的距离不同, 它们互相增强或减弱, 
形成了明暗相间的条纹, 这是光的干涉现象.

如果用白光做上述实验, 得到的条纹是彩色的.
这是因为白光中包含了各种颜色的光, 衍射时不
同色光的亮条纹的位置不同, 于是各种色光就区分开了.

光在没有障碍物的均匀介质中是沿直线传播的. 对衍
射现象的研究表明, 在障碍物或狭缝的尺寸很大时, 衍射
现象不明显, 也可以近似认为光是沿直线传播的. 但是, 
在障碍物或狭缝的尺寸足够小的时候, 衍射现象十分明显, 
这时就不能说光沿直线传播了.

由于单缝衍射的条纹比较宽, 而且距离中央亮条纹较远的
条纹, 亮度也很低, 所以单缝衍射的实用性较低. 
实验表明, 如果增加狭缝的个数, 衍射条纹的宽度将
变窄, 亮度将增加. 光学仪器中用的\textbf{衍射光栅}就是据此制
成的. 它是由许多等宽的狭缝等距离地排列起来形成的光
学元件. 在一块很平的玻璃上刻出一系列等距的平行刻痕, 
刻痕产生漫反射而不太透光, 未刻的部分相当于透光的狭
缝, 这样就做成了透射光栅. 如果在高反射率
的金属上刻痕, 就可以做成反射光栅. 

\subsection{光的偏振}

\subsubsection{波的偏振}
波分为纵波和横波. 在纵波中, 各点的振动方向总与
波的传播方向在同一直线上. 在横波中, 各点的振动方向
总与波的传播方向垂直.

不同的横波, 即使传播方向相同, 振动方向也可能是
不同的. 这个现象称为\textbf{偏振}. 横波的振动方
向称为\textbf{偏振方向}. 

想象在一条绳上的横波, 它可以通过与振动方向相同的狭缝; 但如果
狭缝的方向与振动方向垂直, 绳上的横波则无法通过. 如果是在一条弹簧上传播
的纵波, 无论狭缝的取向如何, 波都能通过.

\subsubsection{光的偏振}
光的干涉和衍射现象说明光具有波动性. 光是横波还是
纵波呢? 研究表明, 光是一种横波. 我们可以用与上述实验
类似的方法来研究光的偏振. 为此, 利用偏振片代替狭缝.

偏振片由特定的材料制成, 每个偏振片都有一个特定的方向, 沿着这个方
向振动的光波能顺利通过偏振片, 偏振方向与这个方向垂
直的光不能通过, 这个方向叫做透振方向. 偏振片对光
波的作用就像狭缝对于机械波的作用一样.

事实上, 只要光的振动方向不与透振方向垂直, 
它都可以不同程度地通过偏振片, 不过强度要比振
动方向与透振方向平行的光弱一些. 

生活中, 太阳光, 日光灯, LED等普通光源发出的光, 包含着在垂直于传播方向上
沿一切方向振动的光, 而且沿着各个方向振动的光波的强度都相同. 
这种光称为\textbf{自然光}.

自然光在通过偏振片时, 只有振动方向与偏振片的
透振方向一致的光波才能顺利通过. 也就是说, 通过偏振
片的光波, 在垂直于传播方向的平面上, 沿着某个特定
的方向振动. 这种光叫做\textbf{偏振光}.

偏振光并不罕见. 除了从太阳, 白炽灯等光源直接发
出的光以外, 我们通常看到的绝大部分光, 都是不同程度
的偏振光. 自然光的反射光和折射光都是偏振光, 入射角变化时偏
振的程度也有变化.

光的偏振现象有很多应用. 例如, 摄影师在拍摄池中
的游鱼、玻璃橱窗里的陈列物时, 由于水面和玻璃表面的
反射光的干扰, 景象会不清楚. 如果在照相机镜头前装一
片偏振滤光片, 转动滤光片, 让它的透振方向与水面和玻
璃表面的反射光的偏振方向垂直, 就可以减弱反射光而使
水下和玻璃后的景象清晰.

电子表的液晶显示, 3D电影等也都利用了偏振光的原理.

\subsubsection{激光}
光是从物质的原子中发射出来的. 但是, 
普通的光源, 例如白炽灯, 灯丝中某个原子在什么时刻发
光、在哪个方向偏振, 完全是随机的, 发出的光传播方向
各异, 频率也不一定相同, 这导致不同原子发出的光没有
确定的相位差, 这导致两个独立的普通光源发出的光不会发生干涉. 

1960 年, 美国物理学家梅曼率先在实验室中制造出了
频率相同,相位差恒定,振动方向一致的光波, 
这就是\textbf{激光}. 由于上述性质, 激光具有高度的相干性, 这是
它的第一个特点. 

我们前面讲过的双缝干涉实验
和衍射实验, 用激光要比用自然光更容易完成. 因此, 激
光被广泛地应用于生产生活和科学研究中. 

激光还能用来传递信息. 光纤
通信就是激光和光导纤维相结合的产物. 

激光的另一个特点是它的平行度非常好, 在传播很远
的距离后仍能保持一定的强度. 激光的这个特点使它可以
用来进行精确的测距. 

激光的亮度很高, 也就是说, 它可以在很小的空间和
很短的时间内集中很大的能量. 因此, 可以利用激光束来切割、焊接, 
以及在很硬的材料上打孔. 



\end{document}