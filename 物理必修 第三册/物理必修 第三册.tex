\documentclass[12pt,a4paper]{ctexart}

\title{物理必修\ \ 第三册}
\author{啊波呲}

\setlength{\parskip}{0em}
\usepackage{amsmath,mathtools,amssymb,geometry,wrapfig,graphicx,empheq,pifont,enumitem,upgreek}
\renewcommand{\baselinestretch}{1.77}
\geometry{left=1.5cm,right=1.5cm,bottom=2.1cm,top=2.5cm}
\setenumerate[1]{itemsep=1pt,partopsep=0pt,parsep=\parskip,topsep=3pt}
\setitemize[1]{itemsep=1pt,partopsep=0pt,parsep=\parskip,topsep=3pt}

\usepackage{tikz}
\usepackage{xcolor}
\newcounter{exam}[section]
\setcounter{exam}{0}
\newcommand{\bre}{\ \ \ }
\newcommand{\examlabel}{\textbf{例\theexam}}
\newcommand{\soln}{\textbf{解}\bre}
\newcommand{\notes}{\textbf{注意}\bre}
\newcommand{\unit}[1]{\ \mathrm{#1}}

\newenvironment{example}{\bigskip\par\refstepcounter{exam}\examlabel\bre}{\par}
\newenvironment{solution}{\par\soln}{\par\bigskip}

\begin{document}
\maketitle
\pagenumbering{roman}
\tableofcontents

\newpage
\pagenumbering{arabic}

\setlength{\abovedisplayskip}{3pt}
\setlength{\belowdisplayskip}{3pt}

\section{静电场}

\subsection{电荷}
\subsubsection{电荷}

人们发现, 很多物体都会由于摩擦而带电, 并称这种方式为\textbf{摩擦起电}.

美国科学家富兰克林通过实验发
现, 雷电的性质与摩擦产生的电的性质完全相
同, 并命名了\textbf{正电荷}和\textbf{负电荷}.自然界的电荷只有两种.

电荷的多少叫做\textbf{电荷量}, 用$Q$或$q$表示. 在国际单位制中, 它的单位是\textbf{库
    仑}, 简称库, 符号是 C, 定义为1 A恒定电流在1 s 时间间隔内所传送的电荷量为1 C. 因此,
电荷量不属于基本物理量, 它是电流强度$I$和时间$t$的导出物理量, 并且$$Q = It.$$

正电荷的电荷量为正值, 负电荷的电荷量为负值.

\subsubsection{起电}

起电, 就是使物体带电. 起电的本质是电子转移.

摩擦可以使物体带电, 那么, 还有其它方法可以使物体带电吗?

\subparagraph{感应起电} 当一个带电体靠近导体时, 由于电荷间相互吸引或排
斥, 导体中的自由电荷便会趋向或远离带电体, 使导体靠
近带电体的一端带异种电荷, 远离带电体的一端带同种电
荷.这种现象叫做\textbf{静电感应}.利用
静电感应使金属导体带电的过程叫做\textbf{感应起电}.

与摩擦起电不同的是, 摩擦起电的两个物体通常都是绝缘体, 这使电荷会留在绝缘体表面, 产生明显的带电现象.相反,导体的电荷会均匀分布整个物体,
不易观察到带电现象.

\subparagraph{接触起电} 感应起电的两个物体不接触, 电荷仅在导体的内部移动.
除此之外, 两个物体在接触时, 如果它们之间存在电位差, 电荷将会在两个物体间转移,
最终达到动态平衡.

一个不带电的导体通过与一个带电体接触后分开, 从而形成带电体的过程, 称为\textbf{接触起电}.

两个完全相同的导体接触后分开, 它们所带的电荷量相同.因此, 除非这两个物体都不带电, 否则在
接触后将会相互排斥.

\subsubsection{电荷守恒定律}

静电感应过程中导体中的自由电荷只是从导体的一部
分转移到另一部分. 而接触起电过程中自由电荷在几个导体间转移.
也就是说, 无论是接触起电还是感应起电都没有创造电荷, 只是电荷的分布发生了变化.

大量实验事实表明, \textbf{电荷既不会创生, 也不会消灭, 它
    只能从一个物体转移到另一个物体, 或者从物体的一部分转
    移到另一部分;在转移过程中, 电荷的总量保持不变.}这个
结论叫做\textbf{电荷守恒定律}.

电荷守恒定律更普遍的表述是: \textbf{一个与外界没有电荷交换的系统, 电
    荷的代数和保持不变.}

\subsubsection{元电荷}
迄今为止, 实验发现的最小电荷量就是电子所带的电
荷量.质子, 正电子所带的电荷量与它相同, 电性相反.
人们把这个最小的电荷量叫做\textbf{元电荷}, 用$e$表示.

元电荷$e$的数值, 最早是由美国物理学家密立根测得
的.在密立根实验之后, 人们又做了许多测量.现在公认的元电荷$e$的值为
$$e = 1.602 176 634 \times 10^{-19}\ \mathrm{C}.$$
在计算中, 可取
$$e = 1.60 \times 10^{-19}\ \mathrm{C}.$$

电子的电荷量$e$与电子的质量$m_e$之比, 叫做电子的\textbf{比
    荷}.比荷也是一个重要的物理量.电子的
质量$m_e = 9.11\times 10^{-31}$ kg, 所以电子的比荷为
$$\frac{e}{m_e} = 1.76 \times 10^{11}\ \mathrm{C/kg}.$$

\subsection{静电力}

通过实验可知, 电荷之间的作用力随着电
荷量的增大而增大, 随着距离的增大而减小.
这看起来与万有引力的规律类似.电荷之间的相互作用力, 会不会与它们电
荷量的乘积成正比, 与它们之间距离的二次方成反比?

法国科学家库仑设计了一个十分精妙的实验(扭秤实验), 对电荷之间的作
用力开展研究, 最终确定: \textbf{真空中两个静止点电荷之间的
    相互作用力, 与它们的电荷量的乘积成正比, 与它们的距
    离的二次方成反比, 作用力的方向在它们的连线上.}这个
规律叫做\textbf{库仑定律}.这种电荷之间的相
互作用力叫做\textbf{静电力}或\textbf{库仑力}.

应该注意, 由于力的大小应该是正值, 而电荷量$q_1$,$q_2$的乘积可以是负值, 因此,
在计算静电力时, 实际上\textbf{应代入电荷量的绝对值}.

假设两个点电荷的电荷量的绝对值分别为$q_1$, $q_2$, 它们的距离为$r$, 那么库仑定律可以表示为
\begin{empheq}[box=\fbox]{equation*}
    F = k\frac{q_1q_2}{r^2}.
\end{empheq}
式中的$k$是比例系数, 叫做\textbf{静电力常量}.当两个点电荷所带
的电荷量为同种时, 它们之间的作用力为斥力; 反之, 为
异种时, 它们之间的作用力为引力.
在国际单位制中, 电荷量的单位是库仑(C), 力的单
位是牛顿(N), 距离的单位是米(m). $k$的值是
$$k = 9.0 \times 10^9\ \mathrm{N\cdot m^2/C^2}.$$

上面的定律中提到了点电荷的概念, 下面我们来介绍一下.

\subparagraph{点电荷} 实验事实说明, 两个实际的带电体间的相互作用力与
它们自身的大小, 形状以及电荷分布都有关系.

当带电体之间的距离比它们自身的大小大得多, 以致带电体的形状, 大小及电荷分布状
况对它们之间的作用力的影响可以忽略时, 这样的带电体可以看作带电的点, 叫做\textbf{点电荷}.
\bigskip

库仑定律描述的是两个点电荷之间的作用力.如果存
在两个以上点电荷, 那么, 每个点电荷都要受到其他所有
点电荷对它的作用力. 两个或两个以上点电荷对某一个点
电荷的作用力, 等于各点电荷单独对这个点电荷的作用力
的矢量和.

库仑定律是电磁学的基本定律之一.库仑定律给出的
虽然是点电荷之间的静电力, 但是任何一个带电体都可以
看成是由许多点电荷组成的. 所以, 如果知道带电体上的
电荷分布, 根据库仑定律就可以求出带电体之间的静电力
的大小和方向.

\subsection{电场}

\subsubsection{电场}

19 世纪 30 年代, 英国科学家法拉第提出一种观点, 认
为在电荷的周围存在着由它产生的电场.处在电场中的其它电荷受到的作用力就是这个电场给
予的.例如, 电荷 A 对电荷 B 的作用力, 就是电荷 A 的电
场对电荷 B 的作用; 电荷 B 对电荷 A 的作用力, 就是电荷 B
的电场对电荷 A 的作用.

物理学的理论和实验证实并发展了法拉第的观点.电场
以及磁场已被证明是客观存在的.场像分子, 原子等实物
粒子一样具有能量, 因而场也是物质存在的一种形式.

静止电荷产生的电场叫做\textbf{静电场}.

把一个电荷放入某个电场中, 来研究这个电场的性质. 这样的
电荷叫做\textbf{试探电荷}.激发电场的带电体所带的电荷叫
作\textbf{场源电荷}, 或\textbf{源电荷}.

在研究电场的性质时, 我们选取的试探电荷应当是电荷量很小的点电荷, 目的是不对所研究的电场产生影响.

\subsubsection{电场强度}

在点电荷$Q$的电场中的$P$点, 放一个试探电荷$q_1$, 它在电场中受到的静电力是$F_1$, 根据库仑定律, 有
\begin{equation}
    F_1 = k\frac{q_1Q}{r^2}.
    \label{静电力1}
\end{equation}

同理, 如果把试探电荷换成$q_2$, 那么它受到的静电力
\begin{equation}
    F_2 = k\frac{q_2Q}{r^2}.
    \label{静电力2}
\end{equation}

由 \eqref{静电力1} \eqref{静电力2} 两式可以看出
\begin{equation}
    \label{静电力与电荷量之比}
    \frac{F_1}{q_1} = \frac{F_2}{q_2} = k\frac{Q}{r^2}.
\end{equation}
放在$P$点的试探电荷所受的静电力与它的电荷量之比, 与产生电场的场源电荷
的电荷量 $Q$及$P$点到场源电荷的距离$r$有关, 而与试探电荷的电荷量无关.

试探电荷所受的静电力与它的电荷量之比反映了电场在各点的性质. 我们定义它为\textbf{电场强度},
用$E$表示, 即
\begin{empheq}[box=\fbox]{equation*}
    E = \frac{F}{q}.
\end{empheq}
这是电场强度的定义式.其中$F$是试探电荷在电场内某点所受的静电力, $q$是这个试探电荷的电荷量, $E$是这一点的电场强度.

由定义式可知, 电场强度的国际单位为\textbf{牛每库}, 符号是N/C.如果 1 C 的电荷在电场中的某点受到的静电力是 1 N,那么
该点的电场强度就是 1 N/C.

电场强度是矢量.物理学规定, 电场中某点的电场强度的方向与正电荷在该点所受静电力的方向相同.

试探电荷在电场中受到的静电力也叫做\textbf{电场力}.

\subsubsection*{点电荷的电场强度}

点电荷是最简单的场源电荷.由 \eqref{静电力与电荷量之比} 可知,
一个电荷量为$Q$的点电
荷, 在与之相距$r$处的电场强度
\begin{empheq}[box=\fbox]{equation}
    E = k\frac{Q}{r^2}.
    \label{点电荷的电场强度}
\end{empheq}

据上式可知, 如果以电荷量为$Q$的点电荷为中心作一
个球面, 则球面上各点的电场强度大小相等. 当$Q$为正电荷
时, 电场强度$E$的方向沿半径向外; 当$Q$为负
电荷时, 电场强度$E$的方向沿半径向内.

\subsubsection*{电场强度的叠加} 我们知道, 两个或两个以上的点电荷对某一个点电荷
的静电力, 等于各点电荷单独对这个点电荷的静电力的矢
量和.由此可以推理, 如果场源是多个点电荷, 则电场中
某点的电场强度等于各个点电荷单独在该点产生的电场强
度的矢量和.

\begin{example}
    在某电场中的$P$点, 放一带电量$q_1 = -3.0\times 10^{-10}\unit{C}$的试探电荷,
    测得该点收到的静电力大小为$F_1 = 6.0\times 10^{-7}\unit{N}$, 方向水平向右. 求

    (1) $P$点的电场强度大小和方向;

    (2) 如果在$P$点放一带电量$q_2 = 1.0 \times 10^{-10}\unit{C}$的试探电荷, 求$q_2$
    受到的静电力$F_2$的大小和方向.
\end{example}
\begin{solution}
    (1) 根据电场强度的定义, $P$点的电场强度为
    $$E = \frac{F_1}{q_1} = \frac{6.0\times 10^{-7}}{-3.0\times 10^{-10}}\unit{N/C} = 2.0\times 10^{3}\unit{N/C},$$
    方向与负点电荷$q_1$受到的静电力方向相反, 即水平向左.

    (2)由电场强度的定义有$$E = \displaystyle\frac{F_2}{q_2}.$$
    由此可得$$F_2 = 2.0\times 10^{-7}\unit{N}.$$因为$q_2$是正点电荷, 所以
    $F_2$的方向与$P$点的场强方向相同,即水平向左.
\end{solution}

\subsubsection*{电场线}

除了用数学公式描述电场外, 形象地了解和描述电场中
各点电场强度的大小和方向也很重要.
法拉第采用了一个简洁的方法来描述电场, 那就是画\textbf{电场线}.

同一幅图中, 电场强度较大的地方电场线较密, 电场强度
较小的地方电场线较疏, 因此在同一幅图中可以用电场线
的疏密来比较各点电场强度的大小.

\subsubsection*{匀强电场}

如果电场中各点的电场强度的大小相等,方向相同,
这个电场就叫做\textbf{匀强电场}.

由于方向相同, 匀强电场中的
电场线应该是平行的; 又由于电场强度大小相等, 电场线
的疏密程度应该是相同的.所以, 匀强电场的电场线可以
用间隔相等的平行线来表示.

\subsection{静电场中的能量}

我们知道, 电荷在电场中会受到静电力, 若电荷发生位移, 则静电力可能会做功. 我们可以以此为突破,
了解电场中的能量.
\subsubsection{电势能}

\subsubsection*{静电力做功的特点}
\label{section:静电力做功的特点}


实验发现, 将一个试探电荷$q$从$A$点移动到$B$点, 无论是沿直线移动, 还是沿折线或曲线移动, 静电力做
的功都相等.这就是说, 在静电场中移动电荷时, 静电力所做的功只与电荷的初末位置有关, 而与电荷经过的路径
无关. 因此, 与重力一样, \textbf{静电力属于保守力}.

\subsubsection*{电势能}

与物体在重力场中具有重力势能类似, 电荷在静电场中具有\textbf{电势能}, 用$E_p$表示.

如果用$W_{AB}$表示电荷由 $A$ 点运动到 $B$ 点静电力所做的
功, 用$E_{\mathrm{p}A}$表示电荷在$A$点所具有的电势能, 用$E_{\mathrm{p}B}$表示电荷在$B$点所具有的电势能,
那么它们之间的关系为
$$W_{AB} = E_{\mathrm{p}A} - E_{\mathrm{p}B}.$$
也可以表示为
$$W_{AB} = -\Delta E_\mathrm{p}.$$

当$W_{AB}>0$时, $E_{\mathrm{p}A} > E_{\mathrm{p}B}$, 静电力做正功, 电势能减小;

当$W_{AB}<0$时, $E_{\mathrm{p}A} < E_{\mathrm{p}B}$, 静电力做负功, 电势能增大.

这满足保守力做功的规律.

\subparagraph{电势能的相对性}
静电力做的功只能决定电势能的变化量,
而不能决定电荷在电场中某点电势能的数值.只有先把电
场中某点的电势能规定为 0, 才能确定电荷在电场中其他
点的电势能.

通常,我们把电荷在离场源电荷无限远处的电势能规定为 0,
或把电荷在大地表面的电势能规定为 0.
\bigskip

规定离场源电荷$Q$无限远处的电势能为 0, 若将一个电荷$q$从$A$点移动到无限远处,
根据$W_{AB} = E_{\mathrm{p}A} - E_{\mathrm{p}B}$可得$$W_{A\rightarrow \infty} = E_{\mathrm{p}A} - 0 =E_{\mathrm{p}A}.$$
由此可知, \textbf{如果规定无限远处的电势能为 0, 那么
    电场中某电荷的电势能, 等于将该电荷从该点移动到无穷远处静电力所做的功}.

如果场源电荷$Q$是点电荷, $A$点到场源电荷的距离为$r$, 根据库仑定律,
通过数学推导可以得到
$$E_{\mathrm{p}A} = W_{A\rightarrow \infty} = Fl = qEl = k\frac{Qq}{r}.$$
式中$l$表示$A$点到无限远处的距离, $k$是静电力常量. 这里$Q$与$q$的正负要代入计算.
推导过程涉及高等数学知识, 故不做展开.

由此可得, 如果规定无限远处的电势能为 0, 那么在点电荷$Q$的电场中的某点, 电荷$q$所具有的电势能为
\setlength{\abovedisplayskip}{10pt}
\setlength{\belowdisplayskip}{10pt}
\begin{empheq}[box=\fbox]{equation}
    E_\mathrm{p} = k\frac{Qq}{r}.
    \label{点电荷附近的电势能}
\end{empheq}
式中, $k$是静电力常量, $r$是场源点电荷$Q$与试探电荷$q$的距离.
\bigskip

应该注意, 电势能是相互作用的电荷所共有
的, 或者说是电荷及对它作用的电场所共有的.我们刚才说某
个电荷的电势能, 只是一种简略的说法.

\subsubsection{电势}

前面我们通过对静电力的研究, 认识了电场强度. 试探电荷在电场中所受的静电力与
试探电荷和场源电荷均有关; 而电场强度与试探电荷无关, 只与场源电荷有关, 是电场本身的性质.

现在我们要通过对电势能的研究来认识另一个物理量——电
势,它同样是表征电场性质的重要物理量.

通过实验可知, 置于某一点的试探电荷$q$, 如果它的电荷
量变为原来的$n$倍,其电势能也变为原来的$n$倍. 电势能
与电荷量之比却是一定的, 它是由电场中该点的性质决定的, 与试探电荷本身无关.

与电场强度的定义类似, 试探电荷在电场中某一点的电势能与它的电荷量之比, 叫做
电场在这一点的\textbf{电势}. 如果用$\varphi$表示电势, 用$E_\mathrm{p}$表示试探电荷$q$的电势能,
则
\begin{empheq}[box=\fbox]{equation*}
    \varphi = \frac{E_\mathrm{p}}{q}.
\end{empheq}

在国际单位制中, 电势的单位是\textbf{伏特}, 符号是V.
在电场中的某一点, 如果电荷量为 1 C 的电荷在这点的电势
能是 1 J, 这一点的电势就是 1 V, 即
1 V = 1 J/C.

假如正的试探电荷沿着电场线的方向向外移动,
它的电势能是逐渐减少的. 可以说, \textbf{沿着电
    场线方向电势逐渐降低}.

与电势能的情况相似, 应该先规定电场中某处的电势
为 0, 然后才能确定电场中其他各点的电势.

当规定无穷远处为零电势点时, 如果场源电荷是点电荷, 根据点电荷附近附近某点的电势能公式 \eqref{点电荷附近的电势能} 可知
\begin{empheq}[box=\fbox]{equation*}
    \varphi = k\frac{Q}{r}.
\end{empheq}
这是\textbf{点电荷的电场内某点的电势计算公式}. 其中$k$为静电力常量, $Q$为场源电荷的电荷量,
$r$为这一点到场源电荷的距离.

由上式可知, \textbf{如果规定无穷远处为零电势点, 那么正电荷附近的电势大于0, 负电荷附近的电势小于0}.

电势只有大小, 没有方向, 是个标量.

\subsubsection*{电势叠加原理}
我们知道, 场源电荷是多个点电荷时, 电场中
某点的电场强度等于各个点电荷单独在该点产生的电场强
度的矢量和.

与电场强度类似,
\textbf{多个点电荷在空间某点产生电场的电势, 为每个点电荷在该点产生电势的代数和.}
这就是电势叠加原理.

\subsubsection{电势差}

选择不同的位置作为零电势点, 电场中某点电势的数
值也会改变, 但电场中某两点之间电势的差值却保持不变.
\setlength{\abovedisplayskip}{0pt}
\setlength{\belowdisplayskip}{0pt}

在电场中, 两点之间电势的差值叫做\textbf{电势差}, 电势差也叫做\textbf{电压}.
设电场中 $A$ 点的电势为 $\varphi_A$, $B$ 点的电势为 $\varphi_B$, 则它们之间的电势差
可以表示为
$$U_{AB} = \varphi_A - \varphi_B.$$
也可以表示为
$$U_{BA} = \varphi_B - \varphi_A.$$
显然
$$U_{AB} = -U_{BA}.$$

电势的值是相对的, 与零电势点的选取有关; 而电势的差值是绝对的, 与零电势点的选取无关.

\subsubsection*{静电力做功与电势差的关系}

将点电荷$q$从$A$点移向$B$点, 静电力做的功$W_{AB}$为电荷$q$在这两点所具有的电势能之差.
由此可以导出静电力做功与电势差的关系.
\begin{align*}
    W_{AB}  = E_{\mathrm{p}A} - E_{\mathrm{p}B}
     & =q\varphi_A - q\varphi_B  \\
     & =q(\varphi_A - \varphi_B) \\
     & =qU_{AB}.
\end{align*}
即
\begin{empheq}[box=\fbox]{equation*}
    U_{AB} = \frac{W_{AB}}{q}.
\end{empheq}
\setlength{\abovedisplayskip}{5pt}
\setlength{\belowdisplayskip}{5pt}
\begin{example}
    在静电场中, 将一带电量$q = -1.5\times 10^{-6}\unit{C}$的电荷从$A$点移向$B$点,
    电势能减少$3.0\times 10^{-4}\unit{J}$. 如果将该电荷从$C$点移向$A$点, 克服静电力
    做功$1.5\times 10^{-4}\unit{J}$. 取无穷远处为零电势点.

    (1) 求$A, B$两点间的电势差, $A, C$两点间的电势差和$B, C$两点间的电势差;

    (2)若将此电荷从$A$点移动至无穷远处, 克服静电力做功为$6.0\times 10^{-4}\unit{J}$,
    求电荷在$A$点的电势能;

    (3)若规定$C$点处电势为0, 求$A$点电势和电荷在$B$点的电势能;

    (4)如果将带电量为$2.0\times 10^{-6}\unit{J}$的正电荷从$A$点移向$B$点, 求静电力做的功.
\end{example}
\begin{solution}
    (1) 电荷从$A$点移向$B$点,
    电势能减少$3.0\times 10^{-4}\unit{J}$,说明静电力做功$3.0\times 10^{-4}\unit{J}$.
    根据静电力做功与电势差的关系, $A, B$间的电势差
    $$U_{AB} = \frac{W_{AB}}{q} = \frac{3.0\times 10^{-4}}{-1.5\times 10^{-6}}\unit{V} = -200\unit{V}.$$

    电荷从$C$点移向$A$点, 克服静电力做功$1.5\times 10^{-4}\unit{J}$, 说明从$A$点移向$C$点,
    静电力做功$1.5\times 10^{-4}\unit{J}$.同样, 根据静电力做功与电势差的关系, $A, C$间的电势差
    $$U_{AC} = \frac{W_{AC}}{q} = -100\unit{V}.$$
    \setlength{\abovedisplayskip}{0pt}
    \setlength{\belowdisplayskip}{0pt}

    根据电势差的定义, 有
    \begin{align*}
        U_{AB} = \varphi_A - \varphi_B, \tag{i - a}\label{例题2-ia} \\
        U_{AC} = \varphi_A - \varphi_C, \tag{i - b}\label{例题2-ib} \\
        U_{BC} = \varphi_A - \varphi_C.\tag{i - c}\label{例题2-ic}
    \end{align*}
    由 \eqref{例题2-ia} \eqref{例题2-ib} 可得, $B, C$两点间的电势差
    $$U_{BC} = U_{AC} - U_{BC} = -100\unit{V} - -200\unit{V} = 100\unit{V}.$$

    (2) 以无穷远处为零电势点时, 电场中某电荷的电势能, 等于将该电荷从该点移动到无穷远处静电力所做的功.
    因此电荷在$A$点的电势能$$E_\mathrm{p}A = W_{A\rightarrow \infty} = -6.0\times 10^{-4}\unit{J}.$$

    (3)规定$C$点处电势为0, 即$\varphi_C = 0$, 那么由 \eqref{例题2-ib} 式可得, $A$点处的电势
    $$\varphi_A = U_{AC} + \varphi_C = U_{AC} = -100\unit{V}.$$

    \setlength{\abovedisplayskip}{5pt}
    \setlength{\belowdisplayskip}{5pt}
    根据电势的定义, $B$点的电势等于电荷$q$在$B$点的电势能除以它的电荷量, 即
    \begin{equation*}
        \varphi_B = \frac{E_\mathrm{p}B}{q}.
        \tag{ii}
        \label{例题2-ii}
    \end{equation*}

    由 \eqref{例题2-ic} 式和 \eqref{例题2-ii} 式可得$E_\mathrm{p}B = -1.5\times 10^{-4}\unit{J}.$
    \setlength{\abovedisplayskip}{0pt}
    \setlength{\belowdisplayskip}{0pt}

    (4) 将带电量为$2.0\times 10^{-6}\unit{J}$的正电荷从$A$点移向$B$点,静电力做的功为
    $$W_{AB} = q^{\prime}U_{AB} = -1.5\times 10^{-6}\unit\times -200 \unit{J} = 3.0\times 10^{-4}\unit{J}.$$
    即静电力做正功$3.0\times 10^{-4}\unit{J}$.

\end{solution}

\subsection{电场的图形描述}

\subsubsection{电场线}
前面我们已经了解过了电场线 现在我们来进一步学习它.

为了形象地描述电场, 我们人为的画出假想曲线.电场线的特点如下.

\begin{enumerate}
    \item 电场线从正电荷或无限远出发, 终止于无限远或
          负电荷;
    \item 电场线在电场中不相交, 这是因为在电场中任意
          一点的电场强度不可能有两个方向;
    \item 电场线上各点的切线方向是该点电场强度的方向;
    \item 电场的疏密反映了电场的强弱, 电场线越密, 电场越强;
    \item 电场线不是客观存在的, 是为了形象描述电场而假想的.
\end{enumerate}

电场线反映了某点电场的强弱.例如, 我们知道, 等量同种电荷连线的中点处电场强度为0, 因此,
在绘制电场线时, 这一点是空出来的, 代表这一点无电场.

\subsubsection{等势面}

电场中电势相同的各点构成的面叫做等势面.

根据等势面的定义, 要使得同一个面上的各点电势相同, 需使得同一电荷在这个面上各点的电势能相同.
即电荷在这个面上运动时, 静电力不做功, 应有静电力的方向时刻跟等势面垂直, 即与电场线垂直. 否则,
电场强度就有一个沿着等势面的分量, 在等势面上移动电荷时静电力就要做功, 这与这个面
是等势面矛盾. 前面已经说过, 沿着电场线方向电势逐渐降低. 因此, 概括起来就是:
\textbf{电场线跟等势面垂直,并且由电势高的等势面指向电势低的等势面}.

等势面的特点如下.
\begin{enumerate}
    \item 等势面一定跟电场线垂直;
    \item 电场线总是从电势高的等势面指向电势低的等势面;
    \item 不同的等势面不会相交;
    \item 如果等势面是等差的, 那么等势面越密, 电场强度越大;
    \item 在同一等势面上移动电荷时, 静电力不做功.
\end{enumerate}


\subsection{电势差与电场强度的关系}

\begin{wrapfigure}{r}{5cm}
    \flushright
    \includegraphics[width=0.32\textwidth]{pic/pic1.6-1.pdf}
    \label{1.6-1}
\end{wrapfigure}

电场线是描述电场强度的,等势线是描述电势的,电场
线和等势线的疏密存在对应关系,表明电场强度和电势之间
存在一定的联系.下面以匀强电场为例讨论它们的关系.
\setlength{\abovedisplayskip}{0pt}
\setlength{\belowdisplayskip}{0pt}

如图所示, 假设电荷$q$在强度为$E$的\textbf{匀强电场}中沿着静电力$F$的方向从$A$点运动到$B$点, 电荷
所受的静电力为$F = qE$. 因为是匀强电场, 所以这个力是恒力, 它所做的功为
\begin{equation}
    W = Fd = qEd.
    \label{静电力做功1}
\end{equation}

除此之外, 静电力做功$W$与$A, B$两点间电势差$U_{AB}$的关系为
\begin{equation}
    W = qU_{AB}.
    \label{静电力做功2}
\end{equation}

比较 \eqref{静电力做功1} 与 \eqref{静电力做功2}, 得到
\begin{empheq}[box=\fbox]{equation*}
    U_{AB} = Ed.
    \label{电势差与电场强度的关系1}
\end{empheq}
即: \textbf{匀强电场中两点间的电势差等于电场强度与这两点沿
    电场方向的距离的乘积.}
\setlength{\abovedisplayskip}{5pt}
\setlength{\belowdisplayskip}{5pt}

电势差与电场强度的关系也可以写作
\begin{empheq}[box=\fbox]{equation*}
    E = \frac{U_{AB}}{d}.
    \label{电势差与电场强度的关系2}
\end{empheq}
它的意义是,在匀强电场中, 电场强度的大小等于两点
之间的电势差与两点沿电场强度方向的距离之比,
也就是说, \textbf{匀强电场的电场强度在数值上等于沿电场强度方向上单位距离的电势差.}
所以说, \textbf{电场强度的
    方向为电场中电势降低最快的方向}.

因为 $d$ 是电荷位移沿电场强度方向的分量, 所以说, \textbf{电场强度的方向就是电场中电势降低最快的方向}.

应该注意, 上面的结论只适用于匀强电场.事实上, 利用微元的思想, 我们在一般电场中有
$$E = \lim_{\Delta d\rightarrow 0} -\frac{\Delta \varphi}{\Delta d}.$$
(式中$\Delta d$为电荷沿电场强度方向的微小位移, $\Delta \varphi$为末电势与初电势的差值)即电场强度反映电势随空间的变化率,
这类似于加速度与速度的关系.对于非匀强电场的情况, 我们也可以用上式做定性判断.

\subsubsection*{利用``等分法''确定匀强电场强度的方向}

由于匀强电场的电场强度与电势差的关系存在上述结论, 因此, 已知电场中任意三点的电势时, 可以将电势差最大的两点连线均分.
我们总能在连线上找到一点, 使它与第三点的电势相等. 连接该点与第三点就得到一条等势线, 而与等势线垂直的方向即为电场方向.

\subsection{静电的防止与利用}
\subsubsection{静电平衡}

把一个不带电的金属导体放到电场强度为 $E_0$ 的电场中.
由于导体内的自由电子受静电力作用而定向移动, 使导体的两个端面出现等量的异种电荷,
这种现象叫做静电感应.

导体两侧出现的正负电荷在导体内部产生与外电场强
度 $E_0$ 方向相反的附加电场, 其电场强度为 $E′$.
这两个电场叠加, 使导体内部的电场减弱. 在叠加后的
电场作用下, 仍有自由电子不断运动, 直到附加电场与外电场完全抵消,
即导体内部各点的电场强度 $E = 0$ 为止,
导体内的自由电子不再发生定向移动. 这时我们说, 导体达到\textbf{静电平衡}状态.

\textbf{处于静电平衡状态的导体, 其内部的电场强度处处为0. 此时整个导体的电势处处相等,
    我们说整个导体是个等势体, 它的表面是个等势面.}

由于处于静电平衡的导体表面是等势面, 因此对于导体产生的附加电场来说, 它表面任意点的电场强度方向与其表面垂直.

\subsubsection{电荷的分布特点}

由于带有同种电荷量的电荷相互排斥, 所以它们尽可能``互相远离'', 因此带电导体的电荷分布在外表面.

除此之外, 受导体的形状影响, 曲率半径大的地方电荷的密度小, 曲率半径小的地方电荷的密度大. 这就导致
细而尖的地方电荷的分布密度大, 于是这里的电场强度也越大.

在一定条件下, 导体尖端周围的强电场足以使空气中残留
的带电粒子发生剧烈运动, 并与空气分子碰撞从而使空气
分子中的正负电荷分离. 这个现象叫做\textbf{空气的电离}. 那
些所带电荷与导体尖端的电荷符号相反的粒子, 由于被吸
引而奔向尖端, 与尖端上的电荷中和, 这相当于导体从尖
端失去电荷. 这种现象叫做\textbf{尖端放电}.避雷针就是利用了这个原理.

\subsubsection{静电屏蔽}

我们把一个带空腔的导体置于电场中. 静电平衡时, 根据电荷的分布特点, 这个导体的自由电荷都分布在外表面, 其内表面
没有电荷. 没有电荷也就没有电场, 所以导体内壁的电场强度为0, 即电场线只能在空腔之外, 不能进入空腔之内.
所以导体壳内空腔里的电场强度也处处为0.也就是说,\textbf{导体内部不受外部电场的影响}.
这种现象叫做\textbf{静电屏蔽}.

上面的例子中, 我们屏蔽了外电场. 那么如果带空腔的导体内部有一个带正电的点电荷, 如何屏蔽这个点电荷的电场呢?
在正常情况下, 导体内壁将被感应出负电荷, 其外壁带有等量的正电荷.这些正电荷将形成电场, 并且这些正电荷所带的
电荷量之和等于该点电荷的电荷量.

如果将这个空腔的外表面接地, 那么正电荷将沿导线流向大地. 这样, 电场终于了空腔内表面的负电荷, 我们成功的
屏蔽了内电场. 概括地说, \textbf{接地的封闭道题刻内部的电场对壳外空间没有影响.}

\subsection{电容}

\subsubsection{电容器}
\textbf{电容器}是一种重要的电学元件.两个彼此绝缘又相距很近的导体, 可以组成一个电容器.
在两个相距很近的平行金属板中间夹上一层绝缘物质——电介质(空
气也是一种电介质), 就组成一个最简单的电容器, 叫做\textbf{平行
    板电容器}.这两个金属板叫做电容器的\textbf{极板}.

通过实验, 我们发现: 电容器充电的过程中, 接在平行板电容器两端的电压表示数迅速增加,
随后稳定在某一数值, 这表明电容器两极板间有一定的电势差;
通过观察电流表的偏转方向可以知道, \textbf{电荷从电源正极流向电容器的正极板,
    同时, 电流从电容器的负极板流向电源的负极}, 这使两极板的电荷量增加,
极板间的电场强度增大, 电源的能量不断储存在电容器中.
随着两极板之间电势差的增大, 充电电流逐渐减小至 0, 此时电容器两极板带有一定的\textbf{等量异种电荷}.
即使断开电源, 两极板上的电荷由于相互吸引而仍然被保存在电容器中.

放电的过程中, 电流从电容器的正极板经过用电器流向电容器的负极板.
此时两极板所带的电荷量减小,
电势差减小, 放电电流也减小, 最后两极板电势差以及放
电电流都等于 0
电容器把储存的能量通过电流做功转化为电路
中其他形式的能量.

\subsubsection{电容}
前面的实验表明, 电容器两极板之间的电势差增大时, 电容器所带的电荷量也在增加.
电容器所带的电荷量跟两极板间的电势差是否存在某种定量关系?

精确的实验表面, 一个电容器所带的电荷量$Q$与两极板间的电势差$U$之比是不变的.不同的电容器, 这个比
一般是不同的, 可见电荷量$Q$与电势差$U$之比表征了电容
器本身的特性.

电容器所带的电荷量$Q$
与电容器两极板之间的电势差$U$之比\footnote{当电容器的两个极板的电荷量分别为$+Q$和$-Q$时, 我们认为该电容器所带的电荷量为$Q$.这个$Q$是一个正值.}, 叫做电容器的\textbf{电容}.
用$C$表示, 则有
\begin{empheq}[box=\fbox]{equation*}
    C = \frac{U}{Q}.
\end{empheq}

上式表示, 电容器的电容在数值上等于使两个极板间电势差为1 V时电容器需要带的电荷量.
这类似于用不同的容器装水, 要使容器中的水深相同, 横截面积大的
容器需要的水多.

国际单位制中, 电容的单位是\textbf{法拉}, 简称
\textbf{法}, 符号是 F.$$1\unit{F} = \frac{1\unit{C}}{1\unit{V}}.$$
实际中常用的单位还有微法($\mathrm{\upmu F}$)和皮法(pF), 它
们与法拉的关系是
$$1 \unit{\upmu F} = 10^{-6} \unit{F},$$
$$1 \unit{pF} = 10^{-12} \unit{F}.$$

加在电容器两极板上的电压不能超过某一限度, 超过
这个限度, 电介质将被击穿, 电容器损坏.这个极限电压
叫做\textbf{击穿电压}.电容器外壳上标的是工作电压, 或称额定
电压, 这个数值比击穿电压低.

\subsubsection{平行板电容器}

平行板电容器是最简单的,也是最基本的电容器.几乎所有电容器都是平行板电容器的变形.
平行板电容器的电容是由哪些因素决定的呢?

通过实验可以得出如下结论: 减小平行板电容器两极板的正对面积, 增大两极板之间的距离都能
减小平行板电容器的电容; 而在两极板之间插入电介质, 却能增大平行板电容器的电容.

理论分析表明, 当平行板电容器的两极板之间是真空时, 电容 $C$ 与极板的正对面积 $S$, 极板间
的距离 $d$ 的关系为$$C = \frac{S}{4\pi kd}.$$
式中$k$是静电力常量.

当两极板之间充满同一种介质时, 电容变大为真空时的 $\varepsilon_r$ 倍, 即
\begin{empheq}[box=\fbox]{equation}
    C = \frac{\varepsilon_rS}{4\pi kd}.
    \label{平行板电容器的电容}
\end{empheq}
这就是\textbf{平行板电容器电容的决定式}.其中$\varepsilon_r$是一个常数, 与电介质的性质有关,
称为\textbf{相对介电常数}.相对介电常数以真空作为标准.空气的相对介电常数与真空非常接近,
我们在计算时通常也取1.

上式也可以表示为$$C\propto \frac{\varepsilon_rS}{d}.$$
即: 当相对介电常数一定时, 平行板电容器的电容与两极板的正对面积$S$成正比,
与两极板间的距离$d$成反比.

\subsubsection*{平行板电容器的电场强度}

平行板电容器的两个极板间带有等量的异种电荷, 我们知道两极板间会产生静电场. 下面我们来
推导平行板电容器的电场强度.

由于平行板电容器的两极板所带正负电荷相等, 两极板平行正对, 各处对应的正负电荷间距离相等,
所以电场线应当是均匀且平行的, 即各点的电场强度的大小和方向均相同.
因此\textbf{平行板电容器两极板间的电场为匀强电场}.\footnote{严格地讲,平行板电容器只有中间部分是匀强电场,而边沿不是.}

根据匀强电场中电势差与电场强度的关系, 电势差为$U$, 距离为$d$两极板间的电场强度 $E = \displaystyle\frac{U}{d}.$\vspace{1ex}
如果这个电容器的电容为$C$, 两极板所带的电荷量为$Q$, 那么由电容的定义式可得 $U = \displaystyle\frac{Q}{C}.$
联立以上两式可得
$$E = \frac{Q}{dC}.$$
代入平行板电容器电容的决定式 \eqref{平行板电容器的电容} , 得到
\begin{empheq}[box=\fbox]{equation}
    E = \frac{4\pi kQ}{\varepsilon_r S}.
    \label{平行板电容器的电场强度}
\end{empheq}
即

值得一提的是, 这个表达式里面, 没有两板间距$d$, 即\textbf{两极板间的电场强度与两极板的距离无关}.
另一方面, 该表达式表明:当相对介电常数一定时, 两极板间的
电场强度正比于极板上的电荷面密度$\displaystyle\frac{Q}{S}$.我们把它记为$\sigma$, 那么上式就可以简单的表示为
$$E \propto \sigma.$$
或者
$$E \propto \frac{Q}{S}.$$

事实上, 带电体附近的电场强度, 本就是直接由带电体上的电荷分布决定的. 由于导体所带电荷只分布在其表面, 因此其附近的电场强度只取决于导体表面的电荷面密度.

根据这一点, 我们就能更直观的分析平行板电容器充放电的机制.

\subsubsection*{平行板电容器的充放电机制}

\begin{wrapfigure}{r}{5cm}
    \flushright
    \includegraphics[width=0.30\textwidth]{pic/1.7-1.pdf}
    \label{1.7-1}
\end{wrapfigure}

先分析充电的机制.

如图, 开关$S$闭合前, 电容器不带电荷.则由平行板电容器的电场强度公式 \eqref{平行板电容器的电场强度} 和
$U=Ed$ 可知, 电容器两极板间的电势差 $U = 0$. 取电源负极与电容器负极板为零电势, 则有电源正极电势高于电容器正极板电势.
因此, 开关闭合, 就必然在导线中形成顺时针方向的充电电流, 直到电容器正极板与电源正极等势.

当极板间距$d$变小时, 设极板上的电荷量不变, 则由平行板电容器的电场强度公式 \eqref{平行板电容器的电场强度} 和
$U=Ed$ 可知, 两板间电势差$U$减小, 电源正极电势高于电容器正极板电势, 又将在导线中形成顺时针方向的充电电流,
直到电容器正极板与电源正极等势.

由上述分析可知, 电容器\textbf{电容器充电的条件是电容器两极板间的压低于与其并联部分两端的电压}.

\bigskip

接下来分析放电机制.

如图所示, 当开关闭合时, 电容器正极板电势高于负极板. 则电容器正极板将通过电阻形成放电电流.

\begin{wrapfigure}{r}{5cm}
    \flushright
    \includegraphics[width=0.30\textwidth]{pic/1.7-2.pdf}
    \label{1.7-2}
\end{wrapfigure}

若使两板间距$d$增大, 设极板上的电荷量不变, 则由平行板电容器的电场强度公式 \eqref{平行板电容器的电场强度} 和
$U=Ed$ 可知, 两板间电势差$U$增大, 电容器正极板电势高于电阻$R$上端的电势, 将在导线中形成逆时针方向充电电流,
直到电容器正极板与电阻$R$上端等势.

从上述分析可知, \textbf{电容器放电的条件是电容器两极板间的电压高于与其并联部分两端的电压}.

\subsubsection{电容器的能量}

\begin{wrapfigure}{r}{5cm}
    \flushright
    \includegraphics[width=0.30\textwidth]{pic/1.7-3.pdf}
    \label{1.7-3}
\end{wrapfigure}

右图是电容器的$U - Q$图像. 当$U$与$Q$成正比时 (即电容$C$不变时), 根据$W = qU$,
图线与$x$轴所称的面积应该表示该电容从不带电到带电量为$Q_0$时电荷所做的功.
如果取该电容器不带电时的电势能为0, 那么其带电量为$Q_0$时所具有的电势能
$$E_\mathrm{p} = \frac12 U_0Q_0 = \frac12CU_0^2.$$
或者$$E_\mathrm{p} = \frac12 U_0Q_0 = \frac12\frac{Q_0}{C}Q_0 = \frac{Q_0^2}{2C}.$$

由此得到电容器的两个电势能公式.
\begin{empheq}[box=\fbox]{equation*}
    E_\mathrm{p} = \frac12CU^2.
\end{empheq}
式中$C$为电容器的电容, $U$为此时电容器两极板间的电势差.

以及
\begin{empheq}[box=\fbox]{equation*}
    E_\mathrm{p} = \frac{Q^2}{2C}.
\end{empheq}
式中$C$为电容器的电容, $Q$为此时电容器两极板所带的电荷.

\subsection{带电粒子在电场中的运动}

分析带电粒子加速的问题, 常常有两种思路: 一种是
利用牛顿第二定律$$qE = ma,$$ 结合匀变速直线运动公式来分析; 另一
种是利用静电力做功, 结合动能定理$$qU = \Delta E_\mathrm{k}$$来分析.

当解决的问题属于匀强电场且涉及运动时间等描述运
动过程的物理量时, 适合运用前一种思路分析; 当问题只
涉及位移, 速率等动能定理公式中的物理量或\textbf{非匀强电场}\footnote{只有在匀强电场中, 粒子做的是匀变速直线运动.}
情景时, 适合运用后一种思路分析.

\end{document}