\documentclass[11pt,a4paper]{ctexbook}

\title{高中物理学}
\author{啊波呲}

\setlength{\parskip}{0em}
\usepackage{amsmath,mathtools,amssymb,geometry,wrapfig,graphicx,empheq,pifont,enumitem,upgreek, subfig}
\renewcommand{\baselinestretch}{1.64}
\geometry{left=1.0cm,right=0.9cm,bottom=1.9cm,top=2.0cm}
\setenumerate[1]{itemsep=1pt,partopsep=0pt,parsep=\parskip,topsep=3pt}
\setitemize[1]{itemsep=1pt,partopsep=0pt,parsep=\parskip,topsep=3pt}

\usepackage{tikz}
\usepackage{xcolor}
\newcounter{exam}[chapter]
\setcounter{exam}{0}
\newcommand{\bre}{\ \ \ }
\renewcommand{\thefootnote}{*}
\newcommand{\examlabel}{\textbf{例\theexam}}
\newcommand{\soln}{\textbf{解}\bre}
\newcommand{\unit}[1]{\ \mathrm{#1}}

\newenvironment{example}{\bigskip\par\refstepcounter{exam}\examlabel\bre}{\par}
\newenvironment{solution}{\par\soln}{\par\bigskip}

\begin{document}
\maketitle
\pagenumbering{roman}
\tableofcontents

\pagenumbering{arabic}

\setlength{\abovedisplayskip}{0pt}
\setlength{\belowdisplayskip}{0pt}

\part{运动与力}

\chapter{运动的描述}
物体的空间位置随时间的变化是自然界中最简单, 最基本的运动形态, 叫作\textbf{机械运动}, 这是物质运动
最低级, 最基本的形式.几乎在任何形式的运动中都包含这种最基本的运动形式.
在物理学中,研究物体做机械运动规律的分支叫作\textbf{力学}.

本书第一部分所研究的力学都叫作\textbf{经典力学}.经典力学有两方面的局限性.
第一, 物体的运动要远远低于光速; 否则物体将遵循另一种运动规律, 即\textbf{相对论力学}.
第二, 物体不能是分子, 原子那样的大小, 即不是所谓微观粒子, 而应是日常中肉眼可见的那种大小;
否则物体将遵循另一种运动规律, 即\textbf{量子力学}.

研究力学, 即研究机械运动, 我们将采取由表及里的研究方式. 
先研究物体怎样运动, 以及如何描述机械运动的现象, 这部分叫作\textbf{运动学}. 
再进一步研究运动内在规律, 即研究运动和力的关系, 这部分叫作\textbf{动力学}.
运动学知识是动力学的基础.

这一章, 我们开始学习机械运动的描述.


\section{参考系\bre 质点}

\subsection{参考系}

运动学最初步的问题是如何确定物体在空间中的位置, 然后要判断它是运动的还是静止的; 
如果是运动的, 再来描述它具体是怎样运动的.

然而, 自然界的一切物体都处于永恒的运动中, 绝对静止的
物体是不存在的. 就此意义而言, 我们说\textbf{运动是绝对的}.
在绝对的运动下, 绝对位置是不存在的. 这样, 要描述一个物体的位置, 
就必须选定另外一个物体作为参考. 地球在不断地绕太阳运动, 太阳也在不断的迁移, 
我们常说的经纬度, 看似是绝对位置, 实际上
是相对于地球的位置而已. 

描述某个物体的位置随时间的变化, 也总是相对于其他物体而言的. 具体地说, 
我们把一个物体当作是静止的, 以此来研究另一个物体的运动. 这便是运动的相对性. 

为了确定物体的位置和描述它的机械运动而选作标准的另一个物体叫作\textbf{参考系}.
生活中我们说一个物体是静止的, 通常是指它相对于地面静止, 即以地面为参考系.
像这样, 在描述机械运动时, \textbf{我们把参考系当作是静止的, 以研究物体相对于参考系的运动.}

应当指出的是, \textbf{同一物体, 相对于不同参考系, 显示出不同的运动状态.}
以火车车厢为参考系, 火车上的乘客是静止的; 可相对于地面来看, 乘客在以与火车相同的速度高速运动着.

\subsection{质点}

如果物体的大小远远小于所研究问题中的有关距离, 问题又不涉及物体的转动, 
我们就可以忽略实际物体的体积, 用一个没有体积大小, 因而也谈不上有什么形状的``点'', 来代替实际物体.
但在物体的机械运动中, 质量起很重要的作用, 因此这个``点''还应该保留有质量, 这就是\textbf{质点}.

例如, 研究地球绕太阳的运行而不涉及地球的自转时, 由于地球的半径远远小于太阳到地球的距离, 完全可以将地球这个庞然大物当做质点; 又例如研究传动机构时涉及转动, 这时哪怕是最小的齿轮,也不能当作质点.
因此, 物体能否被看成质点, 是由所研究问题的性质决定的, 并非依据物体自身的大小和形状来判断.

学习物理学, 应当注意在各种问题中如何区分主要因素,次要因素与偶然因素, 如何选取适当的\textbf{模型}代替实际对象. 质点就是一种理想化的模型. 以适当的抽象模型代替实际对象并不是脱离实际, 反而使人更深刻地抓住问题的本质.
类似的理想化模型还有``轻杆'', ``光滑平面'', ``点电荷'', ``刚体''等等.

\section{时间间隔\bre 时刻}

要描述物体位置随时间的变化,首先要清楚``时间''
一词的含义.
我们在日常生活中所说的时间, 可能是指\textbf{时间间隔}, 也可能指\textbf{时刻}, 
两者的物理意义不同, 必须严格区分. 在物理学中说``时间'', 多是指时间间隔.

为了区分时刻与时间间隔, 我们借助时间轴来理解是最好的了. 在时间轴上, 
一个点表示一个时刻, 一段线段表示一段时间间隔. 形象地说, 
时间间隔能够展示物体运动的一个过程, 好比是一段录像;
时刻可以显示物体运动的一个瞬间, 好比是一张照片. 就像一个一个连续的照片可以组成录像, 
一系列连续时刻的积累便构成时间.

容易理解, 时间轴上的第一个1\ s是从$t = 0$开始, 至$t = 1\unit{s}$结束的.
因此, ``第1\ s初''是一个时刻, 指的是时间轴上
$t = 0$这一点; ``第1\ s末''也是一个时刻, 指的是时间轴上$t = 1\unit{s}$这一点.
``第1\ s内''是一段时间间隔, 指的是时间轴上从$t = 0$到$t = 1\unit{s}$的这段区间;
同理, ``第3\ s内''指的是时间轴上从$t = 2\unit{s}$到$t = 3\unit{s}$的这段区间(图\ref{时间轴}).

\begin{figure}[htbp]
    \centering
	\includegraphics[width=0.56\textwidth]{pic/1-10.pdf}
    \caption{时间轴}
    \label{时间轴}
\end{figure}

\section{位移}

\subsection{位移\bre 路程}

为了定量地描述物体的位置, 需要在参考系上建立适
当的\textbf{坐标系}. 例如, 若想说明地面上
某人所处的位置, 可以采用平面直角坐标系来描述; 如果
物体做直线运动, 可以用一维坐标系来描述.

表示物体在某一时刻的位置是容易的, 那么, 如何描述物体位置的变化呢?
我们在初中已经知道, \textbf{路程}是物体运动轨迹的长度, 但有时候我们并不关注
物体的移动过程如何, 只关心它的位置从哪里变到哪里. 这样, 一根从起点指向
终点的\textbf{有向线段}就足以表示物体位置的变化. 只要物体的初, 末位置确定, 这个有
向线段就是确定的, 它不因路径的不同而改变. 物理学中
用\textbf{位移}来描述物体位置的变化, 并用 $l$ 表示.

位移是有方向的, 它的方向从物体的初位置指向末位置. 在物理学中, 像位移这样的物理量叫作\textbf{矢量}, 它既有
大小又有方向; 像温度, 路程这样的物理量叫作\textbf{标量}, 它
们只有大小, 没有方向.

\subsection{直线运动的位移}

质点相对于参考系做直线运动, 其轨迹是固定于参考系的一条直线. 为了精确表明物体在这条直线上的位置, 
我们选定一个原点$O$, 然后指出质点在原点的哪一边, 距离原点有多远. 若给这条直线规定一个正的指向, 
还可以用正负号来简洁的表明质点在原点的哪一边, 这条直线就是\textbf{坐标轴}, 或称\textbf{一维坐标系}.
现称之为$x$轴. 质点与原点的距离, 赋予``$+$'', ``$-$''号, 叫作质点的\textbf{位置坐标}, 简称\textbf{坐标}, 
记作$x$. 坐标$x$确切的表明了物体相对于参考系的位置.

\setlength{\abovedisplayskip}{0pt}
\setlength{\belowdisplayskip}{0pt}

设物体初位置的坐标为$x_1$, 末位置的坐标为$x_2$, 那么物体在这个过程中的位移就是从$x_1$指向$x_2$的有向线段, 其大小为
$$l = \Delta x = x_2-x_1.$$
若两坐标之差为正, 则位移的方向指向 $x$ 轴的正方向;
若两坐标之差为负, 则位移的方向指向 $x$ 轴的负方向.

今后我们也用$\Delta x$来表示物体在直线运动中的位移.

\subsection{位置—时间图像}

现在, 我们已经能够描述作直线运动的物体的位置. 为了研究物体的位置随时间的变化情况, 我们还应当
建立时间的``坐标轴'', 即选取某个时刻作为\textbf{初始时刻}(或称为``0''时刻), 
并把其他时刻用该时刻在初始时刻之前或之后多久来表明, 这个坐标轴记作$t$轴.

我们还可以把$x$轴和$t$轴结合在一起. 在直角坐标系中选时刻 $t$ 为横轴, 选位置
$x$ 为纵轴, 其上的图线就是\textbf{位置—时间图像}, 也称$x$-$t$图像. 通过它能直观
地看出物体在任意时刻的位置.

这样, 物体的坐标$x$随时间$t$的变化$$x = x(t)$$就精准地表述了物体位置随时间的变化情况, 
即准确的描述了物体的机械运动.

特别地, 如果物体的初始位置为位置坐标的原点$O$, 那么物体的位置坐标即为位移大小, 即$\Delta x = x$.
此时位置—时间图像就成为\textbf{位移—时间图像}.

\section{速度}
\label{速度节}

\subsection{速度}

要描述物体的运动, 物体位置变化的快慢, 即运动的快慢是一个重要的参量\footnote{描述物体运动状态的物理量叫作运动参量.}. 物理学中用位移与发
生这段位移所用时间之比表示物体运动的快慢, 这就是\textbf{速度}, 用$v$表示.

如果在时间 $\Delta t$ 内物体的位移
是 $\Delta x$, 它的速度就可以表示为$$v = \frac{\Delta x}{\Delta t}.$$

速度是矢量, 不仅表明运动快慢, 还表明运动的方向. 速度 $v$ 的方向与时间$\Delta t$内位移 $\Delta x$ 的方向相同. 因此, 
在这段时间里, 如果物体向坐标轴的正方向移动, 则$\Delta x>0$, 因而$v>0$; 
如果物体向坐标轴的负方向移动, 则$\Delta x<0$, 因而$v<0$.

\subsection{平均速度与瞬时速度}
物体在某一段时间内, 运动的快慢通常是
变化的. 而由上式求得的速度$v$, 只是在一段时间$\Delta t$内运动的平均快慢程度, 
称为\textbf{平均速度},通常记作$\overline{v}$. 

如果我们用$x(t)$表示物体位置随时间变化的函数, 由数学知识可以理解, \footnote{本书中, 用符号$f^{\prime}(x)$表示函数$f(x)$在点$x$处的瞬时变化率, 
即$y = f(x)$的图像在$x$处的切线斜率, 定义为$$f^{\prime}(x) = \lim_{\Delta x\rightarrow 0}\frac{f(x+\Delta x)-f(x)}{\Delta x}.$$数学中, 按照这个关系建立的函数$f^{\prime}(x)$称为$f(x)$的导函数, 简称导数.
导数的相关知识在《普通高中教科书\ 数学\ 选择性必修\ 第三册》中有所介绍.}
$$x^{\prime}(t) = \lim_{\Delta t\rightarrow 0} \frac{\Delta x}{\Delta t}= \lim_{\Delta t\rightarrow 0}\frac{x(t+\Delta t)-x(t)}{\Delta t}$$
就表示$t$时刻位置关于时间的瞬时变化率, 我们把它叫作物体在$t$时刻的\textbf{瞬时速度}, 通常记作$v$或$v_t$.
由瞬时速度的定义可知, $x$-$t$图线上某一点切线的斜率就表示物体在这一时刻的瞬时速度.

交通规则更着重于汽车的瞬时速度. 限速60\ km/h的路段上, 一辆汽车以20\ km/h的平均速度行驶了一段时间.
而在这段时间里汽车的速度有时快有时慢, 有时还会停下等待红绿灯, 因而汽车在某些时刻的瞬时速度超过60\ km/h并非不可能.
所以, 交通规则中的限速, 限制的是瞬时速度而非平均速度.

在力学的研究中, 当然也更着重瞬时速度. 如果不加说明, 今后说``速度''时, 指的都是瞬时速度.

\textbf{匀速直线运动}是瞬时速度保持不变的运动. 在匀速直
线运动中, 平均速度与瞬时速度相等. 

瞬时速度的大小通常叫作\textbf{速率}, 而\textbf{平均速率}是物体在一段时间内各个时刻速率的平均值.开车绕400\ m的操场一圈
并回到起点, 汽车每一刻的速率都不为0, 平均速率描述了汽车在这段时间内汽车速率的平均值, 用微元法的思想容易理解, 
汽车的平均速率实际上是汽车运动的路程(而非位移)与所用时间之比.

要注意区分``平均速率''和``平均速度''. 在上一个例子中, 由于汽车运动一周回到起点, 汽车的位移为0, 
所以汽车的``平均速度''就是0.

\section{加速度}
\subsection{加速度}
要比较位置变化的快慢, 可以用位移(即位置的变化量)除以时间. 同理, 要
比较速度变化的快慢, 可以用速度的变化量除以时间.

物理学中把速度的变化量与发生这一变化所用时间之
比, 叫作\textbf{加速度}, 通常用 $a$ 表示. 若用 $\Delta v$ 表
示速度在时间 $\Delta t$ 内的变化量,则有
\begin{equation}
    a = \frac{\Delta v}{\Delta t}.
    \label{加速度的定义}
\end{equation}
国际单位制中, 加速度的单位是\textbf{米每二次方秒}, 符号是$\mathrm{m/s^2}$.

加速度是矢量, 既有大小, 又有方向. 对于做直线运动的物体来说, 如果加速度与速度的
方向相同, 物体的速度将增加; 如果加速度与速度的方向相反, 物体的速度将减小, 直至减小到0后又沿着加速度的方向加速. 
规定了正负号后可以这样说:

如果$a>0$, $v$的值\footnote{这里是带符号比大小, 而不是比绝对值.}将增大; 如果$a<0$, $v$的值将减小.

由上式求得的加速度$a$, 只是一段时间内速度变化的平均快慢程度, 称为平均加速度.
速度在某一时刻的瞬时变化率称为瞬时加速度. 

在中学阶段, 我们研究的大多数运动是加速度恒定的运动. 在这种情况下, 平均加速度和瞬时加速度相等, 
我们不作特别区分, 统称为``加速度''; 其他情况下, 我们所研究的``加速度''大多是瞬时加速度.

\subsection{速度—时间图像}

物体运动的速度随时间变化的情况可以用图像来直观
表示.以时间 $t$ 为横轴, 速度 $v$ 为纵轴, 坐标系中的图像即
为\textbf{速度—时间图像}或 $v$-$t$ 图像.

对于做直线运动的物体来说, 我们选定一个正方向, 并规定沿这个方向的速度为正, 沿反方向的速度为负. 那么
当速度沿正方向时, 图线将画在$t$轴的上方; 当速度沿负方向时, 图线将画在$t$轴的下方; 

在今后的物理学习中, $v$-$t$图像将会是最常用的图像. 因为它能展示许多信息:
\begin{itemize}
    \item 图线上某一点切线的斜率, 就表示物体在这一时刻的瞬时加速度;
    \item 图线的纵轴截距表示物体的初速度;
    \item 图线与$t$轴围成的图形面积, 代表物体的位移.
\end{itemize}

其中, 斜率信息尤为重要. 由数学知识可知, 加速度实际就是速度关于时间的瞬时变化率, 也就是导数.
设速度关于时间的函数为$v(t)$, 则加速度就是
$$a = v^{\prime}(t). $$
如果$v$-$t$图线是上凸的, 即导函数递减, 就表示物体的加速度减小;
如果$v$-$t$图线是下凸的, 即导函数递增, 就表示物体的加速度增大.
利用数学知识, 今后我们还可以发掘$v$-$t$图像的更多含义.

\chapter{匀变速直线运动}

\section{匀变速直线运动的规律}

初中我们研究的运动大多是匀速直线运动. 我们知道, 匀速直线运动的$v$-$t$图像
是一条平行于$t$轴的直线(图\ref{匀直A}), 表示物体的速度恒定不变, 即加速度为0. 如果物体的$v$-$t$图像是一条
倾斜的直线, 物体的运动状态又是如何呢?

考虑做直线运动的某物体, 其速度关于时间的函数$v = v(t)$是斜率不为0的一次函数, 即
$$v(t) = kt+c, $$
则物体的加速度
$$a = v^{\prime}(t) = k.$$
可见, 该物体的加速度为定值$k$, 也就是该一次函数$v(t)$的比例系数(斜率).
像这样, 沿着一条直线, 且加速度不变的运动, 叫作\textbf{匀变速直线运动}.

当$k>0$, 即$a>0$时, 物体做\textbf{匀加速直线运动}, 其速度与时间的关系如图\ref{匀加A} 所示; 

当$k<0$, 即$a<0$时, 物体做\textbf{匀减速直线运动}, 其速度与时间的关系如图\ref{匀减A} 所示.

\subsection{速度与时间的关系}

对于做匀变速直线运动的物体来说, 我们可以取物体开始运动的时刻为初始时刻. 如果从初始时刻到$t$时刻的过程中, 
物体的速度从$v_0$变到$v$, 那么由加速度的定义式 \eqref{加速度的定义} 可得, 物体做匀变速直线运动的加速度是\footnote{前面已经说过, 匀变速直线运动是加速度不变的运动. 因此, 这里的加速度既是平均加速度, 也可以看作是任意时刻的瞬时加速度.}
$$a = \frac{v - v_0}{t-0}.$$
由此立即得到速度$v$与时间$t$的关系
\begin{empheq}[box=\fbox]{equation}
    v = v_0 + at.
    \label{速度与时间的关系}
\end{empheq}

\begin{figure}[thbp]
\centering
\subfloat[匀速直线运动的$v$-$t$图像]{
    \includegraphics[width=0.25\textwidth]{pic/2.1-B.pdf}
    \label{匀直A}
}\hspace{15pt}
\subfloat[匀加速直线运动的$v$-$t$图像]{
    \includegraphics[width=0.25\textwidth]{pic/2.1-A.pdf}
    \label{匀加A}
}\hspace{15pt}
\subfloat[匀减速直线运动的$v$-$t$图像]{
    \includegraphics[width=0.25\textwidth]{pic/2.1-C.pdf}
    \label{匀减A}
}
	\caption{匀速直线运动和匀变速直线运动的$v$-$t$图像}
    \label{v-t简单图}
\end{figure}

从数学的角度看, 这是非常自然的. 对照匀变速直线运动的一般$v$-$t$函数
$$v(t) = kt+c$$
来看, $v = v(t)$的图线斜率$k$就是 \eqref{速度与时间的关系} 式中的加速度$a$, 常数$c$就是 \eqref{速度与时间的关系} 式中初始时刻的速度$v_0$, 称为\textbf{初速度}.
关系式 \eqref{速度与时间的关系} 实际就是画在$v$-$t$图像中的倾斜直线的解析式(如图\ref{v-t图}).

{\kaishu 知道物体的初速度和加速度后, \eqref{速度与时间的关系} 式给出了物体在任意时刻的速度.如果我们
不知道初速度, 但知道物体在某一时刻的速度, 可以求出物体在其他时刻的速度吗?}

在物体做匀变速直线运动的过程中, 取任意两个时刻$t_1 < t_2$, 物体的速度分别为$v_1$, $v_2$,
则根据 \eqref{速度与时间的关系} 式有
$$v_1 = v_0+at_1, \bre\bre v_2 = v_0+at_2. $$
对两式作差, 可以得到
$$v_2 - v_1 = a(t_2-t_1).$$
记$\Delta v = v_2 - v_1, \Delta t = t_2-t_1$, 上式就可以写成
\begin{empheq}[box=\fbox]{equation}
    \Delta v = a\Delta t.
    \label{速度增量与时间间隔的关系}
\end{empheq}
这就是说, \textbf{做匀变速直线运动的物体, 其速度在一段时间内的增量, 等于物体的加速度与这段时间的乘积.}

\subsection{位移与时间的关系}

在瞬时速度的定义中, 我们说, 物体在$t$时刻的瞬时速度$v$是物体的位置$x$在$t$时刻的瞬时变化率. 
从数学的角度来看, 瞬时速度$v$和位置坐标$x$有这样的关系, 即
速度关于时间的函数$v = v(t)$是位置关于时间函数$x = x(t)$的导数, 
用这样的眼光, 我们尝试推导匀变速直线运动位移与时间的关系.\footnote{这里我们再次用到了导数的知识.如果你不具备这部分知识, 可以直接从\eqref{位移与时间的关系} 式之后开始看, 将给出另外一种比较直观的推导方法.}

\setlength{\abovedisplayskip}{3pt}
\setlength{\belowdisplayskip}{3pt}

如果取物体开始运动的时刻为初始时刻, 根据 \eqref{速度与时间的关系} 式, 
$v = v_0 + at$是$x = x(t)$的导函数, 也就是说对$x(t)$关于$t$求导可以得到$v$, 即
$$x^{\prime}(t) = v_0 + at.$$
回忆函数的求导法则, 一次函数$v_0t$的导数是$v_0$\footnote{如果你还没学过导数, 
不妨设想, 一次函数$y = kx+b$在某一点的斜率是多少? 很显然, 这个函数的图像在每一点的
斜率都是$k$, 所以这个函数的导数就是$k$.}, 二次函数$\displaystyle\frac12 at^2$的导数是$at$, 常数$C$的导数是0.
因此, 原函数\footnote{如果函数$f$是函数$F$的导函数, 那么就说函数$F$是$f$的一个原函数.}应该是
$$x = x(t) = v_0t + \frac12 at^2 + C.$$
在初始时刻, 物体的位置坐标$x = x(0) = C$, 可见常数$C$
就是$t = 0$时物体的位置坐标, 称为\textbf{初位置}.如果用$x_0$表示初位置, 那么物体的位置坐标$x$与时间$t$的关系是
$$x = x_0 + v_0t + \frac12 at^2.$$
特别地, 为了简便起见, 我们规定物体的初位置为位置坐标的原点, 即$x_0 = 0$.从而物体的坐标$x$就是物体的位移$\Delta x$.
这样, 我们就得到了匀变速直线运动位移$x$与时间$t$的关系
\begin{empheq}[box=\fbox]{equation}
    x = v_0 t+\frac12 at^2.
    \label{位移与时间的关系}
\end{empheq}

\begin{figure}[t]
    \centering
	\includegraphics[width=0.37\textwidth]{pic/2-1.pdf}

    \caption{匀变速直线运动的$v$-$t$图像}
    \label{v-t图}
\end{figure}

为了验证上述结论的正确性, 我们从另外一个角度对上式进行推导.

图\ref{v-t图} 是做匀变速直线运动的物体的$v$-$t$图像. 可以知道, 图中直线的斜率是
$$k = \frac{v_1-v_0}{t_1}, $$
所以此一次函数的解析式为
$$v = \frac{v_1-v_0}{t_1} t + v_0.$$
由于速度$v$可由位移$x$求导得到, 所以位置函数$x(t)$应是\footnote{可以验证, 对 \eqref{位置函数1} 求导确实能得到上式$v$. 在积分学中, 这个求导的逆运算叫作求$v(t)$的不定积分. 由于常数的
导数为0, 所以在 \eqref{位置函数1} 中我们加上了常数项$x_0$, 称为积分常数.}
\begin{equation}
    x(t) = \frac{v_1-v_0}{2t_1}t^2+v_0t + x_0.
    \label{位置函数1}
\end{equation}

因此, 物体从初始时刻到$t_1$时刻的位移
$$\Delta x = x(t_1)-x(0) = \frac12(v_1-v_0)t_1+ v_0t_1 + x_0 - x_0 = \frac12 v_1t_1 -\frac12 v_0t_1 + v_0t_1= \frac12(v_0+v_1)t_1.$$
不难发现, $\displaystyle\frac12(v_0+v_1)t_1$正是上图中阴影梯形的面积\footnote{这并不是巧合. 用积分学的知识可以证明, 函数$y = f(x)$的图像在一段区间内与$x$轴围成的梯形(或曲边梯形)面积, 正是其原函数在这段区间的函数值变化量. 现在你只需要理解, 因为$x = vt$, 所以物体的$v$-$t$图线与$t$轴围成的梯形面积就是这段时间内物体所走的位移.}.

特别地, 如果规定物体的初位置为位置坐标的原点, 那么 \eqref{位置函数1} 中没有$x_0$这一项, 即
$$x(t) = \frac{v_1-v_0}{2t_1}t^2+v_0t. $$
这样, 物体在$t_1$时刻的位置坐标$x(t_1)$就是物体从初始时刻到$t_1$时刻的位移$x$, 即
$$x(t_1) = \frac12(v_0+v_1)t_1.$$
根据 \eqref{速度与时间的关系}, $v_1 = v_0+at_1$, 代入可得
$$x(t_1) = \frac12(v_0+v_0+at_1)t_1 = v_0t_1 + \frac12 at_1^2, $$
把$t_1$换成$t$, 即物体在$t$时刻的位移$$x = v_0t + \frac12 at^2.$$

\subsection{速度与位移的关系}
将 \eqref{速度与时间的关系} 与 \eqref{位移与时间的关系} 联立求解, 消去$t$, 即可得到
\begin{empheq}[box=\fbox]{equation}
    v^2 - v_0^2 = 2ax.
    \label{速度与位移的关系}
\end{empheq}
其中$v_0$为物体的初速度, $a$为物体的加速度, 规定物体的初位置为位置坐标的原点, 则$x$为物体的位移. 
这就是匀变速直线运动速度$v$与位移$x$的关系式.

如果在所研究的问题中, 已知量和未知量都不涉及时间, 利用
这个公式求解, 往往会更简便. 

\setlength{\abovedisplayskip}{0pt}
\setlength{\belowdisplayskip}{0pt}

\section{自由落体运动\bre 竖直上抛运动}

上一节,我们从理论上分析了匀变速直线运动的运动规律. 生活中的匀变速直线运动有哪些呢?
其实, 我们最常见到的物体自由下落的运动, 就是一种匀变速直线运动.

\subsection{自由落体运动}

物体下落的运动是司空见惯的, 但人们对这个运动的研究却持续了两千多年. 
亚里士多德认为, 物体下落的快慢跟它的轻重有关, 重的物体下落得快. 这一论断很符合人们的直觉, 
在之后的两千多年里都被人们所认同.

伽利略认为, 根据亚里士多德``重的物体下落得快''的论断,
会推出相互矛盾的结论. 在一块大石头上绑一块小石头, 它的下落速度变快还是变慢呢?
直觉上, 整体的质量增大了, 下落的速度应该变快; 但从单个石头来看, 大石头被小石头的速度拖着, 
下落的速度应该变慢. 这种相互矛盾的结论, 说明亚里士多德的看法是错误的.
根据仔细的分析, 伽利略认为物体下落的运动只有一种可能性:重的物体与轻
的物体应该下落得同样快.

其实, 在生活中, 我们认为重的物体下落快, 轻的物体下落慢, 是由于空气阻力对轻的物体影响较大.
实验证明, 在真空中, 轻的物体和重的物体下落得同样快.

物体只在重力作用下从静止开始下落的运动, 叫作\textbf{自由落体运动}. 这种运动只在真空中才能
发生. 在有空气的空间, 如果空气阻力的作用比较小, 可以忽略, 物体的下落可以近似看作自由落体运动.

进一步的实验表明, 在同一地点, 一
切物体自由下落的加速度都相同. 这个加速度叫作\textbf{自由落体加速度}, 也叫作\textbf{重力加速度}, 
通常用 $g$ 表示. 重力加速度的方向竖直向下, 它的大小可以通过多种
方法用实验测定. 精确的实验发现, 在地球表面不同的地方, $g$ 的大小一
般是不同的, 赤道的$g$比两极的要略小\footnote{在第\ref{万有引力章}章会讲述原因.}.
平均来看, 一般我们取$$g = 9.8\unit{m/s^2}.$$ 中学阶段为了简化计算, 我们通常取$$g = 10\unit{m/s^2}.$$

\textbf{自由落体运动是初速度为 0 , 加速度为$g$的匀加速直线运动}, 所以
匀变速直线运动的基本公式及其推论都适用于自由落体运动. 
把初速度 $v_0 = 0$ 和加速度 $a = g$ 分别代入匀变速直线运
动的速度与时间的关系式 \eqref{速度与时间的关系} 和位移与时间的关系式 \eqref{位移与时间的关系}, 可以得
到自由落体的速度$v$,位移$\Delta h$与时间$t$的关系式分别为
$$v = gt,\ \Delta h = \frac12gt^2.$$

以竖直向下为正方向, 则做自由落体运动的物体的位移$\Delta h$为正值. 那么, 
物体做自由落体运动的时间
$$t = \sqrt{\frac{2\Delta h}{g}}.$$
进而可以得到速度$v$与位移$\Delta h$的关系$$v = \sqrt{2g\Delta h}.$$

\subsection{竖直上抛运动}
将物体以一定的初速度竖直向上抛出, 只在重力作用下的运动, 叫作\textbf{竖直上抛运动}.
由于只受重力, 所以物体的加速度仍为重力加速度$g$, 
因此可以确定, 竖直上抛运动是匀变速直线运动.

物体向上运动时, 其速度方向与加速度方向相反, 做匀减速直线运动;
当物体的速度减小到0时, 物体到达最高点; 此后, 物体在重力作用下开始下落, 
做自由落体运动(即匀加速直线运动).

\subsubsection{运动规律}
如果以竖直向上为正方向\footnote{通常, 我们取初速度的方向为正方向.}, 
那么竖直上抛运动的加速度大小为$g$, 方向与$g$的方向相反. 通常, 我们说
竖直上抛运动的加速度为$-g$, 这里的``$-$''号表示方向与规定的正方向相反.

若物体的初速度为$v_0$, 则由 \eqref{速度与时间的关系} 式, 
可得$t$时刻物体的速度
\begin{equation}
    v = v_0 - gt.
    \label{竖直上抛1}
\end{equation}
由上式解出的速度$v$, 可能为正值, 也可能为负值. 如果$v$为正值, 则表示$v$的方向与初速度$v_0$的方向相同, 和$g$的方向相反,
物体正在向上做匀减速直线运动; 如果$v$为负值, 则表示$v$的方向与$v_0$的方向相反, 和$g$的方向相同, 物体正在向下做匀加速直线运动.

进一步地, 以竖直向上为正方向, 建立一维坐标系$Oh$, 并规定物体的初位置为$h_0 = 0$, 
则由 \eqref{位移与时间的关系} 式, 物体的位置$h$与时间$t$的关系为
\begin{equation}
    h = v_0t-\frac12 gt^2.
\label{竖直上抛2}
\end{equation}
由上式解出的位置$h$, 可能为正值, 也可能为负值. 正值表示$h$在初位置$h_0$上方, 
负值表示$h$在初位置$h_0$下方.

当物体到达最高点$h = h_\text{m}$处时, 物体的速度$v = 0$. 代入 \eqref{竖直上抛1} 式, 
可以解得物体达到最高点所用的时间
\begin{equation}
t_1 = \frac{v_0}{g}.
\label{竖直上抛3}
\end{equation}
再由 \eqref{竖直上抛2} \eqref{竖直上抛3} 联立求解, 可以得到
$$h_\text{m} = \frac{v_0^2}{2g}.$$

物体到达最高点后, 由于具有加速度$g$, 将开始向下做匀加速直线运动. 当物体落回到原抛出点时, 
位移为0, 代入 \eqref{竖直上抛2} , 可以解得物体从向上抛出到落回到原抛出点所用的时间\footnote{实际上可以解得两个$t$, 这里舍去了$t=0$这个根.}
$$t_2 = \frac{2v_0}{g}.$$
另一方面, 由图\ref{竖直上抛v-t} 易见, 物体回到原抛出点时, $v$-$t$图像在$t$轴上下围成的阴影面积
上下抵消, 此时的速度恰好是$-v_0$.这样, 把$v = -v_0$代入 \eqref{竖直上抛1} 式也可解得$t_2$.
\begin{figure}[bthp]
\centering
\subfloat[$v$-$t$图像]{
    \includegraphics[width=0.3\textwidth]{pic/2-2.pdf}
    \label{竖直上抛v-t}
}
\subfloat[$h$-$t$图像]{
    \includegraphics[width=0.3\textwidth]{pic/2-3.pdf}
    \label{竖直上抛h-t}
}
	\caption{竖直上抛运动的$v$-$t$图像和$h$-$t$图像}
    \label{竖直上抛图}
    
\end{figure}

\subsubsection{运动的对称性}
由于物体在上升阶段和下降阶段的加速度大小均为$g$, 所以上升和下降的过程可以
视为逆过程, 具有高度的对称性.
\begin{enumerate}
    \item \textbf{速度对称}\bre 物体在上升过程和下降过程中经过同一位置时速度大小相等, 方向相反.
    特别地, 物体下落回原抛出点时的速度与初速度大小相等, 方向相反.
    \item \textbf{时间对称}\bre 物体在上升过程和下降过程中经过同一段高度所用的时间相等.
    特别地, 物体上升到最高点所用的时间与物体从最高点落回原抛出点所用的时间相等.
\end{enumerate}

从图\ref{竖直上抛图} 可以清楚地看到运动的对称性. 在解决问题时, 运用上述的对称性会使问题大大简化.

\section{应用:刹车问题}

本章中提到的刹车过程, 均视为匀减速直线运动, 所以匀变速直线运动的
所有公式都可以使用. 
但是需要注意, 我们以前推导运动学公式时, 都是以一个
理想的物理模型为基础. 当物体减速到0时, 如果没有特殊说明, 我们认为它的加速度仍然不变, 
物体将会反向加速.

而刹车问题是一个生活中的实际问题. 处理这些实际问题时, 一定要考虑到实际情况, 
即汽车的速度减小到0后就停下来, 不再运动了.也就是说, 当汽车的速度减小到0时, 
它的加速度将突然变为0(图\ref{刹车v-t}) 所示. 这往往作为一个容易忽略的考点出现在题目中, 一定要多加注意.

\begin{figure}[bthp]
\centering
\subfloat[不含反应时间的$v$-$t$图像]{
    \includegraphics[width=0.3\textwidth]{pic/2-4.pdf}
    \label{刹车v-t}
}
\subfloat[含反应时间的$v$-$t$图像]{
    \includegraphics[width=0.3\textwidth]{pic/2-5.pdf}
    \label{刹车v-t2}
}
	\caption{刹车问题的$v$-$t$图像}
    \label{刹车v-t3}
    
\end{figure}

对于求解$t$时刻的位移$x$的问题, 我们应该分两步来思考.

第一步, 求刹车时间$t_0$. 如果踩下刹车时的速度为$v_0$, 刹车的加速度为$-a$, \footnote{如果以$v_0$的方向
为正方向, 则加速度的方向与正方向相反.} 当汽车的速度$v = 0$时, 代入 \eqref{速度与时间的关系} 可解得刹车时间
$$t_0 = \displaystyle\frac{v_0}{a}.$$

第二步, 比较$t$与$t_0$的大小. 如果$t<t_0$, 则根据位移与时间的关系式, 代入时间$t$计算位移$x$即可; 
如果$t\geqslant t_0$, 则应代入刹车时间$t_0$来计算位移$x$, 或者利用速度与位移的关系式 \eqref{速度与位移的关系}
$$0 - v_0^2 = -2ax$$
来求解位移$x$.

需要再次强调, 计算刹车时间是解决刹车问题的关键步骤. 解决这类问题时如果不考虑刹车时间, 很大概率会得出错误的结果.

\textbf{使用逆向思维解决问题} \bre 从开始制动到刹停, 刹车过程是末速度为0的匀减速直线运动. 但如果我们倒过来看, 
汽车就是在做初速度为0的匀加速直线运动, 就像``视频倒放''一样. 这个思想可以简化一些计算量.比如, 要计算汽车的刹车位移$x$, 
如果用正向思维来看, 则对应公式$$x = v_0t_0 - \frac12 at_0^2.$$
如果用逆向思维来看, 则对应公式$$x = \frac12 at_0^2.$$
实际上, 这两个式子计算得到的$x$是相同的, 但显然后者计算量更小. 因此, 在处理末速度为0的匀减速直线运动时, 我们通常把它转化为初速度为0的匀加速直线运动, 
来简化计算量.

\textbf{含反应时间的刹车问题} \bre 司机从接收信号到作出反应需要一定的时间, 老司机相比新司机的一个重要区别是反应时间$\Delta t$较短.
在反应时间$\Delta t$内, 汽车做的是匀速直线运动, 之后才开始做匀减速直线运动, 如图\ref{刹车v-t2} 所示.
若刹车时车速为$v_0$, 刹车的加速度为$-a$, 则从接收信号到刹停的位移表达式为
$$x = v_0\Delta t + \frac{v_0^2}{2a}.$$

\begin{example}
    汽车紧急刹车过程中会在路面上留下刹车痕迹.某次紧急刹车后测得刹车痕迹长为$36\unit{m}$, 
    假设制动后汽车做加速度大小恒为$8\unit{m/s^2}$的匀减速直线运动直到停止, 则下列说法正确的是(\bre)

    A. 刹车后前4\ s内的位移大小为32\ m
    
    B. 刹车后第1\ s末的速度大小为16\ m/s

    C. 刹车后第4\ s末的速度大小为8\ m/s

    D. 刚刹车时, 汽车的初速度大小为 24\ m/s
\end{example}

\textbf{答案} \bre B

\textbf{解析} \bre 根据速度与位移的关系式, 对刹车的全过程有$$0 - v_0^2 = -2ax, $$代入数据, 解得汽车的初速度大小
$$v_0 = \sqrt{2ax} = \sqrt{2\times 8\times 36}\unit{m/s} = 24\unit{m/s}.$$
故D错误.

根据速度与时间的关系式, 对刹车的全过程有$$0 = v_0 - at_0, $$解得汽车速度减小到0的时间$$t_0 = \frac{v_0}{a} = \frac{24}{8}\unit{s}= 3\unit{s}.$$
即第3\ s末汽车的速度就减小到0.所以刹车后第4\ s末的速度大小为0, 故C错误. 刹车后前4\ s的位移就是刹车位移$36\unit{m}$, 故A错误.

刹车后第1\ s末的速度大小为$$v_1 = v_0 - at = 24\unit{m/s} - 8\times 1\unit{m/s} = 16\unit{m/s}.$$故B正确.

\section{匀变速直线运动的推论}

\subsection{平均速度\bre 中央位移处的瞬时速度}

在\ref{速度节} 节中, 我们用平均变化率的思想定义了平均速度, 即位移与发生
这段位移所用的时间之比. 如果用函数$x = x(t)$描述物体的位置随时间的变化, 那么该物体从
$t = t_1$时刻到$t = t_2$时刻的平均速度为
\begin{equation}
    \overline{v} = \frac{x(t_2) - x(t_1)}{t_2 - t_1}.
    \label{平均速度}
\end{equation}

对于匀变速直线运动$$x(t) = x_0 + v_0t + \frac12at^2$$
来说, 它的平均速度有什么特别的规律吗? 上式中$v_0$为$t =0$时刻的速度, $x_0$为$t=0$时刻的位置, $a$为匀变速直线运动的加速度. 
把$x(t)$代入 \eqref{平均速度}, 可以得到\vspace{10pt}
\begin{equation}
    \begin{aligned}
    \overline{v} &= \displaystyle\frac{\left(v_0t_2+\displaystyle\frac12at_2^2\right) - \left(v_0t_1+\displaystyle\frac12at_1^2\right)}{t_2 - t_1} 
    = \displaystyle\frac{v_0(t_2-t_1)+\displaystyle\frac12a(t_2+t_1)(t_2-t_1)}{t_2 - t_1}\\[10pt]
    &= v_0+\frac12a(t_2+t_1) = \frac{2v_0 + at_2+at_1}{2}.
\end{aligned}
\label{pjsd1}
\vspace{15pt}
\end{equation}

如果设$t_1$时刻的速率为$v_1$, $t_2$时刻的速率为$v_2$, 则根据速度与时间的关系式 \eqref{速度与时间的关系} 分别有
$$at_1 = v_1 - v_0, $$
$$at_2 = v_2 - v_0. $$
把以上两式分别代入 \eqref{pjsd1} 中, 就得到
\begin{empheq}[box=\fbox]{equation}
    \overline{v} = \frac{v_1 + v_2}{2}.
    \label{平均速度公式}
\end{empheq}
这就是说, \textbf{做匀变速直线运动的物体, 在一段时间内的平均速度, 等于这段时间始末时刻的瞬时速度的均值\footnote{如无特殊说明, 本书中提到两个量的均值, 指的都是它们的算术平均值.}.}

\begin{figure}[bthp]
\centering
\subfloat[$v$-$t$图像]{
    \includegraphics[width=0.3\textwidth]{pic/2-10.pdf}
    \label{中央时刻的瞬时速度}
}
\subfloat[$x$-$t$图像]{
    \includegraphics[width=0.3\textwidth]{pic/2-11.pdf}
    \label{中央位移}
}
	\caption{物体在$t_1\sim t_2$的平均速度}
    
\end{figure}

其实, 根据匀变速直线运动的 $v$-$t$ 图像, 可以很容易地看出这一结论. 由于匀变速直线运动的速度
关于时间是一次函数, 所以匀变速直线运动的 $v$-$t$ 图像是一条直线, 如图\ref{中央时刻的瞬时速度} 所示. 
由图可知, \textbf{物体在任意一段时间$t_1\sim t_2$内的平均速度, 
等于这段时间中央时刻$\displaystyle\frac{t_1+t_2}{2}$的瞬时速度}, 也就是$v_1$与$v_2$的均值$\overline{v}$.
这就是\textbf{平均速度推论}.事实上, 由速度与时间的关系式 \eqref{速度与时间的关系}, 有
$$v\left(\frac{t_1+t_2}{2}\right) = \frac{at_1+at_2}{2} = \frac{v_1+v_2}{2} = \overline{v}.$$

需要注意的是, 上面的推导指出, 物体在某段时间的平均速度, 等于这段时间中央时刻的瞬时速度, 但并非中央位置处的瞬时速度.
由于物体的位移随时间的变化并非线性, 对于匀加速直线运动来说, \vspace{5pt}
$x(t)$是一个向下凸的函数(如图\ref{中央位移})\footnote{匀加速直线运动的位移$x$与时间$t$呈二次项系数为正的二次函数关系, 其$x$-$t$图像开口向上, 所以说它``向下凸'';而匀减速直线运动的$x$-$t$图像开口向下, 即``向上凸'', 结论与上面相反: 当物体运动到中央位置时, 还未到达中央时刻.}
, 因此物体运动到中央时刻$\displaystyle\frac{t_1+t_2}{2}$时, 
还并未到达中央位置$\displaystyle\frac{x(t_1) + x(t_2)}{2}$处.

这时就会问, 中央位移处的瞬时速度又如何表示呢? 考虑物体从坐标原点运动到$x$位置处, 初速度和末速度分别为$v_1$和$v_2$, 
中央位移$\displaystyle\frac{x}{2}$处的速度大小为$v_\text{中}$, 则由速度与位移的关系式 \eqref{速度与位移的关系} 有
$$v_2^2 - v_1^2 = 2ax,$$
$$v_\text{中}^2 - v_1^2 = 2a\frac{x}{2}.$$
联立以上两式, 可以解出中央位移处的速度大小
$$v_\text{中} = \sqrt{\frac{v_1^2 + v_2^2}{2}}.$$
可以发现, 这个表达式中$v_1$和$v_2$的系数相同, 即``权重''相同; 而表达式的次数却仍为一次. 看起来, 这也像是
$v_1$和$v_2$的某种``均值''. 数学中, 这个式子叫作$v_1$和$v_2$的平方平均值, 这样就可以说: \textbf{做匀变速直线运动的物体, 
运动到中央位置时的瞬时速度, 等于其初速度与末速度的平方平均值.}

\begin{example}
一质点做匀加速直线运动, 依次通过$a, b, c$三点.测得$a, b$两点之间的距离为$s$, 
$b, c$两点之间的距离为$4s$.质点通过$ab$端所用时间为$t$, 通过$bc$端所用时间为$2t$, 
求该质点运动的加速度大小.
\end{example}
\begin{solution}
    根据平均速度推论, 质点在$ab$段运动的中央时刻的速度, 等于其在$ab$段运动的平均速度, 为
    $$v_1 = \frac{s}{t}.$$
    同理, 质点在$bc$段运动的中央时刻的速度为
    $$v_2 = \frac{4s}{2t} = \frac{2s}{t}.$$

    $ab$段的中间时刻与$cd$段中间时刻的时间间隔为
    $$\Delta t = \frac{t}{2} + \frac{2t}{2} = \frac{3t}{2}.$$

    根据速度增量与时间间隔的关系 \eqref{速度增量与时间间隔的关系} 有
    $$v_2-v_1 = a\Delta t.$$
    由此可以解得$a = \displaystyle\frac{2s}{3t^2}.$

\end{solution}

\begin{example}
一质点做匀加速直线运动, 在$AB$, $BC$两端相等位移$x$内运动的时间分别为$T_1$, $T_2$, 
试表示质点在$ABC$段的加速度大小.
\end{example}
\begin{solution}
    质点在$AB$段和$BC$段的平均速度分别为
    $$v_1 = \frac{x}{T_1}, \bre\bre v_2 = \frac{x}{T_2}.$$
    两式作差, 即得
    $$\Delta v = v_2 - v_1 = \frac{x(T_2-T_1)}{T_1T_2}.$$
    
    根据平均速度推论, 质点从$v_1$加速到$v_2$所用时间为
    $$\Delta t = \frac{T_1+T_2}{2}.$$
    于是质点的加速度为
    $$a = \frac{\Delta v}{\Delta t} = \frac{2x(T_1-T_2)}{T_1T_2(T_1+T_2)}.$$

\end{solution}

\subsection{相邻且相等的时间间隔的位移差恒定}

我们知道, 做匀变速直线的物体, 其速度随时间呈一次函数的关系变化, 而其位移随时间呈二次函数的关系变化. 
如果你学习过等差数列, 可能会想到等差数列的前$n$项和公式$$S_n = na_1+\frac{n(n-1)}{2}d = \frac{d}{2}n^2+\left(a_1-\frac{d}{2}\right)n$$
是一个没有常数项的二次函数, 但其通项公式$$a_n = a_1+ nd$$却是一个一次函数. 

一方面, 匀变速直线运动速度与时间的关系式 \eqref{速度与时间的关系} 和等差数列的通项公式形式很像;
另一方面, 规定物体的初位置为坐标原点, 则位移与时间的关系式 \eqref{位移与时间的关系} 也是一个没有常数项的二次函数. 
那么, 各个相等时间间隔的位移大小, 是否具有等差数列的特征呢?
答案是肯定的. 事实上, 我们有这样的结论:

\textbf{推论}\bre \textbf{做加速度为$a$的匀变速直线运动的物体, 在任意两个连续且相等的时间间隔$T$内的位移之差恒定, 且}
\begin{empheq}[box=\fbox]{equation}
    \Delta x = aT^2.
    \label{逐差法}
\end{empheq}

\textbf{证明}\bre 设$x_1$和$x_2$是两段连续且时间间隔均为$T$的位移, 则有
$$x_1 = v_0T + \frac12aT^2, \bre\bre x_1 + x_2 = v_0T + \frac12a(2T)^2. $$
用第二式减去两倍的第一式, 即得$x_2 - x_1 = aT^2.$

\setlength{\abovedisplayskip}{0pt}
\setlength{\belowdisplayskip}{0pt}

\begin{example}
    物体从静止开始做匀加速直线运动, 已知第4 s内与第2 s内的位移之差是8 m, 则下列说法错误的是(\bre)

    A. 加速度大小为$8\unit{m/s^2}$
    
    B. 第2 s末的速度为8 m/s

    C. 第2 s内的位移为6 m

    D. 前3 s内的平均速度为6 m/s
\end{example}
\textbf{答案}\bre A

\textbf{解析}\bre 
    设第4 s内与第2 s内的位移分别为$x_4, x_2$, 则由$$x_4-x_2 = 2aT^2, $$
    可知, 物体运动的加速度为
    $$a = \frac{x_4-x_2}{2T^2} = \frac{8}{2\times 1} = 4\unit{m/s^2}.$$
    故A错误.

    物体在第2 s末的速度为$$v_2 = at = 4\time 2\unit{m/s} = 8\unit{m/s}.$$
    故B正确.

    物体在第1 s内的位移为$$x_1 = \frac12 at^2 = \frac12 \times 4\times 1^2\unit{m} = 2\unit{m}.$$
    由A可知$aT^2 = 4\unit{m}$, 则由$$x_2 - x_1= aT^2$$可得$x_2 = 6\unit{m}$, 即第2 s内的位移为6 m, 故C正确.

    类似C, 物体在第3 s内的位移为$$x_3 = x_2+aT^2 = 10\unit{m}.$$
    所以物体在前3 s内的位移为$$x_1+x_2+x_3 = 18\unit{m}.$$
    因此物体在前3 s内的平均速度为$$\overline{v} = \frac{18}{3}\unit{m/s} = 6\unit{m/s}.$$

    物体在前3 s内的位移也可通过$$x_1+x_2+x_3 = \frac12at^2 = \frac{1}{2}\times 4\times 3^2 \unit{m} = 18\unit{m}$$
    求出.故D正确.

\subsection{比例法}
\subsubsection{相邻且相等的时间间隔的位移之比}

对于初速度为0, 加速度为$a$的匀加速直线运动, 它的位移$x$与时间$t$的关系为\footnote{此后如果没有特别说明, 均规定物体的初位置为坐标原点, 并简单地用$x$表示位移.}
\begin{equation}
    x = \frac12 at^2.
    \label{无初速度的x-t关系}
\end{equation}
由此可以得到, 前$t$\ s, 前$2t$\ s, 前$3t$\ s, $\cdots$的位移之比为
$$1^2:2^2:3^2:\cdots.$$
对上面的结果逐一作差, 可以得到第$t$\ s内,第$2t$\ s内, 第$3t$\ s内$\cdots$的位移之比为
$$1:3:5:\cdots.$$
可以看出, 相邻且相等的时间间隔的位移之差恒定\footnote{这里就用到了等差数列的知识.}. 这与我们之前得到的结论一致.

上面的比例关系, 仅适用于初速度为0的匀加速直线运动. 解决自由落体运动等问题时, 利用这个结论可以极大的简化计算量.

\subsubsection{相邻且相等的位移间隔的速度之比}
对 \eqref{无初速度的x-t关系} 式变形, 从中解出$t$, 可得
$$t = \sqrt{\frac{2x}{a}}.$$
可见, 物体通过的位移为$x$\ m, $2x$\ m, $3x$\ m, $\cdots$所用的时间之比为
$$1:\sqrt{2}:\sqrt{3}:\cdots.$$
作差可知, 物体从速度为0开始, 通过连续且相等的位移所用的时间之比为
$$1:(\sqrt{2} - 1):(\sqrt{3}-2):\cdots.$$

用相邻且相等时间的位移差公式 \eqref{逐差法} 也可以推导出相同的结论.

\section{误差和有效数字}
\subsection{误差}

任何测量结果都不可能绝对准确. 测量值跟被测物理量的真实值之间总会有差
异, 这种差异叫作\textbf{误差}. 

由仪器直接读出的数据叫作\textbf{直接测量结果}. 直接测量结果的误差来源于两方
面. 一是仪器本身的缺陷, 例如刻度尺的刻度不够准确, 天平两臂不严格等长; 二
是实验人员操作和读数的不准确, 例如用停表计时, 按下按钮的时刻总会稍有滞后
或提前. 又如, 毫米刻度尺的分度值是 1 mm, 靠目测只能估计到分度值的$\dfrac{1}{10}$, 
即 0.1 mm, 肯定会有误差.

通过改进仪器和谨慎操作, 可以减小误差, \textbf{但任何仪器的测量都不能完全消灭误差.}

\subparagraph{偶然误差与系统误差} 人读仪表时, 眼睛的位置可能稍向左偏或稍向右偏, 电表指针由于轴上的摩擦, 停下来的位置可
能一次偏大, 一次偏小. 这些误差是由偶然因素造成的, 叫作\textbf{偶然误差}.
偶然误差的特点是, 当多次重复测量时, 偏大和偏小的机会比较接近, 可以用取平均值的方法来减小偶然误差.

\textbf{系统误差}是由仪器结构缺陷, 实验方法不完善造成的. 例如, 天
平两臂不完全等长会造成系统误差; 测量重力加速度时由于没有消除空气阻力等影
响也会引起系统误差. 系统误差的特点是, 多次重复测量的结果总是大于(或小
于)被测量的真实值, 呈现单一倾向.

\subparagraph{绝对误差与相对误差} 假设用分度值
为 1 mm 的刻度尺测一段铜丝的直径, 读数为 1.3 mm. 限于刻度尺的精确度, 测
量有一定的误差. 若被测量铜丝直径的真实值为 1.2 mm, 那么测量值与真实值之
差则为
$$1.3 \unit{mm} - 1.2 \unit{mm} -0.1 \unit{mm}.$$
物理学中把测量值与真实值之差, 叫作\textbf{绝对误差}.

除绝对误差之外, 人们通常要考虑这个绝对误差是在测量多大的测量值时产
生的, 如果绝对误差相同, 但测量值大, 这个绝对误差在整个测量中所占的比例
就小. 物理学中把绝对误差与测量值之比, 叫作\textbf{相对误差}\footnote{严格地讲, 相对误差是绝对误差与真实值之比, 在估算时, 分母可用测量值代替.}, 
即$$\text{相对误差}=\frac{\text{绝对误差}}{\text{测量值}}.$$

相对误差是一个比值, 没有单位, 通
常用百分数表示. 在绝对误差相同的情况下, 测
量值越大, 相对误差就越小. 

\subsection{有效数字}
使用毫米刻度尺时, 我们通常估读一位, 即估计到分度值的$\dfrac{1}{10}$.
比如, 在某一次测量中, 待测点$A$的中心位于 11 mm 和 12 mm 之间,  
我们将估读到毫米数的下一位, 假设得到$x_A = 11.3 \unit{mm}$. 其中的末位数 3 是估读的, 不可靠的 .
但是这个不可靠的数字 3 还是有用的, 仍要写出来, 
因为它表示点 A 在 11 mm 和 12 mm 两条刻度线之间, 靠近中点但离 13 mm 刻度线
稍近一些的位置. 

这种带有一位不可靠数字的近似数字叫作\textbf{有效数字}.例
如, 13.3 mm 有三位有效数字, 改写成 1.33 cm 或0.0133 m仍是三位有效数字.

{\kaishu 如果仍用这把刻度尺测量另一点 $B$ 的位置, 点 B 的中心恰好与 19 mm 的刻度
线正对, 这种情况应当怎样处理呢?}

正确的记录是 $x_B = 19.0 \unit{mm}$, 而不应
写成 19 mm. 这种情况下, 1, 9 都是可靠
的, 但若写成 19 mm, 就意味着数字 9 是
不可靠的, 显然与事实上测量仪器的准
确度不相符. 由此可以知道, 最末一位
非零数字后面的数字 0 是有意义的, 不
能随意舍去或添加. 

在小数中, 第一位非零数字前面的 0 是用于表示小数点的位置的, 
不是有效数字. 例如 0.92 s, 0.085 kg, 0.0063 m, 都是两位有效数字. 为了正
确表述有效数字, 特别大或特别小的数字都要用科学记数法, 例如月球到地球的平
均距离为 $3.84\times 10^5$ km,  钨原子的半径
为 $1.37\times10^-10$ m. 

实验的结果往往是由若干直接测量值
经过运算得到的, 每一个直接测量值的误
差都对最后结果产生影响, 应当按一定的
数学方法来确定运算结果的有效数字. 但
是这种处理方法比较复杂, 中学阶段不做
要求, 运算结果一般取两位或三位有效数
字就可以了. 

\chapter{力}

\section{力的分类}

在力学中, 物体间的相互作用抽象为一个概念——\textbf{力}.力的单位是\textbf{牛顿}, 简称\textbf{牛}, 
符号用 N 表示. 力是矢量, 既有大小, 又有方向.
初中时我们就知道, 力可以使物体发生形变, 还可以改变物体的运动状态, 即产生加速度, 这都是物体间相互作用的的效果. 

力可以分为哪几种呢?
在机械运动中, 我们常见的力按性质可以分为三种: 重力, 弹力, 摩擦力. 其中重力是\textbf{场力}, 
它的作用不需要物体与物体的接触, 我们之后会学的静电力, 安培力等等也都是场力; 弹力和摩擦力都是接触力, 它们的产生都要求物体间直接接触.

这一节我们主要研究重力, 弹力, 摩擦力这三种力.

\subsection{重力}

由于地球的吸引而使物体受到的力叫作\textbf{重力}.物体受到的重力 $G$
与物体质量 $m$ 的关系是
$$G = mg.$$
其中 $g$ 是物体做自由落体运动时的加速度, 称为\textbf{重力加速度}.

我们知道, 加速度的单位是$\mathrm{m/s^2}$. 看起来, 物理量$mg$的单位应该是$\mathrm{kg\cdot m/s^2}$, 
这是力的单位吗? 这个单位和牛顿如何换算呢? 实际上, $\mathrm{kg\cdot m/s^2}$和牛顿是同一个单位, 即
$$1\unit{kg\cdot m/s^2} = 1 \unit{N}.$$
在学习了牛顿第二定律之后, 我们将会理解其中的原因.

一个物体的各部分都受到重力的作用, 从效果上看, 
可以认为各部分受到的重力作用集中于一点, 这一点叫作物体的\textbf{重心}.物体的几何形状, 
物体的质量分布都会影响物体中心的位置. 对于质地均匀, 形状规则的物体, 重心在它们的几何中心上.
需要注意的是, 重心不一定在物体上, 比如一个质地均匀的圆环, 它的重心就在空心圆的中心处.

力是矢量, 可以用有向线段表示. 有向线段的长短表示力的大
小, 箭头表示力的方向, 箭尾表示力的作用点. 这
种表示力的方法, 叫作\textbf{力的图示}.
通常, 我们不需要准确标度力的大小, 只需画出力的作用点和方向, 
即画出\textbf{力的示意图}.
在解决问题时, 我们经常要画出力的示意图, 
来分析物体的受力, 这个过程叫作\textbf{受力分析}. 学习了
运动与力的关系之后, 我们将通过受力分析来确定物体的运动状态. 

\subsection{弹力}
\subsubsection{弹力}
物体在力的作用下形状或体积会发生改变, 这种变化
叫作\textbf{形变}. 发生形变的物体, 要恢复
原状, 对与它接触的物体会产生力的作用, 这种力叫作\textbf{弹力}.

放在地板上的物体, 它对地板的压力以及地板对它的
支持力, 都是弹力, 其方向是跟接触面垂直的; 绳子的拉
力, 也是弹力, 其方向是沿着绳子而指向绳子收缩的方向.
一般说来, 弹力的方向总是与施力物体发生形变的方向垂直.

\subsubsection{胡克定律}
物体在发生形变后, 如果撤去作用力能够恢复原状, 
这种形变叫作\textbf{弹性形变}. 如果形变过
大, 超过一定的限度, 撤去作用力后物体不能完全恢复原
来的形状, 这个限度叫作\textbf{弹性限度}.

英国科学家胡克经过研究发现, 在弹性限度内, 弹簧
发生弹性形变时, 弹力 $F$ 的大小跟弹簧伸长(或缩短)的
长度 $\Delta x$ 成正比, 即
\begin{empheq}[box=\fbox]{equation}
    F = k\Delta x.
    \label{胡克定律}
\end{empheq}

这个规律叫作\textbf{胡克定律}. 式中 $k$ 是比例常数, 叫作弹簧的\textbf{劲度系数}, 单位是牛顿每米, 符号是 N/m. 
生活中说有的弹簧``硬'', 有的弹簧``软'', 指的就是
它们的劲度系数不同.

\subsection{摩擦力}

\subsubsection{滑动摩擦力}

两个相互接触的物体, 当它们相对滑动时, 
在接触面上会产生一种阻碍相对运动的力, 这种力叫作\textbf{滑
动摩擦力}. 滑动摩擦力的方向总是
沿着接触面, 并且跟物体相对运动的方向相反. 

初中我们已经知道, 摩擦力与接触面上压力的大小有关, 还与接触面的粗糙程度有关. 
进一步的定量实验证明, \textbf{滑动摩擦力的大小跟压力的大小成正比}.
如果用 $F_\text{f}$ 表示滑动摩擦力的大小, 用 $F_\text{N}$ 表
示压力的大小, 则有
\begin{empheq}[box=\fbox]{equation}
    F_\text{f} = \mu F_\text{N}.
    \label{滑动摩擦力}
\end{empheq}
其中, $\mu$是比例常数, 叫作\textbf{动摩擦因数}. 它的值跟接触面有关, 接触面材料不同, 粗糙程
度不同, 动摩擦因数也不同.

\subsubsection{静摩擦力}

相互接触的两物体处于相对静止时, 它们之间是否可能存在摩擦力呢? 试想一个比较瘦弱的人
用平行于地面的力去推沙发, 但沙发没有被推动.根据二力平衡的知识可知, 这时一定有一个力
与推力大小相等, 方向相反.这个力就是沙发与地面间的摩擦力.

由于这时相互接触的两个物体之间只有相对运动的趋
势, 而没有相对运动.所以这时的摩擦力叫作\textbf{静摩擦力}. 
静摩擦力的方向总是跟物体相对运
动趋势的方向相反. 只要沙发与地面间没有产生相对运动, 
静摩擦力的大小就随着推力的增大而增大, 并与推力保持
大小相等.

静摩擦力的增大有一个限度. 物体所受静摩
擦力的最大值 $F_\text{max}$ 在数值上等于物体即将开始运动时的拉
力. 两物体之间实际产生的静摩擦力 $F$ 在 0 与最大静摩擦力
$F_\text{max}$ 之间, 即
$$0 < F \leqslant F_\text{max}.$$
实际上, 这个力$F_\text{max}$与物体开始运动后受到的滑动摩擦力很接近, 并且略大于滑动摩擦力.
为了简化计算, 在中学阶段, 我们通常认为静摩擦力的最大值$F_\text{max}$等于滑动摩擦力.

瓶子可以拿在手中, 靠的是静摩擦力的作用.皮带运
输机能把货物送往高处, 也是静摩擦力作用的结果.

\section{力的相互作用}

力的作用是相互的. 相互作用的力其大小有
什么关系? 例如, 大人跟小孩掰手腕, 很容易就
把小孩的手压在桌面上. 那么, 大人对小孩的手的作用力, 
是否大于小孩对大人的手的作用力呢?

\subsection{作用力与反作用力}
力是物体对物体的作用. 只要谈到力, 就一定存在着\textbf{受力物体}和\textbf{施力物体}.

坐在椅子上用力推桌子, 会感到桌子也在推我们, 我们的
身体要向后仰. 我们常说, 地面上及地球附近的物体受到
地球的吸引. 其实, 地球也受到它们的吸引, 地球
和物体之间的作用也是相互的.

可见, 两个物体之间的作用总是相互的. 物体间相互作用的这一对力, 
通常叫作\textbf{作用力}和\textbf{反作用力}, 它们是一对\textbf{相互作用力}.

\subsection{牛顿第三定律}

作用力和反作用力的大小之间, 方向之间有什么样的
关系?

把A, B两个弹簧测力计连接在一起, B的
一端固定, 用手拉测力计 A.可以看到, 
两个测力计的示数总是相等, 方向相反.

上面是通过弹力进行的实验, 摩擦力满足上面的
关系吗? 如果是不相互接触的力呢? 牛顿经过研究指出: \textbf{两个物体之间的作用力和反作用
力总是大小相等,方向相反,作用在同一条直线上. }
这就是\textbf{牛顿第三定律}.

在生活和生产中应用牛顿第三定律的例子是很多的. 

汽车的发动机驱动车轮转动, 由于轮胎和地面之间的摩
擦, 车轮向后推地面, 地面给车轮一个向前的反作用力, 使
汽车前进. 汽车受到的驱动力就是这样产生的. 
若把驱动轮架空, 不让它跟地面接触, 这时车轮虽然转动, 
但车轮不推地面, 地面也就不会产生向前推车的力, 汽车
就不会前进. 

陷在泥泞中的汽车, 尽管车轮飞转, 但是如
果泥和车轮之间太滑, 车轮得不到足够的摩擦力, 车也是
出不来的. 

\subsection{对物体进行受力分析}
\label{斜面物块问题}

分析物体的受力, 有两条思路. 一是根据物体运动状
态的变化来分析和判断其受力情况, 这是下一章学习的内
容; 二是根据各种力的特点, 从相互作用的角度来分析物
体的受力. 下面我们先从后一思路来分析物体的受力情况. 它的主要步骤包括: 

\begin{enumerate}
    \item 确定研究对象: 将研究对象从周围物体中隔离出来, 只分析其所受的力, 必要时也可将几个物体看作一个整体进行分析;
    \item 画力的示意图: 先画场力(如重力和电场力等), 再画接触力(如弹力和摩擦力);
    \item 列方程计算: 根据受力图和力的平衡条件, 列出方程进行求解.
\end{enumerate}

\begin{figure}[bthp]
    \centering
    \includegraphics[width=0.3\textwidth]{pic/2-20.pdf}
    \caption{斜面上的物体受力示意图}
    \label{斜面1}
\end{figure}

如图 \ref{斜面1} 所示, 一个木块静止在粗糙斜面上, 我们按重力, 弹力
和摩擦力的顺序来分析它的受力情况. 木块受到重力$G$, 方
向竖直向下; 木块和斜面接触并相互挤压, 木块受到垂直于
斜面向上的弹力$F_\text{N}$; 假设木块和斜面之间没有摩擦, 木块就
会向下滑动, 由此可以判断, 静止的木块相对斜面有向下滑
动的趋势, 所受的静摩擦力 $F_\text{f}$ 是沿斜面向上的.

需要注意, 在这个问题中一共有3组作用力与反作用力出现, 分别是重力$G$与木块对地球的引力, 弹力$F_\text{N}$
和木块对斜面的压力, 静摩擦力 $F_\text{f}$ 和木块对斜面的静摩擦力.
由于我们是在对木块的受力情况进行分析,所以只把这6个力中
木块所受的3个力画出来了.

\subsection{平衡力与相互作用力}

在分析物体的受力情况时,不要把某个力的反作用力
跟这个力的平衡力混淆. 

比如, 人站在地面上, 人受到重力$G$和地面的支持力$F$, 
由于人是静止的, 并且不再受其他力的作用, 因此重力$G$和地面的支持力$F$是一对相互平衡的力.
它们大小相等, 方向相反, \textbf{作用在同一个物体上}.

另一方面, 地面以力$F$向上支持着人, 同时人又以力$F^{\prime}$向下压着地面. 
支持力$F$与压力$F^{\prime}$是一对作用力与反作用力, 它们大小相等, 方向相反, \textbf{作用在两个物体上}.

``一对相互平衡的力''和``一对作用力和反作用力''还有一
个区别: 后者一定是同一种类的力(例如$F$和$F^{\prime}$都是弹力), 而
前者则不一定是同一种类的力(例如$F$是弹力, 而$G$是重力).

\section{力的合成与分解}

如果一个物体受到的力作用于物体上的同一点或者它们的作用线交于一点, 这样的一组力叫作\textbf{共点力}.
对于非共点力来说, 物体所受各力的作用线不交于一点, 在分析时通常需要考虑力矩和合力的方向, 比较复杂.
中学阶段我们只学习共点力的合成与分解.

\subsection{合力与分力}

生活中常常见到这样的事例:一个力的单独作用与两
个或者更多力的共同作用, 其效果相同.
一盏吊灯悬吊在天花板上保持静止, 悬线对吊灯的拉
力是 $F$, 若用两根线共同悬挂吊灯, 悬线上端
分别固定在天花板的左右两处, 线的拉力是 $F_1$ 和 $F_2$, 也能
产生使吊灯保持静止的效果.

当一个物体受到几个力的共同作用时, 我们常常可以求出这样一个力,
这个力产生的效果跟原来几个力的共同效果相同, 这个力就叫作那几个力的\textbf{合力},
原来的几个力叫作\textbf{分力}. 在前面的例子中, 
可以说, $F$是 $F_1$ 和 $F_2$ 的合力, $F_1$ 和 $F_2$ 都是$F$的一个分力.

\subsection{共点力的合成与分解}

\begin{figure}[bthp]
\centering
\subfloat[平行四边形定则]{
    \includegraphics[width=0.3\textwidth]{pic/3-1.pdf}
    \label{平行四边形定则}
}\hspace{15pt}
\subfloat[三角形定则]{
    \includegraphics[width=0.3\textwidth]{pic/3-2.pdf}
    \label{三角形}
}
	\caption{力的合成法则}
    \label{力的合成法则}
\end{figure}

在物理学中, 我们把求几个力的合力的过程叫作\textbf{力的合成}, 把求一个力的分力的过程叫作\textbf{力的分解}.

共点力的合成与分解遵循什么规律呢? 实验表明, \textbf{两个力合成时, 如果以表示这两个力的有向线段为邻边作平行四边形, 这两个
邻边之间的对角线就代表合力的大小和方向.} 这个规律叫作\textbf{平行四边形定则}. 
如图 \ref{平行四边形定则} 所示, 根据平行四边形定则, $F$是 $F_1$ 和 $F_2$ 的合力.

需要指出的是, 如果没有限制, 对于同一条对角线, 
可以作出无数个不同的平行四边形. 也就是说, 
同一个力 $F$ 可以分解为无数对大小, 方向不同的分力. 反过来, 给定两个力$F_1$, $F_2$的大小, 但不规定它们的方向, 
则可以合成出无数个大小, 方向不同的合力$F$.

当$F_1$ 和 $F_2$方向相同时, 它们的合力$F$也沿着这个方向, 并且$F$的大小为$F_1+F_2$. 此时的$F$最大.

当$F_1$ 和 $F_2$方向相反时, 两个力相互抵消, 产生的合力$F$沿着较大的那个分力的方向, 
并且$F$的大小为$|F_1-F_2|$. 此时的$F$最小. 即合力$F$的范围是
$$|F_1 - F_2| \leqslant F \leqslant F_1 + F_2.$$
特别地, 如果$F_1 = F_2$, 且它们的方向相反, 则
这两个的合力为0, 即它们是一对平衡力.

观察图 \ref{平行四边形定则}, 如果把$F_2$平移到平行四边形右侧虚线的位置, 让$F_2$与$F_1$首尾相接, 
则合力$F$的矢量箭头从$F_1$的起点指向$F_2$的终点, 它们正好构成一个封闭的三角形(图 \ref{三角形定则}). 像这样, 
把两个矢量首尾相接从而求出合矢量, 这个规律叫作\textbf{三角形定则}.

三角形定则是平行四边形定则的简化, 它们本质上是一样的. 实际使用时, 三角形定则可以省去平行四边形
的那两条虚线, 绘制起来更简单也更直观.

根据三角形定则可以直观地看到, 两个力的大小不变时, 其合力随着它们夹角的增大而减小, 这与前面通过平行四边形定则得出的结论是一致的.

\subsection{矢量和标量}
力的合成, 可以认为是力的相加. 二力相加时, 不能
简单地把两个力的大小相加, 而要按平行四边形定则来确
定合力的大小和方向.

位移的合成也遵循一样的规律. 考虑一个人先从$A$点走到$B$点, 其位移为矢量$AB$, 又拐弯从$B$点走到了$C$点, 其位移为矢量$AC$.
显然这个人的合位移是由三角形定则所确定的矢量$AC$.

像这样, 既有大小又有方向, 相加时遵从平行四边形定则(或三角形定则)的物
理量叫作\textbf{矢量}. 只有大小, 没有方向, 相加时遵
从算术法则的物理量叫作\textbf{标量}.

区分矢量与标量, 关键就在于看这个物理量的合成遵循平行四边形定则还是算术法则.
除了力和位移以外, 速度, 加速度都是矢量. 质量, 路程, 功, 电流等都是标量.

\section{共点力的平衡}
\subsection{共点力平衡的条件}

\textbf{物体受到几个力作用时, 如果保持静止或匀速直线运
动状态, 我们就说这个物体处于平衡状态.}

受共点力作用的物体, 在什么条件下才能保
持平衡呢?

先考虑二力平衡的情形. 作用在同一物体上的两个力, 如果大小相等, 方向相
反, 并且在同一条直线上, 则这两个力平衡. 二力平衡时物
体所受的合力为 0.

如果物体受到多个共点力作用, 我们可以逐步通过力
的合成, 最终等效为两个力的作用. 如果这两个力的合力
为 0, 则意味着所有力的合力等于 0, 物体将处于平衡状
态. 因此, \textbf{在共点力作用下物体平衡的条件是合力为 0}, 
我们把它称为共点力平衡的条件.

\begin{figure}[bthp]
\centering

\subfloat[力的示意图]{
    \includegraphics[width=0.3\textwidth]{pic/2-21.pdf}
    \label{斜面2}
}
\subfloat[力的三角形]{
    \includegraphics[width=0.15\textwidth]{pic/2-22.pdf}
    \label{斜面3}
}
\subfloat[力的正交分解]{
    \includegraphics[width=0.3\textwidth]{pic/2-24.pdf}
    \label{斜面5}
}
	\caption{斜面上的物体受力示意图}
    \label{斜面4}
    
\end{figure}

\subsection{力的三角形}
\begin{wrapfigure}{o}{6cm}
	\flushright
	\includegraphics[width=0.3\textwidth]{pic/3-3.pdf}
    \caption{力的三角形}
	\label{三角形2}
\end{wrapfigure}

在共点力的平衡问题中, 三个力的平衡问题是最基础的. 相互平衡的三个力, 它们的大小和方向之间有什么关系呢?

回看图\ref{三角形}, $F$是$F_1$与$F_2$的合力. 由共点力平衡的条件知道, 如果我们要找一个力$F^{\prime}$与$F_1$及$F_2$平衡, 就是要
$F^{\prime}$和$F$平衡. 根据二力平衡的知识, $F^{\prime}$和$F$大小相等, 方向相反, 画出来就是一条与$F$重合, 
但箭头指向相反的矢量. 如图\ref{三角形2}, $F_1$, $F_2$和$F^{\prime}$是相互平衡的三个共点力.

观察图\ref{三角形2} 可以发现, $F_1$, $F_2$和$F^{\prime}$这三个力的矢量首尾相接.
可见, \textbf{物体受三个共点力作用而平衡时, 这三个力的矢量箭头首尾相接, 构成一个闭合的三角形.}
我们把它叫作\textbf{力的三角形}或者\textbf{矢量三角形}.

可以这样描述三力平衡的条件: \textbf{物体在三个共点力的作用下平衡, 其中任何一个力与另外两个力的合力大小相等, 方向相反; 并且这三个力的
矢量可以构成一个闭合的三角形.}

通过力的三角形, 我们可以把复杂的平衡问题转化为几何问题来处理. 

回顾 \ref{斜面物块问题} 节的例子, 静止在粗糙斜面上的木块受到重力$G$, 弹力$F_\text{N}$和摩擦力$F_\text{f}$
而静止, 如图\ref{斜面2} 所示. 我们可以把这三个力的矢量首尾相接, 画出力的三角形, 如图 \ref{斜面3} 所示.

根据几何关系, 可以看出$G$, $F_\text{f}$和$F_\text{N}$的关系是
$$F_\text{f} = G\sin\theta,\bre\bre F_\text{N} = G\cos\theta. $$
这样, 如果我们知道木块的质量$m$, 及当地的重力加速度$g$, 就可以解出木块受到的摩擦力和弹力了.

\subsection{力的正交分解}

力的三角形从力的合成的角度来处理共点力的平衡问题, 在分析三力平衡时非常方便, 但分析多个力的平衡时就不适用了.
有没有通用的方法来处理共点力的平衡问题呢?

我们从力的分解的角度, 介绍另外一种处理方法. 仍以粗糙斜面上静止的木块为例. 
在图\ref{斜面5} 中, 沿平行和垂直于斜面的两个方
向建立直角坐标系, 把重力$G$沿两个坐标轴的方向分解为$G_x$和$G_y$.
这种分解方法称为\textbf{正交分解}\footnote{``正交''是垂直的意思. 通过建立直角坐标系来进行力的分解, 就叫作正交分解.}.

设斜面倾角为$\theta$, 根据几何关系, $G_y$与$G$的夹角也为$\theta$, 于是$G_x$, $G_y$与$G$的关系为
$$G_x = G\sin\theta, \bre\bre G_y = G\cos\theta.$$

根据共点力平衡的的知识可知
$$F_\text{f} = G_x = G\sin\theta, \bre\bre F_\text{N} = G_y = G\cos\theta. $$
当然, 这与我们之前得到的结果一致.

正交分解方法非常强大, 因为它属于一种``通法''. 通过把各个方向的力分解到$x$轴和$y$轴两个方向上, 
无论物体受多少个力而平衡, 我们都可以找到各个力之间的关系.

\chapter{运动与力的关系}
\section{牛顿第一定律}

\subsection{伽利略的理想实验}
长期以来, 在研究物体运动原因的过程中, 人们的经
验是: 要使一个物体运动, 必须推它或拉它. 
由这一现象, 亚里士多德得出结论: 必须有力作用在物体上, 物
体才能运动; 没有力的作用, 物体就要静止在某个地方.
这个``错案''维持了很久. 直至近四百年前, 伽利略才推翻了这个结论.

现实中, 当一个球沿斜面向下滚时, 它的速度增大; 而向上滚时, 它的速度减小.
伽利略认为, 当球沿水平面滚动时, 它的速度应不增不减. 
然而, 实际情况却是, 即使沿水平面滚动, 球也会越滚越慢,
最后停了下来. 伽利略认为, 这并非是它的``自然本性'', 
而是由于摩擦阻力的缘故. 

于是他推断: 若没有摩擦, 球将永远运动下去.

为了阐明自己的观点, 伽利略设计了一个
实验: 让一个小球沿斜面从静止状态开始运动, 小球将
冲上另一个斜面. 如果没有摩擦, 小球将到达原来的
高度. 如果第二个斜面倾角减小, 小球仍将到达原来的高
度, 但是运动的距离更长. 由此可以推断, 当斜面最终变
为水平面时, 小球要到达原有高度将永远运动下去(图\ref{伽利略的理想实验}). 这说
明, 力不是维持物体运动的原因. 

我们知道, 阻力不可能完全消除, 第二个斜面也不
可能做得无限长, 所以, 伽利略的实验是一个\textbf{``理想实
验''}. 虽然这个实验无法实现, 但是, 伽利略\textbf{在实验基
础上进一步推理}的方法, 帮助我们找到了解决运动和力
的关系问题的方法. 

伽利略同时代的法国科学家笛卡儿也研究了这个问题. 
他认为, 如果运动中的物体没有受到力的作用, 它将继续
以同一速度沿同一直线运动, 既不会停下来, 也不会偏离
原来的方向. 他还认为, 这应该成为一个原理, 是人类整
个自然观的基石.

\begin{figure}[bthp]
    \centering
    \includegraphics[width=0.4\textwidth]{pic/R-C.png}
    \caption{伽利略的理想实验}
    \label{伽利略的理想实验}
\end{figure}

\subsection{牛顿第一定律}

在伽利略和笛卡儿工作的基础上, 在隔了一代人以
后, 英国科学家牛顿提出了动力学的一条基本定律:
\textbf{一切物体总保持匀速直线运动状态或静止状态, 除非作用在
它上面的力\footnote{无论是亚里士多德, 还
是伽利略和笛卡儿, 都没有提出力的概念. 牛顿的
高明之处在于, 他将物体间复杂多样的相互作用抽象为``力''. 本书为了表述方便, 在陈述亚里士多
德等人的思想时, 借用力的概念.}迫使它改变这种状态.} 这就是\textbf{牛顿第一定律}.
物体这种保持原来匀速直线运动状态或静止状态的性质叫作\textbf{惯性}. 
牛顿第一定律也叫作\textbf{惯性定律}.

任何物体都和周围的物体有相互作用,所以,牛顿第一定律所描述的状态是一
种理想状态.它是利用逻辑思维进行分析的产物,不可能
用实验直接验证.

牛顿第一定律揭示了运动和力的关系: \textbf{力不是维持物
体运动状态\footnote{如果一个物体由静止
变为运动或由运动变为静止, 我们说它的运动状态发生了改变; 
如果一个物体的速度大小或方向改变了, 我们也说它的运动状态发生了改变.}的原因, 
而是改变物体运动状态的原因. }

\subsection{惯性与质量}

我们在改变物体运动状态时, 会感受到物体惯性大小
的不同. 例如, 以相同的方式抛掷质量不同的两个石块, 
让它们获得同样的速度, 质量大的石块需要的力大.

大量事例说明, 不同质量的物体, 惯性的大小是不一
样的. 也就是说, 不同物体维持其原有运动状态的能力
不同, 质量大的物体惯性大. 描述物体惯性的物理量是它
的\textbf{质量}.

质量只有大小, 没有方向, 是标量.在国际单位制中, 质量的单位是千克, 符号为 kg.
\footnote{这里我们把质量定义为描述物体惯性的物理量, 称为\textbf{惯性质量}. 其实, 物体还有另外
一个``质量'', 它描述物体受引力作用的本领, 称为\textbf{引力质量}. 
实际上, 任何物体都具有吸引其他物体的性质(在第\ref{万有引力章}章中会有介绍), 引力质量是物体这种性质的量度. 由于地球的引力质量很大, 所以地球能够吸引其表面的物体, 这就是重力.
惯性质量和引力质量是两个不同的物理量.但日常经验表明, 物体越重, 要改变它的运动状态就越难, 
也就是说, 物体的惯性质量和引力质量成严格的正比例关系. 假如我们选择适当的单位, 就可以使物体的引力质量的数值等于它的惯性质量的数值.
因此, 我们就不再区分惯性质量和引力质量, 而把它们统称为``质量''.}

\section{牛顿第二定律}
由前面的学习我们知道, 力
是改变物体运动状态的原因, 质量大的物体运动状态不容
易改变. 这说明, 加速度与力和质量之间必然存在一定的关系.

大量实验证明, 在我们常见的环境中, 当物体的质量不变时, 其加速度与它受到的作用力成正比, 即
$$\frac{F_1}{a_1} = \frac{F_2}{a_2} = \frac{F_3}{a_3} = \cdots.$$
不同物体所受合外力不变时, 其加速度与质量成反比, 即
$$m_1a_1 = m_2a_2 = m_3a_3 = \cdots.$$

综合以上两个结论可以得到: \textbf{物体加速度
的大小跟它受到的作用力成正比, 跟它的质量成反比, 加
速度的方向跟作用力的方向相同. }这就是\textbf{牛顿第二定律}.

如果质量为$m$物体在力$F$的作用下而运动, 产生的加速度为$a$, 则牛顿第二定律表述为
$$a \propto \frac{F}{m}.$$
也可以写成等式
\begin{equation}
    F = kma.
    \label{牛顿2}
\end{equation}
其中$k$是比例系数. $F$和$a$是矢量, 它们的方向相同.

力的单位就是根据牛顿第二定律来规定的.

取式 \eqref{牛顿2} 中的$k = 1$, 则1\ kg的某个物体要获得1\ $\mathrm{m/s^2}$的加速度, 
作用在它上面的力就是
$$F = ma = 1\unit{kg}\times 1\unit{m/s^2} = 1\unit{kg\cdot m\cdot s^{-2}}.$$

如果我们把这个力作为``一个单位''的力的话, 力 $F$
的单位就是千克米每二次方秒. 后人为了纪念牛顿, 把这个单位称作``牛顿'', 用符号 N 表示, 即
$$1\unit{N} = 1\unit{kg\cdot m\cdot s^{-2}}.$$

在质量的单位取千克(kg), 加速度的单位取米每二次
方秒($\mathrm{m/s^2}$), 力的单位取牛顿(N)时, 牛顿第二定律可以表述为
\begin{empheq}[box=\fbox]{equation}
	F = ma.
    \label{牛顿第二定律}
\end{empheq}

牛顿第二定律表明, 力与加速度总是同时出现, 同时消失; 
力不变则加速度也不变, 力随时间改变, 加速度也随时间改变;
力为 0 则加速度为 0, 这时物体将保持静止或匀速直线运动状态.

通过牛顿第二定律, 我们还能更进一步理解, 描述物
体惯性的物理量是质量的含义. 即在确定的作用力下, 决定
物体运动状态变化难易程度的因素是物体的质量.
例如, 赛车要求能在尽可能短的时间内达到最大速度, 即要有尽可能大的加速度, 
除了装备功率很大的发动机外, 在设计时还要考虑选用轻型材料, 以使赛车的质量尽可能小.

重力的表达式\footnote{在书写牛顿第二定律的表达式时, $mg$和$ma$具有相似的形式, 运算时可以进行合并.因此, 为了方便, 我们以后大多直接用$mg$来表示物体受到的重力, 而不再写出$G = mg$.}
$$G = mg$$其实也符合牛顿第二定律.
它表示:物体在只在重力作用下自由下落, 其加速度恒定不变, 为重力加速度$g$.
初中学习重力时, 我们把$g$看作是$G$与$m$的正比例系数, 所以说它的单位是N/kg;
现在我们从牛顿第二定律的角度来重新定义力, 把$g$看成加速度, $g$的单位就是$\mathrm{m/s^2}$, 所以说
$$1\unit{N/m} = 1\unit{m/s^2}.$$

\section{力学单位制}
物理学中, 有些物理量的单位是基本的, 而有些则是导出的.
例如, 位移的单位是米(m), 时间的单位是秒(s), 
由速度定义式导出的速度单位是米每秒(m/s), 
由加速度定义式导出的加速度单位是米每二次方秒($\mathrm{m/s^2}$).
再如, 在牛顿第二定律中, 质量的单位是千克(kg), 
由 $F = ma$ 导出的力单位是千克米每二次方秒($\mathrm{kg\cdot m\cdot s^{-2}}$), 即牛顿(N).

在物理学中, 只要选定几个物理量的单位, 就能够利
用物理量之间的关系推导出其他物理量的单位. 这些被选
定的物理量叫作\textbf{基本量}, 它们相应的单位叫作\textbf{基本单位}.
由基本量根据物理关系推导出来的其他物理量叫作\textbf{导出量},推导出来的相应单位
叫作\textbf{导出单位}.

为了促进科技交流与贸易往来等, 不同地域的人们逐渐将各自的单位规定进行统一.
1960 年, 第 11 届国际计量大会通过了\textbf{国际单位制(简称 SI)}\footnote{在国际单位制中, 长度, 质量, 时间, 电流, 热力学温度, 物质的量, 发光强度这几个物理量被规定为基本物理量, 它们的单位分别为
米(m), 千克(kg), 秒(s), 安培(A), 开尔文(K), 摩尔(mol), 坎德拉(cd).}, 
并很快被世界上大多数国家采用. 
在国际单位制中, 与力学有关的基本单位有米(m), 千克(kg)和秒(s), 
而速度, 加速度, 力等物理量的单位皆为导出单位. 

\section{超重和失重}

\subsection*{质量的测量}
实验室中称量药品的质量时, 我们通常使用托盘天平.
利用质量相等的物体受到的重力相等, 天平可以直接读出物体的质量.
无论在赤道附近还是两极附近, 甚至是在其他星球, 天平测出的物体质量都是准确的. 

人体的质量通常叫作体重. 在测量体重时, 我们一般使用电子体重计.
站在体重计上并保持静止, 此时人受到的重力等于体重计对人的支持力;
根据牛顿第三定律, 体重计对人的支持力又等于人对体重计的压力.

电子体重计测量人对体重计的压力, 再根据$G = mg$换算出人的体重(质量).\footnote{体重计的$g$通常设定为$9.81\unit{m/s^2}$.}
因此, 电子体重计实际测量的并不是人的质量, 而是人受到的重力.
这导致在赤道测量的值会略小, 在两极测量的值会略大.

\subsection*{超重和失重}
站在体重计上向下蹲, 你会发现, 在下蹲的
过程中, 体重计的示数先变小, 后变大, 再变
小. 当人静止后, 保持某一数值不变. 为什么会这样呢?

选取人为研究对象. 人体受到重力 $mg$ 和
体重计对人的支持力 $F_\text{N}$, 这两个力的共同作用使人在下蹲
的过程中, 先后经历加速, 减速和静止三个阶段.

设竖直向下方向为坐标轴正方向.

人加速向下运动的过程中(图\ref{下蹲1}), 根据牛顿第二
定律, 有
$$mg - F_\text{N} = ma.$$
由此可得$F_\text{N} = m(g-a)<mg.$
即体重计的示数小于人所受的重力.

物体对支持物的压力(或对悬挂物的拉力)小于物体
所受重力的现象,叫作\textbf{失重}现象.

特别地, 如果人在加速下降的过程中加速度 $a = g$, 即自由下落, 那么
$$mg - F_\text{N} = mg.$$
由此可得$F_\text{N} = 0$, 体重计的示数将为 0. 
这时物体对支持物(或悬挂物)完全没有作用力, 这种现象叫作\textbf{完全失重}.

同理, 人减速向下运动的过程中(图\ref{下蹲2}), 加速度
方向与运动方向相反, 有
$$mg - F_\text{N} = -ma.$$
由此可得$F_\text{N} = m(g+a)>mg.$
此时, 体重计的示数大于人所受的重力.

物体对支持物的压力(或对悬挂物的拉力)大于物体
所受重力的现象,叫作\textbf{超重}现象.

当人相对于体重计静止不动时, 有$$F_\text{N} = mg.$$
\begin{figure}[bthp]
\centering

\subfloat[向下加速]{
    \includegraphics[width=0.15\textwidth]{pic/4-10.pdf}
    \label{下蹲1}
}
\subfloat[向下减速]{
    \includegraphics[width=0.15\textwidth]{pic/4-11.pdf}
    \label{下蹲2}
}

	\caption{下蹲的加速阶段和减速阶段}
    \label{下蹲3}
    
\end{figure}

\begin{example}
    \label{电梯例题}
    设某人的质量为 60 kg, 站在电梯内的水平地板上, 当电梯以 $0.25\unit{m/s^2}$ 的加速度匀
加速上升时,求人对电梯的压力.取重力加速度$g = 10\unit{m/s^2}$.
\end{example}
\begin{solution}
    设竖直向上方向为坐标轴正方向.则加速度$a$沿正方向.

    根据牛顿第二定律, 有$$F_\text{N} - mg = ma.$$
    从而$$F_\text{N} = m(g+a) =  60\times(10+0.25)\unit{N} = 615\unit{N}. $$
    
    根据牛顿第三定律, 人对电梯地板的压力 $F_\text{N}^{\prime}$ 为
    $$F_\text{N}^{\prime} = -F_\text{N} = -615\unit{N}.$$
    
    人对电梯的压力大小为 615 N, 方向竖直向下.
\end{solution}

例\ref{电梯例题} 的结果说明, 当人与电梯共同向上加速或向下减
速运动时\footnote{牛顿第二定律的表达式无关于物体的运动状态. 虽然例\ref{电梯例题} 只讨论了加速上升的情况, 但无论加速上升还是减速下降, 其加速度都是向上的, 因此具有相同的结论.}
,\textbf{即加速度的方向向上时}, $F_\text{N}^{\prime}>G$, 人对电梯的压力大于人所受的重力, 
出现超重现象.

同理, 当电梯加速下降或者减速上升时, \textbf{即加速度的方向向下时}, $F_\text{N}^{\prime}<G$, 
人对电梯的压力小于人所受的重力, 出现失重现象.

\subsection*{视重}
物体与支持物相对静止, 并且不受地球和支持物以外其他作用时, 支持物对物体的作用力称为\textbf{视重}.
前面我们说的超重, 失重和完全失重, 分别对应视重大于重力, 视重小于重力和视重为0.

物体内部各部分之间的相互作用情况是由视重直接决定的.

静止在地面上的物体, 它的视重等于弹簧秤测得的重力$G$.
一杯水静止在水平面上有视重, 这时杯子与支持面之间有压力, 
压力的大小就是视重的大小. 水的内部也有压强, 在压强的作用下水稍有压缩.
如果这杯水随支持物做自由落体运动, 这时视重就为0. 这时杯子与支持物之间的压力消失, 
水内部的压强也消失, 水不再被压缩. 虽然重力仍然存在, 但由于视重的消失, 物体内部的相互挤压作用便不再存在.

人站在地上不动时, 由于视重的存在, 不仅脚底与地板之间有相互作用力, 人体内部各部分之间也存在相互作用力, 
如腿骨被挤压, 各种内部器官相互挤压; 血管中的血液除了受由于心脏的工作而产生的压强以外, 
还受由于视重的存在而产生的压强, 即液体压强$\rho gh$. 
人已经习惯于这种相互作用情况.
蹦极时人做自由落体运动, 随着视重的消失, 由于视重存在而产生的那一部分相互作用力将随之消失, 
人体将感到不适. 假设人相对静止于一个加速度向上的加速系统中, 随着视重增大, 
由视重存在而产生的那一部分相互作用力将随之增大, 人体也将感到不适.

乘坐过山车加速下落时会有很强的失重感. 为了乘客的健康, 国标规定过山车的下落加速度不能超过$6g$.
一般过山车的下落加速度在$4g$左右, 根据牛顿第二定律, 此时$$F_\text{N} = m(g-a) = -3mg.$$
即座位靠背和安全杆对人向下的作用力为$3mg$. 如果没有座位靠背和安全杆的作用, 人很容易从座位上飞出去.

\chapter{运动的合成与分解}

\section{曲线运动}

到目前为止, 我们只研究了物体沿着一条直
线的运动. 实际上, 自然界中的曲线运动是很常
见的: 抛出的篮球, 绕太阳公转的地球, 它们的轨迹都是曲线.

运动轨迹是曲线的运动叫作\textbf{曲线运动}. 从现在开始, 我们把目光转向抛体运动, 圆
周运动, 以及更一般的曲线运动.

做曲线运动的物体, 其速度方向在不断变化, 因此曲线运动是一种变速运动.
由几何知识可知, \textbf{质点在某一点的速度方向, 沿曲线在这一点的切线方向.}

物体在什么条件下会做曲线运动呢?

一个钢球在水平面上做直线运动. 从不同方向给它施
加力, 可以发现: 当钢球受到的合力的方向与速度方
向不在同一条直线上时, 钢球做曲线运动. 生活中也有大
量类似的例子. 例如, 向斜上方抛出的石子, 它所受重力
的方向与速度的方向不在同一条直线上, 石子做曲线运动.

理论和事实都表明, \textbf{当物体所受合力的方向与速度的方向不在同一直线上时, 物体做曲线运动.}

根据牛顿第二定律, 物体加速度的方向与它受力的方
向总是一致的. 当物体受力的方向与速度的方向不在同一直
线上时, 加速度的方向也就与速度的方向不在同一直线上
了, 于是物体的速度方向要发生变化, 物体就做曲线运动.

如果物体受到的合力是恒力, 根据牛顿第二定律, 物体的加速度也是恒定的, 
所以物体的速度将随时间均匀变化. 加速度恒定的曲线运动叫作\textbf{匀变速曲线运动}.

曲线运动具有下面几个特点:
\begin{enumerate}
	\item 物体做曲线运动时,速度的方向时刻都在改变,所以物体的加速度一定不为0;但物体的速率可能改变, 也可能不变.
	\item 做曲线运动的物体, 当合力方向与速度方向的夹角为锐角时,物体的速率将增大;
	      当合力方向与速度方向的夹角为钝角时,物体的速率将减小.
    \item 当物体的初速度与所受的合力不在一条直线上时, 物体起初会向初速度的方向运动, 但轨迹会向合力的方向逐渐偏移,
形成曲线. 需要注意的是, \textbf{物体的运动方向永远不会和合力的方向平行}, 因此, 物体做曲线运动的轨迹一定夹在合力方向与速度方向之间. 
	\item 做曲线运动的物体, 位移小于路程.
\end{enumerate}

\section{运动的合成与分解}

\subsection*{小船过河的运动}
若小船在河中始终保持船头朝正前方驶向对岸, 
你认为小船会在对岸的正前方到达, 还是会偏向上
游或下游?

要想定量的研究小船的运动, 就要建立坐标系, 具体分析.

\begin{wrapfigure}[8]{o}{0.33\textwidth}
	\includegraphics[width=0.3\textwidth]{pic/5-1.pdf}
    \caption{小船过河的运动}
	\label{小船过河-0}
\end{wrapfigure}

\subparagraph{坐标系的选取} 首先, 坐标系的选取很重要. 例如, 对于直线运动, 最好沿着这条直线
建立一维坐标系. 研究物体在平面内的运动
时, 可以选择平面直角坐标系. 如果是一艘潜水艇在水中运动, 在运动过程中有下潜和上浮, 
那么我们可以选择空间直角坐标系.

我们不考虑小船的下潜和上浮, 所以只需建立平面直角坐标系. 以小船开始运动的位置为原点$O$, 
垂直于河岸向前的方向为$y$轴方向, 水流的方向为$x$轴方向, 建立直角坐标系(图\ref{小船过河-0}).

\subparagraph{小船运动的轨迹} 要确定小船运动的轨迹, 首先要确定任意时刻小船的位置. 我们设法写出小船的坐标随时间
变化的关系式.小船 $x$ 坐标的值等于它与 $y$ 轴的距离, $y$ 坐
标的值等于它与 $x$ 轴的距离. 若以 $v_x$ 表示水流向右迁移的速度, 
以 $v_y$ 表示小船向前航行的速度,则有
$$x = v_xt, \bre\bre y = v_yt.$$

小船沿着什么样的轨迹运动?在数学中, 包含$x$与$y$的关系式可以描述一条
曲线(包括直线).我们从上面两个式子中消去$t$, 就可以得到
$$y = \frac{v_y}{v_x}x.$$

由于$v_x$, $v_y$都是常量, 所以上面关系式的图像是一条过原点的直线.
也就是说, 小船的运动轨迹是直线.上式就称为小船的\textbf{轨迹方程}. 

\subparagraph{小船运动的速度} 速度$v$与$v_x$, $v_y$的关系已在图\ref{小船过河-0} 中形象地标出.
因此可以根据勾股定理写出它们的关系$$v = \sqrt{v_x^2 + v_y^2}.$$

根据三角函数的知识, 从图\ref{小船过河-0} 中还可以确定速度 $v$
的方向, 即用速度矢量 $v$ 与 $x$ 轴正方向的夹角 $\theta$ 来表示. 它
的正切为$$\tan\theta = \frac{v_y}{v_x}.$$

\subsection*{运动的合成与分解}

在前面的例子中, 我们看到小船向右前方的运动可以看
成由垂直于河岸向前的运动, 和随着水流向右的运动共同构成. 
小船垂直于河岸向前的运动, 和它随着水流向右的运动, 都叫作
分运动; 而小船相对于地面向右前方的运动叫作合运动.

一般地说, \textbf{一个运动可以看成几个运动的
合成或叠加, 这几个运动是同时进行的, 且它们互不干扰.}
这个规律叫作\textbf{运动的叠加原理}.
这几个运动都叫作该物体实际运动的\textbf{分运动},
该物体的实际运动叫作这几个运动的\textbf{合运动}.

由分运动求合运动的过程, 叫作运动的合成;
由合运动求分运动的过程, 叫作运动的分解.
运动的合成与分解遵从矢量运算法则.

\subsubsection{合运动类型的判断}

\setlength{\abovedisplayskip}{3pt}
\setlength{\belowdisplayskip}{3pt}

通过前面的例子, 我们已经知道, 两个相互垂直的匀速直线运动的合成, 仍是匀速直线运动.
下面我们来了解一些其他运动的合成.

\begin{enumerate}
	\item 两个匀速直线运动的合运动是匀速直线运动;
	\item 一个匀速直线运动和一个匀加速直线运动做合成, 当这两个分运动共线时,
	      合运动是匀变速直线运动;当这两个分运动不共线时,合运动是匀变速曲线运动.
	\item 两个匀变速直线运动的合运动,可能是匀变速直线运动,也可能是匀变速曲线运动.\\
	      事实上,当这两个分运动的速度$v_1$, $v_2$,加速度$a_1$, $a_2$对应成比例时,即满足
	      $$\frac{v_1}{v_2}=\frac{a_1}{a_2}$$
	      时,物体做匀变速直线运动; 否则,物体做匀变速曲线运动.
\end{enumerate}

这些结论都是比较符合直觉的, 我们略去证明.

\section{应用:小船过河问题}

在上一节中, 我们简单分析了小船过河的运动, 现在我们来进一步研究小船过河的位移和所用的时间.

\subsection*{最短时间问题}

\begin{figure}[htbp]
\centering

\subfloat[最短时间问题]{
    \includegraphics[width=0.25\textwidth]{pic/pic5.pdf}
    \label{小船过河1}
}
\subfloat[最短位移问题(1)]{
    \includegraphics[width=0.25\textwidth]{pic/pic6.pdf}
    \label{小船过河2}
}
\subfloat[最短位移问题(2)]{
    \includegraphics[width=0.25\textwidth]{pic/pic7.pdf}
    \label{小船过河3}
}
	\caption{小船过河问题}
    \label{小船过河4}
    
\end{figure}

小船过河的时间,是由垂直于河岸的分运动决定的.因此,为使渡河时间最短,应让小船垂直于河岸的分速度最大.
于是小船应该垂直于河岸行驶.

如图\ref{小船过河1}, 河宽为$d$, 小船在静水中的速度为$v_1$, 水流的速度恒为$v_2$, 它们的合速度为$v$.

小船在静水中的速度$v_1$即为小船垂直于河岸的分速度, 此时的过河时间$$t = \frac{d}{v_1}.$$

小船垂直于河岸的分运动与沿水流方向的分运动均为匀速直线运动, 我们知道, 小船的实际运动也是匀速直线运动,
并且轨迹沿着物体的速度$v$的方向.

由于小船做匀速直线运动, 所以小船的位移是
$$l=vt=\sqrt{v_1^2 + v_2^2}\ t = \frac{d\sqrt{v_1^2 + v_2^2}}{v_1} = d\sqrt{1 + \left(\frac{v_2}{v_1}\right)^2}.$$

\subsection*{最短位移问题}

因为在渡河时, 垂直于河岸方向的分位移恒为河宽$d$, 所以要使渡河位移$l$最短, 只需控制平行于河岸方向的分位移最小.

若平行于河岸方向的分位移为0, 即小船实际运动垂直于河岸, 则船速$v_1$在平行于河岸方向上的分矢量, 
需要与水速$v_2$的大小相等, 方向相反.
因此, 应将船头偏向上游, 并与河岸成一定的角度$\theta$, 如图\ref{小船过河2} 所示.
\setlength{\abovedisplayskip}{0pt}
\setlength{\belowdisplayskip}{0pt}

根据几何关系,有 $$v_1 \cos{\theta} - v_2 = 0.$$
因为$0\leqslant\cos \theta\leqslant 1$, 所以只有在船速$v_1$大于水速$v_2$时,
小船的实际位移才有可能垂直于河岸. 因此,需要分两种情况讨论.

(1)当$v_1 > v_2$时, 应使船速$v_1$与水速$v_2$的合速度$v$与河岸垂直, 如图\ref{小船过河2} 所示 .此时船速 $v_1$
方向与河岸方向的夹角$\theta$满足$$\cos{\theta} = \frac{v_1}{v_2}.$$最段位移$l$的大小就是河宽$d$.
合速度的大小$$v = v_1 \sin{\theta}.$$
对应的渡河时间$$t = \frac{d}{v} = \frac{d}{v_1 \sin{\theta}}.$$

(2) 当$v_1 < v_2$时, 由于水流的冲击,小船是无法沿垂直于河岸的方向运动的. 此时合速度$v$指向下游,
与河岸的夹角为$\alpha$ (图\ref{小船过河3}).由几何关系可知,渡河位移$$l = \frac{d}{\sin {\alpha}}.$$
显然, 当$\alpha\leqslant \uppi$时, $l$随角$\alpha$的增大而减小.

如图\ref{小船过河3} 所示, 以水速矢量$v_2$的终点为圆心, 以船速矢量$v_1$的大小为半径作圆. 从水速矢量$v_2$的起点向圆作切线,
切点为合速度矢量$v$的终点. 显然, 沿此方向的航程是最短的.

由几何关系可知, 当位移最短时, 船速 $v_1$ 方向与河岸方向的夹角$\theta$满足 $$\cos{\theta} = \frac{v_1}{v_2}.$$
此时的位移 $$l = \frac{d}{\sin \left (\displaystyle\frac{\uppi}{2}-\theta \right )} = \frac{v_2}{v_1} d.$$

\chapter{抛体运动}
\section{平抛运动}
\setlength{\abovedisplayskip}{0pt}
\setlength{\belowdisplayskip}{0pt}

\subsection{平抛运动的速度}
\begin{wrapfigure}[17]{o}{6cm}
	\flushright
	\includegraphics[width=0.3\textwidth]{pic/4-1.pdf}
    \caption{平抛运动的速度}
	\label{平抛运动1}
\end{wrapfigure}

以一定的速度将物体抛出, 在空气阻力可以忽略的情
况下, 物体只受重力的作用, 这时的运动叫作\textbf{抛体运动}.
如果初速度是沿水平方向的, 这样的抛体运动就叫作\textbf{平抛运动}.

以初速度 $v_0$ 沿水平方向抛出一质量为$m$的物体, 物体做平抛运动. 以物体抛出时的位置为原点$O$, 
建立直角坐标系. $x$ 轴沿初速度的方向, $y$轴竖直向下(图\ref{平抛运动1}). 

由于物体只受到竖直向下的重力, 即在$x$方向不受力. 根据牛顿第二定律, 
它在 $x$ 方向的加速度为 0, 那么在整个运动过程中,
物体在水平方向上的速度 $ v_{x}$ 将保持初速度 $v_0$ 不变, 与时间$t$无关, 即始终有
$$v_x=v_0.$$

在$y$方向上, 物体受到的重力等于$mg$. 用$a$表示物体在$y$方向的加速度, 
应用牛顿第二定律可得$$mg = ma, $$所以$a = g$, 即物体在$y$方向的加速度为重力加速度.

物体的初速度沿$x$方向, 也就是说物体沿$y$方向的初速度为0, 所以, 物体在
$y$方向的分速度$v_y$与时间$t$的关系是$$v_y = gt.$$

根据矢量运算法则, 由图\ref{平抛运动1} 可知, 物体的速度$v$的大小可由勾股定理给出
\begin{equation}
	v=\sqrt{{v_x}^2+{v_y}^2}=\sqrt{v_0^2+\left(gt\right)^2}.
	\label{合速度}
\end{equation}
而$v$的方向可以由图\ref{平抛运动1} 中的夹角$\theta$来表示. 根据几何关系
\begin{equation}
	\tan\theta = \frac{v_y}{v_x} = \frac{gt}{v_0}.
	\label{速度偏转角}
\end{equation}
由上式可知, 随着物体的下落, 角 $\theta$ 从0开始变得越来越大. 
角 $\theta$ 是物体的速度$v$随着运动的进行, 逐渐偏离$v_0$的方向而产生的夹角, 称为\textbf{速度的偏转角}, 简称\textbf{速度角}.

\subsection{平抛运动的位移和轨迹}
\begin{wrapfigure}[15]{o}{0.35\textwidth}
    \begin{centering}
	\includegraphics[scale=0.3]{pic/4-2.pdf}
    \caption{平抛运动的位移}
	\label{平抛运动2}
    \end{centering}
\end{wrapfigure}
物体被抛出后, 其相对于原点$O$的位移的大小和方向都在变化.
因此, 我们要分别研究它在水平和竖直两个方向的分位移$x$和$y$.

我们已经知道, 在$x$方向上, 物体的速度恒为$v_0$不变, 也就是说, 
\textbf{物体在$x$方向做匀速直线运动}. 根据匀速直线运动位移与时间的关系可知, 
它的水平分位移$x$与时间$t$的关系为
\begin{equation}
	\label{水平位移}
	x=v_0t,
\end{equation}

\textbf{物体在$y$方向上的分运动是初速度为0, 加速度为$g$的匀加速直线运动, 即自由落体运动.}
由自由落体运动的知识可知, 物体的竖直分位移$y$与时间$t$的关系是
\begin{equation}
	\label{竖直位移}
	y=\frac12gt^2,
\end{equation}

物体的位置是由它的坐标$x$, $y$描述的, 所以, \eqref{水平位移} \eqref{竖直位移} 两式
确定了物体在任意时刻$t$的位置和位移.

\setlength{\abovedisplayskip}{10pt}
\setlength{\belowdisplayskip}{10pt}
我们当然也可以根据 \eqref{水平位移} \eqref{竖直位移} 两式确定物体的位移大小
$$l=\sqrt{x^2+y^2}=\sqrt{\left(v_0t\right)^2+\left(\frac{1}{2}gt^2\right)^2}.$$
而位移的方向可以由图\ref{平抛运动2} 中的角$\varphi$确定.根据几何关系
\begin{equation}
	\tan\varphi = \frac{y}{x} = \frac{\displaystyle\frac12gt^2}{v_0t} = \frac{gt}{2v_0}.
	\label{位移偏转角}
\end{equation}
由上式可知, 随着物体的下落, 角 $\varphi$ 从0开始变得越来越大. 
角 $\varphi$ 是物体的位移$l$随着运动的进行, 逐渐偏离$x$方向而产生的夹角, 称为\textbf{位移的偏转角}, 简称\textbf{位移角}.

\setlength{\abovedisplayskip}{0pt}
\setlength{\belowdisplayskip}{0pt}

对比 \eqref{速度偏转角} 式与 \eqref{位移偏转角} 式, 可以看出
\begin{empheq}[box=\fbox]{equation*}
	\tan\theta = 2\tan\varphi.
\end{empheq}
也就是说, \textbf{做平抛运动的物体, 其速度偏转角的正切值, 总是等于位移偏转角的正切值的两倍.}
这是平抛运动的一个重要结论, 称为\textbf{平抛运动的偏转角规律}.

\setlength{\abovedisplayskip}{5pt}
\setlength{\belowdisplayskip}{5pt}

物体的位置和位移可以由 $x, y$ 确定, 物体的轨迹方程
也可以由 $x, y$ 确定. 联立 \eqref{水平位移} \eqref{竖直位移}, 消去时间 $t$, 得到
\begin{empheq}[box=\fbox]{equation}
	y=\frac{g}{2v_0^2}x^2.
    \label{平抛运动轨迹方程}
\end{empheq}
\eqref{平抛运动轨迹方程} 式叫作\textbf{平抛运动的轨迹方程}. 在这个式子中, 
重力加速度 $g$, 物体的初速度 $v_0$都是常量, 与$x, y$无关. 
也就是说, \eqref{平抛运动轨迹方程} 式具有二次函数的形式, 它的图像是一条\textbf{抛物线}.

\begin{example}
    \label{平抛运动例题1}
    求证: 做平抛运动的物体, 在任意时刻速度$v$的反向延长线, 总是通过此刻水平位移的中点.
\end{example}

\begin{wrapfigure}[12]{o}{0.35\textwidth}
    \begin{centering}
	\includegraphics[scale=0.3]{pic/6-1-1.pdf}
    \caption{例\ref{平抛运动例题1} 题图}
	\label{平抛运动例1题图}
    \end{centering}
\end{wrapfigure}

\textbf{已知}\bre 如图 \ref{平抛运动例1题图} 所示, 做平抛运动的物体运动到$P$点时, 其速度为$v$, 位移为$OP$, 水平位移为$OA$.
$PM$是$v$的反向延长线.

\textbf{求证}\bre $M$是$OA$的中点.

\textbf{证明}\bre 设速度的偏转角为$\theta$,位移的偏转角为$\varphi$. 
由图可知, $\angle AMP=\theta$, $\angle AOP=\varphi$.
所以$$\tan\theta = \frac{AP}{AM}, \bre\bre\tan\varphi = \frac{AP}{AO}.$$

利用平抛运动的偏转角规律,有$$\tan\theta = 2\tan\varphi.$$

所以$AO = 2AM$, 即$M$是$OA$的中点.

\bigskip
例 \ref{平抛运动例题1} 给出了平抛运动的另一条重要结论, 它是平抛运动的偏转角规律的推论. 通过
几何关系解决问题时, 使用这个推论往往会简化许多计算量.

\subsection{物理量的求解思路}

在求解平抛运动的相关物理量时, 确定哪些量已知很重要. 我们要选择一个方程作为主方程, 这个方程中
有一个量是我们待求的, 其余的量都是已知的, 或者是通过其他辅助方程可求的.下面我们就来举几个例子, 
顺便对平抛运动的规律作进一步分析.

以速度$v_0$从高度$h$处水平抛出一物体, 该物体在竖直方向上做自由落体运动,
根据匀变速直线运动位移$h$与时间$t$的关系$h=\displaystyle\frac{1}{2}gt^2$可知,物体做平抛运动的运动时间
\begin{equation}
	\setlength{\abovedisplayskip}{3pt}
	\setlength{\belowdisplayskip}{3pt}
	\label{运动时间}
	t=\sqrt{\frac{2h}{g}}.
\end{equation}
上式告诉我们,\textbf{物体做平抛运动的飞行时间取决于下落的高度}\footnote{在这个推导中, 我们选择了$h=\displaystyle\frac{1}{2}gt^2$作为主方程.
主方程中有$h$和$t$两个物理量, 其中$h$是已知的, 我们从中解出了$t$.}.上式也可以表示为$$t\propto \sqrt{h}.$$

\setlength{\abovedisplayskip}{10pt}
\setlength{\belowdisplayskip}{10pt}
把 \eqref{运动时间} 代入 \eqref{水平位移}, 我们就得到物体做平抛运动的水平位移
$$x=v_0\sqrt{\frac{2h}{g}}.$$
上式告诉我们,\textbf{物体做平抛运动的水平位移与初速度和下落高度均有关}\footnote{在这个推导中, 可以看出我们选$x = v_0t$作为了主方程. 主方程中有三个物理量, 其中$x$是待求的, 
$v_0$是已知的, $t$是通过辅助方程$h=\displaystyle\frac{1}{2}gt^2$求得的.}.

\setlength{\abovedisplayskip}{0pt}
\setlength{\belowdisplayskip}{0pt}
由于物体在竖直方向上做自由落体运动,根据匀变速直线运动速度 $v_y$ 与位移 $h$ 的关系 $v_y^2 = 2gh$,
结合(\ref{合速度})式可以得到,物体做平抛运动的落地速度
$$v=\sqrt{{v_0}^2+2gh}.$$
上式告诉我们,\textbf{物体做平抛运动的落地速度与初速度和下落高度均有关}.\footnote{在这个推导中, (\ref{合速度})式是主方程, 
$v_y^2 = 2gh$是辅助方程.}

下面的这道例题展示了如何利用上面的结论解题.

\begin{example}
    把一个小球从离地$h=5\ \rm{m}$高度处,以 $v_0=10\ \rm{m/s}$ 的初速度水平抛出,不计空气阻力,
取 $g=10\ \rm{m/s^2}$. 求: 

(1) 小球在空中飞行的时间;

(2) 小球落地点距离抛出点的水平距离;

(3)小球落地时的速度大小.
\end{example}
\begin{solution}
    (1)由题意可知,小球做平抛运动,其竖直方向的分运动是自由落体.
    根据匀变速直线运动位移$h$与时间 $t$ 的关系有 $$h=\frac12gt^2, $$
    所以小球在空中飞行的时间
    $$t=\sqrt{\frac{2h}{g}}=\sqrt{\frac{2\times5}{10}}\ \rm{s}=1\ \rm{s}.$$

    (2)小球的水平位移 $$x=v_0t=10\times 1\unit{m}=10\unit{m}.$$

    (3)小球落地时的竖直分速度 $$v_y=gt=10\ \rm{m/s}.$$
    所以落地时的速度 $$v=\sqrt{{v_0}^2+{v_y}^2}=10\sqrt2\ \rm{m/s}.$$
\end{solution}

本节的这些结论并不能在考试的解答中直接使用, 但可以让我们明确主方程的选定.

\begin{example}
    \label{例题飞镖}
    如图\ref{fig4}, 竖直墙壁上落有两只飞镖,它们是从同一位置水平射出的.飞镖A在飞镖B的上方,两者间的距离为$d$.
    飞镖A与竖直墙壁成$53^{\circ}$角,飞镖B与竖直墙壁成$37^{\circ}$角,假设飞镖的运动是平抛运动,
    求射出点与墙壁间的水平距离.(已知$\sin 37^{\circ}=0.6,\cos 37^{\circ}=0.8$)
\end{example}

\begin{wrapfigure}[10]{o}{4cm}
	\begin{centering}
        \includegraphics[width=0.1\textwidth]{pic/pic4.png}
        \caption{例 \ref{例题飞镖} 题图}
	    \label{fig4}
    \end{centering}
\end{wrapfigure}

\textbf{解}\bre 平抛运动的竖直分运动是自由落体,对于一个飞镖来说,
它的竖直位移$y$与时间$t$的关系是
\begin{equation*}
	y=\frac12gt^2.
	\tag{i}
\end{equation*}

题目中所给的距离$d$,就是两个飞镖的竖直位移差,我们需要分别求出这两个飞镖的竖直位移.
由于 (i) 式中的$t^2$是未知的,我们接下来要把它代换为已知量.

飞镖与竖直墙壁的夹角$\theta$,实际上就是末速度与竖直方向的夹角,即末速度与初速度
夹角的余角.对于一个飞镖来说,设它的初速度为$v_0$,则
\begin{equation*}
	\tan \theta=\frac{v_0}{v_y}=\frac{v_0}{gt}.
	\tag{ii}
\end{equation*}

平抛运动的水平分运动是匀速直线运动,对于一个飞镖来说,它的水平位移$x$和时间$t$的关系是
\begin{equation*}
	x=v_0t.
	\tag{iii}
\end{equation*}

(ii) 式中的$t$在分母上,(iii) 式中的$t$在分子上,将它们联立,刚好可以求出
\begin{equation*}
	t^2=\frac{x}{g\tan \theta}.
	\tag{iv}
\end{equation*}


把 (iv) 式代入 (i) 式中,得
\begin{equation*}
	y=\frac12 g\cdot \frac{x}{g\tan \theta}=\frac{x}{2\tan \theta}.
	\tag{v}
\end{equation*}

根据 (v) 式,依题意有
\begin{equation*}
	\frac{x}{2\tan 37^{\circ}}-\frac{x}{2\tan 53^{\circ}}=d,
	\tag{vi}
\end{equation*}
解得 $x=\displaystyle\frac{24}{7}d.$

因此,射出点与墙壁间的水平距离为$\displaystyle\frac{24}{7}d$.

\section{起落于斜面的平抛运动}

在一个斜面上以不同的速度(使物体均落于斜面)分别水平抛出两个相同物体,哪个物体先落到斜面上呢? 

\begin{wrapfigure}[10]{o}{0.4\textwidth}
	\begin{centering}
        \includegraphics[width=0.38\textwidth]{pic/pic2.png}
    \caption{起落于斜面的平抛运动}
	\label{起落斜面平抛}
    \end{centering}
\end{wrapfigure}

我们先研究抛出一个物体的情况.

如图\ref{起落斜面平抛},以速度$v_0$从斜面上$A$点处水平抛出一物体,该物体做平抛运动.在保证其落于斜面
且不发生反弹的前提下, 水平分位移$x$与竖直分位移$y$的关系是
$$\frac{y}{x}=\tan{\theta}.$$
所以$$\tan{\theta}=\frac{\displaystyle\frac{1}{2}gt^2}{v_0t}=\frac{gt}{2v_0}.$$

从中解出物体在斜面上的运动时间
\begin{equation}
	t=\frac{2v_0\tan{\theta}}{g}.
	\label{斜面上的运动时间}
\end{equation}
在斜面倾角固定的情况下,上式中的$\tan{\theta}$以及重力加速度$g$都是定值.
这就是说,\textbf{物体在做起落于斜面上的平抛运动时,其运动时间与物体的初速度成正比},即
\begin{empheq}[box=\fbox]{equation*}
	t\propto{v_0}.
\end{empheq}

有了运动时间,根据匀速直线运动位移$x$与时间$t$的关系,代入(\ref{斜面上的运动时间})式,
就可以得到物体在斜面上的水平位移
$$x=v_0t=v_0\cdot\frac{2v_0\tan{\theta}}{g}=\frac{2{v_0}^2\tan{\theta}}{g}.$$
再根据几何关系, 容易得到合位移
$$s=\frac{2{v_0}^2\sin{\theta}}{g\cos^2{\theta}}.$$

这就是说,\textbf{物体在做起落于斜面上的平抛运动时,其位移与物体的初速度的二次方成正比},即
\begin{empheq}[box=\fbox]{equation*}
	s\propto{{v_0}^2}.
\end{empheq}

最后来研究物体的末速度.末速度的大小是容易求出的,并且有
$v\propto{v_0}$.我们重点研究末速度的方向.设物体落在斜面上时的速度$v$(水平分量$v_x$,竖直分量$v_y$)
与初速度$v_0$的夹角为$\theta_v$,则有$\tan{\theta_{v}=\displaystyle\frac{v_x}{v_y}}$.根据平抛运动的
运动规律, $v_x=v_0$, $v_y=gt$,代入(\ref{斜面上的运动时间})式,得$v_y=2v_0\tan{\theta}$,
所以
$$\tan{\theta_v}=\frac{2v_0\tan{\theta}}{v_0}=2\tan{\theta}.$$

这就表明,\textbf{物体在做起落于斜面上的平抛运动时,末速度的方向只与斜面倾角有关,与初速度无关}.
这符合平抛运动的偏转角规律.

\section{斜抛运动}
将物体用一定的初速度向斜上方抛出去,这样的抛体运动叫作斜抛运动.
投出的标枪和手榴弹,向斜上方射出的子弹和炮弹,它们的运动都是斜抛运动.
在斜抛运动中,物体由于受到跟速度成角度的重力,所以做曲线运动.
做斜抛运动的物体先是沿着曲线上升,升到一个最高点,然后再沿着曲线下降.
下面我们来具体研究斜抛物体的运动情况.

以初速度$v_0$, 沿着与水平方向成$\theta$角的斜上方, 抛出一个质量为$m$的物体.
以物体抛出时的位置为原点 $O$, 建立直角坐标系. $x$ 轴沿水平方向,与$v_0$共面且成锐角, $y$ 轴竖直向上\footnote{这里与研究平抛运动时不同.因为平抛运动沿水平方向, 以竖直向下为$y$轴的正方向更便于表示位移;而对于斜抛运动, 物体初速度在竖直方向的分速度向上, 所以我们更倾向以竖直向上为$y$轴的正方向.}.

我们把 $v_0$ 分解为水平方向的分速度$v_{0x} = v_0\cos\theta$
和竖直方向的分速度$v_{0y} = v_0\sin\theta.$ 这样, 斜抛运动可以看作是下面两个分运动的合运动:
一个是水平方向的匀速直线运动, 速度为$v_{0x}$, 即$$v_x = v_{0x} = v_0\cos\theta;$$ 另一个是竖直上抛运动, 其速度$v_y$与时间$t$的关系是
$$v_y = v_{0y}-gt = v_0\sin\theta - gt.$$

利用匀速直线运动和竖直上抛运动的知识, 物体的水平分位移$x$和竖直分位移$y$与时间$t$的关系可以分别表示出来
$$x = v_{0x}t = v_0t\cos\theta,$$
$$y = v_{0y}t-\frac12 gt^2 = v_{0}t\sin\theta-\frac12 gt^2.$$

根据这两个公式求出任何时刻物体的位置, 用平滑曲线把这些位置连接起来, 就得到斜抛运动的轨迹; 或者
把上面两个式子联立, 消去时间$t$, 就得到斜抛运动的轨迹方程. 斜抛运动的轨迹也是一条抛物线.

竖直上抛运动和平抛运动都可以看作斜抛运动的特殊情况. 当抛射角$\theta = 90^{\circ}$时, $v_{0x}=0$,
$v_{0y}=v_0$, 物体的运动就是竖直上抛运动; 当抛射角$\theta = 0^{\circ}$时,$v_{0x}=v_0$,
$v_{0y}=0$, 从某一高度抛出的物体的运动就是平抛运动.


\chapter{圆周运动}
\section{圆周运动的运动参量}

相互咬合的大, 小两个齿轮, 它们边缘上的点, 哪个运动的速度更快?这个问题的答案可能会有争议, 
因为到现在为止, 对圆周运动的快慢, 还没有大家都认可的描述方法. 接下来, 我们引入圆周运动中的速度概念.
\subsection{线速度\ \ 角速度}

在圆周运动中,把物体运动沿着一段圆弧运动的瞬时速度叫作\textbf{线速度},用符号$v$表示.
$$v=\lim_{\Delta t\rightarrow 0}\frac{\Delta{s}}{\Delta{t}}, $$
其中$\Delta s$是弧长.线速度的方向为物体做圆周运动时该点的切线方向.
应该注意的是,线速度描述的是物体在做圆周运动时某一时刻的瞬时速度,而不是平均速度.

如果物体沿着圆周运动, 并且线速度的大小处处相等, 
这种运动叫作\textbf{匀速圆周运动}.
应该指出的是, 匀速圆周运动的线速度方向是在时刻变化的, 
因此它是一种变速运动, 这里的``匀速''是指速率不变.

在圆周运动中,把物体在一段时间内转过的角度$\Delta{\theta}$与所用时间$\Delta{t}$的
比值叫作\textbf{角速度}\footnote{角速度是有方向的, 是个矢量. 但在中学阶段我们不研究它的方向, 只研究它的大小. 因此可以认为它是个标量.},用符号 $\omega$ 表示. 一般说角速度, 指的也是瞬时角速度, 即
$$\omega=\lim_{\Delta t\rightarrow 0}\frac{\Delta{\theta}}{\Delta{t}}.$$
在国际单位制中,角速度的单位是$\rm{rad/s}$或$\mathrm{s^{-1}}$.

由于匀速圆周运动是线速度大小不变的运动, 物体在
相等时间内通过的弧长相等, 所以物体在相等时间内转过
的角度也相等. 因此可以说, 匀速圆周运动是角速度不变
的圆周运动.

在$\Delta \theta$以弧度为单位时,因为$\Delta \theta=\displaystyle\frac{\Delta s}{r}$
(其中$s$为弧长, $r$为圆周运动的半径),所以 $$v=\omega r.$$
这表明, \textbf{在圆周运动中, 线速度的大小等于角速度的大小与半径的乘积.}

根据线速度与角速度的定义, 容易知道:
\begin{enumerate}
	\item \textbf{做圆周运动的物体上的各点角速度相同, 线速度与各点相对于绕转中心的距离成正比, 即}$\displaystyle\frac{v_1}{v_2} = \displaystyle\frac{r_1}{r_2}.$
	\item \textbf{同轴转动的不同物体, 它们上面的各点角速度相同, 它们边缘上的点的线速度和物体的半径成正比, 即}$\displaystyle\frac{v_1}{v_2} = \displaystyle\frac{r_1}{r_2}.$
	\item \textbf{用皮带或齿轮传动的两个物体, 它们边缘上的点线速度相同, 角速度和物体的半径成反比, 即}$\displaystyle\frac{\omega_1}{\omega_2} = \displaystyle\frac{r_2}{r_1}.$
\end{enumerate}

回顾节前的问题. 相互咬合的大小齿轮, 如果看线速度, 它们边缘上的点一样快; 如果看角速度, 两个齿轮上的点不一样快, 
小齿轮上的点角速度较大.

\subsection{周期}
做圆周运动的物体, 转过一周所用的时间叫作\textbf{周期},用符号$T$表示.
它的单位与时间的单位相同.

根据周期的定义,可知角速度$\omega$和周期$T$的关系为 $$\omega=\frac{2\uppi}{T};$$

线速度$v$和周期$T$的关系为 $$v=\frac{2\uppi r}{T}, $$
其中$r$是圆周运动的半径.

\subsection{频率\ \ 转速}
\textbf{频率}是指物体在单位时间内转过的圈数,用符号$f$表示,单位是赫兹(Hz).
频率是周期的倒数,即
$$f=\frac{1}{T}.$$

技术上常用转速这个概念. 转速也是指物体在单位时间内转过的圈数,用符号$n$表示,单位是转每秒.
转速在数值上等于频率.

\subsection{向心力\ \ 向心加速度}
做圆周运动的物体,速度的方向时刻都在改变.牛顿第一定律告诉我们,
物体在做圆周运动时,一定受到一个外力, 迫使它的速度改变方向.

一个小球在细线的牵引下, 绕光滑桌面上的图钉做匀速圆周运动. 用剪刀将细线剪断, 发现
小球向断裂瞬间细线指向的方向运动. 忽略小球运动时受到的阻力, 在桌面上做匀速圆周运
动的小球所受的合力为细线的拉力, 拉力即为使小球做圆周
运动的力, 根据拉力的特点可以知道拉力的方向指向圆心. 

大量实例都表明: \textbf{做匀速圆周运动的物体所受的合力总指向圆心.}
这个指向圆心的力就叫作\textbf{向心力},用符号$F_\mathrm{n}$表示.

对于做匀速圆周运动的物体, 物体的速度大小不发生
改变, 因此, 所受合力只改变速度的方向.

物体受到指向圆心的力,是物体做圆周运动的原因.精准的实验表明,向心力的大小可以表示为
\begin{empheq}[box=\fbox]{equation*}
	F_\mathrm{n} = m\frac{v^2}{r},
\end{empheq}
或者
\begin{empheq}[box=\fbox]{equation*}
	F_\mathrm{n} = m{\omega}^{2}r.
\end{empheq}

需要注意的是,向心力属于效果力,实际上就是物体所受力沿半径方向的分力.
如果在分析物体实际所受各力之后,又另加一个向心力,那就不对了.

在一根绳子的一端绑一个小沙袋, 另一端握在手中. 将手举过头顶, 使沙袋在
水平面内做匀速圆周运动, 我们可以直观地感受到向心力. 由于绳子柔软的性质, 
绳子的牵引沙袋的方向就是沙袋所受力的方向.

如果让沙袋加速转动, 可以发现, 绳子牵引沙袋的方向并不与
沙袋运动的方向垂直, 也就是说, 沙袋加速时, 它所受的力并不严格指向轨迹的圆心.

我们可以将绳子对沙袋的拉力$F$分解为指向圆心的分力$F_\text{n}$, 和与轨迹相切的分力$F_\text{t}$.
由几何知识可知它们是垂直的. $F_\text{n}$的方向指向圆心, 提供沙袋做圆周运动的向心力;
$F_\text{t}$与沙袋运动的速度同向, 使沙袋的速度越来越大, 这个与轨迹相切的力叫作\textbf{切向力}.

向心力产生的加速度是\textbf{向心加速度},用符号$a_\mathrm{n}$表示. 根据牛顿第二定律, 
任何做匀速圆周运动的物体,其加速度都指向圆心. 向心加速度的大小描述线速度方向改变的快慢.
$$a_\mathrm{n}=\frac{v^2}{r}={\omega}^{2}r.$$
切向力产生的加速度是\textbf{切向加速度}, 用符号$a_\mathrm{t}$表示.
切向加速度的大小描述线速度大小改变的快慢.

描述物体运动状态的物理量叫作\textbf{运动参量}, 
线速度,角速度,周期,频率,及向心加速度都属于圆周运动的运动参量.

\section{水平面上的圆周运动}

\begin{figure}[htbp]
\centering

\subfloat[小球的运动轨迹]{
    \includegraphics[width=0.33\textwidth]{pic/pic8.pdf}
    \label{圆锥摆-1}
}
\subfloat[小球的受力示意图]{
    \includegraphics[width=0.25\textwidth]{pic/pic9.pdf}
    \label{圆锥摆-2}
}
	\caption{圆锥摆模型}
    \label{圆锥摆-3}
    
\end{figure}

我们已经知道, 向心力是一种效果力, 可以由某一性质的力提供, 也可以由某几个力的合力或某一个力的分力来提供.

分析向心力的一般步骤是:

(1) 确定圆周运动轨道所在的平面;

(2) 找出圆周轨道的圆心的位置;

(3) 分析做圆周运动的物体所受的力, 并作出受力示意图, 其中指向圆心的合力(或分力)就是向心力.

因此, 只要物体在水平面内做匀速圆周运动, 它所受到的合力就一定在圆周运动轨道所在的水平平面上,
并且指向圆周轨道的圆心.

图\ref{圆锥摆-3} 中的模型叫作\textbf{圆锥摆}. 我们以它为例进行分析.

如图\ref{圆锥摆-1}, 长为$L$的细线上栓一质量为$m$的小球, 细线一端固定于$O$点, 让其在水平面内做匀速圆周运动.
这种运动一般称为圆锥摆运动, $h$称为摆高. 细线与竖直方向的夹角为$\theta$.

在找出圆周轨道所在平面以及圆心之后, 我们对小球进行受力分析(图\ref{圆锥摆-2}). 
因为小球在水平面内做匀速圆周运动, 所以小球受到的向心力是水平的. 
由图\ref{圆锥摆-2} 可知, 向心力就是拉力$F_\text{T}$与重力$mg$的合力.
并且有\footnote{\eqref{圆锥摆2} 式其实是牛顿第二定律的表达式. 等号左边$mg\tan\theta$是小球受到的合力, 右边是质量与加速度的乘积. 但这个式子根据图\ref{圆锥摆-2} 的几何关系可以直接得到.}
\begin{align}
	F_\mathrm{T} \cos \theta = mg,
	\label{圆锥摆1}\\
	mg \tan \theta = ma_\mathrm{n}.
	\label{圆锥摆2}
\end{align}

在上面两个关系式中, 已知的量是小球的质量$m$, 重力加速度$g$以及$\cos\theta$, $\tan\theta$; 未知的量是
细线对小球的拉力$F_\text{T}$, 和小球的加速度$a_\text{n}$.

$F_\text{T}$可以从(\ref{圆锥摆1})解出; $a_\text{n}$可以从(\ref{圆锥摆2})解出, 即
\begin{equation}
	a_\mathrm{n} = g\tan \theta.
    \label{圆锥摆向心加速度}
\end{equation}
根据几何关系可知小球做圆周运动的半径
\begin{equation}
    r = h \tan \theta.
    \label{圆锥摆半径}
\end{equation}
把向心加速度公式$a_\mathrm{n}=\omega^2 r$和 \eqref{圆锥摆半径} 式代入 \eqref{圆锥摆向心加速度} 式, 可得
\begin{equation*}
    g\tan \theta = \omega^2 h \tan \theta.
\end{equation*}
由此解得, 小球做圆锥摆运动的角速度
\begin{equation}
    \omega = \sqrt{\frac{g}{h}}.
    \label{圆锥摆角速度}
\end{equation}
这就是说, \textbf{在圆锥摆运动中, 物体的角速度的二次方与摆高成反比},即$$\omega^2 \propto \frac{1}{h}.$$
进而, 小球做圆锥摆运动的周期为$$T = \frac{2\uppi}{\omega} = 2\uppi \sqrt{\frac{h}{g}}.$$

根据几何关系还可以得到$h = L\cos\theta$.把它代入 \eqref{圆锥摆角速度} 式, 可以解出
$$\cos\theta = \frac{g}{l\omega^2}.$$
从此式可以看出, 当小球运动的角速度$\omega$增大时, 夹角$\theta$也随之增大. 


\section{竖直平面上的圆周运动}

\subsection{竖直圆周的几何分析}

\begin{figure}[htbp]
\centering

\subfloat[$\theta\in\left[ 0,\displaystyle\frac{\uppi}{2}\right)$时小球的受力示意图]{
    \includegraphics[width=0.33\textwidth]{pic/7.3-1.pdf}
    \label{竖直圆周-1}
}
\subfloat[任意时刻小球的受力示意图]{
    \includegraphics[width=0.27\textwidth]{pic/7.3-2.pdf}
    \label{竖直圆周-2}
}
	\caption{竖直圆周的受力示意图}
    \label{竖直圆周-3}
    
\end{figure}

用一根轻绳拉着质量为$m$的小球, 另一端固定在竖直平面内的一点, 使小球在竖直平面内做圆周运动.
为了分析小球的向心力, 我们将小球所受的重力$G$分解为沿半径方向的$G_\text{n}$, 和沿切线方向的,垂直于$G_\text{n}$的$G_\text{t}$.

$G_\text{n}$提供小球的向心力,改变速度的方向.
当绳提供的力不为0时, 小球的向心力是$G_\text{n}$与拉力的合力.

$G_\text{t}$是小球的切向力, 改变速度的大小.特别地, 当小球在最高点和最低点时, $G_\text{t} = 0$, 即小球的切向力为0.
因此小球的速率不变. 一般时刻, 在竖直面做圆周运动的小球做\textbf{变速圆周运动}.

不妨假设小球逆时针转动, 圆心为$O$, 小球的位置为$P$. 以竖直向下为正方向, 
设$\theta$是从正方向到$OP$方向的角, 并且$\theta\in[ 0,2\uppi)$, 这样, 由于小球做圆周运动
的半径是固定的$r$, 所以小球在某一时刻的位置就能由角$\theta$唯一确定.

当$\theta\in\left[ 0,\displaystyle\frac{\uppi}{2}\right)$时, 绳子的拉力$F$与$G_\text{n}$反向(图\ref{竖直圆周-1}).
我们写出此时向心力的表达式
$F - G_\text{n} = m\displaystyle\frac{v^2}{r}.$即
\begin{equation}
    F = mg\cos \theta + m\frac{v^2}{r}.
    \label{竖直圆周向心力方程}
\end{equation}
通过数学知识可以验证, \eqref{竖直圆周向心力方程} 式其实就是小球在任意时刻满足的向心力方程.

小球从最低点运动到最高点, $\cos \theta$的值从1变化到$-1$, 而线速度$v$也在不断变小.
因此, 拉力$F$在最低点处取得最大值
\begin{equation*}
	F_\text{max} = m\frac{v_\text{max}^2}{r} + mg.
	\label{最大拉力}
\end{equation*}
在最高点处取得最小值
\begin{equation}
	F_\text{min} = m\frac{v_\text{min}^2}{r} - mg.
	\label{最小拉力}
\end{equation}
其中$v_\text{max}, v_\text{min}$分别是小球在最低点和最高点处的线速度大小.
当然, \eqref{最小拉力} 式成立的前提是小球能上升到最高点, 并且能持续做圆周运动. 
这就需要小球在到达最高点时有速度, 即$v_\text{min}\neq 0$, 并且绳子时刻处于拉直状态, 即
$F_\text{min}\geqslant 0.$

下面我们来研究小球能到达最高点的条件.

\subsection{竖直圆周的临界问题}

通过受力分析可知, 物体在竖直平面内做圆周运动时,一般在通过最高点时处于临界状态.下面我们来
研究一下这个问题.

我们知道, 当$F_\text{min} = 0$时, 小球恰好能到达最高点.由(\ref{最小拉力})可知
临界条件是$mg = m\displaystyle\frac{v_0^2}{r}$ (式中$v_0$是小球恰达最高点时的速度).由此解得
\begin{equation}
	v_0 = \sqrt{gr}.
	\label{临界速度}
\end{equation}

这就是说, \textbf{在竖直平面内被绳子牵引做圆周运动的物体, 能通过最高点并继续运动的条件是, 小球通过最高点时
	的速度不小于$\sqrt{gr}$.}

如果小球在最高点处的速度小于这个值时, 那么$mg \textgreater m\displaystyle\frac{v^2}{r}$.
小球受到的向心力大于圆周运动所需的向心力, 根据牛顿第二定律, 小球将脱离原圆周运动的轨道, 靠近圆心,做\textbf{近心运动}.

上面讨论的情况是小球受绳牵引的情况. 绳与杆的区别是, 绳只能提供拉力;而杆既可以提供拉力, 也可以提供支持力, 
即杆提供的力是双向的. 如果小球用杆牵引,或在管径略大于小球直径的管道内运动, 
那么只需过最高点时的的速度$v_\text{min}\neq 0$, 小球即可完成圆周运动.
特别地, 如果小球到达最高点时, 与支撑物(杆或管道)间无相互作用, 那么小球在这一点的速度$v = \sqrt{gr}$.

学习动能定理后, 解决此类问题将变得容易许多.
\section{生活中的圆周运动}

\subsection*{汽车过拱桥\ \ \ 汽车过地道}

汽车通过拱桥时, 实际是在做竖直面上的圆周运动. 当汽车在拱桥的最高点时, 汽车所受的重力大于支持力, 发生失重现象.
视拱桥为圆弧, 其半径为$r$.
设汽车的质量为$m$, 汽车所受的支持力为$F_\mathrm{N}$, 汽车在最高点时的速度为$v$, 重力加速度为$g$.
对汽车进行受力分析, 可得$$mg - F_\mathrm{N} = m\frac{v^2}{r}.$$
由此可知, 随着汽车速度$v$的增大, 汽车所受的支持力$F_\mathrm{N}$减小. 当支持力减小到0时, 汽车做平抛运动,
此时有$$mg = m\frac{v^2}{r},$$ 可以解出此时的速度$v = \sqrt{gr}.$

汽车通过地道时, 所受的重力小于支持力, 发生失重现象. 视地道为圆弧, 其半径为$r$.
对汽车进行受力分析, 可得$$F_\mathrm{N} - mg = m\frac{v^2}{r}.$$
由此可知, 随着汽车速度$v$的增大, 汽车所受的支持力$F_\mathrm{N}$也随之增大. 如果汽车速度过快, 可能会爆胎.

\subsection*{汽车拐弯}

汽车拐弯时, 实际是在做水平面上的圆周运动. 视弯道为一段圆弧, 其半径为$r$. 汽车拐弯的向心力由静摩擦力$F_\mathrm{f}$
提供, 即$$F_\mathrm{f} = m\frac{v^2}{r}.$$因为$F_\mathrm{f}$有最大值, 所以当速度$v$过大时, 静摩擦力$F_\mathrm{f}$
可能无法提供向心力.

当$F_\mathrm{f} < m\displaystyle\frac{v^2}{r}$时, 摩擦力小于汽车所需的向心力,
汽车将会脱离原圆周运动的轨道,远离圆心, 做\textbf{离心运动}.这是很危险的.

\begin{wrapfigure}[18]{o}{6cm}
	\flushright
	\includegraphics[width=0.3\textwidth]{pic/pic12.pdf}
    \caption{火车的轨道}
	\label{fig12}
\end{wrapfigure}

为了防止发生危险, 弯道一般会做成倾斜的(图\ref{fig12}), 火车轨道更会严格的设计倾斜角度, 这是为了让$mg$与$F_\mathrm{N}$产生
水平合力, 从而提供拐弯所需的向心力, 即$$mg\tan \theta = m\frac{v^2}{r}.$$
由此可得, 当$v = \sqrt{gr\tan \theta}$时, 汽车与地面间无摩擦力.

在这样的设计下, 即使汽车的速度略超过临界速度$\sqrt{gr\tan \theta}$, 也不易发生危险, 因为静摩擦力会补上所需的那部分力.

飞机在转弯时倾斜机身, 也是因为这一点.

\chapter{万有引力与宇宙航行}
\label{万有引力章}

\section{万有引力}

\subsection{行星运动的科学史}

在古代,人们对于天体运动存在着地心说和日心说两种对立的看法. 地心说是托勒密提出的, 日心说是哥白尼提出的. 最终,
日心说战胜了地心说, 被人们所接受.

德国天文学家开普勒后续发现, 行星绕太阳运动的轨道不是圆, 而是椭圆, 并
于1609年和1619年, 发表了他发现的下列规律, 后人称为开普勒行星运动定律.

\textbf{开普勒第一定律\bre 所有行星绕太阳运动的轨道都是椭圆,太阳处在椭圆的一个焦点上.} 

\textbf{开普勒第二定律\bre 对任意一个行星来说,它与太阳的连线在相等的时间内扫过的面积相等.} 

开普勒第二定律告诉我们:当行星离太阳较近的时候,运行的速度较大,而离太阳较远的时候速度较小.

\textbf{开普勒第三定律\bre 所有行星轨道的半长轴的三次方跟它的公转周期的二次方的比都相等.} 

若用 $a$ 代表椭圆轨道的半长轴, $T$代表公转周期,开普
勒第三定律告诉我们$$\frac{a^3}{T^2} = k, $$其中比值$k$是一个对所有行星都相同的常量.
事实上, \textbf{它由太阳的质量决定}.

实际上, 行星的轨道与圆十分接近.在中学阶段,我们仍按圆轨道处理,这样就可以说:
\begin{enumerate}
	\setlength{\itemsep}{0pt}
	\setlength{\parsep}{0pt}
	\setlength{\parskip}{0pt}
	\item \textbf{行星绕太阳运动的轨道十分接近圆,太阳处在圆心.}
	\item \textbf{对某一行星来说,它绕太阳做圆周运动的角速度
	      (或线速度)大小不变,即行星做匀速圆周运动.}
	\item \textbf{所有行星轨道半径 $r$ 的三次方跟它的公转周期 $T$ 的二
	      次方的比值都相等, 即$\displaystyle\frac{r^3}{T^2} = k.$}
\end{enumerate}

\subsection{万有引力}

根据开普勒第三定律, 圆周运动加速度表达式, 可以推出行星与太阳间的作用力$$F = G \frac{m_\text{太}m}{r^2}, $$
其中$G$是常数, $m$是行星的质量, $m_\text{太}$是太阳的质量.太阳与行星间引力的方向沿着二者的连线.

``月地检验''告诉我们, 地面上的物体所受地球的引力,月球所受地球的引力,以及太阳与行星间的引力,都遵从相同的规律.

其实, 任意两个物体间都有引力, 只是因为它们的质量相比于天体很小, 难以发现罢了.
\subparagraph{万有引力定律} 自然界中任何两个物体都互相吸引,引力的方向在它们的连线上,引力的大小
与物体的质量$m_1$, $m_2$的乘积成正比,与它们之间距离$r$的二次方成反比,即
\begin{empheq}[box=\fbox]{equation*}
	F = G\ \frac{m_1 m_2}{r^2},
\end{empheq}
其中质量的单位用千克,距离的单位用米,力的单位用牛.$G$是比例系数,叫作引力常量,适用于任何两个物体, $$G \approx 6.67 \times 10^{-11}\ \mathrm{N\cdot m^2 \cdot kg^{-2}}.$$

\subsection{称量天体的质量}

\textbf{方法 1}\ \ \ 已知引力常数$G$, 和某绕转天体的任意两个运动参量(线速度,角速度,加速度,绕转半径), 由万有引力等于向心力,
可以解出中心天体的质量.

\textbf{方法 2}\ \ \ 已知引力常数$G$和天体的半径$R$, 再测定出该天体表面的的重力加速度$g$, 由$mg = G\displaystyle\frac{Mm}{R^2}$, 可以解出天体的质量$M$.

\section{万有引力与重力}

\setlength{\abovedisplayskip}{3pt}
\setlength{\belowdisplayskip}{3pt}

物体与地球间的万有引力实际上就是物体随地球转动的向心力与物体受到的重力重力的合力.因此,
不计地球自转时,重力就是物体与地球间的万有引力.

\subparagraph{1. 地球表面的重力加速度}地球表面一质量为$m$的物体,在不考虑地球自转的情况下,它的重力就是它与地球间的万有引力.
已知地球的质量为$M$,地球半径(视地球为正球体)为$R$,万有引力常数为$G$,则有
$$mg=G\frac{Mm}{R^2}.$$

即
\begin{empheq}[box=\fbox]{equation}
	g=G\frac{M}{R^2}.
	\label{重力加速度}
\end{empheq}
这是\textbf{不计地球自转时, 地球上一点的重力加速度的决定式}.

\subparagraph{2. 不同高度处的重力加速度}

物体在距地表 $h$处的空中, 那么 (\ref{重力加速度}) 中$R$替换为${R+h}$,即
$$g'=G\frac{M}{(R+h)^2}.$$

\subparagraph{3. 不同深度处的重力加速度}

物体在地下深度为 $H$ 处, 因为物体上方的``球壳''对物体没有引力, 所以
\begin{align*}
	g' & =G\frac{(\frac{R-H}{R})^3M}{(R-H)^2} \\
	   & =\frac{GM(R-H)}{R^3}                 \\
	   & =\frac{GM}{R^2}\cdot\frac{R-H}{R}    \\
	   & =g\ \frac{R-H}{R}.
\end{align*}
其中$g$是地球表面的重力加速度.

于是就有
$$\frac{g'}{g}=\frac{R-H}{R}.$$

如果考虑地球自转,那么物体在地球上受到的重力就等于物体与地球间的万有引力与物体随地球做圆周运动
(一般视为匀速圆周运动)的向心力的差.特别地,当物体在极点上时,不会随地球自转而运动,即仍有
$$mg=G\frac{Mm}{R^2}.$$

当物体在赤道上时,物体做圆周运动的半径就是地球半径,并且向心力与引力同向
$$mg=G\frac{Mm}{R^2}-m\omega^2R.$$

一般地,当物体在地球上某一点时,它们之间的万有引力就是物体受到的重力与向心力的矢量和.

\subparagraph{黄金代换式}

由(\ref{重力加速度})可得
\begin{empheq}[box=\fbox]{equation}
	GM = gr^2.
	\label{黄金代换式}
\end{empheq}
式中$G$是引力常数, $M$是地球质量, $r$是地球中心与物体的距离, $g$是物体在距地球中心$r$处的重力加速度.

这个式子常用于$GM$与$gr^2$之间的代换, 称为黄金代换式.

\section{万有引力与宇宙航行}

本节我们研究卫星绕地球运动的问题.
由于卫星与地球的距离较远, 根据 \eqref{重力加速度} 可知, 卫星所受的重力非常小,
所以我们只研究万有引力与圆周运动的向心力之间的关系.

\subsection{宇宙航行的定量计算}

因为卫星绕地球运动的向心力由卫星与地球间的万有引力提供, 设地球质量为$M$, 卫星质量为$m$,
卫星中心与地球中心的距离为$r$, 卫星的线速度为$v$, 角速度为$\omega$, 向心加速度为$a_\mathrm{n}$,
万有引力常数为$G$, 有
\begin{equation}
	G\frac{Mm}{r^2} = m\frac{v^2}{r},
	\label{宇宙航行1}
\end{equation}
\begin{equation}
	G\frac{Mm}{r^2} = m\omega^2 r,
	\label{宇宙航行2}
\end{equation}
\begin{equation}
	G\frac{Mm}{r^2} = ma_\mathrm{n}.
	\label{宇宙航行3}
\end{equation}

由 \eqref{宇宙航行1} 式可得$v = \sqrt{\displaystyle\frac{GM}{r}}$, 即$v\propto \sqrt{\displaystyle\frac1r}$.\par\vspace{8pt}
由 \eqref{宇宙航行2} 式可得$\omega = \sqrt{\displaystyle\frac{GM}{r^3}}$, 即$\omega \propto \sqrt{\displaystyle\frac{1}{r^3}}$.\par\vspace{8pt}
由 \eqref{宇宙航行3} 式可得$a_\mathrm{n} = \displaystyle\frac{GM}{r^2}$, 即$a_\mathrm{n}\propto \displaystyle\frac{1}{r^2}$.

\subsection{宇宙航行的定性分析}

由前面的分析, 我们可以知道, 当卫星的绕转半径$r$一定时, 它的线速度, 角速度, 向心加速度以及周期都是一定的, 这保证了
卫星间的相对位置稳定, 不会相撞.

当卫星的绕转半径$r$增大时, 根据前面的分析, 卫星的线速度$v$减小; 向心加速度$a_\mathrm{n}$减小; 角速度$\omega$减小,
由$\omega = \displaystyle\frac{2\uppi}{t}$知周期$T$增大. 这就是\textbf{``高轨低速大周期''}.

我们把卫星的线速度, 角速度(周期), 向心加速度, 绕转半径称为卫星的运动参量, 知道它们中的两个, 就可以求出另外两个.
当卫星的质量未知时, 引力值是无法求出的.

\subsection{三种宇宙速度}

\subparagraph{第一宇宙速度} 使卫星能环绕地球运行所需的最小发射速度叫作第一宇宙速度.
当卫星以第一宇宙速度$v_1$发射时, 它将绕地球做匀速圆周运动. 设地球质量为$M$, 卫星的
质量为$m$, 它到地心的距离为$r$. 由于卫星做圆周运动所需的向心力由万有引力提供, 因此
有$$G\frac{Mm}{r^2} = m\frac{v^2}{r},$$解得
\begin{equation}
	v = \sqrt{\frac{GM}{r}}.
	\label{卫星绕转速度}
\end{equation}
此时$r$约等于地球半径$R$,即$v_1 = \displaystyle\sqrt{\frac{GM}{R}}$.
代入数据可解得$$v_1 \approx 7.9\ \mathrm{km/s}.$$

除此之外, 根据黄金代换式 \ref{黄金代换式} 和(\ref{卫星绕转速度}),我们可以得到卫星绕转速度的另一种表达式$$v = \sqrt{gr},$$
但这并不能说明卫星的绕转速度与$\sqrt{r}$成正比,因为在卫星的高度增加时, $g$的值也在变化,
并且$g = \displaystyle\frac{GM}{r^2}$.

事实上, 根据(\ref{卫星绕转速度}), \textbf{卫星的绕转速度$v$与$\sqrt{r}$是成反比的}.

\subparagraph{第二宇宙速度} 在地面附近发射人造卫星, 如果发射速度
$v \in (7.9\ \mathrm{km/s}, 11.2\ \mathrm{km/s})$, 那么它绕地球运行的轨迹将是椭圆.
当发射速度大于等于$11.2\ \mathrm{km/s}$时, 卫星会克服地球的引力而永远离开地球.
我们把它叫作第二宇宙速度.$$v_2 \approx 11.2\ \mathrm{km/s}.$$

\subparagraph{第三宇宙速度} 达到第二宇宙速度的人造卫星还受到太阳的引力. 在地面附近发射一颗人造卫星,
要使它摆脱太阳引力的束缚, 飞到太阳系外, 必须使其发射速度大于等于$16.7\ \mathrm{km/s}$, 这个速度叫作
第三宇宙速度.$$v_3 \approx 16.7\ \mathrm{km/s}.$$
\begin{wrapfigure}{o}{5cm}
	\flushright
	\includegraphics[width=0.28\textwidth]{pic/pic13.pdf}
	\label{fic13}
\end{wrapfigure}

\subsection{变轨问题}

卫星绕中心天体做匀速圆周运动时, 万有引力提供向心力. 当卫星由于某种原因速度突然改变时, 由
(\ref{宇宙航行1})可知, 万有引力不再等于
向心力, 卫星将变轨运行. 卫星的发射和回收就是利用了这一原理.

一个质量为$m$的卫星以第一宇宙速度发射, 它绕地球做匀速圆周运动. 这时, 如果该卫星喷气加速, 那么万有引力不足以
提供向心力, 卫星将做离心运动, 卫星的轨迹是一个椭圆.可以知道, 卫星在椭圆近地点的速度, 大于卫星在原匀速圆周轨道上的速度.
当卫星运行到椭圆的远地点时, 根据 \ref{卫星绕转速度}, 卫星速度减小.

如果卫星在原地点时再次加速, 调整为合适的速度, 那么卫星就可以在更大的圆周轨道上运动. 可以知道, 卫星在大圆周轨道
上的速度, 大于卫星在原椭圆轨道远地点的速度. 另外, 由于大圆周轨道与原小圆周轨道都是圆周轨道, 根据``高轨低速大周期'',
可知卫星在小圆周轨道上的速度大于卫星在大圆周轨道上的速度.

综上所述, 可以得到
$$v_\text{椭圆近地点} \textgreater v_\text{小圆周轨道} \textgreater v_\text{大圆周轨道} \textgreater v_\text{椭圆远地点}.$$

\subsection{双星系统}

在天体运动中,将两颗彼此距离较近的恒星称为双星.它们在相互之间的万有引力作用下,绕两
者连线上的某定点做周期相同的匀速圆周运动.双星系统具有以下三个特点:
\begin{enumerate}
	\setlength{\itemsep}{0pt}
	\setlength{\parsep}{0pt}
	\setlength{\parskip}{0pt}
	\item 两星球做圆周运动所需的向心力由两者间的万有引力提供, 因此两星球做圆周运动的向心
	      力大小相等;
	\item 两星球绕转动中心做圆周运动的角速度(或周期)的大小相等;\label{周期相等}
	\item 质量为$m_1$, $m_2$的两星球绕共同中心转动的半径$r_1$, $r_2$的和等于两星球间的距离$L$,
	      即$$r_1 + r_2 = L,$$ 并且\textbf{两星球绕转半径之比等于它们质量的反比}, 即
	      $$\frac{r_1}{r_2} = \frac{m_2}{m_1}.$$ \label{半径与距离}
\end{enumerate}

以上三个特点是解决很多双星问题(求解运动周期, 角速度, 轨道半径等) 的关键.

由特点\ref{周期相等}, 我们来推导双星系统运动的周期.

设这两个恒星的的质量分别为$m_1$, $m_2$, 绕转半径分别为$r_1$, $r_2$, 它们间的距离为$L$, 根据特点\ref{半径与距离},
有$r_1 + r_2 = L$.

对于$m_1$,有$$\frac{Gm_2}{L^2} = \left(\frac{2\uppi}{T}\right)^2r_1, $$

对于$m_2$, 有$$\frac{Gm_1}{L^2} = \left(\frac{2\uppi}{T}\right)^2r_2, $$

把上面两式相加, 得$$\frac{G(m_1 + m_2)}{L^2} = \left(\frac{2\uppi}{T}\right)^2L, $$

由此可得双星系统的运动周期$$T = 2\uppi \sqrt{\frac{L^3}{G(m_1 + m_2)}}.$$

\chapter{功和能}

\section{功}

\subsection{恒力的功}

物体受到力的作用, 并且在力的方向上发生一段位移,我们就说,\textbf{力对物体做了机械功}.
机械功是物理学中表示力对位移的累积的物理量.

初中时我们知道, 当一个恒力$F$作用在物体上, 使物体沿力的方向发生大小为$l$的位移时, 
力$F$对物体所做的功$W$等于力的大小$F$与位移的大小$l$的乘积, 即
$$W = Fl.$$
实际中, 更普遍的情形是物体运动
的方向与力的方向不一致, 例如马拉雪橇时拉力
方向和雪橇运动方向间有一个角度. 这时应当怎
样计算功呢?

\begin{figure}
    \centering
    \includegraphics[width=0.5\textwidth]{pic/9.1-A.pdf}
    \caption{物体在力$F$的作用下发生位移$l$}
    \label{fig:力与位移的夹角}
\end{figure}
当力 $F$ 的方向与运动方向成某一角度时(图\ref{fig:力与位移的夹角}), 
可以把力 $F$ 分解为两个分力:与位移方向一致的分力$F_1$, 与位移方向垂直的分力
$F_2$.设物体在力 $F$ 的作用下发生的位移的大小是 $l$, 则分力
$F_1$ 所做的功等于 $F_1 l$.分力 $F_2$ 的方向与位移的方向垂直, 物体在 $F_2$ 的方向上没有发生位移, 
$F_2$ 所做的功等于 0. 因此, 力 $F$ 对物体所做的功 $W$ 等于$F_1 l$, 
而 $F_1 = F\cos \theta$, 所以
\begin{empheq}[box=\fbox]{equation}
	W = Fl \cos \theta.
	\label{功的定义式}
\end{empheq}
这就是说, \textbf{功的大小等于力的大小,位移的大小,力与位移夹角的余弦值三者的乘积, 即力矢量与位移矢量的内积.}

功是标量. 在国际单位制中, 功的单位是\textbf{焦耳}, 简称\textbf{焦}, 符号是 J. 1 J 等于 1 N 的力使物体在力的方向上
发生 1 m 位移的过程中所做的功, 所以
$$1 \ \mathrm{J} = 1\ \mathrm{N}\cdot 1\ \mathrm{m} = 1\ \mathrm{N\cdot m}.$$

\subsubsection{功的正负}

虽然功是标量, 但我们对它规定正负. 从公式$W = Fl \cos \theta$的角度来说, 因为$F$和$l$都是正数, 
所以功$W$的正负由力与位移的夹角$\theta$决定. 这要分三种情况讨论.

\begin{enumerate}
	\item 当$0 \leqslant \theta < \displaystyle\frac{\uppi}{2}$ 时, $\cos \theta > 0$, $W > 0$.这表示力$F$对物体做正功,充当动力.\vspace{5pt}

	\item 当$\theta = \displaystyle\frac{\uppi}{2}$ 时, $\cos \theta = 0$, $W = 0$.这说明力$F$与位移$l$垂直时, 力$F$不做功.\vspace{5pt}

	\item 当$\displaystyle\frac{\uppi}{2}<\theta\leqslant \uppi$ 时, $\cos \theta < 0$, $W < 0$.这表示力$F$对物体做负功\footnote{某力对物体做负功,往往也说成``物体克服某力做功''.},充当阻力.\vspace{5pt}
\end{enumerate}

用一个在水平面上运动的小车举例. 人用力拉车前进时, 人的拉力$F$对车做正功; 要使运动的小车减速, 人向后拉车的力$F$对车做负功; 
而小车受到的重力$G$和水平面对它的支持力$F_\text{N}$都与位移方向垂直, 所以这两个力都不做功.

上面我们分析的是物体受一个力时做功的几种情况. 
实际中, 物体通常会受多个力的作用. \textbf{当一个物体在几个
力的共同作用下发生一段位移时,  这几个力对物体所做的
总功,  是各个力分别对物体所做功的代数和.} 可以证明,  
它也就是这几个力的合力对物体所做的功, 称为\textbf{合功}. 

{\kaishu 一对平衡力做的功有什么关系? 一对相互作用力做的功又有什么关系?}

我们知道, 一对平衡力总是大小相等, 方向相反, 作用在同一个物体上. 
因此, 物体受一对平衡力而运动时, 这一对力所做的功一定互为相反数, 即合功为0.

一对作用力与反作用力总是大小相等,方向相反, 但它们分别作用在两个物体上, 
二者两个物体各自发生的位移是不确定的.
所以一个力做功时, 其反作用力可能做功, 也可能不做功; 
可能做正功, 也可能做负功, 它们之间没有必要的联系.

\subsubsection{重力做的功}
\label{重力做的功}
\begin{figure}
    \centering

    \subfloat[沿竖直线运动]{
    \includegraphics[width=0.15\textwidth]{pic/9.3-0.png}
    \label{fig:重力势能-1}
    }
    \subfloat[沿斜线运动]{
    \includegraphics[width=0.28\textwidth]{pic/9.3-1.png}
    \label{fig:重力势能-2}
    }

    \caption{研究物体运动时重力做的功}
    \label{fig:重力势能}
\end{figure}

考虑一个质量为 $m$ 的物体, 从与地面高度为 $h_1$ 的位置 $A$, 
竖直向下运动到高度为 $h_2$ 的位置 $B$ (图\ref{fig:重力势能-1}), 这个过
程中重力做的功是
$$W_\text{G} = mg\Delta h = mgh_1 - mgh_2.$$

再看另一种情况. 质量为 $m$ 的物体仍然从上向下运动, 
高度由 $h_1$ 降为 $h_2$, 但这次不是沿竖直方向, 而是沿着一个
斜面向下运动到 $B^{\prime}$, 再水平运动到 $B$ (图\ref{fig:重力势能-2}).
物体沿斜面运动的距离是$l$, 在这一过程中重力做的功是
$$W_\text{G} =(mg\cos \theta)l= mg\Delta h = mgh_1 - mgh_2.$$
从 $B^{\prime}$沿水平方向运动到 $B$ 的过程中, 重力不做功.

上面两种情况中, 尽管物体运动的路径不同, 但高度
的变化是一样的, 而且重力做的功也是一样的. 

分析表明, \textbf{物体运动时, 重力对它做的功只跟它的起
点和终点的位置有关, 而跟物体运动的路径无关.} 也就是
说, 只要起点和终点的位置不变, 不论物体沿什么路径运
动, 重力所做的功都相同. 具有这种性质的力还有弹簧的弹力, 
电场力, 万有引力等. 我们把这样的力称为\textbf{保守力}.
任何一种恒力都是保守力.

\subsubsection{摩擦力做的功}

一个物块沿着粗糙的斜面上滑, 滑动至最高点时位移为$s$, 则摩擦力$F_\mathrm{f}$所做的功
$$W_\mathrm{f} = F_\mathrm{f}s\cos\uppi = -F_\mathrm{f}s.$$
负号表示摩擦力做负功, 即摩擦力阻碍物块的上滑.

若物块滑动至最高点后又沿反方向下滑, 则摩擦力的大小不变, 方向与先前$F_\mathrm{f}$的方向相反. 下滑至原点时
摩擦力又做功$W_\mathrm{f}$, 从而全程摩擦力做功为$$W_\mathrm{f}^{\prime} = 2W_\mathrm{f} = -2F_\mathrm{f}s.$$

虽然全程的位移为0, 但由于摩擦力的方向改变了, 所以我们不能简单的用摩擦力的大小乘以位移0而得出摩擦力做功为0.

需要注意的是, 摩擦力并不总是做负功. 例如物块放在传送带上由静止开始运动时, 摩擦力就做了正功.

\subsection{直线运动的功}

前面我们研究的都是恒力做的功. 在直线运动中, 如果力随着物体位置的变化而变化, 又该如何定义它的功呢?

建立描述物体位置的一维坐标系$x$. 如果物体在力$F$的作用下在坐标轴上运动, 且力$F$与位置坐标$x$的函数
关系为$F(x)$, 则称满足
$$W^{\prime}(x) = F(x).$$
的$W = W(x)$为物体在力$F$的作用下, 从位置$x = 0$运动到位置$x$时, 力$F$对物体所做的\textbf{功}.\footnote{这里
用反导数的概念定义了功, 即功$W$关于位置$x$的函数$W(x)$就是力$F(x)$的一个反导数. 实际上, 如果
利用定积分, 功$W$可以表示为$$W = \int_0^x F(x)\mathrm{d}x.$$这是大学物理中对功的一个比较严谨的定义.}

由数学知识可知, 当$F(x)$恒为常数$F$时, $W(x) = Fx$, 这就是恒力做功的情况.

\section{功率}

物理学中, 反映做功快慢的物理量是\textbf{功率}.
如果从开始计时到时刻$t$的这段时间内, 力$F$对物体所做的功为$W$, 
则功 $W$ 与完成这些功所用的时间 $t$ 之比叫作力$F$的\textbf{功率}, 用$P$表示, 即
\begin{empheq}[box=\fbox]{equation}
	P = \frac{W}{t}.
	\label{功率的定义式}
\end{empheq}

功率是标量. 在国际单位制中, 功率的单位是\textbf{瓦特}, 简称\textbf{瓦}, 符号
是 W. 瓦这个单位比较小, 技术上常用\textbf{千瓦}(kW)
作功率的单位, 1 kW = 1 000 W. 力, 位移, 时间都与功率相联
系, 这种联系在技术上具有重要意义.

{\kaishu 瞬时速度与瞬时功率有什么关系呢?}

如果物体受到的力是$F$, 从计时开始到时刻 $t$ 这段时
间内, 发生的位移是 $l$, $F$与$l$的夹角是$\theta$, 则力$F$在这段时间内所做的功
$$W = Fl \cos \theta.$$
如果$t$非常小, 我们就把
$$\lim_{t\rightarrow 0} \frac{W}{t} = \lim_{t\rightarrow 0} \frac{Fl \cos \theta}{t} = (F\cos\theta)\lim_{t\rightarrow 0} \frac{l}{t}, $$
叫作力$F$在时刻$t$的\textbf{瞬时功率}. 而$\displaystyle\lim_{t\rightarrow 0} \displaystyle\frac{l}{t}$正是物体在时刻$t$的瞬时速度$v$, 
因此瞬时功率可以表示为
\begin{empheq}[box=\fbox]{equation}
	P = Fv \cos \theta.
	\label{瞬时功率}
\end{empheq}
可见, 某一时刻力对物体做功的功率, 等于力的大小, 
物体在这一时刻的速度及力与位移的夹角的余弦值这三者的乘积.
这就是瞬时速度与瞬时功率的关系.\footnote{当$v$为平均速度时, \eqref{瞬时功率} 式也可以计算对应时间$t$内的平均功率.}

从(\ref{瞬时功率}) 式可以看出, 汽车, 火车等交通工具和各种起重机械, 当发动机的输出功
率 $P$ 一定时, 牵引力 $F$ 与速度 $v$ 成反比. 所以, 在汽车上坡时, 为了获得更大的牵引力, 
司机需要换到低速档.

\section{势能}

我们在初中物理课程中学过, 功和能是两个联系密切
的物理量, 一个物体能够对其他物体做功, 我们就说这个
物体具有\textbf{能量}.能量是表示物体做功本领的物理量, 用$E$表示.
而\textbf{势能}是由于物体所处的位置而具有的能量.
比如重力势能表示物体的重力做功的能力, 物体的质量越大, 所处的位置越高, 重力势能就越大. 
下面我们来进一步定量地研究重力势能.

\subsection{重力势能}
\subsubsection{重力势能}
在 \ref{重力做的功} 节中, 我们已经推导出了重力做功的公式, 并且知道, 只要确定了起点和终点的位置, 
不论运动路径如何, 重力所做的功都相同. 功等于物体所受的重力跟起点
高度的乘积 $mgh_1$ 与跟终点高度的乘积 $mgh_2$ 两者之差.
看起来, 物体所受的重力 $mg$ 与它所处位置的高度 $h$ 的
乘积 $mgh$, 具有特殊的意义. 

$mgh$ 的特殊意义在于它一方面与重力做的功密切相关, 且与功的单位相同; 
另一方面它随着高度的增加而增加, 随着质量的增加而增
加, 恰与前述重力势能的特征一致. 因此, 我们把 $mgh$ 叫作
物体的\textbf{重力势能}, 常用 $E_\mathrm{p}$
表示, 即$$E_\mathrm{p} = mgh.$$

与其他形式的能一样, 重力势能也是标量, 其单位与
功的单位相同, 在国际单位制中都是焦耳, 符号为 J.

\subsubsection{重力做的功与重力势能的关系}
\label{重力势能小节}

有了重力势能的概念, 重力做的功与重力势能的关
系可以写为
$$W_\text{G} = E_{\text{p}1} - E_{\text{p}2}.$$
其中 $E_{\text{p}1}$ 表示物体在初位置的重力势能, $E_{\text{p}2}$ 表示物体在末
位置的重力势能. 也就是说, \textbf{重力做的功等于物体在初位置与末位置的重力势能之差.}

当物体从高处运动到低处时, 重力做正功, 重力势能减小; 
当物体从低处运动到高处时, 重力做负功, 重力势能增大.

\subsubsection{重力势能的相对性} 

物体的重力势能总是相对于某一水平面来说的, 这个
水平面叫作\textbf{参考平面}, 也叫零势能面. 在这个水平面上, 物体的重力势能取
为 0. 选择哪个水平面为参考平面, 可视研究问题的方便而
定. 通常选择地面为参考平面.

选定一个参考平面, 当物体在参考平面上方时, 重力势能
为正值; 当物体在参考平面下方时, 重力势能为负值. 选择不同的参考平面, 物体的重力势能不同,
但重力势能的差值相同.

\subsubsection{势能是系统所共有的}
重力是由于地球的吸引而产生的, 它是地球与物体
组成的系统的内力. 研究重力做功, 本应既考虑重力对物
体所做的功, 也考虑它的反作用力, 即对地球的引力所做
的功, 但由于地球与地面上的物体相比较, 质量可以看作
无穷大, 地球的位移完全可以忽略, 因此不必考虑物体吸
引地球的力对地球运动的影响, 只需考虑重力对物体的影
响, 这就是我们常常说``物体的重力势能''的原因. 但这
只是一种通俗的说法, 实际上, 重力势能是地球和受重力
作用的物体组成的系统所共有的, 不是地球上的物体单独
具有的. 这一点从重力势能的表达式$E_\mathrm{p} = mgh$可以看得很
清楚: $m$是物体的质量, $g$是重力加速度, 它是由于地球
的吸引而产生的, $h$则是物体相对于地球表面上某一平面
(参考平面)的高度. 如果没有地球, 就没有$g$, 也谈不
上$h$了.

与重力势能一样, 弹性势能, 电势能等也是系统所共有的, 不是某个物体单独具有的.

\subsection{势能}
前面我们说过, 重力是一种保守力, 即它做的功只与物体所处的位置有关,
而与物体运动的路径无关. 
当某个保守力对物体做功时, 物体相应的能量就发生改变. 我们该怎么表述这种保守力做功时所对应的能量呢?

由于保守力做的功只取决于始末的相对位置, 所以一定存在某一个关于物体位置的状态函数$E(x)$.
我们把$E_\text{p} = E(x)$称为系统在$x$位置处的\textbf{势能}.每一种保守力都可以用一个势能函数来描述.

在\ref{重力势能小节} 节中我们已经给出了重力做的功与重力势能的关系, 事实上, 对于任何保守力$F$,
都有类似的关系. 设物体在初位置处的势能为$E_\text{p1}$, 在末位置处的势能为$E_\text{p2}$, 则$F$
对物体做的功为
$$W_\mathrm{F} = E_\mathrm{p1} - E_\mathrm{p2}.$$
也就是说, \textbf{保守力做的功等于物体在初位置与末位置的对应势能之差.} 这个规律称为\textbf{势能定理}.

\subsection{弹性势能}
拉长或压缩的弹簧, 卷紧的发条, 拉开的弓, 正在击
球的网球拍, 撑竿跳高运动员手中弯曲的竿, 
等等, 这些物体都发生了弹性形变, 每个物体的各部分之
间都有弹力的相互作用. 发生弹性形变的物体的各部分之间, 由于有弹力的相互作用, 也具有势能, 
叫作\textbf{弹性势能}.
当然, 弹性势能也是标量, 单位是焦耳(J).

前面我们提到过, 重力与弹簧的弹力都是保守力, 它们做的功只与物体所处的位置有关. 
我们知道, 弹簧的弹力与弹簧的形变量成正比, 即弹力$F$与形变量$x$满足下面的函数关系
$$F(x) = kx,$$
根据直线运动功的定义, 由数学知识可得
$$\left(\frac12 kx^2\right)^{\prime} = F(x), $$
所以, 弹力做的功与形变量$x$的关系为$$W(x) = \frac12 kx^2.$$

如果规定形变量$x = 0$时弹簧的弹性势能为0, 则形变量为$x$时弹簧的弹性势能的表达式为
$$E_\mathrm{p} = \frac12 k x^2.$$
由此可知, 弹性势能的大小与弹簧的形变量有关, 形变量越大, 弹性势能越大; 弹性势能的大小还与弹簧的劲度系数$k$有关,
$k$越大, 弹性势能越大. 

\section{动能和动能定理}

\subsection{动能}

初中时我们已经知道, 物体由于运动而具有的能量叫作\textbf{动能}. 
运动物体的动能与物体的质量和速度有关.
人类活动中, 利用动能的实例很多, 比如长江三峡水
利枢纽, 主要就是利用水流由高处下落后的巨大动能发电.

质量为 $m$ 的物体沿水平面向右做匀加速直
线运动. 物体在初始位置 $A$ 时的瞬时速度为 $v_A$, 经过位置 $B$ 时
的瞬时速度为 $v_B$. 物体的位移为 $l$.

由于物体做匀加速直线运动, 所以物体所受合力$F$为恒力. 设物体的加速度为$a$, 
则根据牛顿第二定律有
$$F = ma.$$
由匀变速直线运动速度与位移的关系式 \eqref{速度与位移的关系} 可以得到加速度$a$, 代入上式即
$$F = m\left(\frac{v_B^2-v_A^2}{2l}\right)$$
整理后得
\begin{equation}
    Fl = \frac12 mv_B^2-\frac12 mv_A^2.
    \label{动能定理前身}
\end{equation}

\eqref{动能定理前身} 式的左边为合力对物体做的功$W$, 右边是两个单项式之差. 一方面, 这两个单项式与合力
的功密切相关, 且与功的单位相同; 另一方面, 这两个单项式与物体的质量和速度有关, 与动能的特征一致.

我们把物体的质量与速度的二次方乘积的一半,称为物体的\textbf{动能},用符号 $E_\text{k}$ 表示, 即
\begin{empheq}[box=\fbox]{equation*}
	E_\mathrm{k} = \frac12 m v^2.
	\label{动能定理}
\end{empheq}

\eqref{动能定理前身} 式右边的两个单项式, 分别是物体在$B$位置
的动能 $E_{\mathrm{k}B}$ 和在 $A$ 位置的动能 $E_{\mathrm{k}A}$.
这样, \eqref{动能定理前身} 式就可以写成
\begin{equation}
    W = E_{\mathrm{k}B} - E_{\mathrm{k}A}
    \label{动能定理1}
\end{equation}

动能是标量, 单位与其他能量一样都是焦耳(J).动能和速度一样,是描述物体运动状态的重要物理量.

\subsection{动能定理}
合力对物体所做的功产生了什么效果?

具有动能的物体可以做功, 对物体做功也可以改变物体的动能.

\eqref{动能定理1} 式反映了合力对物体所做的功与物体动能变化之间的关系. 
这个关系虽然是从受恒力作用的物体做匀加速直线运动的过程推导得到的, 但是进
一步的理论推导可以证明, 这个关系对于变力做功和物体做曲线运动的过程依然成立.

只需要简单的微积分知识, 我们就可以利用直线运动中功的定义, 来推导直线运动中这个结论的正确性.
由于过程比较复杂, 这里不再赘述. 读者有兴趣可以参考大学物理教材.

\textbf{物体所受的合力在一个过程中所做的功, 等于物体在这个过程中动能的变化量.}
这个规律叫作\textbf{动能定理}. 它的表达式可以写成
\begin{empheq}[box=\fbox]{equation*}
	W = \Delta E_\mathrm{k} = E_\mathrm{k2} - E_\mathrm{k1}.
	\label{动能定理2}
\end{empheq}
式中$W$是合力做的功, $E_\mathrm{k2}$为末动能, $E_\mathrm{k1}$为初动能.

当 $W>0$ 时, $\Delta E_\mathrm{k}>0$, 表示合力做正功, 动能增加;
当 $W<0$ 时, $\Delta E_\mathrm{k}<0$, 表示合力做负功, 动能减小.

前面我们已经学习过一些求解变力做功的方法, 除此之外, 我们还可以利用动能定理求解变力做功. 利用动能定理,
我们只需要知道物体的初, 末动能就可以求解整个过程的做功情况, 而不需要了解运动过程的细节, 因此可以使
某些复杂问题简化.

\begin{example}
    将一个可视为质点的物体从距离水平地面高度为$h$处以恒定速率$v_0$斜向上
抛出,设抛出速度与水平方向的夹角为$\theta$, 不计空气阻力, 试分析:

(1) 物体质量$m$与落地速度$v$的关系;

(2) $\theta$与落地速度$v$的关系;

(3) $h$与落地速度$v$的关系;

(4) $\theta$与飞行时间$t$的关系;
\end{example}
\begin{solution}
    物体下落的过程中只受重力作用. 对物体应用动能定理, 有
$$mgh = \frac12mv^2 - \frac12mv_0^2.$$

因为$m$可以消去, 所以物体质量与落地速度无关. 事实上, 根据上式, 我们解出落地速度$$v = \sqrt{v_0^2 + 2gh},$$
可见, 落地速度$v$与物体的质量, 抛出时的倾角均无关; $h$越大, 落地速度$v$越大.

对物体的运动进行分解.物体上抛时, 竖直方向的分速度
\setlength{\abovedisplayskip}{0pt}
\setlength{\belowdisplayskip}{0pt}
$$v_y = v\sin \theta.$$ $\theta$越大, 则竖直方向的分速度越大, 由竖直上抛的知识可知, 物体的运动时间$t$越长.
\end{solution}

\section{应用:机车启动问题}

若质量为$m$的机车启动时做直线运动,受到的阻力$F_\mathrm{f}$恒定, 在某一时刻速率为$v$, 加速度为
$a$, 牵引力大小为$F$, $F$的瞬时功率为$P$, 则根据牛顿第二定律有
\begin{equation}
	F - F_\mathrm{f} = ma,
	\label{机车启动牛二律}
\end{equation}
根据(\ref{瞬时功率})式有
\begin{equation}
	P = Fv.
	\label{机车启动功率}
\end{equation}

以上两个方程是解决本节问题的关键.

\begin{figure}[bthp]
\centering

\subfloat[恒定功率的机车启动问题]{
	\includegraphics[width=0.3\textwidth]{pic/pic10.pdf}
    \label{机车启动1}
}
\subfloat[恒定加速度的机车启动问题]{
	\includegraphics[width=0.31\textwidth]{pic/pic11.pdf}
    \label{机车启动2}
}
	\caption{机车的启动时的$v$-$t$图像}
    \label{机车启动3}
    
\end{figure}

\subsection*{恒定功率的机车启动问题}

当牵引力做功的功率$P$恒定时,根据(\ref{机车启动功率})式,因为机车的速率$v$持续增大,
所以牵引力$F$持续减小. 由(\ref{机车启动牛二律})式可知, \textbf{机车的加速度$a$也持续减小}.

我们知道, 机车加速是由于机车有加速度, 当机车的加速度为0时,机车的速度达到最大值.此时有
\begin{equation}
	F_0 - F_\mathrm{f} = 0,
	\label{恒定功率启动的牛二律}
\end{equation}
\begin{equation*}
	P = F_0v_\mathrm{max}.
\end{equation*}
由此解得
\begin{equation}
	v_\mathrm{max} = \frac{P}{F_\mathrm{f}}.
	\label{最大速度}
\end{equation}
由以上分析可知, 当机车的牵引力$F$与所受阻力$F_\mathrm{f}$相等时, 机车速度达到最大值$v_\mathrm{max}$.

图\ref{机车启动1} 是质量为$m$的机车以恒定功率$P$启动的$v$-$t$图像. 在$OA$段, 机车的加速度逐渐减小;
在$A$点处, 机车的加速度降为0, 机车的牵引力$F$等于阻力$F_\mathrm{f}$.

如何求出机车的位移呢? 因为机车做的不是匀变速直线运动, 我们没有直接的公式可以计算;
$v$-$t$图线与坐标轴围成的图形也不规则, 不易计算面积. 不妨试试动能定理.
根据动能定理, 机车的牵引力做功与所受阻力做功的代数和, 等于机车动能的变化量, 即
$$Pt_0 - F_\mathrm{f}x = \frac12 m v_\mathrm{max}^2 - 0.$$
由此即可解出机车的位移$x$.

\subsection*{恒定加速度的机车启动问题}

当机车以恒定加速度$a$启动时, 机车的速率$v$持续增加. 因为合力$ma = F - F_\mathrm{f}$, 且
阻力$F_\mathrm{f}$是定值, 所以当加速度恒定时, 牵引力$F$也是恒定的. 根据(\ref{机车启动功率})式,
机车的功率持续增加. 由于机车有额定功率$P_\mathrm{max}$, 因此机车以恒定加速度
启动时, 最终一定会达到最大功率$P_\mathrm{max}$, \textbf{从而也会进入恒定功率的状态}.

图\ref{机车启动2} 是质量为$m$的机车以恒定加速度$a$启动的$v$-$t$图像, 直线$OA$是机车以加速度$a$做
匀加速直线运动的$v$-$t$图线, 其牵引力$F$恒定不变; 在$A$点处, 机车达到额定功率$P_\mathrm{max}$.
机车的匀加速运动结束于$A$点, 此时机车的速率为$v_1$, 类似(\ref{机车启动牛二律}) (\ref{机车启动功率})有
$$F_1 - F_\mathrm{f} = ma,$$ $$P_\mathrm{max} = F_1v_1.$$
由此解得$$v_1 = \frac{P_\mathrm{max}}{ma+F_\mathrm{f}}.$$
再根据$v_1 = at_1$, 可以求出$t_1$.

曲线$AB$是机车以额定功率$P_\mathrm{max}$继续运动的$v$-$t$图线, 因为汽车的速率继续增大, 根据(\ref{机车启动功率}),
机车的牵引力逐渐减小; 在点$B$处, 机车的加速度降为0, 机车的牵引力$F$等于阻力$F_\mathrm{f}$,
机车达到最大速度$v_\mathrm{max}$. 类似(\ref{最大速度}), 容易知道$$v_\mathrm{max} = \frac{P_\mathrm{max}}{F_\mathrm{f}}.$$

\section{机械能守恒定律}

我们在初中已经知道, 动能和势能(包括重力势能和
弹性势能)统称为\textbf{机械能}.动能和势
能是可以相互转化的, 如果只有它们相互转化, 尽管各自
的大小会变化, 但机械能的总和不变, 即机械能守恒.

那么, 在什么条件下才只有动能和势能的相互转化
呢? 或者说, 在什么条件下机械能守恒? 

\subsection{动能和势能的相互转化}

动能和势能的相互转化是日常生活中常见的现象.

重力势能可以与动能相互转化. 小朋友们荡秋千时, 
下降过程中重力做正功, 
他们的重力势能减少, 而速度逐渐变大, 动能增加
了, 这时重力势能转化成动能. 当他们上升时, 一个逆向
的能量转化过程就开始了: 重力做负功, 他们的重力势能
增加, 动能减少, 这时动能转化成重力势能. 可见, 
\textbf{重力势能与动能的互相转化是通过重力做功来实现的}.

弹性势能可以与动能相互转化. 网球触拍减速时, 克服
网线弹力做功, 网线的弹性势能增大, 网球的动能减少, 
这时动能转化为弹性势能. 当网球反弹加速时, 网线弹力
对网球做正功, 网线的弹性势能减少, 网球的动能增大, 
这时弹性势能转化为动能. 这说明, \textbf{弹性势能与动能的互相转化是通过弹力做功来实现的}.

从上述实例我们看到, \textbf{动能和势能\footnote{这里的势能包括重力势能和弹性势能.}之间的转化是通过
重力或弹力做功来实现的}. 重力或弹力做正功的过程就是
势能减少的过程, 但减少的势能是否一定转化为动能呢?
这是我们要讨论研究的问题.

\subsection{机械能守恒定律}
功与能量的转化密切相关, 我们从力做功的角度分析
能量转化过程中的定量关系. 

先来考虑只有重力做功的物体系统. 设物体在初位置的动能为$E_\mathrm{k1}$, 势能为$E_\mathrm{p1}$;
在末位置处的动能为$E_\mathrm{k2}$, 势能为$E_\mathrm{p2}$, 则由重力做的功与重力势能的关系和动能定理有
$$W_\text{G} = E_\mathrm{p1} - E_\mathrm{p2},$$
$$W_\text{合力} = E_\mathrm{k2} - E_\mathrm{k1}.$$

如果只有重力对物体做功, 则$W_\text{合力} = W_\text{G}$, 由此可得
\begin{equation}
    E_\mathrm{p1} - E_\mathrm{p2} = E_\mathrm{k2} - E_\mathrm{k1},
    \label{机械能守恒定律推导1}
\end{equation}
整理可得\begin{equation}
    E_\mathrm{k1}+E_\mathrm{p1} = E_\mathrm{k2} + E_\mathrm{p2}.
    \label{机械能守恒定律推导2}
\end{equation}

\eqref{机械能守恒定律推导1} 式表明, 系统势能的减少等于动能的增加, 即
$$-\Delta E_\text{p} = \Delta E_\text{k};$$
而 \eqref{机械能守恒定律推导2} 式表明, 系统不同时刻的动能和势能之和, 即总机械能保持不变.
这说明, \textbf{在只有重力做功的过程中, 系统的机械能守恒}. 

只有弹力做功的过程与此类似, 系统的动能与弹性势
能之间相互转化而总机械能守恒.

\textbf{在只有重力和弹力做功的物体系统中, 系统的动能和势能可以互相转化,而它们的总和保持不变}. 
这就是\textbf{机械能守恒定律}.

需要注意的是, 在多个物体组成的物体系统中, 即使该系统不受其他外力作用,
单个物体的机械能也不一定守恒, 但该系统内所有物体的的机械能总和一定守恒.

如果一个物体系统除了重力和系统内力(即系统内各物体间的相互作用)外还有其他力做功, 那么该系统的机械能不守恒,
并且``其他力''做的功等于该系统机械能的变化量.如果用$\Delta E_\mathrm{m}$表示机械能的变化量, 用$W_\text{其他}$
表示``其他力''做功,那么上述内容可以表示为$$W_\text{其他}= \Delta E_\mathrm{m}.$$

\begin{example}
    用轻绳拉着小球在竖直面内做半径为$R$的圆周运动, 轻绳的另一端固定在圆心$O$处. 
    小球的质量为$m$, 在最低点处的速率为$v_0$, 重力加速度为$g$.
    某一时刻的摆角为$\theta$. 
    利用机械能守恒定律, 试推导小球的速率$v$与$\theta$的关系.
\end{example}
\begin{solution}
    在小球做圆周运动的过程中, 只有重力对小球做功, 所以小球的机械能守恒.
    因此小球在任意时刻的机械能都等于小球在最低点的机械能.

    以最低点所在水平面为重力势能的参考平面, 则由机械能守恒定律有
    $$\frac12 mv_0^2 = \frac12 mv^2+mgh.$$
    其中$h = R - R\cos\theta.$
    
    由此解得$$v = \sqrt{v_0^2-2gR(1-\cos\theta)}.$$
\end{solution}

\begin{example}\label{例题刚性轻杆}
    如图\ref{5.6-2} 所示, A, B 两小球分别固定在一刚性轻杆的两端, 两球球心间相距$L$, 两球质量分别为$m_A = 4.0\ \mathrm{kg}$,
$m_B = 1.0\ \mathrm{kg}$,
杆上距A球球心$0.4L$处有一水平轴$O$, 杆可绕轴无摩擦转动. 现先使杆保持水平, 然后从静止释放. 当杆转到竖直位置时, 求:

(1)当杆转到竖直位置时两球的速度$v_\mathrm{A}$, $v_\mathrm{B}$;

(2)杆对A球的作用力$F$;

(3)转动过程中杆对A球做的功$W$.
\end{example}

\begin{figure}
    \centering
	\includegraphics[width=0.3\textwidth]{pic/exam_5.6-2.pdf}
    \caption{例 \ref{例题刚性轻杆} 题图}
	\label{5.6-2}
\end{figure}

\textbf{解}\bre (1) 由圆周运动的知识可知, A, B 两小球运动的角速度相等.设它们到轴$O$的距离分别为$L_\mathrm{A}$,
$L_\mathrm{B}$, 根据线速度与角速度的关系有$$\frac{v_\mathrm{A}}{v_\mathrm{B}} = \frac{L_\mathrm{A}}{L_\mathrm{B}}.$$
\setlength{\abovedisplayskip}{5pt}
\setlength{\belowdisplayskip}{10pt}

取杆的初位置为重力势能的参考平面, 以两球组成的系统为研究对象, 由机械能守恒定律得
$$-m_\mathrm{A}gL_\mathrm{A}+m_\mathrm{B}gL_\mathrm{B} = \frac12 m_\mathrm{A} v_\mathrm{A}^2 + \frac12 m_\mathrm{B}v_\mathrm{B}^2.$$

联立解得$v_\mathrm{A} = \displaystyle\frac{4\sqrt{5}}{5}\ \mathrm{m/s}$
, $v_\mathrm{B} = \displaystyle\frac{6\sqrt{5}}{5}\ \mathrm{m/s}$.

\setlength{\abovedisplayskip}{3pt}
\setlength{\belowdisplayskip}{3pt}

(2) 当杆转到竖直位置时,对小球A应用牛顿第二定律得
$$F - m_\mathrm{A}g = m_\mathrm{A}\frac{v_\mathrm{A}^2}{L_\mathrm{A}}.$$
由此可得$F= 72$\ N.

(3) 当杆转到竖直位置时,对A应用动能定理得$$m_\mathrm{A}gL_\mathrm{A} + W = \frac12 m_\mathrm{A}v^2_\mathrm{A}.$$
由此解得$W = -9.6$\ J.

\section{能量的守恒与转化}

不同形式的能量之间的转化通过做功来实现, 即做功的过程
是能量转化的过程. 物体做了多少功, 就有多少能量发生了转化, 而
功的正负表示能量的转化方向(即增加与减少).因此, \textbf{功是能量转化的量度}.

\subsection{摩擦生热}

两个粗糙的物体发生$\Delta x$的相对滑动, 如果它们之间的滑动摩擦力做功为$W_\mathrm{f}$, 那么
该过程产生的热量为$$Q = W_\mathrm{f} \cdot \Delta x.$$

因为摩擦力是重力和弹力之外的``其他力'', 所以当有摩擦力做功时, 机械能不守恒. 事实上, 机械能
转化为了热量$Q$, 这就是\textbf{摩擦生热}.

\subsection{能量守恒}

初中时我们就知道能量是守恒的, 并且知道:

\textbf{能量守恒定律\bre 能量既不会凭空产生, 也不会凭空消失, 它只会从一种形式转化为另一种形式,
或者从一个物体转移到另一个物体, 而在转移或转化的过程中, 能量的总量保持不变.}

能量守恒定律告诉我们:\textbf{某种形式的能量减少, 一定存在其他形式的能量增加; 某个物体的
	能量减少, 一定存在其他物体的能量增加. 增加量和减少量一定相等.}

这启示我们可以利用初时刻总能量等于末时刻总能量列式, 或者通过能量的增加量等于能量的减少量
列式. 应用能量守恒定律的关键是分析清楚系统中有哪几种形式的能量, 发生了哪些转化或转移过程.\\

把一个物块轻放在恒定速度运转的水平传送带上, 假设传送带足够长, 物块将持续加速, 最终与传送带共速,
它的动能是从哪里来的呢? 如果是倾斜的传送带, 物块从传送带的最低点升至最高点, 那么物块的重力势能又
是从何而来的呢?

事实上, 物块随传送带运动时传送带的耗电量, 是比传送带自己转动时的耗电量更多的. 传送带自己转动时,电能转化为了
焦耳热和传送带机械内部的摩擦生热;当物块被放在传送带上时, 电能还转化为物块的机械能以及物块与传送带的摩擦生热.
因此, 根据能量守恒定律, 传送带多消耗的电能为
$$\Delta E_\text{电} = Q + \Delta E_\text{物块},$$
式中$\Delta E_\text{电}$是传送带多消耗的电能, $\Delta E_\text{物块}$是
物块机械能的变化量, $Q$是物块与传送带的摩擦生热, 等于物块受到的摩擦力乘以物块与传送带的相对位移.

\begin{example}\label{传送带问题}
    如图\ref{fic14} 所示, 倾角$\theta = 37^{\circ}$的斜面与足够长的水平传送带相连接(接口处平滑)
, 传送带逆时针转动, 速度$v = 6\ \mathrm{m/s}$.某时刻质量$m = 1\ \mathrm{kg}$的物块(可视为质点)
从斜面上距离传送带水平面高$H = 3\ \mathrm{m}$处的$A$点由静止滑下.物块与传送带和物块与斜面间的动摩擦
因数均为$\mu = 0.3$. 已知$\sin 37^{\circ} = 0.6, \cos 37^{\circ} = 0.8$, 重力加速度$g$取$10\ \mathrm{m/s^2}$.
求物块在传送带上第一次相对地面静止时, 由于物块滑上传送带而使传送带多做的功$\Delta W$.
\end{example}

\begin{figure}
    \centering
	\includegraphics[width=0.35\textwidth]{pic/exam_5-3.pdf}
    \caption{例\ref{传送带问题} 题图}
	\label{fic14}
\end{figure}
\textbf{解}\ \ \ 设物块运动到$B$点时速度为$v_B$, 由动能定理得
\begin{equation*}
	mgH - \mu mg\cos \theta \ \frac{H}{\sin \theta} = \frac12 mv^2_B-0.
	\tag{i}
	\setlength{\abovedisplayskip}{5pt}
	\setlength{\belowdisplayskip}{10pt}
\end{equation*}

根据能量守恒定律, 从物块到达$B$点到物块第一次相对地面静止, 物块的动能$E_\mathrm{k} = \displaystyle\frac12 m v_B^2$
以及传送带多做的功$\Delta W$,
都转化为物块与传送带的摩擦生热$Q = \mu mg\Delta s$, 即
\begin{equation*}
	\frac12 m v_B^2 + \Delta W = \mu mg \Delta s,
	\setlength{\abovedisplayskip}{5pt}
	\setlength{\belowdisplayskip}{5pt}
	\tag{ii}
\end{equation*}
其中$\Delta s$是物块与传送带的相对位移.

物块在传送带上做匀减速运动,根据牛顿第二定律和运动学公式有 
\setlength{\abovedisplayskip}{0pt}
\setlength{\belowdisplayskip}{0pt}
\begin{equation*}
	-\mu mg = ma,
	\tag{iii}
\end{equation*}
\begin{equation*}
	0 = v_B - at.
	\tag{iv}
\end{equation*}

根据匀变速直线运动位移与时间的关系, 物块相对地面的位移大小为
\begin{equation*}
	s_1 = v_Bt-\frac12 at^2,
	\tag{v}
\end{equation*}
传送带在这段过程中相对地面的位移大小为
\begin{equation*}
	s_2 = vt.
	\tag{vi}
\end{equation*}
因为物块和传送带在这段过程中是反向运动的, 所以它们的相对位移是
\begin{equation*}
	\Delta s = s_1 + s_2.
	\tag{vii}
\end{equation*}

由此解出$\Delta W = 36\ \mathrm{J}$.


\chapter{动量守恒定律}
\section{动量}

\subsection{碰撞中的不变量}

做相对运动的两个或几个物体相遇并发生相互作用, 在\textbf{很短}的时间内, 
它们的运动状态会发生显著变化, 这一过程叫作\textbf{碰撞}.大到宇宙天, 小
到微观粒子, 都时常发生碰撞. 这些碰撞现象
的背后蕴藏着什么样的规律呢? 荷兰物理学家惠更斯做了下面这个实验.

用两根长度相同的细线, 分别悬挂两个完全
相同的钢球 A, B, 且两球并排放置.拉起 A 球, 
然后放开, 该球与静止的 B 球发生碰撞.可以看
到, 碰撞后 A 球停止运动而静止, B 球开始运动, 
最终摆到和 A 球被拉起时同样的高度.

碰撞后, A 球的速度大小不变地传给了 B 球. 所有的碰撞都有这样的特点吗?

将上面实验中的 A 球换成大小相同的 C 球, 
使 C 球质量大于 B 球质量, 用手拉起 C 球至某一高度后放
开, 撞击静止的 B 球.我们可以看到, 碰撞后 B 球获得较
大的速度, 摆起的最大高度大于 C 球被拉起时的高度.

可以看出, 质量大的 C 球与质量小的 B 球碰撞
后, B 球得到的速度比 C 球碰撞前的速度大, 两球碰撞前
后的速度之和并不相等.

经过多次实验验证, 我们发现: 碰撞前后, 
\textbf{两个小球的动能之和并不相等\footnote{碰撞时小球会发热, 这说明有一部分能量转化为了内能.}, 但是质量与速度的乘积之和却基本不变.}

上面的实验提示我们, 质量与速度的乘积
$mv$ 这个物理量具有特别的意义.

\subsection{动量}

物理学中把质量和速度的乘积$mv$叫作物体的\textbf{动量}, 用字母$p$表示, 即
\begin{empheq}[box=\fbox]{equation*}
    p = mv.
\end{empheq}

速度是矢量, 动量也是矢量, 动量的方向与速度的方
向相同. 在国际单位制中, 动量的单位是\textbf{千克米每秒}, 
符号是$\mathrm{kg\cdot m/s}$. 

由于速度反映物体的运动状态, 
所以动量是状态量. 我们说物体在某一时刻的动量, 
是指物体的质量$m$与它在这一时刻速度$v$的乘积.
这里$v$是物体在这一时刻的瞬时速度, 而不是某段时间内的平均速度.

物体在一个过程中\textbf{动量的变化量}是
它的末动量与初动量的矢量差, 也叫作\textbf{动量的增量}. 如果初末动量在一条直线上, 即运动是一维的, 那么选定正方向后,
动量的方向可以用正负号表示, 从而将矢量运算转化为代数运算, 即
$$\Delta p = p^{\prime}-p = m\Delta v = mv^{\prime} - mv.$$
此时动量的增量$\Delta p$为负表示动量向负方向增加.

\section{动量定理}

\subsection{冲量}
物理学中把力的作用对时间的累积效果叫作\textbf{冲量}, 用$I$表示. 如果力$F$与时间$t$的函数关系是$F(t)$, 
则称满足$$I^{\prime}(t) = F(t)$$的$I = I(t)$为力$F$作用时间$t$的冲量.

高中阶段我们研究的力大多是恒力. 当$F(t)$恒为常数$F$时, 力的冲量就是力与力作用时间的乘积, 即\footnote{之后没有特别说明, 我们所说的冲量都是指恒力的冲量.}
\begin{empheq}[box=\fbox]{equation}
    I = Ft.
    \label{恒力冲量}
\end{empheq}

冲量是矢量, 冲量的方向与力的方向相同. 冲量的单位是\textbf{牛秒}($\mathrm{N\cdot s}$).事实上, 它与动量的量纲相同, 并且
$$1\ \mathrm{N\cdot s} = 1\ \mathrm{kg\cdot m/s}.$$

\subsection{动量定理}

为什么要引入冲量的概念呢? 冲量和我们前面学过的动量有什么关系呢?
下面我们就来研究这个问题.

根据牛顿第二定律, 恒力冲量的表达式 \eqref{恒力冲量} 可以写为
$$I = mat.$$
其中$a$是物体在恒力$F$作用下的加速度, $m$是物体的质量, $t$是力的作用时间.

设物体在初始时刻的速度为$v$, 经过时间$t$后, 物体的速度变为$v^{\prime}$.
考虑到$v^{\prime}-v = at$,\footnote{这是速度的增量与时间间隔的关系, \eqref{速度增量与时间间隔的关系} 式给出了这个关系.} 上式可以写成
$$I = m(v^{\prime}-v).$$
展开后就得到
\begin{empheq}[box=\fbox]{equation}
    I = mv^{\prime}-mv.
    \label{动量定理1}
\end{empheq}
可以发现, 上式的右边正是末时刻与初始时刻物体的动量之差. 这个结论不仅对于恒力成立, 对于变力也同样适用,
并且利用简单的微积分知识就可以证明.

\textbf{物体在一个过程中所受力的冲量, 等于它在这个过程始末的动量变化量}, 这就是\textbf{动量定理}.如果用
$\Delta p$表示物体动量的变化量, 则动量定理可以写成
\begin{empheq}[box=\fbox]{equation*}
    I = \Delta p.
\end{empheq}

值得注意的是, 动量定理中所提到的``力的冲量''指的是物体所受合力的冲量.除此之外, 物体受到的任何一个分力
都有冲量, 比如质量为$m$的物体重力的冲量为$$I_\mathrm{G} = mg\Delta t.$$
其中$g$是重力加速度, $\Delta t$是重力的作用时间.然而重力无时无刻不作用在物体上, 所以我们可以
取任意的$\Delta t$, 并说$mg\Delta t$是物体的重力在时间$\Delta t$内的冲量.

动量定理的表达式还可以写成$$F = \frac{\Delta p}{\Delta t},$$
上式表明, \textbf{物体所受的合力等于物体动量随时间的变化率}\footnote{实际上, 这个式子叫作牛顿第二定律的动量表述形式: 物体的动量随时间的变化率与受力成正比.
当物体的质量不变时, 才推得了我们之前学习的牛顿第二定律.}.

一定质量的物体, 改变一定的速度, 动量的改变量是一定的. 此时$$F\propto \displaystyle\frac{1}{\Delta t},$$
即力的作用时间越长, 力越小. 物体相互碰撞时出现的力称为\textbf{冲力}. 为了安全, 我们通常延长冲力的作用时间
从而减小冲力的大小, 这就是\textbf{``缓冲''}.易碎物品运输时要用柔软材料包装, 跳高时运动员
要落在软垫上, 就是这个道理.

% \begin{wrapfigure}{o}{4cm}
%     \flushright %右侧对齐
%     \includegraphics[width=0.25\textwidth]{路径}
%     \label{fig}
% \end{wrapfigure}
\setlength{\abovedisplayskip}{3pt}
\setlength{\belowdisplayskip}{3pt}

\begin{example}\label{拉力水平}
    一个质量为$m = 2\ \mathrm{kg}$的物体,在$F_1 = 8\ \rm{N}$的水平拉力作用下,
    从静止开始沿水平面运动了$t_1 = 5\ \rm{s}$,然后拉力减小为$F_2=5\ \rm{N}$,方向不变,物体又运动了
    $t_2=4\ \rm{s}$后撤去外力,物体再经过$t_3=6\ \rm{s}$停下来. 试求物体在水平面上所受的摩擦力.
\end{example} 

\textbf{解}\bre 设拉力的方向为正方向, 则摩擦力沿负方向.
注意到全程中摩擦力的大小$F_\mathrm{f}$是恒定不变的, 对全过程应用动量定理得
$$F_1t_1 + F_2t_2 - F_\mathrm{f}(t_1 + t_2 + t_3) = 0-0.$$
由此解得$F_\mathrm{f} = 4\ \mathrm{N}$.

需要注意, 例\ref{拉力水平} 中拉力都是水平的, 如果拉力是倾斜的, 并且在整个过程中
大小发生了变化, 要注意摩擦力也会随之变化.


\begin{example}\label{平底鞋起跳}
    某人所受重力为$G$, 穿着平底鞋起跳, 竖直着地过程中,
    双脚与地面间的作用时间为$t$, 地面对他的平均冲击力大小为$4G$. 若他穿上带有减震气垫
    的鞋起跳, 以与第一次相同的速度着地时, 双脚与地面间的作用时间变为$2.5t$,
    求地面对他的平均冲击力$\bar{F}$变为多少.
\end{example}

\textbf{解}\bre 由两过程的末速度相同, 知两过程的动量变化量相同. 由动量定理知
两过程的合力冲量相同, 即
$$(4G-G)t = (\bar{F} - G)2.5t.$$
由此解得$\bar{F} = 2.2G.$

值得强调的一点是, 在动量定理$I = \Delta p$中, $I$是合力的冲量, 而不是物体受到的某一个分力的冲量.
在例\ref{平底鞋起跳} 中, 要注意不要把重力当作合力计算.

\subsection{流体的冲力}
\setlength{\abovedisplayskip}{3pt}
\setlength{\belowdisplayskip}{3pt}
流体以速度$v$冲击到某一点上, 根据动量定理, 流体的冲力$F$满足$$Ft = mv.$$
其中$m$为流体质量, $t$为流体某一点的运动时间.

因为流体的体积$V = vt\cdot S$($S$为流体截面), 所以
流体的质量$m = \rho V = \rho\cdot vt \cdot S$($\rho$为流体的密度).因此
$$Ft = \rho\cdot vt \cdot S \cdot v.$$

由此得到流体的冲力公式
\begin{empheq}[box=\fbox]{equation*}
    F = \rho Sv^2.
\end{empheq}
式中$F$为流体的冲力, $\rho$为流体密度, $S$为流体截面, $v$为流体速度.

\section{动量守恒定律}

前面我们学习了动量定理. 恒力$F$作用在质量为$m$的物体上, 作用时间为$t$,
物体在这个过程始末的速度分别为$v$, $v^{\prime}$, 则有
$$Ft = mv^{\prime} - mv.$$

当外力$F$为0时, 有$$mv^{\prime} = mv.$$
即物体的动量保持不变, 我们可以说这个物体的动量是守恒的.这个结论我们很好理解, 应用牛顿第一定律就可以解释.

上面的讨论是对单个物体而言的, 那么对多个物体有没有``动量守恒''的规律呢?
在研究这个问题之前, 我们先介绍一个概念.

当我们对两个(或多个)物体进行研究时, 它们组成一个\textbf{力学系统}.
系统内两个(或多个)物体的相互作用力称为\textbf{内力}, 系统以外的物体对系统的作用力称为\textbf{外力}.

下面我们结合一个具体情境来研究, 系统所受合外力为0时, 系统的总动量如何变化.

质量为$m_1$和$m_2$的两个小球A和B, 在光滑水平面上沿同一方向
做匀速直线运动, 速度分别是$v_1,\ v_2$, 且$v_2>v_1$. 经过一段时间后,
B追上了A, 两球发生碰撞, 碰撞后A, B的速度分别变为$v_1^{\prime},\ v_2^{\prime}$.

设碰撞过程中, B对A的作用力为$F_1$, A对B的作用力为$F_2$, 由牛顿第三定律可知,
$F_1$与$F_2$大小相等, 方向相反, 即$F_1 = -F_2$.

对A应用动量定理得$$F_1t = m_1v_1^{\prime} - m_1v_1.$$

对B应用动能定理得$$F_2t = m_2v_2^{\prime} - m_2v_2.$$

联立以上三个式子, 得
\begin{equation}
    \label{动量守恒1}
    m_1v_1^{\prime}-m_1v_1 = -(m_2v_2^{\prime} - m_2v_2).
\end{equation}
整理得
\begin{equation}
    \label{动量守恒2}
    m_1v_1 + m_2v_2 = m_1v_1^{\prime} + m_2v_2^{\prime}.
\end{equation}
我们发现, 两个球碰撞前后, 系统的总动量是不变的, 也可以说系统的动量是守恒的.
由此, 我们得到如下结论.

\textbf{如果一个系统不受外力或所受外力的矢量和为0, 那么这个系统的总动量保持不变,
    即系统的动量守恒.}这就是\textbf{动量守恒定律}.

\eqref{动量守恒1} 和 \eqref{动量守恒2} 分别是动量守恒定律的两种表达形式, 即
\begin{empheq}[box=\fbox]{equation*}
    \Delta p_1 = -\Delta p_2.
\end{empheq}
其中$\Delta p_1$, $\Delta p_2$分别为两个物体在一个过程中的动量变化量,
即系统内一个物体动量的增加量等于另一个物体动量的减少量.

或者
\begin{empheq}[box=\fbox]{equation*}
    p = p^{\prime}.
\end{empheq}
其中$p$, $p^{\prime}$分别为系统的初动量, 末动量, 即系统的总动量不变. 特别地, 
当系统中有两个物体时, 上式可以写成
\begin{empheq}[box=\fbox]{equation*}
    m_1v_1 + m_2v_2 = m_1v_1^{\prime} + m_2v_2^{\prime}.
\end{empheq}

{\kaishu 如果把系统中的所有物体看成一个整体, 动量守恒定律可以写成什么形式呢?}

对于整个系统来说, 因为系统的总动量$p = m_\text{系统}v_\text{质心}$,
所以
$$m_\text{系统}v_\text{质心} = m_\text{系统}v_\text{质心}^{\prime},$$
由此可得$$v_\text{质心} = v_\text{质心}^{\prime},$$ 即\textbf{系统质心速度不变}.
因此系统动量守恒等价于系统质心的速度守恒.

随着物理学的发展, 人们认识到动量守恒定律具有普
适性, 它比牛顿运动定律适用的范围要广得多. 无论在微
观, 宏观还是高速领域, 无论是何种形式的相互作用, 只
要系统所受的合外力为0, 动量守恒定律都是适用的. 动
量守恒定律是自然界中最普遍, 最基本的规律之一. 如果
系统内相互作用的物体不只是两个, 而是三个或者多个, 
只要系统不受外力或所受的合外力为0, 系统的动量也是
守恒的. 

应用动量守恒定律时, 有几点注意事项:
\begin{enumerate}
    \item 动量守恒指的是总动量在相互作用的过程中时刻守恒, 而不是只有始末状态才守恒. 因此在实际使用时, 可以任选两个状态来列方程.
    \item 系统的动量守恒, 不能说明单体的动量守恒. 在应用动量守恒定律时, 一定要明确是哪些物体构成的系统.
    \item 动量守恒定律的表达式是矢量式, 对于一维的问题可以先规定正方向, 再进行计算.
    \item 动量与参考系的选择有关, 一般以地面为参考系.
    \item 动量守恒不能说明机械能守恒, 机械能守恒也不能说明动量守恒, 它们的判定条件不同, 没有必然联系.
    \item 动量守恒定律不但适用于宏观低速运动的物体, 而且还适用于微观高速运动的粒子. 它与牛顿运动定律相比, 适用范围要广泛得多. 并且动量守恒定律不需要考虑物体间的作用细节, 在解决问题上比牛顿运动定律更简捷.
\end{enumerate}

\subsection*{动量守恒的条件}

虽然动量守恒定律要求系统所受的合外力为0, 但是在以下两种情况下, 我们也可以用
动量守恒定律解决问题:
\begin{enumerate}
    \item 系统所受合外力不为0, 但在系统各部分相互作用的瞬时过程中, 系统内力
          远远大于外力, 外力相对来说可以忽略不计, 这时系统的动量近似守恒. 例如爆炸, 反冲等过程.
    \item 系统所受合外力不为0, 但系统在某一方向上不受外力或该方向上外力之和为0.
          则系统在该方向上的动量守恒.例如物块在斜面上下滑的问题. 在分析这类问题时, 要注意先把速度分解成沿该方向的速度再代入计算.
\end{enumerate}

\begin{example}
    两个木块A, B置于光滑水平面上, 它们的质量$m_\mathrm{A} = m_\mathrm{B} = 2$\ kg.
B与一轻质弹簧的一端相连,
弹簧的另一端固定在墙上. 当A以$v = 4$\ m/s的速度向B撞击时, 由于有橡皮泥(质量不计)而粘在
一起运动. 求弹簧被压缩到最短时, 弹簧的弹性势能$E_\mathrm{p}$.
\end{example}

\textbf{解}\bre 在弹簧压缩前, 两木块的动量守恒, 并且最终具有相同的速度$v_\text{共}$,
根据动量守恒定律有$$m_\mathrm{A}v + 0 = (m_\mathrm{A}+m_\mathrm{B})v_\text{共}.$$

弹簧压缩的过程中, 根据能量守恒定律, 两物体的动能转化为弹簧的弹性势能, 即
$$\frac12 (m_\mathrm{A}+m_\mathrm{B}) v_\text{共}^2 = E_\mathrm{p}.$$

联立以上两式, 代入数据可得$E_\mathrm{p} = 8$\ J.

\section{反冲}
系统在内力的相互作用下, 当一部分向某一方向运动时, 剩余部分
将向相反方向运动, 这种现象叫作\textbf{反冲}.

发射炮弹时, 炮弹从炮筒中飞出, 炮身则向后退, 
这就是反冲现象. 射击前, 炮弹静止在炮筒中, 它们的总动量为
0. 炮弹射出后以很大的速度向前运动, 根据动量守恒定
律, 炮身必将向后运动. 只是由于炮身的质量远大于炮弹
的质量, 所以炮身向后的速度很小.

用枪射击时, 子弹向前飞去, 枪身发生反冲向后运动. 枪身的反冲会影
响射击的准确性, 所以用步枪射击时要把枪身抵在肩部, 
以减少反冲的影响.

以发生反冲的系统为研究对象, 系统两部分间的相互作用力是内力,
在系统外力(如重力, 空气阻力等)可以忽略的情况下, 我们知道,
系统的动量守恒.

爆炸也是反冲运动的一种. 爆炸过程中,
系统内部在极短时间内释放出大量的能量, 内力远远大于外力,
因此, 对整个系统而言, 我们可以利用动量守恒来解决问题.

在光滑水平面上, 质量为$m_1$, $m_2$的两个物体紧靠在一起, 它们之间有少许炸药(质量不计), 炸药爆炸后,
两个物体分开而向相反的方向运动, 速度大小分别为$v_1$, $v_2$. 以
$v_1$的方向为正方向, 根据动量守恒, 有
\begin{equation}
    0 = m_1v_1 - m_2v_2.
    \label{爆炸1}
\end{equation}
由此可得$$\frac{v_1}{v_2} = \frac{m_2}{m_1}.$$
这就是说, \textbf{原本静止的系统因爆炸而分成两部分, 这两部分的速度大小之比是它们质量比的反比.}

研究完速度, 我们再来研究动能和动量. 根据 \eqref{爆炸1}, 显然这两个物体的动量相等, 那它们的动能又有什么关系呢?

设这两个物体的动能分别为$E_\mathrm{k1}$, $E_\mathrm{k2}$, 则
$$\frac{E_\mathrm{k1}}{E_\mathrm{k2}} = \frac{\frac12 m_1v_1^2}{\frac12m_2v_2^2} = \frac{m_1v_1^2}{m_2v_2^2} = \frac{v_1}{v_2}.$$
即它们的动能之比等于速度之比.

当爆炸产生的化学能全部转化为动能时, 则
$$E_\text{化学} = \Delta E_\mathrm{k} = \left(\frac12m_1v_1^2 + \frac12m_2v_2^2\right)-0.$$
当转化比为$\eta$时, 则
$$\eta E_\text{化学} = \Delta E_\mathrm{k} = \left(\frac12m_1v_1^2 + \frac12m_2v_2^2\right)-0.$$

\begin{example}
    在粗糙水平面上, 质量为$m_1$, $m_2$的两个物体紧靠在一起,
它们之间有少许炸药(质量不计), 炸药爆炸后,
两个物体分别滑动$x_1$, $x_2$的距离而停止,
两物体与水平面间的动摩擦因数均为$\mu$.
求两物体的质量比, 以及爆炸后瞬间的速度大小之比,动能大小之比.
\end{example}

\textbf{分析}\bre 爆炸过程中产生的内力极大, 时间极短, 摩擦力相对来说可以忽略不计,
因此系统的动量可以看作是守恒的.

\textbf{解}\bre 设爆炸后瞬间两物体的速度大小分别为$v_1$, $v_2$.
炸药爆炸后, 分析两物体滑行的过程, 易知两个物体的
加速度恒定且均为$a = \mu g$. 由匀变速直线运动速度与位移的关系可知
$$0 - v_1^2 = -2ax_1,$$ $$0 - v_2^2 = -2ax_2.$$
可得两物体爆炸后瞬间的速度大小之比$v_1:v_2 = \sqrt{x_1}:\sqrt{x_2}$.

炸药爆炸过程中, 取$v_1$的方向为正方向, 应用动量守恒定律得
$$0 = m_1v_1 - m_2v_2.$$
解得$m_1:m_2 = v_2:v_1 = \sqrt{x_2}:\sqrt{x_1}$.

爆炸后两物体的动能之比为
$$\frac{E_\mathrm{k1}}{E_\mathrm{k2}} = \frac{\frac12 m_1v_1^2}{\frac12m_2v_2^2}=\frac{m_1v_1\cdot v_1}{m_2v_2\cdot v_2}.$$
由上可知$m_1v_1 = m_2v_2$, 所以$$\frac{E_\mathrm{k1}}{E_\mathrm{k2}} =\frac{v_1}{v_2} = \frac{\sqrt{x_2}}{\sqrt{x_1}}.$$

\section{碰撞}

碰撞是两物体间极短的相互作用. 发生碰撞的系统所受合外力为0, 因此系统的
动量守恒, 也就是\textbf{系统质心的速度守恒}.

\subparagraph{完全非弹性碰撞} 在非弹性碰撞过程中, 物体往往会发生形变, 还会发声发热.
因此, 在非弹性碰撞过程中会有动能损失, 转化为其他形式的能, 即动能不守恒.

系统碰撞后, 物体结合在一起 (即共速), 此时动能损失最大. 这种碰撞叫作\textbf{完全非弹性碰撞}.

质量为$m_1$, $m_2$, 初速度为$v_1$, $v_2$的两个物体发生完全非弹性碰撞,
系统亏损的动能为$$\Delta E_\mathrm{k} = \left(\frac12 m_1 v_1^{\prime 2} + \frac12 m_2 v_2^{\prime 2}\right) - \left(\frac12 m_1 v_1^2 + \frac12 m_2 v_2^2\right).$$
系统的动量守恒, 有$$m_1v_1 + m_2v_2 = (m_1+m_2)v_\text{共}.$$

由上式可知, 它们的末速度$v_\text{共}$可以用加权平均数表示
\begin{equation}
    v_\text{共} = \frac{m_1v_1 + m_2v_2}{m_1+m_2}.
\end{equation}
事实上, 这也是系统质心速度的公式.

\subparagraph{完全弹性碰撞} 在理想情况下, 物体碰撞后, 形变能够完全恢复, 并且不发声发热.
因此系统没有动能损失, 这种碰撞称为\textbf{完全弹性碰撞}.

质量为$m_1$, $m_2$, 初速度为$v_1$, $v_2$的两个物体发生完全非弹性碰撞,
根据动量守恒, 动能守恒分别有
$$m_1v_1 + m_2v_2 = m_1v_1^{\prime} + m_2v_2^{\prime}, $$
$$\displaystyle\frac12 m_1 v_1^2 + \displaystyle\frac12 m_2 v_2^2 = \displaystyle\frac12 m_1 v_1^{\prime 2} + \displaystyle\frac12 m_2 v_2^{\prime 2}.$$


特别地, 当两物体质量相同时, 即$m_1 = m_2$时, 容易得到$$v_1 + v_2 = v_1^{\prime}+ v_2^{\prime},$$
$$v_1^2 + v_2^2 = v_1^{\prime 2}+ v_2^{\prime 2}.$$
解得$v_1^{\prime} = v_2$, $v_2^{\prime} = v_1$, 即\textbf{两物体的速度互换}.

\subparagraph{部分弹性碰撞}
\section{应用:板块模型}

\chapter{机械振动}

\section{简谐运动}
我们把物体或物体的一部分在一个位置附近的往复运动, 
称为\textbf{机械振动}, 简称\textbf{振动}. 

把一个有孔的小球连接在弹簧的一端, 弹簧的另一端固定, 小球和弹簧套在光滑的杆上, 能够自由滑动. 弹簧的质量可以忽略, 小球运动时的空气阻力也可以忽略. 


小球静止时所受合力为0, 处于\textbf{平衡位置}. 向一侧拉动小球, 然后放开, 它就在平衡位置附近运动起来. 这样的理想化模型称为\textbf{弹簧振子}. 

由于没有阻力, 小球在运动过程中没有机械能的损失, 因而能够在平衡位置附近做周期性的往复运动, 即周期性的振动. 

\subsection{简谐运动}

为了研究弹簧振子的运动规律, 我们利用频闪照相机得到它的位移—时间图像, 即$x$-$t$图像, 也称为\textbf{振动图像}. 

我们发现, 小球位移与时间的关系可以用正弦函数来表示, 并可通过实验进行证明. 弹簧振子的这种运动是最简单的振动, 我们把这种运动叫作\textbf{简谐运动}. 

\textbf{如果物体的位移与时间的关系遵从正弦函数的规律, 即它的振动图像($x$-$t$图像)是一条正弦曲线, 这样的振动就是简谐运动. }

因此, 位移$x$关于时间$t$的函数表达式可以写为
\begin{equation}
    x = x(t) = A\sin(\omega t+\varphi_0),
    \label{位移函数}
\end{equation}
其中$A$, $\omega$, $\varphi_0$是参量, 下面我们就来介绍它们的物理意义.

\subsection{振幅}

因为$\sin(\omega t+\varphi_0)\in [-1,1]$, 所以$x\in [-A,A]$.
即$$0 \leqslant |x|\leqslant A.$$
可以看出, $A$是小球离开平衡位置的最大距离, 称为振幅. 振幅是表示物体振动幅度的物理量. 如果记平衡位置为$O$点, $M$和$N$分别是右端和左端的最远位置,  那么$$|OM|=|ON|=\frac12|MN| = A.$$振幅是小球的运动范围的一半. 

\subsection{周期\bre 频率}

做简谐运动的小球, 如果在经过$O$点时开始计时, 那么它将向右经过$M$点, 然后向左回到$O$点, 又继续运动到$N$点, 之后又向右回到$O$. 这样一个完整的振动过程称为一次\textbf{全振动}. 事实上, 不管从哪里开始计时, \textbf{做简谐运动的物体完成一次全振动的时间总是相同的}. 

做简谐运动的物体完成一次全振动所需要的时间, 叫作简谐运动的\textbf{周期}. 
在一个周期内, 小球完成一次全振动, \textbf{经过的路程是振幅的四倍}. 

物体完成全振动的次数与所用时间之比, 叫作简谐运动的\textbf{频率}, 数值上等于单位时间内完成全振动的次数. 
用$T$表示周期, $f$表示频率, 则有
$$f = \frac{1}{T}.$$

在国际单位制中, 周期的单位是秒, 频率的单位是\textbf{赫兹}(Hz). $1\ \text{Hz} = 1\ \text{s}^{-1}.$
周期和频率都是描述振动快慢的物理量. 周期越小, 频率越大, 表示振动越快. 

对于函数$x(t)$, 我们知道, $(\omega t+\varphi_0)$在每增加$2\uppi$的过程中, 函数值$x(t)$(即位移$x$)周期性变化一次, 函数的周期$T$就是简谐运动的周期$T$. 于是有
$$\omega(t+T)+\varphi_0 = (\omega t+\varphi_0) +2\uppi.$$
由上式解得$$\omega = \frac{2\uppi}{T}.$$
根据周期与频率的关系, 上式还可以写为
\begin{empheq}[box=\fbox]{equation*}
    \omega = 2\uppi f.
\end{empheq}

可见, $\omega$是一个与频率成正比, 与周期成反比的物理量, 它也可以描述
简谐运动的快慢, 叫作\textbf{圆频率}. 

实验证明, \textbf{简谐运动的周期与其振幅无关}.

\subsection{相位}

对于函数$x(t)$, 在一个周期内, 每一个$(\omega t+\varphi_0)$对应着一个函数值. 也就是说, 
当$(\omega t+\varphi_0)$确定时, 位移$x$也就确定了. 因此, $(\omega t+\varphi_0)$可以描述
物体正处于运动周期中的哪个状态, 物理学中把它叫作\textbf{相位}.

$\varphi_0$是$t = 0$时的相位, 称为\textbf{初相位}, 简称\textbf{初相},用$\varphi$表示.

两个具有相同频率的简谐运动, 我们通常研究它们的\textbf{相位差}. 
如果两个简谐运动的频率相同, 它们的相位差就是初相之差. 即
$$\Delta\varphi = \varphi_1 - \varphi_2.$$
此时, 我们说1的相位比2超前$\Delta\varphi$.

我们把两个相同的弹簧振子并列悬挂. 把它们拉到同一位置, 然后放开.
可以发现, 两个小球同时释放时, 除了振
幅和周期都相同外, 还总是向同一方向运动, 同时经过平
衡位置, 并同时到达同一侧的最大位移处.
在一个周期内, 如果不同时释放小球,它们的步调就不一致.

同时放开的两个小球振动步调总是一致, 我们
说它们的相位是相同的; 而对于不同时放开的两个小球, 
我们说第二个小球的相位落后于第一个小球的相位.

\subsection{位移\bre 速度}
做简谐运动的物体, 偏离平衡位置的距离, 叫作这个物体做简谐运动的位移.规定某一方向为正方向, 则物体偏向另一方向时, 
物体的位移为负值.

通过前面的分析知道, 根据一个简谐运动的振幅$A$, 周期$T$, 初相位$\varphi_0$, 可以
确定物体在任意时间$t$的位移
\begin{empheq}[box=\fbox]{equation}
    x = A\sin\left(\frac{2\uppi}{T}t+\varphi_0\right).
    \label{位移}
\end{empheq}

特别地, 如果规定物体处于平衡位置时$t = 0$, 即初相位$\varphi_0 = 0$, 那么
物体在时间$t$的位移$$x = A\sin \frac{2\uppi}{T}t.$$
如果规定物体处于正方向最大位移处时$t = 0$, 则有$\sin\varphi_0 = 1, $
取$\varphi_0 = \displaystyle\frac{\uppi}{2}$, 由三角函数的知识可知, 物体在时间$t$的位移
$$x = A\cos \frac{2\uppi}{T}t.$$

我们知道, 做机械运动的物体在某一时刻的速度, 就是它在这一时刻位移的瞬时变化率.
由数学知识可知, 速度就是位移关于时间的导函数.

如果物体做简谐运动, \eqref{位移函数} 式是物体的位移关于时间的函数, 那么该物体在时间$t$的速度
$$v = x^{\prime}(t) = A\omega\cos(\omega t+\varphi_0).$$

\subsection{回复力\bre 加速度}
根据牛顿运动定律, 在简谐运动中, 一定存在一个力, 在物体远离平衡位置时, 它迫使物
体的运动速度逐渐减小直到减为0, 然后, 物体在这个力
的作用下, 运动速度又由 0 逐渐增大并回到平衡位置; 物
体由于具有惯性, 到达平衡位置后会继续向另一侧运动, 
这个力使它再一次回到平衡位置.

正是在这个力的作用下, 物体才能在平衡位置附近做往复运动. 我们把这样的力叫作\textbf{回复力}.

对于水平方向的弹簧振子来说, 小球做简谐运动的回复力是弹簧对小球的弹力.大小是$F = kx$, 其中
$k$是弹簧的劲度系数.

理论上可以证明, 做简谐运动的物体都受到这样的回复力的作用, 它的大小与物体相对平衡位置的
位移成正比, 方向与位移方向相反, 这个力可以用$$F = -kx$$表示.其中符号表示$F$与$x$反向.
反过来说, \textbf{如果物体在运动方
向上所受的力与它偏离平衡位置位移的大小成正比, 并且
总是指向平衡位置, 物体的运动就是简谐运动.}

应当指出的是, 类似于向心力的概念, 回复力是一种效果力, 而不是拉力, 弹力这些性质力.
在分析物体所受的力时, 我们只需考虑物体实际所受的性质力, 不用再外加一个``回复力''.
这一点我们应该在学习向心力的时候就已经清楚了.

我们知道, 做简谐运动的物体的周期与其振幅无关. 事实上, 更进一步的研究表明, 
力学中一切做简谐运动的物体, 其往复运动的周期$T$均符合
\begin{equation}
    T = 2\uppi\sqrt{\frac{m}{k}},\label{简谐运动周期公式}
\end{equation}
其中$m$是做简谐运动的物体质量, $k$为回复力表达式中的比例系数.

有了回复力的概念, 我们重新来定义简谐运动中的平衡位置. 做简谐运动的物体所受回复力为0的位置,
叫作简谐运动的\textbf{平衡位置}.
事实上, 当物体处于平衡位置时, 所受的回复力一定为0, 但合力却不一定为0. 接下来要学习的
单摆就给出了一个例子.

由牛顿第二定律可知, 做简谐运动的物体, 其加速度大小也与位移成正比,
方向与位移方向相反. 另一方面, 加速度是速度的瞬时变化率, 也就是位移关于时间的二阶导数. 
如果物体做简谐运动, \eqref{位移函数} 式是物体的位移关于时间的函数, 那么
该物体在时间$t$的加速度
$$a = x^{\prime}(t) = -A\omega^2\sin(\omega t+\varphi_0).$$

\setlength{\abovedisplayskip}{0pt}
\setlength{\belowdisplayskip}{0pt}

\section{单摆}

生活中经常可以看到悬挂起来的物体在竖直平面内摆动.将一个小球用细线悬挂起来,把它拉离
最低点一段距离,然后放开, 小球就会来回摆动. 如果细线的长度不可改变, 细线的质量与小球相比可
以忽略, 球的直径与线的长度相比也可以忽略; 与小球受到的重力及
线的拉力相比,空气等对它的阻力可以忽略, 这样的装
置就叫作\textbf{单摆}.单摆是一种理想化模型.显然,
单摆摆动时摆球在做振动, 那么单摆的运动是否是简谐运动呢?

\subsection{单摆的运动}

\begin{wrapfigure}{o}{5cm}
    \flushright
    \includegraphics[width=0.2\textwidth]{pic/2.2-1.pdf}
    \label{2.2-1}
\end{wrapfigure}

如图, 单摆摆长为 $l$,摆球质量为 $m$. 将摆球拉离
平衡位置 $O$ 后释放, 摆球沿圆弧做往复运动. 当摆球沿圆
弧运动到某一位置 $P$ 时, 摆线与竖直方向的夹角为$\theta$. 此时
摆球受到重力 $G$ 和摆线拉力 $F_\text{T}$ 的作用. 重力 $G$ 沿圆弧切线
方向的分力 $$F = mg\sin\theta.$$ 正是这个力充当回复力, 迫使摆
球回到平衡位置 $O$. 而重力沿绳方向的分力$F_1$与绳子的拉力$F_\text{T}$
为摆球做圆周运动提供向心力.
需要注意的是, 与弹簧振子不同, 单摆在平衡位置的合力并不为0.因为摆球在
竖直平面内做圆周运动, 所以在$O$点处绳子的拉力略大于小球的重力.

回复力 $F$ 与摆球从 $O$ 点到 $P$ 点的位移 $x$ 并不成正比也不
反向.这样来看, 单摆似乎并不做简谐运动. 
但是, 当摆角 $\theta$ 很小时, 摆球运动的圆弧可以看成
直线, 可认为 $F$ 指向平衡位置 $O$, 与位移 $x$ 反向. 圆弧$\overset{\LARGE{\frown}}{OP}$
的长度可认为与摆球的位移 $x$ 大小相等, 即
$$\sin\theta \approx \theta = \frac{\overset{\LARGE{\frown}}{OP}}{l} \approx \frac{x}{l}.$$
因此, 回复力$F$就可以表示为$$F = -\frac{mg}{l}x.$$
式中符号表示回复力方向与位移$x$方向相反.对于一个确定的单摆来说, 
$\displaystyle\frac{mg}{l}$是一个确定的值.于是上式符合$F = -kx$的形式. 所以,单摆
在摆角很小\footnote{一般认为摆角小于$5^{\circ}$时, 单摆的运动是简谐运动.}的情况下做简谐运动.

\setlength{\abovedisplayskip}{5pt}
\setlength{\belowdisplayskip}{5pt}
\subsection{单摆的周期}
一条短绳系一个小球, 它的振动周期较短. 悬绳较长的
秋千, 周期较长. 实验表明: 单摆做简谐运动的周期与摆长有关, 摆长
越长, 周期越大; 单摆的周期与摆球质量和振幅无关.

那么, 单摆的周期与摆长有什么定量关系呢?
荷兰物理学家惠更斯进行了详尽的研究, 发现单摆做简谐运动的周
期 $T$ 与摆长 $l$ 的二次方根成正比, 与重力加速度 $g$ 的二次方
根成反比, 而与振幅和摆球质量无关. 惠更斯确定了计算
单摆周期的公式
\begin{eqnarray}
    T = 2\uppi\sqrt{\frac{l}{g}}.\label{单摆周期公式}
\end{eqnarray}

这一周期公式同样符合简谐运动的周期公式 \eqref{简谐运动周期公式}. 
把$k = \displaystyle\frac{mg}{l}$代入 \eqref{简谐运动周期公式},
就得到了上式.

\subsection{实验:用单摆测量重力加速度}

惠更斯在推导出单摆的周期公式后, 用一个单摆测出
了巴黎的重力加速度.我们也可以采用同样的办法, 测量
所在地区的重力加速度数值. 

当摆角较小时, 单摆做简谐运动, 根据其周期公式 \eqref{单摆周期公式}
可得
$$g = \frac{4\uppi^2l}{T^2}.$$
为了提高实验精度, 我们一般要求摆角$\theta<5^{\circ}$.

将细线穿过球上的小孔并打结固定, 然后把细线上端固定在铁
架台上, 就制成一个单摆. 在实验过程中,需要注意以下问题.
\begin{enumerate}
    \item 应选择密度较大的摆球, 最好是金属小球;
    \item 摆球要在同一竖直平面内摆动, 不能形成圆锥摆;
    \item 要从摆球经过平衡位置时开始计时,因为此处摆球速度最小, 计时误差最小;
    \item 通常测量多次全振动的时间来计算周期, 并且在数``0''的同时按下秒表, 此后
    每当摆球从同一方向经过平衡位置时计数1次.
    \item 摆长为绳长加摆球半径, 不要加摆球直径或只算绳长.
\end{enumerate}

本实验可以采用图像法来处理数据, 即用横轴表示摆长$l$,用纵轴表示$T^2$.
改变摆长, 多次测量周期$T$, 进行描点连线.这样得到的数据理论上为一条过原点的直线. 如果错把绳长作为摆长计算, 则直线
在纵轴上的截距为正; 如果把摆长算成绳长加摆球直径, 则直线在纵轴上的截距为负.
但无论如何, 直线的斜率均为$$k = \frac{4\uppi^2}{g}.$$即测量的重力加速度为准确值.

\section{外力作用下的振动}

\subsection{固有振动}

通过对弹簧振子及单摆的研究,我们知道弹簧振子
与单摆在没有外力干预的情况下做简谐运动,周期或频
率与振幅无关,仅由系统自身的性质决定. 我们把这种
振动称为\textbf{固有振动}, 其振动频率称为\textbf{固有频率}.

如果系统受到外力作用, 它将如何运动?

\subsection{阻尼振动}

由于实际的振动系统都会受到摩擦力, 黏滞力等阻
碍作用, 在振动过程中要不断克服外界阻力做功, 消耗能量, 
振幅必然逐渐减小. 这种振幅随时间逐渐减小的
振动称为\textbf{阻尼振动}.当阻力很小时, 在不长的时间内看不出
明显的振幅减小, 这样我们就可以把它当作简谐运动处理.

细化来说, 振动系统能量衰减的方式可以分为两种.一种是由于振
动系统受到摩擦阻力的作用,使振动系统的机械能逐渐转
化为内能. 例如单摆运动时受到空气的阻力. 

另一种是由于振动系统引起邻近介质中各质点的振动,使能量向四周
辐射出去, 从而自身机械能减少. 例如音叉发声时, 一部
分机械能随声波辐射到周围空间, 导致音叉振幅减小.

\subsection{受迫振动}
阻尼运动最终要停下来, 那么怎样才能产生持续的振动呢?
最简单的办法是使周期性的外力作用于振动系统, 
外力对系统做功, 补偿系统的能量损耗, 使系统的振动维
持下去. 这种周期性的外力. 叫作\textbf{驱动力}, 系统在驱动力作
用下的振动. 叫作\textbf{受迫振动}.机器运转时
底座发生的振动, 扬声器纸盆的振动, 都是受迫振动.

大量实验证明: \textbf{物体做受迫振动时, 振动稳定后的频率等于驱动力的频率, 与物体的
固有频率没有关系.}

\subsection{共振}

\begin{wrapfigure}{o}{5cm}
    \flushright
    \includegraphics[width=0.22\textwidth]{pic/2.3-1q.pdf}
    \label{2.3-1q}
\end{wrapfigure}

我们知道, 做简谐运动的物体, 其运动周期与振幅无关. 
如果物体在周期性变化的驱动力作用下振动, 物体的振幅是否也与其固有频率无关呢? 

实验结果告诉我们, 物体在做受迫振动时, 驱动力的频率与物体的
固有频率相差越小, 受迫振动的振幅越大; 当驱动力的频
率与物体的固有频率相等时, 受迫振动的振幅达到最大.

右图反映了受迫振动振幅 $A$ 与驱动力频率 $f$ 之间的
关系. 图中 $f_0$ 等于物体的固有频率, 可以看出, 当驱动力
的频率等于固有频率时, 物体做受迫振动的振幅达到最大
值, 这种现象称为\textbf{共振}.

\chapter{机械波}

\section{波的产生和传播}

\subsection{波的形成}

取一条较长的软绳, 用手握住一端拉平后
向上抖动一次, 可以看到绳上形成一个凸起部分, 这个凸
起部分向另一端传去. 向下抖动一次, 可以看到绳上形成
一个凹下部分, 这个凹下部分也向另一端传去. 连续向上, 
向下抖动长绳, 可以看到一列波产生和传播的情形.

在绳上做个标记, 在波传播的过程中, 这个标记
怎样运动? 它是否随着波向绳的另一端移动? 

仔细观察会发现, 这个标记只是在上下
振动, 没有向前运动. 相应地, 绳子上的各个点都只是在
上下运动, 但振动却传播出去了. 
振动的传播称为\textbf{波动}, 简称\textbf{波}.

一条绳子可以分成一个个小段, 一个个小段可以看作一个个相连的质点.
这些质点之间存在着相互作用. 当手握住绳子上下振动时, 绳端带动相邻的质点,
使它也上下振动; 这个质点又带动更远一些的质点, 绳子上的质点逐渐都跟着振动起来,
只是后面的质点总比前面的质点迟一些开始振动,也就是说, 
\textbf{后一个质点振动的相位总比前一个质点落后一些}.

这样,绳端这种上下振动的状态就沿绳传播出去了, 从整体上看, 就是一些凹凸相间的波形.

\subsection{横波\bre 纵波}

前面提到的, 在绳子上传播的波中, 质点上
下振动, 波向右传播, 二者的方向相互垂直. 像这样, 
质点的振动方向与波的传播方向相互垂直的波, 叫作\textbf{横波}.
在横波中, 凸起的最高处叫作\textbf{波峰},
凹下的最低处叫作\textbf{波谷}.

下面我们来看另外一种波. 

将一根长而软的弹簧水平放置在光滑平
面上, 在左端沿弹簧轴线方向不断推拉弹簧, 观察到, 
弹簧圈密集的部分和稀疏的部分交替
向右传播, 在弹簧上形成一种与横波不一样的波.

这种波又是如何形成的呢? 我们把一系列弹簧圈看成
一系列质点, 它们之间由弹力联系着. 手执弹簧一端左右
振动起来以后, 近端的质点依次带动远端的质点左右振动, 
但后一个质点总比前一个质点迟一些开始振动. 这样, 
弹簧一端左右振动的状态就沿弹簧传播开来. 从整体上看, 
就形成了疏密相间的波.

在这个例子中, 弹簧上的质点左右振动, 波向右传播, 
二者的方向在同一直线上. 质点的振动方向与波的传播方
向在同一直线上的波, 叫作\textbf{纵波}.
在纵波中, 质点分布最密的位置叫作\textbf{密部},质点分布最疏的
位置叫作\textbf{疏部}. 

声波是一种纵波. 发声体振动时也带动空气振动, 在空气中产生纵波.
例如振动的音叉, 它的叉股向一侧振动时, 压缩邻近的空气, 使
这部分空气变密, 叉股向另一侧振动时, 又使这部分空
气变得稀疏. 这种疏密相间的状态向外传播就形成声波. 
声波传入人耳, 使鼓膜振动, 就引起声音的
感觉. 声波不仅能在空气中传播, 也能在液体或固体中
传播, 但不能在真空中传播. 

\subsection{机械波}
绳上和弹簧上的波是在绳和弹簧上传播的, 水波是在
水面传播的, 声波通常是在空气中传播的. 绳, 弹簧, 水,
空气等是波借以传播的物质, 叫作\textbf{介质}. 

组成介质的质点之间有相互作用, 一个质点的振动会引起相
邻质点的振动. 机械振动在介质中传播, 形成了\textbf{机械波}.
声音不能在真空中传播, 是由于真空中没有能使声波传播的介质, 
因此我们说, 声音的传播需要介质, 实质上是振动的传播需要介质.

\textbf{介质中有机械波传播时, 介质本身并不随波一起传播.}
例如绳上或弹簧上有波传播时, 它们的质点在各自的平衡位置附近发生振动, 但
并不随波迁移, 传播的只是振动这种运动形式.

介质中本来静止的质点, 随着波的传来而发生振动, 
这表示它获得了能量. 这个能量是从波源通过前面的质
点依次传来的, 所以波在传播``振动''这种运动形式的
同时, 也将波源的能量传递出去. 因此, 
\textbf{波是传递能量的一种方式}.

波不但传递能量, 而且可以传递信息. 我们用语言进
行交流, 就是利用声波传递信息.

\section{波的描述}
\subsection{波长\bre 频率 \bre 波速}

在波动中, 各个质点的振动周期或频率是相同的, 它
们都等于波源的振动周期或频率, 这个周期或频率也叫作
波的周期或频率. 所以, \textbf{波的频率由波源决定, 与介质无关.}

在波的一个周期中, 介质上一个质点$P$的振动传到更远一些
的另一个质点$Q$, 使$Q$开始振动. 这时质点$P$恰好结束了一次全振动
而开始下一次全振动, 此后, 这两个质点的振动步调总是一致, 
也就是说, 质点$P$和$Q$的相位相同, 它们在任何时刻相对平衡位置的
位移的大小和方向总是相同的.

在波的传播方向上, 振动相位总是相同的两个相邻质
点间的距离, 叫作\textbf{波长}, 通常用 $\lambda$ 表示.
在横波中, 两个相邻波峰或两个相邻波谷之间的距离
等于波长; 在纵波中,两个相邻密部或两个相邻疏部之间
的距离等于波长.

由质点 $P$ 发出的振动, 经
过一个周期传到质点 $Q$, 也就是说, 经过一个周期 $T$, 振
动在介质中传播的距离等于一个波长 $\lambda$, 所以机械波在介质
中传播的速度为$$v = \frac{\lambda}{T}.$$
由于$f = T^{-1}$, 上式也可以写成
\begin{empheq}[box=\fbox]{equation*}
    v = \lambda f.
\end{empheq}

\textbf{机械波在介质中的传播速度由介质本身的性质决定, 
与波源的振动无关.} 在不同的介质中, 波速是不同的. 
以声速为例, 声波在空气中传播的速度为$332\unit{m/s}$, 
在水中传播的速度为$1450\unit{m/s}$, 
在玻璃中传播的速度为$5000$至$6000\unit{m/s}$, 在橡胶中
传播的速度为$30$到$50\unit{m/s}$.

\subsection{波的图像}

\begin{wrapfigure}{o}{9cm}
    \flushright
    \includegraphics[width=0.47\textwidth]{pic/2.4-1q.pdf}
    \label{2.4-1q}
\end{wrapfigure}

过去我们研究的是单个质点的运动情况, 用 $x$-$t$ 图像可
以很方便地描述质点在任意时刻的位移. 而波却是很多质
点的运动, 在同一时刻各个质点的位移都不尽相同, 不方
便用 $x$-$t$ 图像来描述. 能否用其他的图像来描述波呢? 我们
以横波为例研究波的图像.

一条波动的绳子, 它上面的质点呈现凹凸相间的波形.
在某一时刻拍一张照片, 照片记录了绳上各质点在该时刻的具体位置.
如果建立直角坐标系, 把该时刻绳上各质点的具体位置反映在坐标系中, 
就可以得到这一时刻绳子上\textbf{波的图像}, 也叫\textbf{波形图}.

在上图中, 用横坐标 $x$ 表示在波的传播方向上绳中
各质点的平衡位置, 纵坐标 $y$ 表示某一时刻绳中各质点偏
离平衡位置的位移. 我们规定, 位移向上时 $y$ 取正值, 向
下时 $y$ 取负值. 这样, 该图像看起来跟``照片''中的波形一致.

如果波的图像是正弦曲线, 这样的波叫作正弦波, 也
叫\textbf{简谐波}.可以证明, 介质中有简谐波传播时, 
介质上的质点做简谐运动.

简谐波的波形图与质点的振动图像都是正弦曲线, 但
它们的意义是不同的. 波形图表示介质中的``各个质点''
在``某一时刻''的位移, 振动图像则表示介质中``某一质
点''在``各个时刻''的位移.
形象地来比喻, 波形图是记录着许多人在某一时刻的集体照片, 
而振动图像是一个人在一段时间内的录像带.
可以说, \textbf{波形图研究某时刻所有质点的空间分布规律, 
振动图像研究一个质点的位移随时间的变化规律.}

\subsection*{判断波形图上任意点的振动方向}

要确定质点$P$的振动方向, 首先要明确波的传播方向, 确定波源在哪个
位置. 如上图所示, 假设波向右传播, 那么我们逆着波的传播方向, 
在波形图上往前找一点(即向左找一点, 这点与$P$点的距离不超过四分之一个波长).
分析波的传播过程可知, 如果这一点在$P$点的上方, 那么$P$点在下一时刻
将向上振动; 如果这一点在$P$点的下方, 那么$P$点在下一时刻
将向下振动. 在上图中, $P$点左边的点在$P$点上方, 所以$P$点
向上振动; 如果波向左传播, 那么$P$点向下振动.

\begin{wrapfigure}{o}{7cm}
    \flushright
    \includegraphics[width=0.35\textwidth]{pic/2.4-2q.pdf}
    \label{2.4-2q}
\end{wrapfigure}

下面再介绍另外一种判断方法.我们可以作出微小时间$\Delta t$后的波形,
这样就知道了各质点在$\Delta t$时间后的位置, 振动的方向也就知道了.
右图虚线所示波形是实线在$\Delta t$后的波形, $P^{\prime}$是$P$点
在$\Delta t$时间后的位置, 由此可知$P$点向上振动.

\subsection*{关于波动与振动的几个推论}

(1) 在波的传播方向上, 当两质点平衡位置的距离为波长的整数倍(即半波长的偶数倍)时, 
这两个质点的速度和位移总是相同的, 即振动步调相同(或相位相同). 反之, 当两个质点
的相位相同时, 它们平衡位置之间的距离一定是整数倍的波长, 
即$$\Delta x = n\lambda, n = 1,2,3,\cdots, $$
或者$$\Delta x = \frac{n\lambda}{2}, n = 2,4,6,\cdots.$$

(2)在波的传播方向上, 当两质点平衡位置的距离为半波长的奇数倍时, 
这两个质点的速度和位移总是大小相等, 方向相反的, 即振动步调相反. 
反之, 当两个质点
的振动步调相反时, 它们平衡位置之间的距离一定是半波长的奇数倍,即 
$$\Delta x = \frac{n\lambda}{2}, n = 1,3,5,\cdots.$$

(3)介质中任一质点起振的方向必然与波源起振的方向一致. 因此可以说,
波最前沿的质点的振动方向就是波源的起振方向.

(4)对于简谐振动与简谐波而言, 无论计时起点如何, 经过一个周期$T$, 
质点都完成了一次完整振动, 其路程为$4A$($A$为振幅或波峰的纵坐标);
相应地, 无论计时起点如何, 经过$\displaystyle\frac{T}{2}$后, 
质点相对于平衡位置的位移与速度都与计时开始时等大反向, 其路程恰好是$2A$.
但是, 经过$\displaystyle\frac{T}{4}$后, 质点所经过的路程却不是定值, 
并不必然等于$A$.

\section{波的性质}

\subsection{波的反射}
我们以水波为例来研究波的反射.

在水槽中一端有一振动发生器, 振动发生器在水槽中能够产
生水波. 在水槽中斜向放置一块挡板,
当水波遇到挡板时会发生\textbf{反射}. 
如果用一条射线代表水波的入射方向(入射
线), 用另一条射线代表水波的反射方向(反射线), 我们
发现水波的反射与初中学过的光的反射遵循同样的规律:
\textbf{反射线,法线与入射线在同一平面内, 反射线与入射线分
居法线两侧, 反射角等于入射角.}
即:\textbf{三线共面, 两线分居,两角相等}.

\subsection{波的折射}

我们知道光从一种介质进入另一种介质时会发生折射,
机械波会发生折射吗? 理论和实验证明, 一切波都会发生
折射现象. 一列水波在深度不同的水域传播时, 在交界面
处将发生\textbf{折射}.

水波发生折射的根本原因是, 深度不同的水域, 相当于不同的介质;
而不同介质的波速$v$不同. 深水域的水波波速较小, 浅水域的水波波速较大.

波的频率$f$只由波源决定, 在
折射过程中不会变化; 由$v = \lambda f$可知, 波长$\lambda$在折射过程中发生变化.

波的折射定律可以表述如下:

\begin{enumerate}
    \item \textbf{反射线,法线与入射线在同一平面内, 反射线与入射线分
    居法线两侧;}
    \item \textbf{波从波速较大的介质射入波速较小的介质中, 折射角小于入射角; 
    波从波速较小的介质射入波速较大的介质中, 折射角大于入射角.}
\end{enumerate}

事实上, 折射角和入射角存在一个定量关系. 在学习光的折射时, 我们再详细给出这一关系.

\subsection{惠更斯原理}

我们知道, 波是波源的振动在介质中的传播.从宏观上看, 波是往前不断推进的一个过程:
后面的波推前面的波, 前面的波再往前走, 对于水波来说, 就是``长江后浪推前浪''.
从微观上看, 一个波源形成一个波面, 旧的波面再产生新的波面, 这又是怎样的一个过程呢?
如果前面的波面遇到了障碍物, 它以后的运动形式是怎么样的? 对于这个问题, 我们可以用惠更斯原理来解释.

假设水面上的某一点有一个波源, 水波向四周传开.由于向各个方向的波速都一样, 
因此向四面八方传播的波峰组成了一个个圆, 波谷也组成了一个个圆.
实际上, 任何振动状态相同的点都组成了一个个圆.
我们把这些圆叫作一个个\textbf{波面}, 而与波面垂直的那些线, 代表波的传播方向, 叫作\textbf{波线}.

\subparagraph{惠更斯原理} 介质中任意波面上的各点, 都可以看作发射子波的波源.
而在其后的任意时刻, 这些子波在波的传播方向的包络面, 就是新的波面.

惠更斯原理不是从某个定律推导出来的, 也不是直接由实验总结出来的. 它之所以正确, 是因为由它得出来的结论
都与事实相符.

\subsection{波的衍射}

波在传播过程中如果被较大的障碍物挡住, 就会在障碍物的后方形成一个没有波的阴影区.
但是当障碍物较小时, 波能绕过障碍物的边缘, 在障碍物的阴影区内继续传播.
这种波绕过障碍物继续传播的现象叫作波的\textbf{衍射}.

在波的传播方向上放上两块挡板,挡板中间留有适当的空隙. 
如果波只是沿着直线传播, 那么应该只有空隙后面才有直线传播.
而实际上,在挡板后面本来不应该有波的区域却能看到波的传播, 这也是波的衍射现象.

在不改变波源的情况下, 将挡板的缝隙调整得很大, 我们会发现波基本上是沿着直线传播的.
在挡板后方, 波的传播被挡板挡住了, 衍射现象不明显.

接下来,缩小挡板的缝隙, 此时缝隙的大小与波长近似, 在挡板的后方就出现了明显的衍射现象. 

我们继续减小挡板的缝隙, 将缝隙减小到比波长还要小. 挡板后面波的传播范围不但没有减小, 反而扩大了, 
也就是说波的衍射现象变得更加明显了.

实验证明: \textbf{当障碍物或孔的尺寸比波长小, 或者与波长相差不大的时候, 
波会产生明显的衍射现象.}

不只是水波, 声波也能发生衍射. 通常的声波, 波长为 1.7\ cm到
17\ m, 跟一般障碍物的尺寸相当, 所以声波能绕过一般的
障碍物, 使我们能听到障碍物另一侧的声音.

一切波都能发生衍射.衍射是波特有的现象.

\subsection{波的干涉}

在介质中常常有几列波同时传播.两列波相遇时,会
不会像两个小球相碰时那样,改变各自的运动特征呢?

事实表明, \textbf{几列波相遇时能够保持各自的运动特征
继续传播而互不干扰. 在它们重叠的区域里, 介质的质点同时参与这
几列波引起的振动, 质点的位移等于这几列波单独传播时引起的位移的矢量和.}
这就是\textbf{波的叠加原理}.

如果相遇的两列波周期相同, 在它们重叠的区域里会发
生什么特别现象吗?

在水槽外放置一个振动片, 振动片上固定两根相同的细杆. 振动片振动时, 两根细杆周期性地击打水面,
形成两个波源. 这两列波的频率相同, 振动方向也相同. 由于两根细杆是同步振动的, 所以它们振动的相位
差恒为0.

这两列波相遇后, 水面上出现了一条条相对平静的区域和激烈振
动的区域, 并且这两类区域在水面上的位置是稳定的.

\begin{figure}[htbp]
    \centering
    \includegraphics[width=13cm]{pic/3.3-1.jpg}
    \label{3.3-1}
\end{figure}

怎样解释上面的现象呢?如上图所示, 
用两组同心圆表示从波源发出的两列波, 
蓝线圆表示波峰, 黑线圆表示波谷. 在一组同心圆中, 相
邻的蓝黑线圆间的距离等于半个波长, 
同色线圆之间的距离等于一个波长.

\subparagraph{加强点} 如果在某一时刻, 两列波的波峰与波峰相遇(比如上图中的$M$点), 也就是说, 两列波
在这一点引起的振动具有相同的相位. 那么经过半个周期后, 一定是
波谷和波谷在这一点相遇.

根据波的叠加原理, 波峰与波峰相遇时, 两列波引起的振动在$M$点叠加, 
引起的位移是两列波的振幅之和, 并且在这一点向上的位移最大;
波谷与波谷相遇时, 引起的位移也是两列波的振幅之和, 
在这一点向下的位移最大.由于两列波在这一点
的相位相同, 所以它们引起的振动总是相互增强而非抵消, 我们把这样的
点叫作振动的\textbf{加强点}.

\subparagraph{减弱点} 如果在某一时刻, 两列波的波峰与波谷相遇
(比如上图中的$N$点), 也就是说, 两列波
在这一点引起的振动相位相反. 那么经过半个周期后, 仍然是
波谷和波峰在这一点相遇.两列波的振动在这一点相互抵消, 
引起的位移相互削弱.如果两列波的振幅相同, 那么质点
的位移之和就总等于0.由于两列波在这一点
的相位相反, 所以它们引起的振动总是相互削弱, 我们把这样的
点叫作振动的\textbf{减弱点}.

\subparagraph{加强区与减弱区} 在上图中标出波峰与波峰, 波谷与波谷相遇的点, 
并将它们连起来(如上图中实线), 根据前面的分析可知, 这些点的振动总是相互增强, 是振动的加强点.
事实上, 在这些线上的点都是振动的加强点, 我们称这些线为振动的加强区.

类似地, 标出波峰与波谷相遇的点, 并将它们连起来(如上图中虚线), 这些线
上的点振动相互削弱, 是振动的减弱点, 我们称这些线为振动的减弱区.

\subparagraph{波的干涉与干涉条件} 通过理论分析可知, 
频率相同,相位差恒定,振动方向相同的两列
波叠加时, 某些区域的振动总是加强, 某些区域的振动总
是减弱, 这种现象叫作波的\textbf{干涉}. 形成的这
种稳定图样叫作干涉图样.

需要注意的是, 我们说的波的增强与减弱, 是指能量的增强与减弱, 
由叠加波的振幅$A$体现, 但位移可以是$-A$到$A$的任意值.

不仅水波会发生干涉现象, 声波, 电磁波等一切波, 
只要满足上述条件都能发生干涉. 跟衍射一样, 干涉也是
波特有的现象.主动降噪耳机就是利用了波的干涉原理, 
在耳机内产生与噪声波相互削弱的声波, 使我们听不到噪音.

\subsection{多普勒效应}

仔细听急救车的鸣笛声, 会发现一个现
象: 当车从你身边疾驰而过的时候, 鸣笛的音
调会由高变低. 这是怎么回事呢?

1842 年, 奥地利物理学家多普勒带着女儿在铁道旁散
步时就注意到了类似上面描述的现象. 他经过认真的研究, 
发现波源与观察者相互靠近或者相互远离时, 接收到的波
的频率都会发生变化. 人们把这种现象叫作\textbf{多普勒效应}.

为了理解多普勒效应, 我们做如下模拟实验: 让一列人沿街行走, 
观察者站在侧面不动, 假设每分钟有30人从他身体通过, 这种情况下
的``过人频率''是30人每分钟. 
如果现在观察者逆着队伍行走, 每分钟与观察者相遇的人数增加, 也就是频率
增加; 反之, 如果观察者顺着队伍走, 频率降低.

对于机械波也有类似的结论:
\begin{enumerate}
    \item 波源与观察者相对静止时, 接收频率等于波源频率;
    \item 波源与观察者相互靠近时, 接收频率大于波源频率;
    \item 波源与观察者相互远离时, 接收频率小于波源频率.
\end{enumerate}

多普勒效应在科学技术中有广泛的应用.
交通警察向行进中的车辆发射频率已知的超声波, 同时测量反射波的
频率, 根据反射波频率变化的多少就能知道车辆的速度.这种
测速仪叫作多普勒测速仪.

医生向人体内发射频率已知的超声波, 超声波被血管
中的血流反射后又被仪器接收. 测出反射波的频率变化, 
就能知道血流的速度.这种方法俗称``彩超''.

理论和实验都证明, 光波或电磁波都有多普勒效应, 
宇宙中的星球都在不停地运动.测量星球上某些元素发出的光波的频
率, 然后与地球上这些元素静止时发光的频率对照, 就可
以算出星球靠近或远离我们的速度.

\part{电与磁}
\chapter{静电场}

\section{电荷}
\subsection{电荷}

人们发现, 很多物体都会由于摩擦而带电, 并称这种方式为\textbf{摩擦起电}.

美国科学家富兰克林通过实验发现, 雷电的性质与摩擦产生的电的性质完全相
同, 并命名了\textbf{正电荷}和\textbf{负电荷}.迄今为止,人们没有发现对这两种电荷都排斥或都
吸引的电荷, 自然界的电荷只有两种.

电荷的多少叫作\textbf{电荷量}, 用$Q$或$q$表示. 在国际单位制中, 它的单位是\textbf{库仑}, 简称库, 符号是 C.

库仑是一个非常大的电荷量单位. 通常,一把梳子和衣袖摩擦后所带的电荷量不到百万
分之一库仑; 但天空中发生闪电之前,巨大的云层中积累
的电荷量可达几百库仑.

库仑这个单位是一个导出单位, 规定 1 A 的恒定电流在1 s 时间间隔内所传送的电荷量为1 C. 因此,
电荷量不属于基本物理量, 它是电流强度$I$和时间$t$的导出物理量\footnote{电流的本质是电荷的移动, 为什么不规定电荷量为基本物理量, 并以它和时间来导出电流呢? 规定如此. 事实上, 应该是习惯性和便捷性的重要性超过了逻辑性. 电流更早被定义为基本物理量, 应用又极其广泛, 如果改变为电荷量, 会对人类社会造成不便的影响.}, 并且$$Q = It.$$

正电荷的电荷量为正值, 负电荷的电荷量为负值.

\subsection{起电}

起电, 就是使物体带电. 起电的本质是电子转移.

摩擦可以使物体带电, 这是因为不同物质对电子的束缚能力不同. 那么, 还有其他方法可以使物体带电吗?

\subparagraph{感应起电} 当一个带电体靠近导体时, 由于电荷间相互吸引或排
斥, 导体中的自由电荷便会趋向或远离带电体, 使导体靠
近带电体的一端带异种电荷, 远离带电体的一端带同种电
荷.这种现象叫作\textbf{静电感应}.利用
静电感应使金属导体带电的过程叫作\textbf{感应起电}.

与摩擦起电不同的是, 摩擦起电的两个物体通常都是绝缘体, 这使电荷会留在绝缘体表面, 产生明显的带电现象.相反,导体的电荷会均匀分布整个物体,
摩擦后不易观察到带电现象.

\subparagraph{接触起电} 感应起电的两个物体不接触, 电荷仅在导体的内部移动.
除此之外, 两个导体在接触时, 如果它们之间存在电势差, 电荷将会在两个物体间转移,
最终达到动态平衡.

一个不带电的导体通过与一个带电体接触后分开, 从而形成带电体的过程, 称为\textbf{接触起电}.

两个完全相同的导体接触后分开, 它们所带的电荷量相同.因此, 除非这两个导体都不带电, 否则在
接触后将会相互排斥.

\subsection{电荷守恒定律}

静电感应过程中导体中的自由电荷只是从导体的一部
分转移到另一部分. 而接触起电过程中自由电荷在几个导体间转移.
也就是说, 无论是接触起电还是感应起电都没有创造电荷, 只是电荷的分布发生了变化.

大量实验事实表明, \textbf{电荷既不会创生, 也不会消灭, 它
    只能从一个物体转移到另一个物体, 或者从物体的一部分转
    移到另一部分;在转移过程中, 电荷的总量保持不变.}这个
结论叫作\textbf{电荷守恒定律}.

然而, 近代物理实验发现, 在一定条件下, 带电粒子可以产
生或湮没. 例如, 一个高能光子在一定条件下可以产生一
个正电子\footnote{正电子与电子质量相同, 与电子的电荷量相等但符号相反.下文说的``负电子''就是电子.}和一个负电子; 一对正负电子可以同时湮没, 
转化为光子. 不过在这些情况下, 带电粒子总是成对产生
或湮没的, 两个粒子带电数量相等但电性相反, 而光子不带电, 所以电荷的代数和仍然不变. 
因此, 电荷守恒定律更普遍的表述是:\textbf{一个与外界没有电荷交换的系统, 电
    荷的代数和保持不变.}

\subsection{元电荷}
迄今为止, 实验发现的最小电荷量就是电子所带的电
荷量.质子, 正电子所带的电荷量与它相同, 电性相反.
人们把这个最小的电荷量叫作\textbf{元电荷}, 用$e$表示.

元电荷$e$的数值, 最早是由美国物理学家\textbf{密立根}测得
的.在密立根实验之后, 人们又做了许多测量.现在公认的元电荷$e$的值为
$$e = 1.602 176 634 \times 10^{-19}\ \mathrm{C}.$$
在计算中, 可取
$$e = 1.60 \times 10^{-19}\ \mathrm{C}.$$

电子的电荷量$e$与电子的质量$m_e$之比, 叫作电子的\textbf{比
    荷}.比荷也是一个重要的物理量.电子的
质量$m_e = 9.11\times 10^{-31}$ kg, 所以电子的比荷为
$$\frac{e}{m_e} = 1.76 \times 10^{11}\ \mathrm{C/kg}.$$

\section{静电力}

实验事实说明, 两个实际的带电体间的相互作用力与
它们自身的大小, 形状以及电荷分布都有关系.

当带电体之间的距离比它们自身的大小大得多, 以致带电体的形状, 大小及电荷分布状
况对它们之间的作用力的影响可以忽略时, 这样的带电体可以看作带电的点, 叫作\textbf{点电荷}.

\subsection{库仑定律}
通过实验可知, 电荷之间的作用力随着电
荷量的增大而增大, 随着距离的增大而减小.
这看起来与万有引力的规律类似.电荷之间的相互作用力, 会不会与它们电
荷量的乘积成正比, 与它们之间距离的二次方成反比?

电荷之间的作用力与万有引力是否相似的问
题早已引起当年一些研究者的注意, 英国科学家卡文迪什
和普里斯特利等人都确信``平方反比''规律适用于电荷间
的力.最终是由
法国科学家库仑设计了一个十分精妙的实验(\textbf{扭秤实验}), 对电荷之间的作
用力开展研究, 确定: \textbf{真空中两个静止点电荷之间的
    相互作用力, 与它们的电荷量的乘积成正比, 与它们的距
    离的二次方成反比, 作用力的方向在它们的连线上.}这个
规律叫作\textbf{库仑定律}.这种电荷之间的相
互作用力叫作\textbf{静电力}或\textbf{库仑力}.

假设两个点电荷的电荷量分别为$q_1$, $q_2$, 它们的距离为$r$, 那么库仑定律可以表示为
\begin{empheq}[box=\fbox]{equation*}
    F = k\frac{q_1q_2}{r^2}.
\end{empheq}
式中的$k$是比例系数, 叫作\textbf{静电力常量}. $k$的值是
$$k = 9.0 \times 10^9\ \mathrm{N\cdot m^2/C^2}.$$
当两个点电荷所带的电荷量为同种时, 式中算得的$F$为正值, 
它们之间的作用力为斥力; 反之, 为
异种时, 式中算得的$F$为负值, 它们之间的作用力为引力. 
在国际单位制中, 电荷量的单位是库仑(C), 力的单
位是牛顿(N), 距离的单位是米(m). 

\begin{example}
    \label{比较静电力与万有引力}
    在氢原子内, 氢原子核与电子之间的最短距离为 $5.3\times 10^{-11}$ m. 试比较氢原子核
与电子之间的静电力和万有引力(质子质量为 $1.67\times 10^{-27}$ kg,电子质量为 $9.1\times 10^{-31}$ kg).
\end{example}
\begin{solution}
    根据库仑定律,它们之间的静电力
    $$F_\text{库} = k\frac{q_1q_2}{r^2} = 9.0\times 10^9\times
    \frac{(1.6\times 10^{-19})^2}{(5.3\times10^{-11})^2}\unit{N} = 8.2\times 10^{-8}\unit{N}.$$

    根据万有引力定律, 它们之间的万有引力
    $$F_\text{引} = G\frac{m_1m_2}{r^2} = 6.7\times 10^{-11}\times
    \frac{1.67\times 10^{-27}\times 9.1\times 10^{-31}}{(5.3\times10^{-11})^2}\unit{N} = 3.6\times 10^{-47}\unit{N}.$$

    因此$$\frac{F_\text{库}}{F_\text{引}} = 2.3\times 10^{39}.$$
    即氢原子核与电子之间的静电力是万有引力的 $2.3\times 10^{39}$ 倍.
\end{solution}

由例 \ref{比较静电力与万有引力} 可见, 微观粒子间的万有引力远小于库仑力. 因此, 
在研究微观带电粒子的相互作用时, 可以把万有引力忽略.

库仑定律描述的是两个点电荷之间的作用力.如果存
在两个以上点电荷, 那么, 每个点电荷都要受到其他所有
点电荷对它的作用力. 两个或两个以上点电荷对某一个点
电荷的作用力, 等于各点电荷单独对这个点电荷的作用力
的矢量和.\footnote{实验表明, 两个点电荷
之间的作用力不因第三个点电荷的存在而改变.}

库仑定律是电磁学的基本定律之一.库仑定律给出的
虽然是点电荷之间的静电力, 但是任何一个带电体都可以
看成是由许多点电荷组成的. 所以, 如果知道带电体上的
电荷分布, 根据库仑定律就可以求出带电体之间的静电力
的大小和方向.

\subsection{应用: 静电力的平衡问题}

在真空中有两个相距不远的点电荷A和B, 显然, 
无论它们电性如何, 在静电力的相互作用下,都不可能平衡.

现在我们另有一点电荷C, 把它放置在A, B所在直线上某一位置, 能使它们三者受力平衡吗?
应该放在哪里?

我们知道, 当每个电荷受另外两个电荷的合静电力均为0时, 它们三个均平衡. 根据这一点, 我们可以确定: 

(1) 放在中间的电荷, 与放在两边的电荷电性相反; 

(2) 放在两边的电荷电性相同.

如果不满足(1), 那么两边的电荷无法平衡; 如果不满足(2), 那么中间的电荷无法平衡. 
这个规律可以简单记为``两同夹一异''.

我们不妨假设A, B带正电, C带负电, 那么C应该放在A, B的中间. 应该更靠近谁呢?
容易知道, C应该更靠近电荷量较小的那个. 可以列出C的受力平衡方程, 从而解出这个距离.
这个规律可以简单说成``近小远大''.

除此之外, A, B的平衡对C的电荷量也有要求. 如果A, B固定, 那么在A, B所在的直线上, 总有一点
使C受力平衡, 但这一点不一定能使A, B也平衡. 事实上, 当A, B, C均平衡时, 由库仑定律可以得到
$$\sqrt{q_1q_3} = \sqrt{q_1q_2}+\sqrt{q_2q_3},$$
其中$q_1$, $q_3$是放在两边的两个电荷的电荷量, $q_2$是放在中间的电荷的电荷量.
由此可知, 放在两边的电荷, 它们的电荷量都大于中间电荷的电荷量, 简单说成``两大夹一小''.

综合起来, 我们有: \textbf{``两同夹异, 两大夹小, 近小远大''}.

\begin{example}
    两个带正电小球,电荷量分别为 $Q$ 和
$9Q$,在真空中相距 0.4 m.如果引进第三个带
电小球,正好使三个小球仅在静电力的作用下
处于平衡状态,那么第三个小球应放在什么地
方?带的是哪种电荷?电荷量是 $Q$ 的几倍?
\end{example}
\begin{solution}
    根据本节知识, 容易知道引入的第三个小球必须带负电, 放在前两个小球的连线上, 并且距离
    电荷量为$Q$的小球较近.

    设第三个小球的电荷量为$q$, 放在与电荷量为$Q$的小球的距离为$r$ 的地方. 由平衡条件和库仑定律有
    $$k\frac{Qq}{r^2} = k\frac{9Qq}{(0.4 - r)^2}$$
    解得$r = 0.1\unit{m}.$

    对电荷量为$Q$的小球进行受力分析, 由平衡条件可知
    $$k\frac{Qq}{0.1^2} = k\frac{9Q^2}{0.4^2}.$$
    解得$q = 0.5625Q.$

    因此, 第三个小球带负电, 电荷量为$0.5625Q$; 放在两个带正电的小球之间, 距离电荷量为$Q$的小球0.1 m处.
\end{solution}

\section{电场}

\subsection{电场}

19 世纪 30 年代, 英国科学家法拉第提出一种观点, 认
为在电荷的周围存在着由它产生的电场.处在电场中的其他电荷受到的作用力就是这个电场给
予的.例如, 电荷 A 对电荷 B 的作用力, 就是电荷 A 的电
场对电荷 B 的作用; 电荷 B 对电荷 A 的作用力, 就是电荷 B
的电场对电荷 A 的作用.

物理学的理论和实验证实并发展了法拉第的观点.电场
以及磁场已被证明是客观存在的.场像分子, 原子等实物
粒子一样具有能量, 因而场也是物质存在的一种形式.

应该指出, 只有在研究运动的电荷, 特别是运动状态迅
速变化的电荷时, 上述场的物质性才突显出来. 本章我们只研究静止电荷产生的电场, 
叫作\textbf{静电场}.

\subsection{电场强度}

把一个电荷放入某个电场中, 来研究这个电场的性质. 这样的
电荷叫作\textbf{试探电荷}或检验电荷.激发电场的带电体所带的电荷叫
作\textbf{场源电荷}, 或\textbf{源电荷}.

在研究电场的性质时, 我们选取的试探电荷应当是电荷量很小的点电荷, 目的是不对所研究的电场产生影响.

在点电荷$Q$的电场中的$P$点, 放一个试探电荷$q_1$, 它在电场中受到的静电力是$F_1$, 根据库仑定律, 有
\begin{equation}
    F_1 = k\frac{q_1Q}{r^2}.
    \label{静电力1}
\end{equation}

同理, 如果把试探电荷换成$q_2$, 那么它受到的静电力
\begin{equation}
    F_2 = k\frac{q_2Q}{r^2}.
    \label{静电力2}
\end{equation}

由 \eqref{静电力1} \eqref{静电力2} 两式可以看出
\begin{equation}
    \label{静电力与电荷量之比}
    \frac{F_1}{q_1} = \frac{F_2}{q_2} = k\frac{Q}{r^2}.
\end{equation}
放在$P$点的试探电荷所受的静电力与它的电荷量之比, 与产生电场的场源电荷
的电荷量 $Q$及$P$点到场源电荷的距离$r$有关, 而与试探电荷的电荷量无关.

试探电荷所受的静电力与它的电荷量之比反映了电场在各点的性质. 物理学中把它定义为\textbf{电场强度},
用$E$表示, 即
\begin{empheq}[box=\fbox]{equation*}
    E = \frac{F}{q}.
\end{empheq}
这是电场强度的定义式.其中$F$是试探电荷在电场内某点所受的静电力, $q$是这个试探电荷的电荷量, $E$是这一点的电场强度.

由定义式可知, 电场强度的国际单位为\textbf{牛每库}, 符号是N/C.如果 1 C 的电荷在电场中的某点受到的静电力是 1 N,那么
该点的电场强度就是 1 N/C.

电场强度是矢量.物理学规定, \textbf{电场中某点的电场强度的方向与正电荷在该点所受静电力的方向相同.}

试探电荷在电场中受到的静电力也叫作\textbf{电场力}.

我们知道, 两个或两个以上的点电荷对某一个点电荷
的静电力, 等于各点电荷单独对这个点电荷的静电力的矢
量和.由此可以推理, 如果场源是多个点电荷, 则电场中
某点的电场强度等于各个点电荷单独在该点产生的电场强
度的矢量和.

\subsubsection{点电荷的电场强度}

点电荷是最简单的场源电荷.由 \eqref{静电力与电荷量之比} 式可知,
一个电荷量为$Q$的点电
荷, 在与之相距$r$处的电场强度
\begin{empheq}[box=\fbox]{equation}
    E = k\frac{Q}{r^2}.
    \label{点电荷的电场强度}
\end{empheq}

据上式可知, 如果以电荷量为$Q$的点电荷为中心作一
个球面, 则球面上各点的电场强度大小相等. 当$Q$为正电荷
时, 电场强度$E$的方向沿半径向外; 当$Q$为负
电荷时, 电场强度$E$的方向沿半径向内.

我们知道, 两个或两个以上的点电荷对某一个点电荷
的静电力, 等于各点电荷单独对这个点电荷的静电力的矢
量和.由此可以推理, 如果场源是多个点电荷, 则电场中
某点的电场强度等于各个点电荷单独在该点产生的电场强
度的矢量和.


\begin{example}
    在某电场中的$P$点, 放一带电量$q_1 = -3.0\times 10^{-10}\unit{C}$的试探电荷,
    测得该点受到的静电力大小为$F_1 = 6.0\times 10^{-7}\unit{N}$, 方向水平向右. 求

    (1) $P$点的电场强度大小和方向;

    (2) 如果在$P$点放一带电量$q_2 = 1.0 \times 10^{-10}\unit{C}$的试探电荷, 求$q_2$
    受到的静电力$F_2$的大小和方向.
\end{example}
\begin{solution}
    (1) 根据电场强度的定义, $P$点的电场强度为
    $$E = \frac{F_1}{q_1} = \frac{6.0\times 10^{-7}}{-3.0\times 10^{-10}}\unit{N/C} = 2.0\times 10^{3}\unit{N/C},$$
    方向与负点电荷$q_1$受到的静电力方向相反, 即水平向左.

    (2)由电场强度的定义有$$E = \displaystyle\frac{F_2}{q_2}.$$
    由此可得$$F_2 = 2.0\times 10^{-7}\unit{N}.$$因为$q_2$是正点电荷, 所以
    $F_2$的方向与$P$点的场强方向相同,即水平向左.
\end{solution}

\subsubsection{电场线}

除了用数学公式描述电场外, 形象地了解和描述电场中
各点电场强度的大小和方向也很重要.
法拉第采用了一个简洁的方法来描述电场, 那就是画\textbf{电场线}.

同一幅图中, 电场强度较大的地方电场线较密, 电场强度
较小的地方电场线较疏, 因此在同一幅图中可以用电场线
的疏密来比较各点电场强度的大小.

\subsubsection{匀强电场}

如果电场中各点的电场强度的大小相等,方向相同,
这个电场就叫作\textbf{匀强电场}.

由于方向相同, 匀强电场中的
电场线应该是平行的; 又由于电场强度大小相等, 电场线
的疏密程度应该是相同的.所以, 匀强电场的电场线可以
用间隔相等的平行线来表示.

例如,相距很近的一对带等
量异种电荷的平行金属板,它们之间的电场除
边缘外,可以看作匀强电场.

\subsection{两个等量点电荷形成的电场}

把两个带电量相等的点电荷置于空间中不远的两点, 两个点电荷形成的电场将会互相叠加.
我们主要研究这个电场分别在两点电荷连线和中垂线上的性质.

记这两个点电荷连线的中点为$O$点. 通过数学分析可知, 无论这两个点电荷
的电性如何, 在它们的连线上, $O$点的电场强度总是最小的.这个分析的过程比较复杂, 
我们不做展开, 只要记住这个结论即可; 当然, 我们也可以直接绘制电场线, 观察电场线的
疏密来判断电场强度的大小.

\subsubsection{等量同种点电荷}

在这两个电荷的连线上, $O$点的电场强度是最小的. 假设一个正点电荷放置在$O$点,
那么左右两个电荷对它的静电力等大反向, 因此\textbf{$O$点的电场强度为0}.

在这两个电荷连线的中垂线上, $O$点的电场强度也是最小的. 假设一个正点电荷放置在$O$点上方,
那么它同时受左右两个电荷的斥力(或引力), 并且这两个力是等大的. 它们的合力
方向(即电场强度的方向)垂直于连线向上(或向下).

我们知道, 中垂线上无限远处的电场强度为0, $O$点处的电场强度也为0,
而它们之间的电场强度却不为零. 因此, 在中垂线上存在一点为电场强度在这条线上的最大值.
\textbf{这一点我们无法确定}.

\subsubsection{等量异种点电荷}

在这两个电荷的连线上, $O$点的电场强度是最小的. 假设一个正点电荷放置在$O$点,
那么左右两个电荷对它的静电力等大同向, 它们的合力不为0.
因此, \textbf{在连线上, $O$点的电场强度最小但不为0}.

在这两个电荷连线的中垂线上, $O$点的电场强度是最大的. 假设一个正点电荷放置在$O$点上方,
那么它同时分别受两个电荷的引力和斥力, 并且这两个力是等大的. 它们的合力
方向(即电场强度的方向)平行于它们的连线向左或向右.

因为它放在$O$点上方时与两个电荷的距离变大了, 所以它受到的静电力变小了. 因此, \textbf{在中垂线上,
    $O$点的电场强度最大; 沿中垂线向外,电场强度减小}.

\section{静电场中的能量}

一个正电荷在电场中只受到静电力$F$的作用, 
它在电场中由$A$点运动到$B$点时, 静电力做了正功
$W_{AB}$. 由动能定理可知, 该电荷的动能增加了$W_{AB}$. 
从能量转化的角度思考, 物体动能增加了, 
意味着有另外一种形式的能量减少了. 这是一种
什么形式的能量呢? 
\subsection{电势能}

\subsubsection{静电力做功的特点}
\label{section:静电力做功的特点}

我们知道, 电荷在电场中会受到静电力, 若电荷发生位移, 则静电力可能会做功. 我们可以以此为突破,
了解电场中的能量.

实验发现, 将一个试探电荷$q$从$A$点移动到$B$点, 无论是沿直线移动, 还是沿折线或曲线移动, 静电力做
的功都相等.这就是说, 在静电场中移动电荷时, 静电力所做的功只与电荷的初末位置有关, 而与电荷经过的路径
无关. 因此, 与重力一样, \textbf{静电力属于保守力}.

\subsubsection{电势能}

我们知道, 功和能量的变化密切相关. 例如, 重力做功
等于重力势能的减少量. 节前提到的电荷所减少的能量, 
必定跟静电力做的功相关. 静电力做功具有跟重力做功一样
的特点, 即静电力做功的多少与路径无关, 只与电荷在电场
中的始末位置有关. 电荷在电场中也具有势能, 我们称这种形式的能为\textbf{电势能}, 用$E_\text{p}$表示.

如果用$W_{AB}$表示电荷由 $A$ 点运动到 $B$ 点静电力所做的
功, 用$E_{\mathrm{p}A}$表示电荷在$A$点所具有的电势能, 用$E_{\mathrm{p}B}$表示电荷在$B$点所具有的电势能,
那么它们之间的关系为
$$W_{AB} = E_{\mathrm{p}A} - E_{\mathrm{p}B}.$$
也可以表示为
$$W_{AB} = -\Delta E_\mathrm{p}.$$

当$W_{AB}>0$时, $E_{\mathrm{p}A} > E_{\mathrm{p}B}$, 静电力做正功, 电势能减小;

当$W_{AB}<0$时, $E_{\mathrm{p}A} < E_{\mathrm{p}B}$, 静电力做负功, 电势能增大.

这满足保守力做功的规律.

应该注意, 电势能是相互作用的电荷所共有
的, 或者说是电荷及对它作用的电场所共有的.我们刚才说某
个电荷的电势能, 只是一种简略的说法.

\subsubsection{电势能的相对性}

静电力做的功只能决定电势能的变化量,
而不能决定电荷在电场中某点电势能的数值.只有先把电
场中某点的电势能规定为 0, 才能确定电荷在电场中其他
点的电势能.

通常,我们把电荷在离场源电荷无限远处的电势能规定为 0,
或把电荷在大地表面的电势能规定为 0.

规定离场源电荷$Q$无限远处的电势能为 0, 若将一个电荷$q$从$A$点移动到无限远处,
根据$W_{AB} = E_{\mathrm{p}A} - E_{\mathrm{p}B}$可得$$W_{A\rightarrow \infty} = E_{\mathrm{p}A} - 0 =E_{\mathrm{p}A}.$$
由此可知, \textbf{如果规定无限远处的电势能为 0, 那么
    电场中某电荷的电势能, 等于将该电荷从该点移动到无穷远处静电力所做的功}.

\subsection{电势}

通过实验可知, 置于某一点的试探电荷$q$, 如果它的电荷
量变为原来的$n$倍,其电势能也变为原来的$n$倍. 电势能
与电荷量之比却是一定的, 它是由电场中该点的性质决定的, 与试探电荷本身无关.

与电场强度的定义类似, 试探电荷在电场中某一点的电势能与它的电荷量之比, 叫作
电场在这一点的\textbf{电势}. 如果用$\varphi$表示电势, 用$E_\mathrm{p}$表示试探电荷$q$的电势能,
则
\begin{empheq}[box=\fbox]{equation*}
    \varphi = \frac{E_\mathrm{p}}{q}.
\end{empheq}

在国际单位制中, 电势的单位是\textbf{伏特}, 符号是V.
在电场中的某一点, 如果电荷量为 1 C 的电荷在这点的电势
能是 1 J, 这一点的电势就是 1 V, 即
1 V = 1 J/C.

假如正的试探电荷沿着电场线的方向向外移动,
它的电势能是逐渐减少的. 可以说, \textbf{沿着电
    场线方向电势逐渐降低}.

与电势能的情况相似, 应该先规定电场中某处的电势
为 0, 然后才能确定电场中其他各点的电势.
\textbf{如果规定无穷远处为零电势点, 那么正电荷附近的电势大于0, 负电荷附近的电势小于0}.

电势只有大小, 没有方向, 是个标量.\textbf{多个点电荷在空间某点产生电场的电势, 为每个点电荷在该点产生电势的代数和.}
这就是电势叠加原理.

\subsection{电势差}

选择不同的位置作为零电势点, 电场中某点电势的数
值也会改变, 但电场中某两点之间电势的差值却保持不变.
\setlength{\abovedisplayskip}{0pt}
\setlength{\belowdisplayskip}{0pt}

在电场中, 两点之间电势的差值叫作\textbf{电势差}, 电势差也叫作\textbf{电压}.
设电场中 $A$ 点的电势为 $\varphi_A$, $B$ 点的电势为 $\varphi_B$, 则它们之间的电势差
可以表示为
$$U_{AB} = \varphi_A - \varphi_B.$$
也可以表示为
$$U_{BA} = \varphi_B - \varphi_A.$$
显然
$$U_{AB} = -U_{BA}.$$

电势的值是相对的, 与零电势点的选取有关; 而电势的差值是绝对的, 与零电势点的选取无关.

\subsubsection{静电力做功与电势差的关系}

将点电荷$q$从$A$点移向$B$点, 静电力做的功$W_{AB}$为电荷$q$在这两点所具有的电势能之差.
由此可以导出静电力做功与电势差的关系.
\begin{align*}
    W_{AB}  = E_{\mathrm{p}A} - E_{\mathrm{p}B}
     & =q\varphi_A - q\varphi_B  \\
     & =q(\varphi_A - \varphi_B) \\
     & =qU_{AB}.
\end{align*}
即
\begin{empheq}[box=\fbox]{equation*}
    U_{AB} = \frac{W_{AB}}{q}.
\end{empheq}
或者
\begin{empheq}[box=\fbox]{equation*}
    W_{AB} = qU_{AB}.
\end{empheq}

因此, 知道了电场中两点的电势差, 就可以很方便地
计算在这两点之间移动电荷时静电力做的功, 而不必考虑
静电力和电荷移动的路径. 正是因为这个缘故, 在物理学
中, 电势的差值往往比电势更重要.

\setlength{\abovedisplayskip}{5pt}
\setlength{\belowdisplayskip}{5pt}
\begin{example}
    \label{电势差的减法}
    在静电场中, 将一带电量$q = -1.5\times 10^{-6}\unit{C}$的电荷从$A$点移向$B$点,
    电势能减少$3.0\times 10^{-4}\unit{J}$. 如果将该电荷从$C$点移向$A$点, 克服静电力
    做功$1.5\times 10^{-4}\unit{J}$. 取无穷远处为零电势点.

    (1) 求$A, B$两点间的电势差, $A, C$两点间的电势差和$B, C$两点间的电势差;

    (2)若将此电荷从$A$点移动至无穷远处, 克服静电力做功为$6.0\times 10^{-4}\unit{J}$,
    求电荷在$A$点的电势能;

    (3)若规定$C$点处电势为0, 求$A$点电势和电荷在$B$点的电势能;

    (4)如果将带电量为$2.0\times 10^{-6}\unit{J}$的正电荷从$A$点移向$B$点, 求静电力做的功.
\end{example}
\begin{solution}
    (1) 电荷从$A$点移向$B$点,
    电势能减少$3.0\times 10^{-4}\unit{J}$,说明静电力做功$3.0\times 10^{-4}\unit{J}$.
    根据静电力做功与电势差的关系, $A, B$间的电势差
    $$U_{AB} = \frac{W_{AB}}{q} = \frac{3.0\times 10^{-4}}{-1.5\times 10^{-6}}\unit{V} = -200\unit{V}.$$

    电荷从$C$点移向$A$点, 克服静电力做功$1.5\times 10^{-4}\unit{J}$, 说明从$A$点移向$C$点,
    静电力做功$1.5\times 10^{-4}\unit{J}$.同样, 根据静电力做功与电势差的关系, $A, C$间的电势差
    $$U_{AC} = \frac{W_{AC}}{q} = -100\unit{V}.$$

    根据电势差的定义, 有
    \begin{align*}
        U_{AB} = \varphi_A - \varphi_B, \tag{i - a}\label{例题2-ia} \\
        U_{AC} = \varphi_A - \varphi_C, \tag{i - b}\label{例题2-ib} \\
        U_{BC} = \varphi_A - \varphi_C.\tag{i - c}\label{例题2-ic}
    \end{align*}
    \setlength{\abovedisplayskip}{0pt}
    \setlength{\belowdisplayskip}{0pt}    
    由 \eqref{例题2-ia} \eqref{例题2-ib} 两式可得, $B, C$两点间的电势差
    $$U_{BC} = U_{AC} - U_{AB} = -100\unit{V} - (-200\unit{V}) = 100\unit{V}.$$

    (2) 以无穷远处为零电势点时, 电场中某电荷的电势能, 等于将该电荷从该点移动到无穷远处静电力所做的功.
    因此电荷在$A$点的电势能$$E_{\mathrm{p}A} = W_{A\rightarrow \infty} = -6.0\times 10^{-4}\unit{J}.$$

    (3)规定$C$点处电势为0, 即$\varphi_C = 0$, 那么由 \eqref{例题2-ib} 式可得, $A$点处的电势
    $$\varphi_A = U_{AC} + \varphi_C = U_{AC} = -100\unit{V}.$$

    \setlength{\abovedisplayskip}{5pt}
    \setlength{\belowdisplayskip}{5pt}
    根据电势的定义, $B$点的电势等于电荷$q$在$B$点的电势能除以它的电荷量, 即
    \begin{equation*}
        \varphi_B = \frac{E_{\mathrm{p}B}}{q}.
        \tag{ii}
        \label{例题2-ii}
    \end{equation*}

    由 \eqref{例题2-ic} 式和 \eqref{例题2-ii} 式可得$E_{\mathrm{p}B} = -1.5\times 10^{-4}\unit{J}.$
    \setlength{\abovedisplayskip}{0pt}
    \setlength{\belowdisplayskip}{0pt}

    (4) 将带电量为$2.0\times 10^{-6}\unit{J}$的正电荷从$A$点移向$B$点,静电力做的功为
    $$W_{AB} = q^{\prime}U_{AB} = -1.5\times 10^{-6}\unit\times (-200 \unit{J}) = 3.0\times 10^{-4}\unit{J}.$$
    即静电力做正功$3.0\times 10^{-4}\unit{J}$.

\end{solution}

由例 \ref{电势差的减法} 的 \eqref{例题2-ia} \eqref{例题2-ib} \eqref{例题2-ic} 三式, 可以总结出
三点中各两点间电势差的关系.这个关系看起来就像数学中矢量的加减法, 即
$$U_{AB} + U_{BC} = U_{AC}, $$
$$U_{BC} - U_{AC} = U_{AB}. $$

\section{电场的图形描述}

\subsection{电场线}
前面我们已经了解过了电场线, 现在我们来进一步学习它.

为了形象地描述电场, 我们人为的画出假想曲线.电场线的特点如下.

\begin{enumerate}
    \item 电场线从正电荷或无限远出发, 终止于无限远或
          负电荷.
    \item 电场线在电场中不相交, 这是因为在电场中任意
          一点的电场强度不可能有两个方向.
    \item 电场线上各点的切线方向是该点电场强度的方向.
    \item 电场的疏密反映了电场的强弱, 电场线越密, 电场越强.
    \item 电场线不是客观存在的, 是为了形象描述电场而假想的.
\end{enumerate}

电场线反映了某点电场的强弱.例如, 我们知道, 等量同种电荷连线的中点处电场强度为 0, 因此,
在绘制电场线时, 这一点是空出来的, 代表这一点无电场.

\textbf{电荷运动的轨迹与电场线一般不重合.}若电荷只受静电力的作用,仅在以下情况下两者重合:
\begin{enumerate}
    \item 电荷由静止释放; 或者 
    \item 电荷有初速度, 且初速度方向与电场线方向平行.
\end{enumerate}

\subsection{等势面}

电场中电势相同的各点构成的面叫作等势面.

根据等势面的定义, 要使得同一个面上的各点电势相同, 需使得同一电荷在这个面上各点的电势能相同.
即电荷在这个面上运动时, 静电力不做功, 应有静电力的方向时刻跟等势面垂直, 即与电场线垂直. 否则,
电场强度就有一个沿着等势面的分量, 在等势面上移动电荷时静电力就要做功, 这与这个面
是等势面矛盾. 前面已经说过, 沿着电场线方向电势逐渐降低. 因此, 概括起来就是:
\textbf{电场线跟等势面垂直,并且由电势高的等势面指向电势低的等势面}.

等势面的特点如下.
\begin{enumerate}
    \item 等势面一定跟电场线垂直.
    \item 电场线总是从电势高的等势面指向电势低的等势面.
    \item 不同的等势面不会相交.
    \item 如果等势面是等差的, 那么等势面越密, 电场强度越大.
    \item 在同一等势面上移动电荷时, 静电力不做功.
\end{enumerate}


\section{电势差与电场强度的关系}

电场线是描述电场强度的,等势面是描述电势的.根据上一节的分析我们知道, 电场
线和等势面的密集程度都与电场强度正相关, 换句话说, 电场线和等势面的疏密存在对应关系. 
这表明电场强度和电势之间存在一定的联系. 下面我们以匀强电场为例讨论它们的关系.
\setlength{\abovedisplayskip}{0pt}
\setlength{\belowdisplayskip}{0pt}

\begin{wrapfigure}[13]{o}{6cm}
    \centering
    \includegraphics[width=0.28\textwidth]{pic/pic1.6-1.pdf}
    \caption{讨论匀强电场的电势差与电场强度的关系}
    \label{1.6-1}
\end{wrapfigure}

如图 \ref{1.6-1}, 假设电荷$q$在强度为$E$的\textbf{匀强电场}中沿着静电力$F$的方向从$A$点运动到$B$点, 电荷所受的静电力为$F = qE$. 因为是匀强电场, 所以这个力是恒力, 它所做的功为
\begin{equation}
    W = Fd = qEd.
    \label{静电力做功1}
\end{equation}

除此之外, 静电力做功$W$与$A, B$两点间电势差$U_{AB}$的关系为
\begin{equation}
    W = qU_{AB}.
    \label{静电力做功2}
\end{equation}

比较 \eqref{静电力做功1} 与 \eqref{静电力做功2}, 得到
\begin{empheq}[box=\fbox]{equation*}
    U_{AB} = Ed.
    \label{电势差与电场强度的关系1}
\end{empheq}
即: \textbf{匀强电场中两点间的电势差等于电场强度与这两点沿
    电场方向的距离的乘积.}
\setlength{\abovedisplayskip}{5pt}
\setlength{\belowdisplayskip}{5pt}

电势差与电场强度的关系也可以写作
\begin{empheq}[box=\fbox]{equation*}
    E = \frac{U_{AB}}{d}.
    \label{电势差与电场强度的关系2}
\end{empheq}
它的意义是,在匀强电场中, 电场强度的大小等于两点
之间的电势差与两点沿电场强度方向的距离之比,
也就是说, \textbf{匀强电场的电场强度在数值上等于沿电场强度方向上单位距离的电势差.}
所以说, \textbf{电场强度的
    方向为电场中电势降低最快的方向}.

应该注意, 上面的结论只适用于匀强电场. 对于非匀强电场, 我们可以利用微元的思想得到
$$E = \lim_{d\rightarrow 0} \frac{-\Delta\varphi}{d} = \lim_{d\rightarrow 0} \frac{U}{d}.$$
即电场强度反映电势随空间的变化率,
这类似于加速度与速度的关系.对于非匀强电场的情况, 我们也可以用上式做定性判断.

上式表明, 两相邻等势线之间的电势差 $U$ 相同时, 电
场强度 $E$ 越大的地方, 两相邻等势线之间的距离 $d$ 越小, 这
就是电场线较密的地方等势线也较密的原因.

由于电场强度与电势差的关系存在上述结论, 因此, 已知电场中任意三点的电势时, 可以将电势差最大的两点连线均分.
我们总能在连线上找到一点, 使它与第三点的电势相等. 连接该点与第三点就得到一条等势线, 而与等势线垂直的方向即为电场方向.

\section{静电的防止与利用}
\subsection{静电平衡}

把一个不带电的金属导体放到电场强度为 $E_0$ 的电场中.
由于导体内的自由电子受静电力作用而定向移动, 使导体的两个端面出现等量的异种电荷,
这种现象叫作静电感应.

导体两侧出现的正负电荷在导体内部产生与外电场强
度 $E_0$ 方向相反的附加电场, 其电场强度为 $E^{\prime}$.
这两个电场叠加, 使导体内部的电场减弱. 在叠加后的
电场作用下, 仍有自由电子不断运动, 直到附加电场与外电场完全抵消,
即导体内部各点的电场强度 $E = 0$ 为止,
导体内的自由电子不再发生定向移动. 这时我们说, 导体达到\textbf{静电平衡}状态.

\textbf{处于静电平衡状态的导体, 其内部的电场强度处处为0. 此时整个导体的电势处处相等,
    我们说整个导体是个等势体, 它的表面是个等势面.}

由于处于静电平衡的导体表面是等势面, 因此对于导体产生的附加电场来说, 它表面任意点的电场强度方向与其表面垂直.

\subsection{电荷的分布特点}

由于带有同种电荷量的电荷相互排斥, 所以它们尽可能``互相远离'', 因此带电导体的电荷分布在外表面.

除此之外, 受导体的形状影响, 曲率半径大的地方电荷的密度小, 曲率半径小的地方电荷的密度大. 这就导致
细而尖的地方电荷的分布密度大, 于是这里的电场强度也越大.

在一定条件下, 导体尖端周围的强电场足以使空气中残留
的带电粒子发生剧烈运动, 并与空气分子碰撞从而使空气
分子中的正负电荷分离. 这个现象叫作\textbf{空气的电离}. 那
些所带电荷与导体尖端的电荷符号相反的粒子, 由于被吸
引而奔向尖端, 与尖端上的电荷中和, 这相当于导体从尖
端失去电荷. 这种现象叫作\textbf{尖端放电}.避雷针就是利用了这个原理.

\subsection{静电屏蔽}

我们把一个带空腔的导体置于电场中. 静电平衡时, 根据电荷的分布特点, 这个导体的自由电荷都分布在外表面, 其内表面
没有电荷. 没有电荷也就没有电场, 所以导体内壁的电场强度为0, 即电场线只能在空腔之外, 不能进入空腔之内.
所以导体壳内空腔里的电场强度也处处为0.也就是说,\textbf{导体内部不受外部电场的影响}.
这种现象叫作\textbf{静电屏蔽}.

上面的例子中, 我们屏蔽了外电场. 那么如果带空腔的导体内部有一个带正电的点电荷, 如何屏蔽这个点电荷的电场呢?
在正常情况下, 导体内壁将被感应出负电荷, 其外壁带有等量的正电荷.这些正电荷将形成电场, 并且这些正电荷所带的
电荷量之和等于该点电荷的电荷量.

如果将这个空腔的外表面接地, 那么正电荷将沿导线流向大地. 这样, 电场终于了空腔内表面的负电荷, 我们成功的
屏蔽了内电场. 概括地说, \textbf{接地的封闭导体壳内部的电场对壳外空间没有影响.}

\subsection{静电吸附}

静电虽然会有危害, 但也可以利用. 在电场中, 带电
粒子受到静电力的作用, 向着电极运动, 最后会被吸附在
电极上. 这一原理在生产技术上被广泛应用. 

\textbf{静电除尘} \bre 设法使空气中的尘埃带电, 在静电力作用
下, 尘埃到达电极而被收集起来, 这就是静电除尘. 
静电除尘用于粉尘较多的各种场所, 
除去有害的微粒, 或者回收物资, 如回收水泥粉尘. 

\textbf{静电喷漆} \bre 接负高压的涂料雾化器喷出的油漆微粒带
负电, 在静电力作用下, 这些微粒向着作为正极的工件运
动, 并沉积在工件的表面, 完成喷漆工作. 

\textbf{静电复印} \bre 激光复印机也应用了静电吸附. 复印机的核心
部件是有机光导体鼓, 它是一个金属圆柱, 表面涂覆有机
光导体(OPC)\footnote{20 世纪 90 年代以前没有有机光导体, 那时金属圆柱表面镀硒, 具有同样的
功能, 圆柱叫作硒鼓. 现在仍然有人沿用这个名称. }. 没有光照时, OPC 是绝缘体, 受到光照
时变成导体. 复印时, 有机光导体鼓被激光照射, 形成静电潜像. 
未被照射到的位置保持正电荷, 被照射到的位置形成暗点. 
暗点经过粉末盒, 吸附复印粉, 形成最终的打印图像.

\section{电容}

\subsection{电容器}
\textbf{电容器}是一种重要的电学元件.两个彼此绝缘又相距很近的导体, 可以组成一个电容器.
在两个相距很近的平行金属板中间夹上一层绝缘物质——电介质(空
气也是一种电介质), 就组成一个最简单的电容器, 叫作\textbf{平行
    板电容器}.这两个金属板叫作电容器的\textbf{极板}.

通过实验, 我们发现: 电容器充电的过程中, 接在平行板电容器两端的电压表示数迅速增加,
随后稳定在某一数值, 这表明电容器两极板间有一定的电势差;
通过观察电流表的偏转方向可以知道, \textbf{电流从电源正极流向电容器的正极板,
    同时, 电流从电容器的负极板流向电源的负极}, 这使两极板的电荷量增加,
极板间的电场强度增大, 电源的能量不断储存在电容器中.
随着两极板之间电势差的增大, 充电电流逐渐减小至 0, 此时电容器两极板带有一定的\textbf{等量异种电荷}.
即使断开电源, 两极板上的电荷由于相互吸引而仍然被保存在电容器中.

放电的过程中, 电流从电容器的正极板经过用电器流向电容器的负极板.
此时两极板所带的电荷量减小,
电势差减小, 放电电流也减小, 最后两极板电势差以及放
电电流都等于 0.
电容器把储存的能量通过电流做功转化为电路
中其他形式的能量.

\subsection{电容}
前面的实验表明, 电容器两极板之间的电势差增大时, 电容器所带的电荷量也在增加.
电容器所带的电荷量跟两极板间的电势差是否存在某种定量关系?

精确的实验表面, 一个电容器所带的电荷量$Q$与两极板间的电势差$U$之比是不变的.不同的电容器, 这个比
一般是不同的, 可见电荷量$Q$与电势差$U$之比表征了电容
器本身的特性.

电容器所带的电荷量$Q$
与电容器两极板之间的电势差$U$之比\footnote{当电容器的两个极板的电荷量分别为$+Q$和$-Q$时, 我们认为该电容器所带的电荷量为$Q$.这个$Q$是一个正值.}, 叫作电容器的\textbf{电容}.
用$C$表示, 则有
\begin{empheq}[box=\fbox]{equation}
    C = \frac{Q}{U}.
    \label{电容的定义式}
\end{empheq}

上式表示, 电容器的电容在数值上等于使两个极板间电势差为1 V时电容器需要带的电荷量.
这类似于用不同的容器装水, 要使容器中的水深相同, 横截面积大的
容器需要的水多.

国际单位制中, 电容的单位是\textbf{法拉}, 简称\textbf{法}, 符号是 F.
$$1\unit{F} = \frac{1\unit{C}}{1\unit{V}}.$$
实际中常用的单位还有微法($\mathrm{\upmu F}$)和皮法(pF), 它
们与法拉的关系是
$$1 \unit{\upmu F} = 10^{-6} \unit{F},$$
$$1 \unit{pF} = 10^{-12} \unit{F}.$$

加在电容器两极板上的电压不能超过某一限度, 超过
这个限度, 电介质将被击穿, 电容器损坏.这个极限电压
叫作\textbf{击穿电压}.电容器外壳上标的是工作电压, 或称额定
电压, 这个数值比击穿电压低.

\subsection{平行板电容器}
\subsubsection{平行板电容器}

平行板电容器是最简单的,也是最基本的电容器.几乎所有电容器都是平行板电容器的变形.
平行板电容器的电容是由哪些因素决定的呢?

通过实验可以得出如下结论: 减小平行板电容器两极板的正对面积, 增大两极板之间的距离都能
减小平行板电容器的电容; 而在两极板之间插入电介质, 却能增大平行板电容器的电容.

理论分析表明, 当平行板电容器的两极板之间是真空时, 电容 $C$ 与极板的正对面积 $S$, 极板间
的距离 $d$ 的关系为$$C = \frac{S}{4\uppi kd}.$$
式中$k$是静电力常量.

当两极板之间充满同一种介质时, 电容变大为真空时的 $\varepsilon_r$ 倍, 即
\begin{empheq}[box=\fbox]{equation}
    C = \frac{\varepsilon_rS}{4\uppi kd}.
    \label{平行板电容器的电容}
\end{empheq}
这就是\textbf{平行板电容器电容的决定式}.其中$\varepsilon_r$是一个常数, 与电介质的性质有关,
称为\textbf{相对介电常数}.相对介电常数以真空作为标准.空气的相对介电常数与真空非常接近,
我们在计算时通常也取1.

上式也可以表示为$$C\propto \frac{\varepsilon_rS}{d}.$$
即: 当相对介电常数一定时, 平行板电容器的电容与两极板的正对面积$S$成正比,
与两极板间的距离$d$成反比.

\subsubsection{平行板电容器的电场强度}

平行板电容器的两个极板间带有等量的异种电荷, 我们知道两极板间会产生静电场. 下面我们来
推导平行板电容器的电场强度.

由于平行板电容器的两极板所带正负电荷相等, 两极板平行正对, 各处对应的正负电荷间距离相等,
所以电场线应当是均匀且平行的, 即各点的电场强度的大小和方向均相同.
因此\textbf{平行板电容器两极板间的电场为匀强电场}.\footnote{严格地讲,平行板电容器只有中间部分是匀强电场,而边沿不是.}

根据匀强电场中电势差与电场强度的关系, 电势差为$U$, 距离为$d$两极板间的电场强度 $E = \displaystyle\frac{U}{d}.$\vspace{1ex}
如果这个电容器的电容为$C$, 两极板所带的电荷量为$Q$, 那么由电容的定义式可得 $U = \displaystyle\frac{Q}{C}.$
联立以上两式可得
$$E = \frac{Q}{Cd}.$$
代入平行板电容器电容的决定式 \eqref{平行板电容器的电容} , 得到
\begin{empheq}[box=\fbox]{equation}
    E = \frac{4\uppi kQ}{\varepsilon_r S}.
    \label{平行板电容器的电场强度}
\end{empheq}

值得一提的是, 这个表达式里面, 没有两板间距$d$, 即\textbf{两极板间的电场强度与两极板的距离无关}.
另一方面, 该表达式表明:\textbf{当相对介电常数一定时, 两极板间的
电场强度正比于极板上的电荷面密度$\displaystyle\frac{Q}{S}$}, 即
$$E \propto \frac{Q}{S}.$$
如果用$\sigma$表示电荷的面密度, 那么上式就可以简单的表示为
$$E \propto \sigma.$$

事实上, 带电体附近的电场强度, 本就是直接由带电体上的电荷分布决定的. 由于导体所带电荷只分布在其表面, 因此其附近的电场强度只取决于导体表面的电荷面密度.

根据这一点, 我们就能更直观的分析平行板电容器充放电的机制.

\subsubsection{平行板电容器的充放电机制}

\begin{figure}[bthp]
\centering

\subfloat[充电]{
	\includegraphics[width=0.3\textwidth]{pic/1.7-1.pdf}
    \label{平行板电容器1}
}
\subfloat[放电]{
	\includegraphics[width=0.31\textwidth]{pic/1.7-2.pdf}
    \label{平行板电容器2}
}
	\caption{平行板电容器的充放电}
    \label{平行板电容器3}
    
\end{figure}
先分析充电的机制.

如图 \ref{平行板电容器1}, 开关$S$闭合前, 电容器不带电荷.则由平行板电容器的电场强度公式 \eqref{平行板电容器的电场强度} 和
$U=Ed$ 可知, 电容器两极板间的电势差 $U = 0$. 取电源负极与电容器负极板为零电势, 则有电源正极电势高于电容器正极板电势.
因此, 开关闭合, 就必然在导线中形成顺时针方向的充电电流, 直到电容器正极板与电源正极等势.

当极板间距$d$变小时, 设极板上的电荷量不变, 则由平行板电容器的电场强度公式 \eqref{平行板电容器的电场强度} 和
$U=Ed$ 可知, 两板间电势差$U$减小, 电源正极电势高于电容器正极板电势, 又将在导线中形成顺时针方向的充电电流,
直到电容器正极板与电源正极等势.

由上述分析可知, 电容器\textbf{电容器充电的条件是电容器两极板间的电压低于与其并联部分两端的电压}.

接下来分析放电机制.

如图\ref{平行板电容器2}, 当开关闭合时, 电容器正极板电势高于负极板. 则电容器正极板将通过电阻形成放电电流.

若使两板间距$d$增大, 设极板上的电荷量不变, 则由平行板电容器的电场强度公式 \eqref{平行板电容器的电场强度} 和
$U=Ed$ 可知, 两板间电势差$U$增大, 电容器正极板电势高于电阻$R$上端的电势, 将在导线中形成逆时针方向充电电流,
直到电容器正极板与电阻$R$上端等势.

从上述分析可知, \textbf{电容器放电的条件是电容器两极板间的电压高于与其并联部分两端的电压}.

\subsection{电容器的动态分析}

\begin{wrapfigure}{o}{5cm}
    \flushright
    \includegraphics[width=0.27\textwidth]{pic/1.8-3.pdf}
    \caption{电容器的动态分析}
    \label{1.8-3}
\end{wrapfigure}

\setlength{\abovedisplayskip}{0pt}
\setlength{\belowdisplayskip}{0pt}

将电容为\(C\), 两板间距为\(d\)的平行板电容器接在
电路中, 电源提供的电压恒为$U$.
并将下极板接地(图\ref{1.8-3}).

如果我们将$A$极板向下移动, 那么两极板的间距$d$减小. 由平行板电容器电容的决定式 \eqref{平行板电容器的电容}
可知, 电容$C$增大. 两板间的电压$U$是恒定的, 根据匀强电场中电势差与电场强度的关系$U = Ed$, 可知
电容器中的电场强度$E$减小. 将这个装置竖直放置在重力场内, 再在$M$点放置一带电粒子, 如果移动$A$板前粒子静止(由此可知粒子带负电), 
那么移动$A$板后, 粒子将向下移动.

下面我们来分析$M$点的电势如何变化.

规定大地的电势为0, 则$B$板的电势为0. 移动$A$板前, $A$板与$M$点的电势差
$$U_{AM} = U - U_{MB} = U - \varphi_M.$$
由此可得$M$点的电势
$$\varphi_M = U - U_{AM} = U - Ed_{AM}.$$
因为$U$不变, $E$和$d_{AM}$都减小了, 所以$M$点的电势增加.


\begin{wrapfigure}{o}{5cm}
    \flushright
    \includegraphics[width=0.27\textwidth]{pic/1.8-4.pdf}
    \caption{电容器的动态分析}
    \label{1.8-4}
\end{wrapfigure}

现在, 我们把电容器从电路中取出, 仍把下极板接地(图 \ref{1.8-4}). $A$板带有电荷量为$Q$的电荷.

我们将$A$极板向下移动, 那么两极板的间距$d$减小. 由平行板电容器电容的决定式 \eqref{平行板电容器的电容}
可知, 电容$C$增大. 此时, 两板间的电压不再恒定, 但所带的电荷量$Q$是恒定的. 
根据 \eqref{平行板电容器的电场强度}, 电容器中的电场强度$E$不变. 将这个装置竖直放置在重力场内, 
再在$M$点放置一带电粒子, 如果移动$A$板前粒子静止, 
那么移动$A$板后, 粒子仍静止.

类似前面的分析过程, $M$点的电势为
$$\varphi_M = U - U_{AM} = U - Ed_{AM}.$$

因为$E$不变, $d_{AM}$减小, 所以$M$点的电势减小.

\section{带电粒子在匀强电场中的运动}

\setlength{\abovedisplayskip}{5pt}
\setlength{\belowdisplayskip}{5pt}
分析带电粒子加速的问题, 常常有两种思路: 一种是
利用牛顿第二定律$$qE = ma,$$ 结合匀变速直线运动公式来分析; 另一
种是利用静电力做功, 结合动能定理$$qU = \Delta E_\mathrm{k}$$来分析.

当解决的问题属于匀强电场且涉及运动时间等描述运
动过程的物理量时, 适合运用前一种思路分析; 当问题只
涉及位移, 速率等动能定理公式中的物理量或\textbf{非匀强电场}\footnote{只有在匀强电场中, 粒子做的是匀变速直线运动.}
情景时, 适合运用后一种思路分析.

\subsection{匀强电场中的直线运动}

在匀强电场中,由静止释放一个质量为$m$, 电荷量为$q$的粒子(可视为点电荷). 该粒子从$A$点运动到$B$点,且$A$,$B$两点之间
的电势差为$U_{AB}$. 对粒子从$A$点到$B$点的过程应用动能定理, 有
$$qU_{AB} = \frac12 mv^2.$$
所以, 粒子到达$B$点时的速度$$v = \sqrt{\frac{2qU_{AB}}{m}}.$$

如果这个匀强电场由平行板电容器产生, 两极板间的电压为$U$, 两板间距为$d$, 匀强电场的强度为$E$,
粒子在极板间由静止释放, 加速度为$a$, 则
$$a = \frac{F}{m} = \frac{qE}{m}.$$
由匀强电场中电场强度与电势差的关系$U = Ed$, 得
\begin{empheq}[box=\fbox]{equation*}
    a = \frac{qU}{md}.
\end{empheq}

\subsection{匀强电场中的类平抛运动}

在匀强电场中, 以垂直于电场线的初速度$v_0$发射一个带电量为$q$的粒子.
若不计粒子的重力, 则该粒子在电场中只受垂直于初速度方向的电场力作用.
这类似于平抛运动, 只是重力在这里变成了电场力.

\begin{wrapfigure}[11]{o}{7cm}
    \centering
    \vspace*{8pt}
    \includegraphics[width=0.30\textwidth]{pic/1.8-1.pdf}
    \caption{匀强电场中的类平抛运动}
    \label{匀强电场中的类平抛运动}
\end{wrapfigure}

仍然假设这个匀强电场由平行板电容器产生. 两极板间的电压为$U$,
板长为$L$, 两板间距为$d$ (图\ref{匀强电场中的类平抛运动}). 根据平抛运动的知识, 我们考虑将粒子的运动分解成
沿初速度$v_0$方向的匀速直线运动, 以及沿电场力$F$方向的匀加速直线运动. 粒子沿电场力方向有加速度$a$. 根据牛顿第二定律,
得$$a = \frac{qE}{m} = \frac{qU}{md}.$$

如果粒子沿垂直于电场线的方向从平行板电容器的一侧飞入, 并从另一侧飞出, 那么粒子在电场中沿$v_0$方向
的位移为板长$L$. 沿$F$方向的位移$y$称为粒子在电场中的\textbf{侧移量}.

粒子沿$v_0$方向的运动是匀速直线运动, 有$L = v_0t$.所以粒子的运动时间$$t = \frac{L}{v_0}.$$

粒子沿$F$方向的运动是匀加速直线运动, 加速度$a = \displaystyle\frac{qU}{md}$. 所以粒子的侧移量
\begin{empheq}[box=\fbox]{equation}
    y = \frac12at^2 = \frac{qUL^2}{2mdv_0^2}.
    \label{侧移量方程}
\end{empheq}
这就是理想情况下粒子在平行板电容器内做类平抛运动的\textbf{侧移量方程}.
式中, 板长$L$也可换成任意的水平位移$x$.

粒子沿电场力$F$方向的位移为侧移量$y$, 所以电场力做的功$W = Fy = qEy$. 
由动能定理得$$qEy = \frac12mv^2 - \frac12mv_0^2.$$
所以粒子离开电场时的速度$$v = \sqrt{\frac{2qEy}{m}+v_0^2}.$$

\subparagraph{偏转角规律} 设粒子在电场中的速度偏转角为$\theta$, 位移偏转角为$\varphi$,则根据几何关系有
\begin{equation}
    \label{匀强电场速度偏转角}
    \tan\theta = \frac{v_y}{v_0} = \frac{at}{v_0} = \frac{qUL}{mdv_0^2}.
\end{equation}
$$\tan\varphi = \frac{y}{x} = \frac{\frac12at}{v_0t} = \frac{qUL}{2mdv_0^2}.$$
所以$$\tan\theta = 2\tan\varphi.$$

刚刚我们主要研究的是侧移量$y$. 当粒子沿$v_0$方向的位移不是板长$L$时, 
我们可以换个角度入手, 用$y$表示$x$. 这类似于我们之前研究平抛运动的思路.

由$\displaystyle y = \frac12at^2$, 可得粒子的运动时间$$t = \sqrt{\frac{2y}{a}}=\sqrt{\frac{2my}{qE}}.$$
所以粒子沿$v_0$方向的位移$$x = v_0t = v_0\sqrt{\frac{2my}{qE}}.$$

\subsection{匀强电场中的加速与偏转}

\begin{figure}[htbp]
    \centering
    \includegraphics[width=0.63\textwidth]{pic/1.9-2.pdf}
    \caption{匀强电场中的加速与偏转}
    \label{匀强电场中的加速与偏转}
\end{figure}

现在, 我们将直线运动和类平抛运动结合起来. 带电量为$q$, 质量为$m$的粒子由静止开始, 
先通过电压为$U_1$的加速电场, 再通过板长为$L$, 两板间距为$d$, 电压为$U_2$的偏转电场(图\ref{匀强电场中的加速与偏转}).

如果粒子进入偏转电场时的速度为$v_0$, 由动能定理有$$qU_1 = \frac12mv_0^2.$$
由上式解得$$v_0 = \sqrt{\frac{2qU_1}{m}}.$$

设粒子离开偏转电场时的侧移量为$y$, 把$v_0$代入 \eqref{侧移量方程} 式, 可得
$$y = \frac{qU_2L^2}{2mdv_0^2} = \frac{U_2L^2}{4U_1d}.$$

设粒子离开偏转电场时的速度$v$与$v_0$的夹角为$\theta$, 把$v_0$代入 \eqref{匀强电场速度偏转角} 式可得
$$\tan\theta = \frac{qU_2L}{mdv_0^2} = \frac{U_2 L}{2U_1 d}.$$

由以上过程可知, \textbf{粒子的偏移量$y$, 偏转角度$\theta$与粒子的带电量$q$和质量$m$均无关;
与偏转电场的电压$U_2$成正比, 与加速电场的电压$U_1$成反比}, 即$$y\propto \frac{U_2}{U_1},\bre\bre \tan \theta\propto\frac{U_2}{U_1}.$$

\chapter{电路}
\setlength{\abovedisplayskip}{0pt}
\setlength{\belowdisplayskip}{0pt}

\section{电源和电流}

\subsection{电源}
\label{电源一节}
\begin{wrapfigure}{o}{6cm}
    \flushright
    \includegraphics[width=0.35\textwidth]{pic/2.3-1.pdf}
    \caption{电源的示意图}
    \label{2.3-1}
\end{wrapfigure}

有A, B两个完全相同的金属导体, 分别带有正,负电荷. 用导线H把它们相连,
B失去电子, A得到电子, 导线中产生由A到B的电流.
很快, A, B之间的电势差消失, 成为等势体. 这种情况下,导体H中的电流只是瞬间的.

自由电子的定向运动使两个带电体成为等势体, 自由
电子不能持续定向流动.如何才能使导线 H 中存在持续的
电流呢?

如果在A, B间再增加一个装置P (图 \ref{2.3-1}),这个装置可以把A中的电子取走给B, 使A, B
之间始终存在一定数量的正,负电荷, 所以, A的电势将高于B,
它们之间存在一定的电势差.

能把电荷从A搬运到B的装置P就是\textbf{电源}. A和B分别是电源的正极和负极.
在导线H中, 电子从电源负极移向电源正极.

电闪雷鸣时, 强大的电流使天空不时发出
耀眼的闪光, 但它只能存在于一瞬间; 而手电
筒中的小灯泡却能持续发光, 这是因为手电筒的电路中有电源, 
使得电流能够持续存在.

\subsection{恒定电场}

我们知道, 电流是自由电荷定向移动形成的. 在电源 P 的外部, 导线 H 中的电荷发生定向移动, 这是因为空间内存在电场; 在电源 P 的内部, 有静电力之外的其他力做功\footnote{\ref{电动势一节} 节介绍了这个内容.}, 迁移着电源内部的电荷, 使A, B间存在恒定的电势差. 这样, 形成电场的电荷得以稳定分布着.

当形成电场的电荷稳定分布时, 这个电场也是稳定的. 尽管这些电荷也
在运动, 但有的流走了, 另外的又来补充. 这种稳定分布的电荷形成的稳定的电场, 叫作\textbf{恒定电场}.

需要指出的是, 恒定电场是由运动的电荷形成的; 而静电场是由静止电荷形成的电场. 所以恒定电场并不是静电场.

由于在恒定电场中, 任何位置的电荷分布和电场强度都不随时间变化, 所
以它的基本性质与静电场相同.在静电场中所讲的电势, 电势差及其与电场
强度的关系等, 在恒定电场中同样适用.

处在恒定电场中的导体, 其中的自由电荷发生定向移动. 在移动的过程中,电荷与
导体中不动的粒子不断发生碰撞, 阻碍了电荷的移动, 结果是大量自由电荷定向运动的平
均速率不随时间变化. 如果在这个电路中并联一个电流表, 那么电流表的示数将保持恒定.
我们把大小和方向都不随时间变化的电流称为\textbf{恒定电流}.下面我们就来研究恒定电流.

\subsection{电流}

恒定电流的强弱程度体现为单位时间内通过导体某一横截面的电荷量, 这个物理量叫作\textbf{电流}.
单位时间内通过导体横截面的电荷量越多,电流就
越大. 如果用 $I$ 表示电流, $q$ 表示在时间 $t$ 内通过导体横截
面的电荷量,则有
\begin{empheq}[box=\fbox]{equation*}
    I = \frac{q}{t}.
\end{empheq}

在国际单位制中,电流的单位是\textbf{安培}, 符号是A. 电流
是七个基本物理量之一. 由上式可知
$$1\unit{C} = 1\unit{A}\cdot1\unit{s}.$$

我们规定, 正电荷定向移动的方向为电流的方向. 若电流是
自由电子的定向移动形成的,则电流的方向与自由电子的移动方向相反.
尽管电流有方向, 但这只是我们人为规定的. 电流是个标量, 在运算时
遵循代数法则.

电流的形成条件有三个: 一是导体内存在自由电荷, 二是导体两端存在电势差,
三是电路闭合.

\subsection{电流的微观本质}

通常情况下, 金属中的自由电子不断地做无规则的热运动, 它们朝任何方向运动的机会都一
样.从宏观上看, 没有自由电子的定向移动, 因而也没有电流. 如果导体两端有电势差,
在导体内部就建立了电场, 导体中的自由电子就要受到静电力的作用. 这样, 自由电子在导体中
除了做无规则的热运动外, 还要在静电力的作用下定向移动, 从而形成电流.

设导体的横截面积为 $S$ , 自由电子数密度(单位体积内的自由电子数)为 $n$,
自由电子定向移动的平均速率为 $v$, 则时间 $t$ 内通过某一横截面的自由电子数为 $nSvt$.
由于电子电荷量为 $e$, 因此, 时间 $t$ 内通过横截面的电荷量 $q = neSvt$.根据电流
的公式$I = \displaystyle\frac{q}{t}$, 就可以得到电流和自由电子定向移动平均速率
的关系
$$I = neSv.$$
如果在导体中移动的不是自由电子, 而是平均电荷量为$q$的自由电荷,那么这个关系为
\begin{empheq}[box=\fbox]{equation}
    I = nqSv.
    \label{电流和自由电荷定向移动平均速率的关系}
\end{empheq}
这就是电流的微观解释式.

\begin{example}
    一根横截面积为$S$的均匀长直橡胶棒上带有均匀的电荷,
    每单位长度上的电荷量为$Q$.当此棒沿轴线方向做速度为$v$的匀速直线运动时,
    求由于棒运动而形成的等效电流大小.
\end{example}
\begin{solution}
    棒沿轴线方向做速度为$v$的匀速直线运动, 就相当于棒内的电荷定向移动的
    平均速率为$v$.则电荷通过单位长度$d$的时间$$t = \frac{d}{v}.$$

    由公式 $I = \displaystyle\frac{q}{t}$ 得
    $$I = \frac{qv}{d}.$$ 其中$q$是电荷在单位时间$t$内通过导体的电荷量.
    由题意得 $Q = \displaystyle\frac{q}{d}$, 所以等效电流$$I = Qv.$$

    另一方面, 根据公式 $I = nqSv$\ (其中$n$为电荷数密度, $q$为每个电荷的电荷量, $S$为橡胶棒的横截面积),
    由题意可知, 每单位长度上的电荷数为\(\displaystyle\frac{Q}{q}\), 即\(nS = \displaystyle\frac{Q}{q}\), 因此也有$I = Qv.$
\end{solution}
\begin{example}
    一个半径为$r$的均匀橡胶圈上带有均匀的电荷, 总电荷量为$Q$.当此圈
    以圆心为轴做线速度为$v$的匀速运动时,求由于圈转动而形成的等效电流大小.
\end{example}
\begin{solution}
    圈以圆心为轴做线速度为$v$的匀速圆周运动, 就相当于其中的电荷做先速度为$v$的匀速圆周运动.
    则电荷转过一圈的时间$$t = \frac{2\uppi r}{v}.$$
    由已知可得, 等效电流$I$,圈上的总电荷量$Q$与电荷转过一圈的时间$t$的关系为$I = \displaystyle\frac{Q}{t}$.因此$$I = \frac{Qv}{2\uppi r}.$$
\end{solution}

\section{导体的电阻}
\subsection{电阻}
选取一个导体,研究导体两端的电压随电流的变化情况.实验发现,同一个导体,\vspace{5pt}
无论电流$U$, 电压$I$怎样变化, $\displaystyle\frac{U}{I}$都是一个常量.
可以看出, 当电压$U$不变时,$\displaystyle\frac{U}{I}$越大, 电流$I$越小.
可见,这个量反映了导体对电流的阻碍程度, 物理学把它叫作导体的\textbf{电阻}\footnote{一个导体接在电路中, 通常通过自身发热将电能转化为热, 这样的元件称为\textbf{电阻器}, 简称电阻. 说到``电阻''时, 要根据上下文判断指的是导体本身, 还是导体的电阻.},
用$R$表示, 即
\setlength{\abovedisplayskip}{5pt}
\setlength{\belowdisplayskip}{5pt}
\begin{empheq}[box=\fbox]{equation*}
    R = \frac{U}{I}.
\end{empheq}

从上式中解出电流$I$, 可以得到
\begin{empheq}[box=\fbox]{equation*}
    I = \frac{U}{R}.
\end{empheq}
这就是我们初中学过的\textbf{部分电路的欧姆定律}, 它表示\textbf{在一段电路中, 
通过某段导体的电流跟这段导体两端的电压成正比, 跟这段导体的电阻成反比.}\footnote{正是因为有这个规律, 所以我们才用电压与电流之比来定义电阻.}

\subsection{影响电阻的因素}

实验发现, \textbf{在一定温度下, 同种材料的导体, 其电阻$R$与它的长度$l$成正比,
    与它的横截面积$S$成反比;导体电阻还与它的材料有关}.这个规律称为\textbf{电阻定律}.
写成公式则是
\begin{empheq}[box=\fbox]{equation}
    R = \rho\frac{l}{S}.
    \label{电阻定律}
\end{empheq}
其中, $\rho$与导体的材料性质有关.不同材料的
导体$\rho$一般不同.由上式可知, 在长度,横截面积不变的条件下,
$\rho$越大, 导体的电阻越大. $\rho$叫作这种材料的\textbf{电阻率}.

在国际单位制中, 电阻率的单位是$\Omega/\mathrm{m}$.

电阻率是反映导体材料导电性能的物理量. 各种材料的电阻率都随温度变化而变化.
金属的电阻率随温度的升高而增大; 有些合金(比如锰铜,镍铜)的电阻率几乎不受温度影响;
半导体和电介质的电阻率随温度的升高而减小.

\subsection{导体的伏安特性曲线}
在实际应用中, 常用横坐标表示电压 $U$, 纵坐标表示电流 $I$, 这样画出的 $I$-$U$ 图像叫作
导体的\textbf{伏安特性曲线}. 对于金属导体, 在温度没有显著变化时, 电阻几乎是不变的(不随电流和电压改变), 它的伏安特性曲线是一条过原点的直线, 也就是电流 $I$ 与电压 $U$ 成正比(图\ref{14.2-10}).具有这
种伏安特性的电学元件叫作\textbf{线性元件}.

实验表明, 除金属导体外, 欧姆定律对电解质溶液也适用, 但对气态导体 (如日光灯管, 霓虹灯管
中的气体) 和半导体元件并不适用. 也就是说, 在这些情况下电流与电压不成正比, 
这类电学元件叫作\textbf{非线性元件}.

\begin{figure}[bthp]
\centering
	\includegraphics[width=0.31\textwidth]{pic/14.2-10.pdf}
	\caption{线性元件的伏安特性曲线}
    \label{14.2-10}
\end{figure}

\section{电路}

导线, 电源和用电器相连, 组成了电流可以流过的路径, 叫作\textbf{闭合电路}.
用电器和导线组成\textbf{外电路},电源内部是\textbf{内电路}.

\subsection{串联电路与并联电路}

如图\ref{电路图1}, 把几个导体(用电器)依次首尾连接, 再接入电路.这样的连接方式叫作\textbf{串联}.

如图\ref{电路图2}, 把几个导体(用电器)的一端接在一起, 另一端也接在一起, 再将两端接入电路. 这样的连接方式叫作\textbf{并联}.并联电路中两个导体(用电器)共用的那部分电路叫\textbf{干路},单独使用的那部分电路叫\textbf{支路}.

\begin{figure}[bthp]
\centering

\subfloat[串联电路]{
	\includegraphics[width=0.3\textwidth]{pic/14.3-0.png}
    \label{电路图1}
}
\subfloat[并联电路]{
	\includegraphics[width=0.31\textwidth]{pic/14.3-1.png}
    \label{电路图2}
}
	\caption{串联和并联}
    \label{电路图3}
    
\end{figure}

\subsubsection{串并联电路中的电流}

我们知道, 恒定电流电路中各处电荷的分布是稳定
的, 任何位置的电荷都不可能越来越多或越来越少.在图 \ref{电路图1} 所示的串联电路中, 
既然电路中各处的电荷分布保持不变, 相同时间内通过0, 1, 2, 3各点的电荷量必然相
等. 因此, \textbf{串联电路中的电流处处相等}.

在图 \ref{电路图2} 所示的并联电路中,只有在相同时间内流过干
路的电荷量等于进入各支路各点的电荷量之
和,才能保证电路各处的电荷量的分布保持不变.因此,
\textbf{并联电路的总电流等于各支路电流之和.}

\subsubsection{串并联电路中的电压}

在图 \ref{电路图1} 所示的串联电路中, 用$\varphi_i$表示$i$点处的电势, 用$U_{ij}$表示$i$, $j$两点之间的电势差, 那么
$$U_{01} = \varphi_0 - \varphi_1,\bre U_{12} = \varphi_1 - \varphi_2,\bre U_{23} = \varphi_2 - \varphi_3.$$
因此$$U_{01}+U_{12}+U_{23} = \varphi_0 - \varphi_3 = U_{03}.$$
即, \textbf{串联电路两端的总电压等于各部分电路两端电压之和}.

在图 \ref{电路图2} 所示的并联电路中, 不考虑导线的电阻, 0, 1, 
2, 3 各点之间没有电势差, 它们具有相同的电势. 同样, 
几个电阻右边的电势也相同. 因此, \textbf{并联电路的总电压与
各支路电压相等. }

\subsubsection{串并联电路中的电阻}

\begin{figure}[bthp]
\centering

\subfloat[电阻的串联]{
	\includegraphics[width=0.42\textwidth]{pic/14.3-20.pdf}
    \label{电路图4}
}
\subfloat[电阻的并联]{
	\includegraphics[width=0.3\textwidth]{pic/14.3-21.pdf}
    \label{电路图5}
}
	\caption{电阻的串联和并联}
    \label{电路图6}
    
\end{figure}

两个电阻 $R_1$, $R_2$ 串联起来接到电路里,作为一个整体,
它相当于一个电阻 $R$ (图\ref{电路图4}).这个电阻的大小与原来
两个电阻的大小有什么关系?

由于 $R_1$ 与 $R_2$ 是串联的, 它们两端的总电压 $U$ 等于两个
电阻两端电压 $U_1$, $U_2$ 之和, 即$$U = U_1 + U_2.$$

由$R = \displaystyle\frac{U}{I}$, 通过这两个电阻的电流 $I$ 相同, 上式两边都
除以电流 $I$, 即
$$\frac{U}{I} = \frac{U_1}{I}+\frac{U_2}{I}.$$
可得
\begin{empheq}[box=\fbox]{equation*}
    R = R_1+R_2.
\end{empheq}

不难证明, 如果 $n$ 个电阻串联, 那么
$$R = R_1+R_2+\cdots +R_n.$$
即, \textbf{串联电路的总电阻等于各部分电路电阻之和}.

两个电阻 $R_1$, $R_2$ 并联接到电路里, 作为一个整体,
它相当于一个电阻 $R$ (图\ref{电路图5}). 通过它们的总电流 $I$ 等于
通过两个电阻的电流 $I_1$, $I_2$ 之和, 即$$I = I_1+I_2.$$

由$R = \displaystyle\frac{U}{I}$, 两个电阻上的电压 $U$ 相同, 上式两边都
除以电压 $U$, 即$$\frac{I}{U} = \frac{I_1}{U}+\frac{I_2}{U}.$$
可得
\begin{empheq}[box=\fbox]{equation*}
    \frac{1}{R} = \frac{1}{R_1}+\frac{1}{R_2}.
\end{empheq}
上式也可以整理为
\begin{empheq}[box=\fbox]{equation*}
    R = \frac{R_1R_2}{R_1+R_2}.
\end{empheq}

不难证明, 如果 $n$ 个电阻并联, 那么
$$\frac{1}{R} = \frac{1}{R_1}+\frac{1}{R_2}+\cdots + \frac{1}{R_n}.$$
即, \textbf{并联电路总电阻的倒数等于各支路电阻的倒数之和}.

\subsubsection{分压原理\bre 分流原理}
我们知道, 在串联电路中, 电流处处相等. 把电阻为$R_1$和$R_2$的两个导体串联起来接到电路里(图\ref{电路图4}), 
它们两端的电压$U_1$和$U_2$有什么关系呢?

设流经这段电路的电流为$I$, 根据部分电路的欧姆定律,有 $$U_1 = IR_1, \bre \bre U_2 = IR_2,$$
把它们联立求解, 消去$I$, 就得到
\begin{empheq}[box=\fbox]{equation*}
    \frac{U_1}{U_2} = \frac{R_1}{R_2}.
\end{empheq}
不难证明, 如果是$n$个电阻串联, 那么
$$U_1:U_2:\cdots :U_n = R_1:R_2:\cdots :R_n.$$
即, \textbf{串联电路中各导体两端的电压之比等于它们的电阻之比.}这就是\textbf{串联电路的分压原理}.

除此之外, 在并联电路中, 各支路的电压相等.把电阻为$R_1$和$R_2$的两个导体并联接在一起(图\ref{电路图5}), 根据部分电路的
欧姆定律, 流经这两个支路的电流$I_1$和$I_2$满足$$I_1R_1= I_2R_2.$$
即\begin{empheq}[box=\fbox]{equation*}
    \frac{I_1}{I_2} = \frac{R_2}{R_1}.
\end{empheq}
同理, 如果 $n$ 个电阻并联, 那么
$$I_1R_1= I_2R_2 = \cdots = I_nR_n.$$
即, \textbf{并联电路中流经各支路的电流与各支路上的电阻成反比}.这就是\textbf{并联电路的分流原理}.

\subsection{电压表和电流表}

在电路实验当中, 常常需要使用电流表和电压表分别测量电路的电流和电压.
在这一小节, 我们学习如何用表头分别改装成电流表和电压表.

表头是量程很小的电流表, 又叫作灵敏电流计, 通常用G表示. 表头自身有电阻$R_\text{g}$, 称为表头的内阻.
表头接入电路时, 与普通的电阻无异, 只是它能显示出流过自身的电流.

当表头的指针转到最大刻度线时, 通过表头的电流$I_\text{g}$叫作表头的满偏电流, 
此时加在它两端的电压$U_\text{g}$叫作表头的满偏电压.$R_\text{g}$, $I_\text{g}$, 
$U_\text{g}$三者之间的关系满足
\begin{equation}
    U_\text{g} = I_\text{g}R_\text{g}.
    \label{表头的内阻公式}
\end{equation}

\subsubsection{电流表的改装}
我们知道, 将表头串联在电路中, 它可以显示出流过自身的电流, 但它的量程一般都很小.
因此, 在测量较大的电流时, 应并联一个小电阻$R$作为分流电阻, 
能够使得表头满偏时, 通过电流表两端的电流$$I = \frac{U_\text{g}}{\dfrac{1}{R_\text{g}}+\dfrac{1}{R}}= U_\text{g}\frac{R_\text{g}+R}{R_\text{g}R}$$
远大于表头的满偏电流$I_\text{g}$, 也就是并联后的总电阻远小于表头的内阻$R_\text{g}$.
这样, 我们就能利用表头测出比较大的电流. 

{\kaishu 并联一个阻值$R$为分流电阻, 电流表的量程相比于表头扩大了几倍?}

如果我们要将电流表的量程从$I_\text{g}$扩大到$I$, 设要并联的电阻为$R$, 那么
根据并联电路的分流原理有$$\frac{I_\text{g}}{I-I_\text{g}} = \frac{R}{R_\text{g}}.$$
由此可得, 分流电阻$$R = \frac{I}{I-I_\text{g}}R_\text{g}.$$

若用$n$表示量程的扩大倍数, 那么上式就是$$R = \frac{1}{n-1}R_\text{g}.$$
从中解出扩大倍数$n$, 有$$n = \frac{R_\text{g}+R}{R}.$$
此时改装成的电流表的内阻为$$R_\text{A} = \dfrac{1}{R_\text{g}}+\dfrac{1}{R}=\frac{R_\text{g}R}{R_\text{g}+R} = \frac{1}{n}R_\text{g}.$$
因此, 电流表的电阻$R_\text{A}$一般很小, 接近且略小于分流电阻$R$. $R$越小, 
电流表的量程越大.

\subsubsection{电压表的改装}
表头可测出流过自身的电流, 我们可以通过 \eqref{表头的内阻公式} 式进而计算出它两端的电压.
为了测量电路某一部分的电压, 我们把表头并联在这部分电路的两端, 从而保证表头上的电压与
这部分电路的电压相等. 因此, \textbf{电压表是并联使用的.}

由于表头的满偏电压一般都很小, 在测量较大的电压时, 应串联一个电阻$R$作为分压电阻, 
使得表头满偏时, 加在电压表两端的电压
$$U = I_\text{g}(R_\text{g}+R)$$
远大于表头的满偏电压$U_\text{g}$, 也就是串联后的总电阻远大于表头的内阻$R_\text{g}$.
这样, 我们就能利用表头测出比较大的电压.

{\kaishu 串联一个阻值$R$为分压电阻, 电压表的量程相比于表头扩大了几倍?}

如果我们要将电压表的量程从$U_\text{g}$扩大到$U$, 设要串联的电阻为$R$, 那么
根据串联电路的分压原理有$$\frac{U_\text{g}}{U-U_\text{g}} = \frac{R_\text{g}}{R}.$$
由此可得, 分压电阻$$R = \left(\frac{U}{U_\text{g}}-1\right)R_\text{g}.$$

若用$n$表示量程的扩大倍数, 那么上式就是$$R = \left(n-1\right)R_\text{g}.$$
从中解出扩大倍数$n$, 有$$n = \frac{R_\text{g}+R}{R_\text{g}}.$$
此时改装成的电压表的内阻为$$R_\text{V} = R+R_\text{g} = nR_\text{g}.$$
因此, 电压表的电阻$R_\text{V}$一般很大, 接近且大于分压电阻$R$. $R$越大, 
电压表的量程越大.


\subsection{等效电路}

\begin{figure}[bthp]
\centering

\subfloat[简化前的电路]{
	\includegraphics[width=0.65\textwidth]{pic/2.4-1.pdf}
    \label{电路图7}
}\\
\subfloat[用``节点法''标注的电路]{
	\includegraphics[width=0.65\textwidth]{pic/2.4-2.pdf}
    \label{电路图8}
}\\
\subfloat[等效电路]{
	\includegraphics[width=0.65\textwidth]{pic/2.4-3.pdf}
    \label{电路图9}
}
	\caption{等效电路}
    \label{电路图10}
    
\end{figure}

连接如图\ref{电路图7} 所示的电路, 电流从1流向2. 为了算出1, 2间的等效电阻,
我们需要画出1, 2间一个简化的\textbf{``等效电路''}. 

通过分析可知, $R_1$与$R_5$串联, $R_2$与$R_5$串联; $R_1$与$R_2$并联,
$R_4$与$R_5$并联; $R_3$直接接在1,2的两端, 如图\ref{电路图9} 所示.

除了一个一个的分析, 我们还有技巧性的方法.
我们将电路中电势相等的点用相同的字母表示, 如图\ref{电路图8} 所示.
其中$\varphi_A>\varphi_B>\varphi_C$.

因为电路中从1到2电势逐渐降低, 所以它等效于图\ref{电路图9} 所示的电路.

这与前面分析的结果是一致的.并且因为相同的字母表示电势相等的点,
所以上支路和下支路的总电阻相等.
这种方法称为``节点法''.

\section{电路中的能量转化}

\subsection{电功}

我们知道, 电能转化为其他形式的能, 是通过电流做
功来实现的.电流做的功也叫作\textbf{电功}.

\textbf{电流做功的实质是, 导体中的恒定电场对自由电荷的静电力在做功. 自由电荷在静电力的作用下做定向移动, 
结果是电荷的电势能减少, 其他形式的能增加. }

考虑一段电路中有电荷定向移动, 电流为$I$, 则在时间$t$内流过该电路的电荷量为
$$q = It.$$

如果这段电路两端的电势差为$U$, 那么静电力做的功就是
$$W = Uq = UIt.$$

上式表示\textbf{电流在一段电路中所做的功, 等于这段电路
两端的电压 $U$,电路中的电流 $I$,通电时间 $t$ 三者的乘积.}

\subsection{电功率}
电流在一段电路中所做的功与通电时间之比叫作\textbf{电功率},用 $P$ 表示.
由$P = \displaystyle\frac{W}{t}$, 进而得到
\begin{empheq}[box=\fbox]{equation*}
    P = UI.
\end{empheq}

这个公式表示,\textbf{电流在一段电路中做功的功率 $P$ 等于
这段电路两端的电压 $U$ 与电流 $I$ 的乘积.}

其中, 电流, 电压和时间的单位分别是安培(A), 伏
特(V)和秒(s), 电功和电功率的单位分别是焦耳(J)
和瓦特(W).

\subsection{焦耳定律}
我们知道, 电流做功一定会产生热. 如果电能全部转化为导体的内能, 
那么电流在这段电路中做的功$W$等于这段电路产生的热量$Q$, 即
$$Q = W = UIt.$$
这样的电路称为\textbf{纯电阻电路}.电热水器这类用电器内部的电路就是纯电阻电路,
它消耗的电能全部转化为了内能.

由欧姆定律$U = It$, 可以得到热量$Q$的表达式
\begin{empheq}[box=\fbox]{equation*}
    Q = I^2Rt.
\end{empheq}
即, \textbf{电流通过导体产生的热量跟电流的二次方成正比, 跟
导体的电阻及通电时间成正比.}
这个结论是焦耳通过实验直接得出的, 称为\textbf{焦耳定律}, 电流做功产生的热称为\textbf{焦耳热}. 
实验表明, 焦耳定律不仅对纯电阻电路适用, 对一般的电路也适用. 下一小节我们将讨论一般电路中的能量转化, 
便可以理解为什么焦耳定律对一般电路也适用.

由于$W = Q$, 所以此时的电功率
$$P = \frac{W}{t} = \frac{Q}{t}$$
就是电流通过导体发热的功率
\begin{empheq}[box=\fbox]{equation}
    P_\text{热} = I^2R.
    \label{焦耳定律}
\end{empheq}
这是焦耳定律更常用的一种形式.

\subsection{电路中的能量转化}

在上面的推导过程中, 我们用到了$W = Q$这个条件, 即电流做的功``全部变成热''.
如果电路中只有电阻, 没有其他形式的能量输出, 那么电流做的功就会全部变成热, 比如电热水器.

实际中, 有些电路除含有电阻外还含有其他负载, 如电动
机. 下面我们以电动机为例, 讨论一下电路中的能量转化.

设电动机消耗的功率为 $P_\text{电}$, 电动机对外做功, 输出的
功率为 $P_\text{机}$, 另外, 电动机工作时自身也有能量损失, 对应
的功率为 $P_\text{损}$, 它们之间满足$$P_\text{电} = P_\text{机}+P_\text{损}.$$

设电动机两端的电压为$U$, 通过电动机线圈的电流为$I$, 则
$$P_\text{电} = UI.$$

电动机刚停止工作时, 我们发现外壳是热的, 说明工
作时有电能转化为内能. 这是为什么呢? 事实上, 电动机内部的线圈有电阻$R$, 
这导致电动机工作时必然会把一部分电能转化为内能. 我们可以把电动机看作是一个
无内阻的理想电动机和一个电阻为$R$的导体的串联, 这样, 理想电动机就是一个两端有电压\footnote{对于一般的导体, 其两端有电势差是由于减少的电势能转化为了电流通过导体发热的内能; 对于理想电动机来说, 减少的电势能转化为了电动机输出的机械能.}, 但无电阻的理想导体. 如果理想电动机两端的电压为$U_1$, 电阻$R$两端的电压为$U_2$, 那么电动机两端的电压$$U = U_1+U_2.$$
对于电阻$R$有部分电路的欧姆定律$$I = \frac{U_2}{R}, $$
所以电流通过电阻$R$发热的功率为$$P_\text{热} = I^2R = \frac{U_2^2}{R} = IU_2.$$
由于串联电路中电流处处相等, 因此可以说, $P_\text{热} = I^2R$就是电动机发热的功率\footnote{但是$\displaystyle\frac{U^2}{R}$或$IU$却并不是电动机发热的功率.}.

电动机的转子与轴承均有摩擦,另外还有空气阻力.但若忽略这部分能量损
失,只考虑线圈发热产生的能量损失,则有$P_\text{损} = P_\text{热}$, 
于是$$P_\text{机} = P_\text{电} - P_\text{热} = UI - I^2R.$$

同样,对于正在充电的电池,电能除了转化为化学能之
外, 还有一部分转化为内能.

\section{电源的电动势和内阻}

\subsection{电源内的能量转化}

电子向某一方向的定向移动相当于正电荷沿相反方向的定向移动. 为了方便, 我们下面以正电荷为例讨论.

如 \ref{2.3-1} 节的图 \ref{2.3-1} 所示, 在导线H中, 正电荷从电源正极移向电源负极;
在电源P内, 正电荷要从电源负极移向电源正极. 但是, 电源正极的电势高于负极的电势, 
正电荷受到的静电力阻碍电子继续向正极移动.因此, 在电源内部要使正电荷向正极移动,
就一定要有与静电力方向相反的其他力做功才行, 这个力叫作\textbf{非静电力}.

在干电池中, 非静电力是化学作用;
在发电机中, 非静电力是电磁作用. 电源内部的非静电力做功,
将正电荷移送到电势高的电极, 使电荷的电势能增加. 所以, 从能量转化的角度看,
\textbf{电源是通过非静电力做功将其他形式的能转化为电势能的装置}.
非静电力对电荷做的功等于电荷电势能的增量.

\subsection{电动势}
\label{电动势一节}

在电源内部, 电源移动电荷, 增加电荷的电势能.在物
理学中, 我们用非静电力所做的功与所移动的电荷量之比来
表示电源的这种特性,叫作\textbf{电动势}.

电动势在数值上等于非静电力把1\ C的正电荷在电源内从负极搬运到正极所做的功,
如果移动电荷量$ q $时非静电
力所做的功为 $W$,那么,电动势$ E $表示为\footnote{根据国家标准, 电动势也用字母$E$表示, 使用时要注意与电场强度$E$的区别.}
\begin{empheq}[box=\fbox]{equation}
    E = \frac{W}{q}.
    \label{电动势的定义}
\end{empheq}
电动势由电源内部非静电力的特性决定, 与外电路无关.

电动势与电压的单位相同, 但它们的物理意义不同.电动势是非静电力做功产生的,
电压是静电力做功产生的.

\subsection{内阻}
电源内部也是由导体组成的, 所以也有电阻, 叫作电源的\textbf{内阻},用$r$表示.
当电源接入电路时, 有内阻$r$的非理想电源等效于一个无内阻的理想电源与电阻$r$串联.

电动势和内阻都是由电源本身决定的.

电源的内阻一般是很小的. 如果把电源两端用导线连接起来, 那么电源的内阻将流过
很大的电流, 从而产生很多的热量,称为\textbf{电源短路}.这不仅对电源有损害, 还有可能
发生火灾.

\section{闭合电路欧姆定律}

\subsection{闭合电路欧姆定律}

我们已经知道部分电路的欧姆定律$$I = \frac{U}{R}.$$
它表示:流过导体的电流与导体两端的电压成正比, 与导体的电阻成反比.

对于闭合电路而言, 在外电路中, 正电荷在恒定电场
的作用下由正极移到负极;在电源内部, 非静电力把正电
荷由负极移到正极.

正电荷在静电力的作用下从电势高的位置向电势低的
位置移动, 电路中正电荷的定向移动方向就是电流的方向, 
所以, 在外电路中, \textbf{沿电流方向电势降低}.

我们知道, 电源可以看作一个没有电阻的理想
电源与电阻的串联, 这个电阻的电势也会沿电
流方向降低. 

下面我们来推导, 在只有电阻的闭合电路中, 电流 $I$ 跟
电源的电动势 $E$, 内阻 $r$和外电路的电阻 $R$ 之间的关系.

对于电源来说, 非静电力做功将其他形式的能
转化为电能. 所以非静电力做功与电源输出的电能相等.
在时间$t$内, 电源输出的电能为
\begin{equation}
    W = Eq = EIt.
    \label{欧姆定律推导1}
\end{equation}

电流通过电阻$R$时, 电流做功生热, 电能转化为内能. 在时间$t$内, 外电路转化的
内能为
\begin{equation}
    Q_\text{外} = I^2Rt.
    \label{欧姆定律推导2}
\end{equation}
同理, 当电流通过电源内阻$r$时, 内电路转化的内能为
\begin{equation}
    Q_\text{内} = I^2rt.
    \label{欧姆定律推导3}
\end{equation}

因为外电路是纯电阻电路, 所以电能全部转化为热, 即
\begin{equation}
    W = Q_\text{外}+Q_\text{内}.
    \label{欧姆定律推导4}
\end{equation}

把 \eqref{欧姆定律推导1}, \eqref{欧姆定律推导2},
 \eqref{欧姆定律推导3} 三式代入 \eqref{欧姆定律推导4}, 
有$$EIt = I^2Rt + I^2rt,$$
即$$E = IR + Ir.$$

也就是
\begin{empheq}[box=\fbox]{equation}
    I = \frac{E}{R + r}.
\end{empheq}

上式表示:\textbf{闭合电路的电流跟电源的电动势成正比, 
跟内外电路的电阻之和成反比}.这个结论叫作\textbf{闭合电路
的欧姆定律}.

如果用$U_\text{外}$表示$IR$, $U_\text{内}$表示$Ir$, 
则闭合电路的欧姆定律也可以写为
\begin{empheq}[box=\fbox]{equation}
    E = U_\text{外} + U_\text{内}.
    \label{闭合电路欧姆定律2}
\end{empheq}
这就是说, 电源的电动势等于内外电路电势降落之和\footnote{这个规律的本质是能量守恒定律, 
因此不仅适用于纯电阻电路, 也适用于非纯电阻电路.}.

\subsection{路端电压和负载的关系}

我们把外电路的电势降落叫作\textbf{路端电压}, 外电路的用电器叫作\textbf{负载}.

当外电路的电阻$R$增大时, 根据闭合电路欧姆定律$$I = \frac{E}{R+r}, $$
电路中的总电流$I$减小; 根据部分电路欧姆定律, 内电路电压$U_\text{内} = Ir$, 
所以内电路的电压$U_\text{内}$减小.
\begin{figure}[bthp]
\centering
\subfloat[电源的$U$-$I$图像]{
    \includegraphics[width=0.3\textwidth]{pic/2.6-1.pdf}
    \label{路端电压1}
}
\subfloat[工作点]{
    \includegraphics[width=0.3\textwidth]{pic/2.6-2.pdf}
    \label{路端电压2}
}
	\caption{路端电压$U$与电流$I$的关系}
    \label{路端电压3}
    
\end{figure}

我们把路端电压记为$U$, 根据 \eqref{闭合电路欧姆定律2}, 有
\begin{empheq}[box=\fbox]{equation}
    U = E - Ir.
    \label{路端电压}
\end{empheq}
这就是路端电压的表达式\footnote{这个式子对纯电阻电路和非纯电阻电路都适用.}. 由此可知, 当外电路的电阻$R$增大时, 路端电压也增大.

由 \eqref{路端电压} 可知, 路端电压$U$与电流$I$成线性关系. 画出$U$与$I$的关系图像(图 \ref{路端电压1}), 我们称之为电源的$U$-$I$图像. 它表示这个电源外电路的特性曲线, 即路端电压$U$随电流$I$变化的曲线. 

曲线在$U$轴上的截距为电源的电动势$E$, 这表明, 电源在未接入
电路时, 其两端的电压就是电源的电动势. 曲线的斜率是$-r$, 
电源的内阻越大, 斜率的绝对值越大.

将外电路电阻的伏安特性曲线也画在该坐标系中, 它和电源的$U-I$曲线有一个交点$P$ (图\ref{路端电压2}). 这个交点$P$对应的横纵坐标即为
该电阻接在此电源下实际工作的电流和电压. 我们把这个点称为该电阻在此电源下的\textbf{工作点}.
\setlength{\abovedisplayskip}{0pt}
\setlength{\belowdisplayskip}{0pt}
\section{电源的功率与效率}

电源中的非静电力做功的功率, 称为电源的总功率或输入功率, 如果用$P_{E}$表示, 则
$$P_{E} = \frac{W}{t} = \frac{Eq}{t} = EI.$$

电流通过电源内阻$r$做功会产生热,这个功率称为电源内阻的热功率, 也称为电源的损耗功率.
如果用$P_{r}$表示, 则$$P_{r} = I^2r.$$

电源的总功率与损耗功率之差, 就是电源的输出功率, 也即电流通过外电路做功的功率.
如果用$P_{R}$表示电源的输出功率, $U$表示路端电压, 那么
$$P_{R} = UI = IE - I^2r.$$
这是一个关于$I$的二次函数, 当$I = \dfrac{E}{2r}$时, $P_R$取得最大值$\dfrac{E^2}{4r}$.

\begin{figure}
    \centering
    \includegraphics[width=0.35\textwidth]{pic/2.7-1.pdf}
    \caption{电源的输出功率与电阻的关系}
    \label{2.7-1}
\end{figure}

特别地, 当外电路是纯电阻电路时, 设外电路总电阻为$R$, 由欧姆定律有$U = IR$, 所以
\vspace{3pt}$$P_{R} = UI = I^2R = \frac{E^2R}{(R+r)^2} = \frac{E^2}{\displaystyle\frac{(R-r)^2}{R}+4r}.\vspace{10pt}$$
此时$P_{R}$与$R$的关系如图\ref{2.7-1} 所示. 由此可知, 在电源内阻$r$不变的情况下, 当$R = r$时, 电源的输出功率最大,最大值为$\displaystyle\frac{E^2}{4r}$;\vspace{2pt} 除去这一点之外, 
每个输出功率$P_{R}$都有两个外电阻$R_1$, $R_2$与之对应, 并且可以证明$R_1R_2 = r^2.$

\subsection{功率与电流的关系}

我们知道, 电源的总功率$P_{E}$与电流$I$的关系为
$$P_{E} = EI.$$
电源的损耗功率(即内阻的发热功率) $P_{r}$与电流$I$和内阻$r$的关系为
$$P_{r} = I^2r.$$
当外电路为纯电阻电路时, 电源的输出功率(即外电路电阻的发热功率)$P_{R}$与电流$I$和外电路电阻$R$的关系为
$$P_{R} = I^2R.$$

根据上面的关系, 我们选取三个特殊点.

当$R$无穷大时, 电流$I = 0$.此时电源的总功率, 损耗功率, 输出功率均为0.

当$R = r$时, 电流$$I = \frac{E}{R+r} = \frac{E}{2r}.$$
此时三个功率分别为
$$P_{E} = EI = \frac{E^2}{2r}, $$
$$P_{r} = I^2r = \frac{E^2}{4r}, $$
$$P_{R} = I^2R = I^2r = \frac{E^2}{4r}. $$
可以发现, 当内外电路的电阻相等时, 电源的损耗功率和输出功率也相等.

当$R = 0$时, 即电源短路时, 电流$$I = \frac{E}{R+r} = \frac{E}{r}.$$
此时三个功率分别为
$$P_{E} = EI = \frac{E^2}{r}, $$
$$P_{r} = I^2r = \frac{E^2}{r}, $$
$$P_{R} = I^2R = 0. $$
可以发现, 当外电路电阻为0时, 输出功率也为0; 电源的总功率及损耗功率相等,
并且容易证明它们都取得最大值.

有了上面三个特殊点, 结合表达式, 我们可以画出纯电阻电路的
功率随电流变化的图像, 如图\ref{2.7-2} 所示:

\begin{figure}[htbp]
    \centering
    \includegraphics[width=6cm]{pic/2.7-2.pdf}
    \caption{功率与电流的关系}
    \label{2.7-2}
\end{figure}
其中$a$图线表示$P_{E}$, $b$图线表示$P_{r}$, $c$图线表示$P_{R}$.
\setlength{\abovedisplayskip}{5pt}
\setlength{\belowdisplayskip}{5pt}

\subsection{电源的效率}

电源的输出功率与总功率的比值, 叫作电源的效率, 用$\eta$表示.
即
\begin{empheq}[box=\fbox]{equation*}
    \eta = \frac{P_R}{P_E},
\end{empheq}
其中$P_R$是电源的输出功率, $P_E$电源的总功率.

由$P_R = UI$, $P_E = EI$, 可得
$$\eta = \frac{U}{E},$$
其中$U$是路端电压, $E$是电源的电动势.

如果外电路是纯电阻电路, 则有$U = IR$, $E = I(R+r)$, 那么
$$\eta = \frac{R}{R+r},$$
其中$r$, $R$分别是内,外电路电阻.

\section{应用: 动态电路\bre 含电容器的电路}

\subsection{动态电路问题}
动态电路分析类问题是由于开关的闭合和断开,滑动变阻器滑片的移动等造成电路结构或电阻发生变化, 
从而引起电路发生连锁变化的问题. 这类问题通常只需要定性分析, 不涉及具体数值和大小关系的计算.

解决这类问题常需根据欧姆定律以及串并联电路的性质, 来分析电路中某电阻的变化而引起整个电路中
各部分物理量的变化情况.

解决问题的一般步骤为:
\begin{enumerate}
    \item 确定外电路的总电阻$R_\text{总}$如何变化;
    \item 根据闭合电路欧姆定律$I = \displaystyle\frac{E}{R_\text{总}+r}$确定电路的电流如何变化;
    \item 由$U = E - Ir$确定路端电压如何变化;
    \item 由部分电路的欧姆定律确定支路(或干路上某一段)电压的变化;
    \item 根据串并联规律确定支路各物理量的变化情况.
\end{enumerate}
\bigskip

一般地, 当电源内阻不为0时, 我们有以下结论:
\begin{enumerate}
    \item 当某一电阻增大(或减小)时, 与它串联或间接串联的那部分电路的电流, 电压, 以及用电器的电功率都减小(或增大);
    \item 当某一电阻增大(或减小)时, 与它并联或间接并联的那部分电路的电流, 电压, 以及用电器的电功率都增大(或减小).
\end{enumerate}
这个规律可以简单说成\textbf{``串反并同''}.

\subsection{含电容器的电路问题}

解决含电容器的电路的基本思路为:
\begin{enumerate}
    \item 首先分析电路稳定状态时的连接方式, 此时\textbf{电容器所在支路可以看作断路}, 简化电路时可以去掉;
    \item 分析清楚电容器两端的电压与哪部分电路的电压相同, 即与哪部分电路并联;
    \item 当分析涉及充放电的问题时,电容器上电荷量的变化可由$\Delta Q = C\Delta U$计算.
\end{enumerate}

当电容器处于稳定状态时(即不充电也不放电), 电容器所在的支路相当于断路.此时
的电路具有以下两个特点:
\begin{enumerate}
    \item 电容器所在的支路无电流, 与电容器直接串联的导体相当于一根无电阻的导线;
    \item 电容器上的电压就是与电容器所在支路并联部分电路的电压.
\end{enumerate}

需要注意的是, 当电容器两端电压发生变化时, 电容器会发生充放电.此时电容器所在支路有电流通过, 
不能作为断路处理.

\section{实验:伏安法测电阻}
\subsection{电流表的内接法与外接法}

\begin{figure}[bthp]
\centering
\subfloat[电流表的内接法]{
    \includegraphics[width=0.3\textwidth]{pic/2.8-1.pdf}
    \label{内接法}
}
\subfloat[电流表的外接法]{
    \includegraphics[width=0.3\textwidth]{pic/2.8-2.pdf}
    \label{外接法}
}
	\caption{电流表的内接法与外接法}
    \label{内外接}
    
\end{figure}

在使用伏安法测量电阻时, 我们要用到电压表和电流表. 初中时, 我们只讨论理想电压表和理想电流表, 它们的
内阻分别是无穷大和0. 当电流表和电压表均为理想表时, 以上两种电路的连接方法是等效的, 
否则, 以上两种连接方法便会产生不同的读数.

图\ref{内接法} 的连接方法称为\textbf{电流表的内接法}.设电流表的内阻为$R_\text{A}$, 则使用内接法的电阻测量值为
$$R_\text{测} = \frac{U}{I} = \frac{U_\text{A}+U_R}{I} = R_\text{A} + R_x.$$
其中, $U$, $I$分别是电压表和电流表的读数, $U_\text{A}$是电流表两端的电压, $U_R$是
电阻$R_x$两端的电压. 显然, $R_\text{测}$>$R_x$, 即\textbf{测量值比真实值偏大}, 并且
$$R_x = R_\text{测} - R_\text{A}.$$

当$R_x\gg R_\text{A}$时, 根据上式, $R_x$与$R_\text{测}$近似相等. 也就是说, 
\textbf{电流表的内接法适合测量大电阻}.

图\ref{外接法} 的连接方法称为\textbf{电流表的外接法}.设电压表的内阻为$R_\text{V}$, 则使用外接法的电阻测量值为
$$R_\text{测} = \frac{U}{I} = \frac{U}{I_\text{V}+I_R} = \frac{U}{\displaystyle\frac{U}{R_\text{V}}+\displaystyle\frac{U}{R_x}} = \frac{R_\text{V}R_x}{R_\text{V}+R_x} = \frac{R_x}{1+\displaystyle\frac{R_x}{R_\text{V}}}.$$
其中, $U$, $I$分别是电压表和电流表的读数, $I_\text{V}$是流过电压表的电流, $I_\text{R}$是流过电阻$R_x$的电流. 显然, $R_\text{测}$<$R_x$, 
即\textbf{测量值比真实值偏小}, 并且
$$R_x = \frac{R_\text{测}}{1-\displaystyle\frac{R_\text{测}}{R_x}}.$$

当$R_x\ll R_\text{A}$时, 根据上式, $R_x$与$R_\text{测}$近似相等. 也就是说, 
\textbf{电流表的外接法适合测量小电阻}.

以上结论可以简单记为\textbf{``大内小外''}, 或者\textbf{``大内偏大, 小外偏小''}.

\bigskip
如何判断电阻是``大电阻''还是``小电阻''呢? 除了定性判断, 我们还有以下两种办法:
\begin{enumerate}
    \item 比较相对大小. 即比较$\displaystyle\frac{R_\text{V}}{R_x}$与$\displaystyle\frac{R_x}{R_\text{A}}$的大小关系; 
也即比较$R_x$与$\sqrt{R_\text{A}R_\text{V}}$的大小关系.
    \item 实验判断. 分别使用电流表的内接法和外接法记录数据. 如果电流有较大差异, 则使用内接法;如果电压有较大差异, 则使用外接法.
\end{enumerate}

\subsection{滑动变阻器的限流接法和分压接法}

\begin{figure}[bthp]
\centering
\subfloat[滑动变阻器的的限流接法]{
    \includegraphics[width=0.26\textwidth]{pic/2.8-3.pdf}
    \label{限流接法}
}\bre
\subfloat[滑动变阻器的的分压接法]{
    \includegraphics[width=0.26\textwidth]{pic/2.8-4.pdf}
    \label{分压接法}
}
	\caption{滑动变阻器的限流接法和分压接法}
    \label{分压和限流}
    
\end{figure}
图\ref{限流接法} 的连接方法称为\textbf{滑动变阻器的限流接法}.当滑动变阻器$R_0$的滑片移至$b$端时, 如果不记电源内阻, 那么根据串联电路的分压规律, $R_x$两端的电压
$U_x = \displaystyle\frac{R_x}{R_0+R_x}E$; $R_x$上的电流$I_x = \displaystyle\frac{E}{R_0+R_x}$.\vspace{10pt}

当滑动变阻器$R_0$的滑片移至$a$端时, 负载$R_x$两端的电压$U_x = E$; $R_x$上的电流$I_x = \displaystyle\frac{E}{R_x}$.

因此, 若不计电源内阻, 使用限流接法时, 负载$R_x$上电压的调节范围为$$\frac{R_x}{R_0+R_x}E\leqslant U_x \leqslant E;$$
电流的调节范围为$$\frac{E}{R_0+R_x}\leqslant I_x \leqslant \frac{E}{R_x}.$$
电路消耗的功率为$EI_x$.

图\ref{分压接法} 的连接方法称为\textbf{滑动变阻器的分压接法}.当滑动变阻器$R_0$的滑片移至$a$端时, $R_x$被短路, 电压和电流均为0.

当滑动变阻器$R_0$的滑片移至$b$端时, $R_0$与$R_x$并联, 如果不计电源内阻, 那么$R_x$两端的电压$U_x$等于电源电动势$E$, 
$R_x$上的电流$I_x = \displaystyle\frac{E}{R_x}$.

因此, 若不计电源内阻, 使用分压接法时, 负载$R_x$上电压的调节范围为$$0\leqslant U_x \leqslant E;$$
电流的调节范围为$$0\leqslant I_x\leqslant \frac{E}{R_x}.$$\vspace{5pt}
设滑动变阻器接入的电阻为$R_P$, 则电路消耗的功率为$E\left(I_x+\displaystyle\frac{E}{R_P}\right)$.

综上所述, 采用限流接法时, 负载$R_x$上电压和电流的调节范围更小, 能耗较低; 采用分压接法时, 负载$R_x$上电压和电流的调节范围更大, 能耗较高.

为了节约能源, 我们通常采用限流接法. 但是\textbf{下列情况必须采用分压接法}:
\begin{enumerate}
    \item 被测电阻的电压或电流一定要从0开始调节;
    \item 采用限流电路时, 电路中的最小电流仍超过元件允许的最大电流;
    \item 滑动变阻器的电阻远小于被测电阻.
\end{enumerate}

\section{实验: 测量电源的电动势及内阻}
\setlength{\abovedisplayskip}{0pt}
\setlength{\belowdisplayskip}{0pt}

\subsection{实验思路}
电动势$E$和内阻$r$是电源的重要参数. 在学习闭合电路欧姆定律后, 我们发现它的表达式
$$E = U + Ir$$恰好包含了$E$和$r$两个物理量.
如果能测出$U$, $I$的两组数据, 就可以列出两个关于 $E$, 
$r$ 的方程, 从中解出 $E$ 和 $r$.因此, 用电压表, 电流表加上
一个滑动变阻器$R$, 就能测定电源的电动势$E$及内阻$r$.

\subsection{数据处理}
为便于分析, 我们把闭合电路欧姆定律的表达式写成
\begin{equation}
    U = -rI + E.
    \label{测量电动势2}
\end{equation}
以$U$为纵坐标, $I$为横坐标, 建立平面直角坐标系, 把实验获得的数据记录在坐标系中.

我们发现, 直线在$U$轴上的截距为短路时的路端电压, 它就等于电源的电动势$E$; 直线在$I$轴上的
截距为短路电流$I_\text{短}$.由关系式$$r = \frac{E}{I_\text{短}}$$可以求出电源的内阻.

另一方面, 由 \eqref{测量电动势2} 式可知, 直线的斜率的绝对值即为电源的内阻$r$, 即
$$r = \left|\frac{\Delta U}{\Delta I}\right|.$$

应注意当内阻$r$较小时, $U$的变化量较小, 在作图时$x$轴上方有较大空白. 为避免由此带来的误差,
可使纵轴不从0开始,并把纵坐标的比例放大. 此时, 图线与纵轴的交点仍代表电源电动势$E$, 
但图线与横轴的交点不再代表短路状态. 此时只能利用斜率求出电源内阻.

\subsection{电路连接及误差分析}

\begin{figure}[bthp]
\centering
\subfloat[采用外接的电路]{
    \includegraphics[width=0.26\textwidth]{pic/2.9-1.pdf}
    \label{测量电动势与内阻1}
}\bre
\subfloat[实验得到的电源的$U$-$I$图像]{
    \includegraphics[width=0.3\textwidth]{pic/2.9-2.pdf}
    \label{测量电动势与内阻2}
}
	\caption{采用外接测量电源的电动势与内阻}
    \label{测量电动势与内阻3}
    
\end{figure}

测量电源电动势及内阻的实验电路, 不涉及限流接法与分压接法的选择, 但是涉及电流表内接与外接的选择.

由于电源内阻很小, 故\textbf{电流表相对于电源而言一般采用外接}\footnote{对于水果电池,老旧电池等内阻较大的电池有时也采用内接.}(图\ref{测量电动势与内阻1}).
否则, 当采用内接时,$$r_\text{测} = r+R_\text{A}.$$
电源内阻$r$相比于电流表内阻$R_\text{A}$并不很大, 会产生较大的误差.

下面我们对外接法的误差进行分析.

偶然误差来源于读数和描点, 系统误差来源于电压表的分流. 对于电流表(相对于电源)的外接法, 我们近似地将电流表的
示数看作了干路电流, 但实际上的干路电流比电流表的示数略大.

由实验得到的数据直接作出图像(图\ref{测量电动势与内阻2} 实线).考虑到电压表的分流后, 每一个电压值$U_1$对应的电流表
读数$I_\text{1测}$都比真实值$I_\text{1真}$略小(只有短路电流是准确的), 所以真实的$U-I$图像应该是右图虚线.
由此可见, 按外接法测量出的电源电动势$E_\text{测}<E_\text{真}$, 内阻$r_\text{测}<r_\text{真}$.也就是说,
\textbf{真实值全部大于测量值}.

\setlength{\abovedisplayskip}{5pt}
\setlength{\belowdisplayskip}{5pt}

定量来看, 我们获得的数据实际上满足
$$E_\text{真} = U+\left(I+\frac{U}{R_\text{V}}\right)r.$$
其中$U,I$是我们在电表上的读数, $R_\text{V}$是电压表内阻.
写成斜截式的形式就是$$U = -\frac{rR_\text{V}}{r+R_\text{V}}I+\frac{R_\text{V}}{r+R_\text{V}}E_\text{真}.$$
这就是上图中实线符合的方程, 并且$$E_\text{测} = \frac{R_\text{V}}{r+R_\text{V}}.$$
可以看出, 斜率的绝对值和纵轴截距相比虚线都偏小.

要减小误差, 所选择的电压表内阻应适当大些, 使得$R_\text{V}\gg r$. 

\setlength{\abovedisplayskip}{0pt}
\setlength{\belowdisplayskip}{0pt}

如果我们使用内接法, 也可以采用类似方法分析. 我们获得的数据满足$$E = (U+IR_\text{A}) +Ir, $$即
$U = -(r+R_\text{A})I+E.$由此可以看出, 我们测得的数据图线斜率的绝对值偏大, 而电动势不变.也就是
测得的内阻偏大, 而电动势准确.

\chapter{磁场}

\section{磁场}

\subsection{磁场}

我们知道, 磁体存在北极, 南极两个磁极, 就像自然界存在正负两种电荷.
不仅如此, 磁极之间的相互作用, 与电荷
之间的相互作用具有相似的特征:同名磁极或同种电荷相
互排斥, 异名磁极或异种电荷相互吸引.
但是,直到 19 世纪初,库仑和安培等都认为电与磁是互不相关的两回事;
丹麦物理学家奥斯特却深信电与磁之间存在联系, 并开始了不懈的探索.

当时人们见到的力都沿着
物体连线的方向, 奥斯特在寻找电和
磁的联系时, 总是把磁针放在通电导线的延长线上, 结果
实验均以失败告终.

在一次讲课中, 他偶然地把导线放置在
一个指南针的上方, 通电时磁针转动了, 这个
现象虽然没有引起听众的注意, 但却是奥斯特盼望已久的.
他连续进行了大量研究,同年 7 月发表论文, 宣布发现了
电流的磁效应, 这首次揭示了电与磁的联系.

在这次实验之后, 安培等人也发现, 不仅通电导线对磁体有作用力, 磁体对通电导
线也有作用力. 例如, 把一段直导线悬挂在蹄形磁体的两
极间, 通以电流, 导线就会移动.他们还发
现, 任意两条通电导线之间也有作用力.

事实上, 正如电荷间的相互作用是由电场发生的, 磁体与磁体间, 磁体与通电导体间, 
通电导体与通电导体间的相互作用是
由\textbf{磁场}发生的. 磁场是客观存在于磁体或电流附近的一种物质.

应当指出的是, 产生磁场的磁体或电流不会受到自己的磁场的作用.

\subsection{磁感线}
小磁针有两个磁极, 它在磁场中静止后就会显示出这
一点的磁场对小磁针 N 极和 S 极作用力的方向. 
物理学中把小磁针静止时 N 极所指的方向规定为该点磁场的方向.
实验中我们常用铁屑的分布来反映磁场的分布.

在磁场中画一些有向曲线, 使曲线上任意一点的切线方向为该点处的磁场方向, 
这样的曲线叫作\textbf{磁感线}. 铁屑的分布就类似于磁感线.

磁感线是假想的曲线, 并不是客观存在的. 没有画磁感线的地方, 那里的磁感应强度不一定为0.

磁感线有以下特点:
\begin{enumerate}
    \item 磁感线上每一点的方向为该点磁场的切线方向, 磁感线越密的地方, 磁场越强.
    \item 任何两条磁感线都不相交.
    \item 磁感线是闭合的曲线.在磁体外部, 磁感线由N极指向S极;在磁体内部, 
    磁感线由S极指向N极.
\end{enumerate}

\subsection{安培定则}

实验表明, 直线电流的磁感线是一圈圈的同心圆, 这些同心圆都在跟导
线垂直的平面上. 改变电流的方向, 各点
的磁场方向都变成相反的方向. 

直线电流的方向跟它的磁感线方向之间的关系
可以用\textbf{安培定则}(也叫\textbf{右手螺旋定则})来判断: 

\textbf{安培定则} \textbf{用右手握住导线, 让伸直
的拇指所指的方向与电流方向一致, 弯曲的四指所指的方
向就是磁感线环绕的方向. }\bigskip

在初中, 我们学过安培定则的另一种形式: \textbf{让右手弯曲的四
指与环形电流(或螺线管)的方向一致, 伸直的拇指所
指的方向就是环形导线(或螺线管)轴线上磁场的方向.}
这两个形式本质是相同的, 都描述了电流方向与磁场方向的位置关系.
需要强调的是, 安培定则所确定的方向是环形导线(或螺线管)轴线上磁场的方向.
绘制磁感线可以知道, \textbf{它两侧的磁场方向与上述方向相反}. 

与天然磁体的磁场相比,电流磁场的强弱容易控制,
因而在实际中有很多重要的应用.

\subsection{地磁场}

指南针之所以能够指示方向, 是由于地球内部存在磁场, 称为\textbf{地磁场}. 地球内部的磁场类似于
一个很大的条形磁体的磁场, 有以下特点:

\begin{enumerate}
    \item 地磁场的N极在地球的(地理)南极附近, S极在地球的(地理)北极附近; 但N, S极与
    南, 北极点并不重合, 而是存在一个偏角.
    \item 地磁场$B$的水平分量$B_x$总是从地球的(地理)南极指向(地理)北极;竖直分量
    $B_y$在南半球垂直于地面向上, 在北半球垂直于地面向下.
    \item 在赤道平面上, 距离地球表面相等的各点, 磁感应强度相等, 且方向水平向(地理的)北.
\end{enumerate}

\subsection{安培的分子电流假说}

磁体和电流都能产生磁场, 它们的磁场是否有联系? 我们知
道, 通电螺线管外部的磁场与条形磁体的磁场十分相似.安培由
此受到启发, 提出了以下假说. 

安培认为, 在物质内部, 
存在着一种环形电流——\textbf{分子电流}, 分子电流使每个物质微粒都
成为微小的磁体, 它的两侧相当于两个磁极.

一根铁棒未经磁化的时候, 内部分子电流的取向是杂乱无章的, 
它们的磁场互相抵消,因此对外不显磁性; 当
铁棒受到外界磁场的作用时,各分子电流
的取向变得大致相同, 铁棒被磁化, 两
端对外界显示出较强的磁性, 形成磁极.磁体受到高温或猛烈撞击
时会失去磁性, 这是因为激烈的热运动或震
动使分子电流的取向又变得杂乱无章了.

\section{磁感应强度}

\subsection{电流元}

用小磁针可以判断空间某点磁场的方向, 但很难对它
进行进一步的定量分析. 若以通电导线作为磁场的检验物
体, 则既可以知道导线中电流的大小, 又能测量导线的长
度, 从而可以进行定量的研究.

为研究空间某点的磁场, 可以考虑在该处放一段很短
的通电导线, 分析它受到的力. 物理学把很短一段
通电导线中的电流$I$与导线长度 $l$ 的乘积 $Il$ 叫作\textbf{电流元}. 

然而在现实中, 孤立的电流元是不存在的. 如果要研究的那部分磁场的强弱和方向
都是一样的, 我们也可以用比较长的通电导线进行实验, 
从结果中推知电流元的受力情况.

\subsection{磁感应强度}

根据实验事实, \textbf{垂直于磁场方向的通电导线, 在磁场中受到的力$F$
既与流过它的电流$I$成正比, 又与它的长度$l$成正比.} 除此之外, 
在不同的磁场中, 或在不均匀磁场的不同位置, 一般来说, 导线受的力也不同.
这个关系可以写成$$F = IlB.$$

\setlength{\abovedisplayskip}{5pt}
\setlength{\belowdisplayskip}{5pt}

上式中, $B$是与$I$和$l$都无关的量, 能表征磁场在某一点的强弱.
物理学把这个量定义为\textbf{磁感应强度}.于是, 在通电导线与磁场方向
垂直的条件下, 有关系式
\begin{empheq}[box=\fbox]{equation*}
    B = \frac{F}{Il}.
\end{empheq}

在国际单位制中, 磁感应强度的单位是\textbf{特斯拉}, 简称\textbf{特}, 
符号为T.
$$1\unit{T} = \frac{1\unit{N}}{1\unit{A}\cdot 1\unit{m}}.$$

磁感应强度是矢量, 它的方向就是磁场方向, 即该处小磁针静止时
N 极所指的方向. 

若空间中存在多个磁场, 则总磁场由这几个磁场叠加而成, 总的磁感应强度等于这几个磁场的
磁感应强度的矢量和.

\subsection{匀强磁场}
类似匀强电场的概念, 如果磁场中各点的磁感应强度的大小相等, 方向相同, 
那么这个磁场叫作\textbf{匀强磁场}. 匀强电场可以用平行等距的磁感线表示.

距离很近的两个平行异名磁极之间
的磁场, 除边缘部分外, 可以认为是匀强磁场, 比如蹄型磁体内部的磁场.
两个平行放置较近的线圈通电时,其中间区域的磁场近似为匀
强磁场.

\section{磁通量}

磁感线的疏密程度表示了磁场的强弱, 这种不同是如何体现
的呢? 如果在垂直于纸面方向取同样的面积,
穿过相同面积磁感线条数多的就密,磁感应强度就大.
类似于面密度的概念, 在电磁学中, 我们做下述定义.

\setlength{\abovedisplayskip}{0pt}
\setlength{\belowdisplayskip}{0pt}

设在磁感应强度为 $B$ 的匀强磁场中, 有一个与磁场方
向垂直的平面, 面积为 $S$, 我们把 $B$ 与 $S$ 的乘积
叫作穿过这个面积的\textbf{磁通量},简称\textbf{磁通}.
用字母$\Phi$表示磁通量, 则$$\Phi = BS.$$

如果磁感应强度 $B$ 不与我们研究的平面$S$垂直, 
那么我们用这个面在垂直于磁感应强度 $B$
的方向的投影面积 $S^{\prime}$与 $B$ 的乘积表示磁通量.
也可以表示为$$\Phi = BS\cos\theta,$$
其中$S\cos\theta$是面积$S$在垂直于磁感线方向上的投影
面积$S^{\prime}$.

在国际单位制中,磁通量的单位是\textbf{韦伯}, 简
称\textbf{韦}, 符号是 Wb.
$$1\unit{Wb} = 1\unit{{T}\cdot{m^2}}.$$

由磁通量的定义式可得$$B = \frac{\Phi}{S}.$$这个式子表示, 
磁感应强度在数值上等于穿过单位面积的磁通量. 因此, 磁感应强度又叫作磁通密度.

磁通量是标量, 但是有正负. 如果我们规定磁感线从某一面穿入时磁通量为正值, 
那么磁感线从此面穿出时磁通量为负值.
特别地, 如果磁感线沿相反方向穿过同一平面, 且正向磁通量为$\Phi_1$, 
反向磁通量为$\Phi_2$, ($\Phi_1>0$, $\Phi_2<0$) 那么穿过该平面的磁通量$$\Phi = \Phi_1 + \Phi_2.$$

有时候, 穿过一个平面的磁通量可能由多个磁场共同提供.此时
我们需要把每个磁场产生的磁通量分别计算, 再算出它们的代数和.
这叫作穿过这个平面的\textbf{净磁通量}, 简称\textbf{净磁通}.

类似电势与电势差的关系, 在电磁感应中, 我们也研究磁通量的变化量.
末态时的磁通量$\Phi_\text{末}$与初态时的磁通量$\Phi_\text{初}$之差, 称为磁通量的
变化量, 即$$\Delta \Phi = \Phi_\text{末} - \Phi_\text{初}.$$

显然, 引起磁通量变化的原因有以下三种:
\begin{enumerate}
    \item 磁感应强度$B$发生了变化;
    \item 面积$S$发生了变化;
    \item 面与磁感线的夹角发生了变化.
\end{enumerate}

\section{磁场对通电导线的作用力}

在前面的内容中, 我们已经知道了磁场对通电导线有作用
力, 并从这个现象入手定义了物理量——磁感应强度.

安培在研究磁场与电流的相互作用方面作出了杰出的贡
献,为了纪念他,
人们把通电导线在磁场中受的力称为\textbf{安培力}, 
把电流的单位定为安培. 

\subsection{安培力的方向}

通过实验, 我们发现通电导线受力的方向与电流的方向, 磁场的方向均有关.
众多事实表明, 通电导线在磁场中所受安培力的方向
与电流方向, 磁感应强度的方向都垂直, 即\textbf{安培力的方向垂直于电流方向和磁场方向确定的平面}.

安培力的方向可用以下定则判定: 

\textbf{安培力的左手定则\bre 伸开左手, 使拇指与其余
四个手指垂直, 并且都与手掌在同一个平面内; 让磁感
线从掌心垂直进入, 并使四指指向电流的方向, 这时拇
指所指的方向就是通电导线在磁场中所受安培力的方向.}

\subsection{安培力的大小}

\setlength{\abovedisplayskip}{0pt}
\setlength{\belowdisplayskip}{0pt}

我们已经知道, 在垂直于磁场 $B$ 的方
向放置的长为$l$的一段导线, 当通过的电流为 $I$ 时, 它所受
的安培力$$F = IlB.$$

当磁感应强度 $B$ 的方向与通电导线的方向平行时,导
线受力为 0.

当磁感应强度$B$的方向与通电导线的方向成$\theta$角时, 导线所受的安培力又如何呢?
我们考虑把磁感应强度做矢量分解, 分解为与电流平行和与电流垂直两个分量. 其中平行分量
对导线没有作用力.

一般地, 通电导线所受的安培力大小等于电流$I$, 导线长度$l$, 以及磁感应强度垂直于导线方向的分量$B_\bot$这三者的乘积.
设磁感应强度$B$的方向与通电导线的方向的夹角为$\theta$, 则通电导线在磁场中所受的安培力为
\begin{empheq}[box=\fbox]{equation*}
    F = IlB\sin \theta.
\end{empheq}

\section{磁场对运动电荷的作用力}

我们知道, 磁场对通电导线有作用力;我们还知道, 带电粒子的定向移动形成了电流. 那
么, 磁场对运动电荷有作用力吗?如果有, 力的方向和大小又是怎样的呢?

在磁场内发射电子束, 实验表明, 电子束受到磁场的力的作用, 径迹发生了
弯曲. 运动电荷在磁场中受到的力称为\textbf{洛伦兹力}.
通电导线在磁场中受到的安培力, 实际是洛伦兹力的宏观表现; 而洛伦兹力是
安培力的微观本质.

\subsection{洛伦兹力的方向}

由于安培力的本质就是洛伦兹力, 所以, 洛伦兹力的方向也符合左手定则.

\textbf{洛伦兹力的左手定则\bre 伸开左手, 使拇指与其余四个手指垂直, 并且都与手掌在同一个平面内; 让磁感线从
掌心垂直进入, 并使四指指向正电荷运动的方向, 这时拇指所指的方向就是运动的正电荷在磁场中所受洛伦兹力的方向.}

负电荷受力的方向与正电荷受力的方向相反, 可以用右手来判断. 

\subsection{洛伦兹力的大小}

接下来, 我们用安培力的表达式来推导洛伦兹力的表达式.

考虑一根横截面积为$S$的静止导线置于磁感应强度为$B$的匀强磁场
中, 带电粒子以速度 $v$ 定向移动. 单
位体积内的带电粒子数为$n$. 则在时间$t$内, 流过一段导线的粒子数为
$$N = nSvt.$$

记粒子的电荷量为$q$, 
根据 \eqref{电流和自由电荷定向移动平均速率的关系}, $q$与电流
$I$的关系为 $$I = nqSv.$$

如果电流与磁场方向垂直, 即粒子的速度$v$与磁感应强度$B$垂直, 那么导线所受安培力的大小$$F_\text{安} = IlB = nqSv^2tB.$$

我们知道, 导线受到安培力作用, 本质是它内部的粒子受到洛伦兹力作用. 每一个粒子所受的力为\vspace{3pt}
$$F = \frac{F_\text{安}}{N} = \frac{nqSv^2tB}{nSvt} = qvB.$$
这就是粒子速度$v$的方向与磁感应强度$B$的方向垂直时, 粒子所受的洛伦兹力.

仿照上节, 在一般情况下, 当电荷运动的方向与磁场的方向夹角为$\theta$时, 电荷所受的洛伦兹力大小为
\begin{empheq}[box=\fbox]{equation*}
    F = qvB\sin \theta.
\end{empheq}

\section{带电粒子在磁场中的运动}
我们知道, 带电粒子在磁场中运动要受到洛
伦兹力的作用. 如果带电粒子初速度的方向和洛伦兹力的
方向都在与磁场方向垂直的平面内, 那么粒子将在这个平面
内运动. 

洛伦兹力总是与粒子的运动方向垂直, \textbf{总是只改变粒子速
度的方向, 不改变粒子速度的大小}. 由于粒子速度的大小
不变, 粒子在匀强磁场中所受洛伦兹力的大小也不改变, 
洛伦兹力对粒子起到了向心力的作用. 所以, 沿着与磁场
垂直的方向射入磁场的带电粒子, 在匀强磁场中做匀速圆
周运动.

\subsection{带电粒子在匀强磁场中的匀速圆周运动}

考虑到带电粒子做匀速圆周运动的向心力由洛伦兹力提供, 
列出方程就可以得出几个物理量之间的关系式.

假设一个电荷量为 $q$ 的粒子, 在磁感应强度为 $B$ 的匀强
磁场中以速度 $v$ 运动, 洛伦兹力提供向心力
$$qvB = m\frac{v^2}{r}.$$
由此可得粒子做匀速圆周运动的半径
\begin{empheq}[box=\fbox]{equation}
    r = \frac{mv}{qB}.
    \label{匀强磁场匀速圆周半径}
\end{empheq}
可以看出, 粒子运动的半径与它的质量, 速度成正比; 与电荷量, 磁感应强度成反比. 

回忆匀速圆周运动的知识, 匀速圆周运动的周期$T = \displaystyle\frac{2\uppi r}{v}$.
把 \eqref{匀强磁场匀速圆周半径} 代入, 可以得到粒子做匀速圆周运动的周期
\begin{empheq}[box=\fbox]{equation}
    T = \frac{2\uppi m}{qB}.
    \label{匀强磁场匀速圆周周期}
\end{empheq}
可以看出, 粒子的周期与磁感应强度和它的比荷成反比, \textbf{与轨道半径和运动速度无关}.
这是一个很重要的结论, 
\ref{电磁科技一节} 节要介绍的回旋加速器就是依据这个原理制成的.

\subsection{带电粒子在有界匀强磁场中的运动}

我们已经知道, 带电粒子在匀强磁场中将做匀速圆周运动.
为了确定粒子的运动, 我们需要知道轨迹的圆心, 半径和运动时间.

\subparagraph{圆心和半径}由于洛伦兹力提供向心力, 因此洛伦兹力总是指向轨迹的圆心, 也就是圆心
总在各点处粒子所受洛伦兹力的方向上. 如果我们已知轨迹上的任意两点(一般是进入磁场和离开磁场的两点)
及这两点的洛伦兹力方向, 它们的交点就是圆心. 进一步地, 如果我们知道这两点的速度方向, 根据左手定则,
我们可以判断这两点的洛伦兹力方向, 进而确定圆心位置.

另一方面, 如果我们已知轨迹上的两点, 根据垂径定理, 这两点连线的中垂线一定过圆心. 因此, 
我们有以下结论:

\textbf{粒子入射方向的垂线, 出射方向的垂线, 入射点与出射点连线的中垂线, 这三条直线中任意两条的交点即为粒子轨迹的圆心.}

知道了圆心和轨迹上的两点后, 我们可以利用勾股定理, 三角函数等数学方法求解轨迹圆的半径. 

\subparagraph{运动时间} 一般来说, 粒子在有界磁场中的运动轨迹是一段圆弧. 假设粒子进入磁场时的速度方向与
离开磁场时的速度方向夹角为$\theta$ (称为速度的\textbf{偏转角}), 则由几何关系可以证明: \textbf{速度方向的偏转角等于
轨迹圆弧所对的圆心角.} 今后我们就把速度的偏转角和圆心角都用$\theta$表示.

由于粒子做匀速圆周运动, 所以粒子的运动时间为$$t = \frac{\theta}{2\uppi}T.$$
其中$\theta$的单位是弧度, $T$是粒子做匀速圆周运动的周期.
把 \eqref{匀强磁场匀速圆周周期} 代入, 可以得到
$$t = \frac{m\theta}{qB}.$$
可以看出, \textbf{同一粒子在已知磁场内的运动时间只与轨迹圆弧所对的圆心角$\theta$有关, 与粒子的速度无关.}

另一方面, 根据匀速圆周运动线速度的定义, 可得
$$t = \frac{\theta r}{v}.$$
其中$\theta$的单位是弧度, $r$是粒子的运动半径, $v$是粒子的速度.

\bigskip
在解决这类问题时, 通常涉及许多几何知识.下面是一些常用的几何关系.
\begin{enumerate}
    \item 在运动平面内, 速度方向的垂线经过圆心.
    \item 轨迹上两点连线的垂直平分线经过圆心.
    \item 轨迹上两点速度方向的偏转角等于这两点间的圆弧所对的圆心角.
    \item 轨迹上两点的连线与其中一点速度方向的(锐)夹角, 等于这两点间的圆弧所对的圆心角的一半.
\end{enumerate}

\begin{figure}
    \centering
    \includegraphics[width=0.35\textwidth]{pic/3.5-1.pdf}
    \caption{例\ref{例题匀强电场磁场} 题图}
    \label{3.5-1}
\end{figure}

\begin{example}
    \label{例题匀强电场磁场}
    如图 \ref{3.5-1}, 平面直角体坐标系$xOy$的第一象限内有平行于$y$轴的匀强电场,
    方向沿$y$轴正方向; 在第四象限的正三角形$abc$区域内有匀强电场, 
    方向垂直于$xOy$平面向里, 正三角形边长为$L$, 且与$ab$边$y$轴平行.
    一质量为$m$, 电荷量$q$为的粒子, 从$y$轴上的$P(0, h)$点, 
    以大小为$v_0$的速度沿$x$轴正方向射入电场, 通过电场后从$x$轴的$a$点进入第四象限,
    又经过磁场从$y$轴上的某点进入第三象限, 且速度与$y$轴负方向成$45^{\circ}$角.
    不计粒子所受的重力.求:

    (1)电场强度$E$的大小;
    
    (2)粒子到达$a$点时, 速度$v_a$的大小和方向;

    (3) $abc$区域内磁场的磁感应强度的最小值$B_0$.

\end{example}
\begin{solution}
    (1) 粒子在第一象限内做类平抛运动, 显然$q<0$. 设粒子在第一象限的运动时间为$t_1$, 则水平方向上有
    \begin{equation*}
        2h = v_0t_1.\tag{i}
    \end{equation*}
    竖直方向上有
    \begin{equation*}
        h = \frac12 at_1^2 = \frac12\frac{qE}{m}t_1^2.\tag{ii}
    \end{equation*}
    联立 (i)(ii) 解得$E = \displaystyle\frac{mv_0^2}{2qh}.$

    (2)设粒子到达$a$点时竖直方向的速度为$v_y$, 则有
    \begin{equation*}
        v_y = at_1 = \frac{qE}{m}t_1.\tag{iii}
    \end{equation*}
    联立(i)(ii)(iii)得$v_y = v_0$.

    所以粒子到达$a$点时的速度
    \begin{equation*}
        v_a = \sqrt{v_0^2 +v_y^2} = \sqrt{2}v_0.\tag{iv}
    \end{equation*}
    由$v_0 = v_y$知$v_a$与$x$轴夹角$\theta$的正切$\tan\theta = 1$, 所以$\theta = 45^{\circ}$,
    即$v_a$指向斜下方, 与$x$轴成$45^{\circ}$角.

    (3)粒子进入磁场后, 受到的洛伦兹力与粒子的速度方向时刻垂直, 因此洛伦兹力改变速度的方向而不改变大小, 
    使粒子做匀速圆周运动. 经过分析可知, 当粒子从$b$点离开磁场时, 对应的磁感应强度最小.
    根据几何关系, 粒子做匀速圆周运动的半径
    \begin{equation*}
        r = L\cos\theta = \frac{\sqrt{2}}{2}L.\tag{v}
    \end{equation*}

    洛伦兹力提供粒子做匀速圆周运动的向心力, 即
    \begin{equation*}
        qvB_0 = m\frac{v_a^2}{r}.\tag{vi}
    \end{equation*}
    
    联立 (iv)(v)(vi) 解得$B_0 = \displaystyle\frac{2mv_0}{qL}.$

\end{solution}

\subsection{带电粒子在复合场中的运动}

复合场是指在空间某一区域内, 电场, 磁场和重力场同时存在, 
或者其中两种同时存在; 组合场是指电场, 磁场存在于空间中的不同区域中.
组合场的粒子我们前面已经见过, 下面我们分析复合场的情况.

粒子在经过复合场空间时可能受到的力有重力, 电场力和洛伦兹力.抓住三个力的特点
是分析和解决相关问题的基础.

\subparagraph{重力} 电子, 质子, 离子, $\alpha$粒子等基本粒子一般不需要考虑重力. 其他带电粒子一般需要
考虑重力.

一般来说, 如果题目中给出了重力加速度$g$, 就说明需要考虑粒子的重力$mg$, 方向竖直向下.

\subparagraph{静电力} 带电粒子在电场中一定受到静电力作用. 在匀强电场中, 静电力为恒力, 
大小为$qE$, 方向与电场强度的方向相同或相反.

\subparagraph{洛伦兹力} 带电粒子运动, 且运动的方向不与磁感应强度方向平行时, 粒子受洛伦兹力作用, 
大小为$qvB$ ($v$是粒子在某一刻的瞬时速度).洛伦兹力的方向时刻与速度方向垂直, 因此, 洛伦兹力永远不做功, 
也不会改变粒子的动能.
\bigskip

下面研究带电粒子在复合场中的几种典型运动.
\subparagraph{直线运动} 如果自由的带电粒子(无轨道约束)在受洛伦兹力作用的复合场中做直线运动, 
那么它一定做的是匀速直线运动. 除非运动方向沿匀强磁场的方向.这是因为电场力和重力都是恒力,
如果物体不处于平衡状态, 当速度变化时(无论大小还是方向), 洛伦兹力的变化将导致合力也相应地发生变化. 从而粒子的运动方向改变.

\subparagraph{匀速圆周运动} 当带电粒子进入匀强电场, 匀强磁场和重力场共存的复合场中, 电场力和重力相平衡, 
且粒子运动方向与匀强磁场的方向相垂直时, 带电粒子就在库仑兹力的作用下做匀速圆周运动. 这种情况可等效为仅在洛伦兹力作用下的匀速圆周运动.

\subparagraph{曲线运动} 当带电粒子所处的合外力是变力, 且与初速度方向不共线时, 粒子做非匀变速曲线运动.
这时, 粒子的运动轨迹既不是圆弧, 也不是抛物线.

\section{电磁科技}
\label{电磁科技一节}
\subsection{速度选择器}
\begin{figure}[htbp]
    \centering
    \includegraphics[width=10cm]{pic/3.3-1.png}
    \caption{速度选择器}
    \label{3.3}
\end{figure}
在图\ref{3.3} 的平行板器件中, 电场强度 $E$ 和磁感应强度 $B$ 相互垂直. 具有不同水
平速度的带电粒子射入后发生偏转的情况不同. 这种装置能把具有某一特定速度的粒子选择出
来, 所以叫作\textbf{速度选择器}.

根据粒子做匀速直线运动, 有$qvB = qE$, 所以可通过粒子的速度为$$v = \frac{E}{B}.$$
并且, 粒子能否通过与它的电荷量, 电性, 质量均无关, 只与速度有关.

\subsection{质谱仪}
\begin{figure}[htbp]
    \centering
    \includegraphics[width=7cm]{pic/zhipuyi.png}
    \caption{质谱仪}
    \label{质谱仪}
\end{figure}
质谱仪可以用于测量粒子的质量或比荷, 由粒子发射器, 加速电场和偏转磁场组成(图\ref{质谱仪}).

如图 \ref{质谱仪} 所示, 质量为 $m$, 电荷量为 $q$ 的粒子, 从容
器 A 下方的小孔 $\mathrm{S_1}$ 飘入电势差为 $U$ 的加速电场, 其初速度
几乎为 0, 然后经过 $\mathrm{S_3}$ 沿着与磁场垂直的方向进入磁感应
强度为 $B$ 的匀强磁场中, 最后打到照相底片 D 上.

粒子进入磁场时的速度 $v$ 等于它在电场中被加速而得
到的速度.由动能定理得
$$qU = \frac12mv^2.$$

然后, 粒子进入磁感应强度为$B$的偏转磁场, 做匀速圆周运动, 由洛伦兹力提供向心力可得
$$qvB = m\frac{v^2}{r}, $$
其中$r$为粒子做圆周运动的半径.
\setlength{\abovedisplayskip}{5pt}
\setlength{\belowdisplayskip}{5pt}

由以上两个式子可以解出粒子在磁场中做圆周运动的半径
$$r = \frac{1}{B^2}\sqrt{\frac{2mU}{q}}.$$

如果容器 A 中粒子的电荷量相同而质量不同, 它们进
入匀强磁场后将沿着不同的半径做圆周运动, 因而被分开, 
并打到照相底片的不同地方.

实际工作中, 往往让中性的气体分子进入电离室 A, 
在那里被电离成离子, 这些离子从电离室的小孔``飘''出, 
从缝 $\mathrm{S_1}$ 进入加速电场中被加速.然后让离子垂直进入匀强
磁场中做匀速圆周运动, 最后打在照相底片 D 上.从离子
打在底片上的位置可以测出圆周的半径 $r$, 进而可以算出离
子的比荷
$$\frac{q}{m} = \frac{2U}{B^2r^2}.$$

\subsection{回旋加速器}
研究原子时常常用到各种各样的粒子加速器. 
由于库仑力可以对带电粒子做功, 从而增加粒子的动
能, 因此, 人们首先想到加速器中一定要用到电场. 加速
电压越高, 粒子获得的动能就越高. 然而产生过高的电压
在技术上是很困难的, 于是人们就会进一步设想, 能不能
采用多次加速的方法呢?

如果以直线排列多个电场, 那么要获得较高动能的粒子, 其加速装置要很长.人们进一步思考, 
如果带电粒子在一次加速后又转回来被第二次加速, 如此
往复``转圈圈''式地被加速, 加速器装置所占的空间不是
会大大缩小吗? 而磁场正好能使带电粒子``转圈圈''! 于
是, 人们依据这个思路设计出了用磁场控制轨道, 用电场
进行加速的\textbf{回旋加速器}.

回旋加速器由两个D形的金属盒和一块大型的电磁铁组成. 两个D形盒之间加有
高频振荡交变电压$U$.

粒子大致从装置的中心位置释放, 在两盒之间被电场加速. 由于处在匀强磁场中, 粒子将
做圆周运动. 经过半个圆周之后, 当粒子再次到达两
盒间的缝隙时, 这时控制两盒间的电势差, 使其恰好改变
正负, 于是粒子经过盒缝时再一次被加速. 之后, 粒子又进入D形盒内做圆周运动, 但
根据\eqref{匀强磁场匀速圆周半径} 式, 
粒子运动的半径比原来增大了.

如此, 粒子在做圆周运动的过程中一次一次地经过盒缝, 而两盒间的
电势差一次一次地改变正负, 粒子的速度就能够增加到
很大. 直到粒子的圆周运动半径大于D形盒的半径, 
粒子将射出回旋加速器.

D形盒是金属制的, 它的作用是静电屏蔽, 使带电粒子在圆周运动的过程中只处在磁场中, 
而不受电场的干扰, 以保证粒子做匀速圆周运动, 直到粒子运动到两个D形盒之间.

为了粒子能够不断地被电场加速而非减速, 我们要求电场强度的方向要与粒子的运动方向相同
(如果粒子带负电, 则要相反).根据\eqref{匀强磁场匀速圆周周期} 式, 
对一定的带电粒子和一定的磁场来说, 尽管粒子的速率和半径一次比一次大, 运动周期却始终不变.
这样, 如果在两盒间加一个交变电场, 使它也以同样的周期往复变化, 那就可以保证粒子每经过
电场时, 都正好赶上适合的电场方向而被加速.
也就是说, \textbf{交变电压的周期应该等于粒子做圆周运动的周期}. 

由$qvB = m\displaystyle\frac{v^2}{r}$, 可得粒子的速度$$v = \frac{qBr}{m},$$
其中$m$是粒子的质量, $q$是粒子的电荷量, $B$是D形盒内的磁感应强度, $r$是粒子
在某一时刻的圆周运动半径. 当粒子的运动半径最大时, 粒子的速度也达到
最大值. 因此, 粒子的最大动能$$E_\mathrm{km} = \frac12mv^2 = \frac{q^2B^2R^2}{2m},$$
其中$R$是D形盒的半径. 由此可知, 在粒子的比荷一定的情况下, 粒子的最大动能与加速电压$U$无关, 
与D形盒内的磁感应强度以及D形盒的半径有关, D形盒半径越大, 粒子的末动能就越大.

\setlength{\abovedisplayskip}{0pt}
\setlength{\belowdisplayskip}{0pt}

另一方面, 由动能定理可得$$nqU = \frac12mv^2. $$
所以粒子的速度与加速次数$n$及粒子的比荷有关, 并且加速次数$$n = \frac{qB^2R^2}{2mU}.$$

\subsection{电磁流量计}
一个非磁材料制成的圆柱导管直径为$d$, 导管内有
可以导电的液体径向流动. 把导管置于磁感应强度为$B$, 方向垂直于液体流动方向的匀强磁场中, 
导电液体中的自由电荷(正, 负离子)在洛伦兹力的作用下纵向偏转, 导管的
上下表面出现电势差$U$.

当自由电荷所受的电场力$qU$与洛伦兹力$qvB$平衡时, 自由电荷就不再偏转, 
导管上下表面的电势差$U$也就保持稳定. 此时有
$$q\frac{U}{d} = qvB.$$
由此可以得到自由电荷定向移动的速度, 从而可以得到液体的流量\footnote{流量: 单位时间内通过横截面的液体体积}
$$Q = \frac{V}{t} = \frac{Svt}{t} = Sv = \frac{\uppi d^2}{4}\cdot\frac{U}{Bd} = \frac{\uppi dU}{4B}.$$
\chapter{电磁感应}

\section{电磁感应现象}
\label{电磁感应1}

\subsection{电磁感应现象的发现}
1820年, 丹麦物理学家奥斯特发现了电流的磁效应, 即``电生磁'', 振动了整个科学界,它
证实电现象与磁现象是有联系的. 既然电流能够引起磁针的运动, 那么, 为什么不能用磁体
使导线中产生电流呢?

英国物理学家法拉第敏锐的察觉到, 磁与电之间也应该有这种``感应''.
最初, 法拉第认为, 很强的磁体或很强的电流可能会在邻近的闭合导线中感应出电
流. 他进行了很多次尝试, 没有得到预想的结果. 1831 年, 
法拉第把两个线圈绕在一个铁环上, 一个线圈接电源, 另一个线圈接电流表. 
当给一个线圈通电或断电的瞬间, 在另一个线圈上出现了电流. 法拉第从中领悟到, 
``磁生电''是一种在变化, 运动的过程中才能出现的效应, 他把这些现象定名为\textbf{电磁感应},
产生的电流叫作\textbf{感应电流}. 

\subsection{产生感应电流的条件}
初中时我们知道, 当闭合电路的一部分导体在磁场中做切
割磁感线的运动时, 导体中就会产生感应电流. 那么, 
切割磁感线是产生感应电流的唯一方法吗? 还有其他
方法吗? 这些方法有什么内在联系?

切割磁感线的运动, 实际上是闭合电路在磁场中的的面积发生了变化, 也就是说, 穿过闭合电
路的磁通量发生了变化.感应电流的产生是否与磁通
量的变化有关呢? 答案是肯定的. 以其他形式使穿过闭合电路的磁通量发生变化时, 比如改变
磁感应强度的大小时, 电路中同样产生了感应电流. 以上实验和其他事实都表明:\textbf{当穿过闭合导体回路的磁
通量发生变化时, 闭合导体回路中就产生感应电流.}这就是产生感应电流的条件.



\section{电磁波}

\subsection{电磁场}
\label{电磁场理论}

英国物理学家麦克斯韦系统地总结了人类直至19世纪随电磁规律的研究成果, 
建立了经典电磁场理论. 下面我们定性地介绍麦克斯韦关于电磁场的一些观点.

在变化的磁场中放入一个闭合电路, 电路里会产生感应电流, 这是法拉第
发现的电磁感应现象. 既然产生了感应电流, 一定是有了电场, 它促使
导体中的自由电荷做定向移动; 即使变化的磁场内没有闭合电路, 电场依然存在.
因此, 麦克斯韦认为:\textbf{变化的磁场产生电场}.

那么, 变化的电场会产生磁场吗? 麦克斯韦相信自然的和谐性, 他大胆的假设, 
变化的电场就像导线中的电流一样, 会在空间中产生磁场, 即\textbf{变化的电场产生磁场}.

按照这个理论, 变化的电场与磁场总是互相联系, 形成一个不可分割的统一的\textbf{电磁场}.

事实上, 麦克斯韦的预言完全正确. 1866年, 德国科学家赫兹通过实验捕捉到了电磁波. 后来他又
做了大量的实验, 证实了麦克斯韦的电磁场理论:
\begin{itemize}
    \item 变化的电场(磁场)产生磁场(电场);
    \item 均匀变化的电场(磁场)产生恒定的磁场(电场);
    \item 周期性变化的电场(磁场)产生周期性变化的磁场(电场).
\end{itemize}
\subsection{电磁波谱}
在一列水波中, 突起的最高处叫作\textbf{波峰}, 凹下的最低处叫作\textbf{波谷}.
邻近的两个波峰(或波谷)的距离叫作\textbf{波长}, 单位时间内波峰(或波谷)
的数量叫作\textbf{频率}. 而\textbf{波速}是描述波传播快慢的物理量.

对于电磁波也是如此. 我们用$\lambda$表示电磁波的波长, 
$f$表示它的频率, 那么, 电磁波的波速$c$与$\lambda$, $f$的关系是
$$c = \lambda f.$$
在真空中, 电磁波传播的速度$$c = 3\times 10^8\unit{m/s}.$$

电磁波的频率范围很广. 按照电磁波的频率或波长的顺序排列起来, 就得到\textbf{电磁波谱}.
不同的电磁波由于具有不同的波长, 因而具有不同的特性.

电磁波谱按照波长从大到小的顺序可以排列如下:
$$\text{长波}\rightarrow\text{中波}\rightarrow\text{短波}\rightarrow\text{微波}\rightarrow\text{红外线}\rightarrow\text{可见光}\rightarrow\text{紫外线}\rightarrow\text{X射线}\rightarrow\upgamma\text{射线}.$$
其中, 长波,中波,短波属于无线电波, 用于广播以及传播其他信号; 微波可以用于卫星通信, 电视信号传输等; 红外线可以用于加热理疗;
可见光让我们看见这个世界, 也可以用于通信; 紫外线可以消毒; X射线用于诊断病情; $\upgamma$射线可以摧毁病变细胞.

容易看出, 电磁波的频率越大, 具有的能量就越高.

电磁场不仅仅是一种描述方式, 而是真正的物质存在.

生活中常用微波炉加热食物, 食物中的水分子在微波的作用下热运动加剧, 温度升高.
可见, 电磁波具有能量. 例如, 光是一种电磁波——传播着的电磁场. 光具有能量.

除了可见光外, 其他的电磁波我们虽然看不见, 却能通过它们的能量感受到它们.
为什么播音员的声音能到达收音机?因为电台发射的电磁波在收音机的天线里感应出了电流, 
有电流就有能量.

所有这一切都表明, \textbf{电磁波具有能量, 电磁波是一种物质.}

\section{楞次定律}

在电磁感应的实验中, 我们注意到, 不同情况下产生的感应电流方向是不同的.
那么, 感应电流的方向由哪些因素决定?遵循什么规律?
我们知道, 穿过闭合回路的磁通量变化是产生感应电
流的条件, 看来感应电流的方向可能与磁通量的变化有关.

1834 年, 俄国物理学家楞次在分析了许多实验事实后, 
得到了关于感应电流方向的规律: \textbf{感应电流具有这样的方向, 它产生的磁场总是
阻碍引起感应电流的磁通量的变化}.这就是\textbf{楞次定律}.

理解楞次定律需要注意以下几点:
\begin{enumerate}
    \item 谁阻碍谁?——是感应电流的磁通量阻碍原磁通量.
    \item 阻碍什么?——阻碍的是磁通量的变化, 也就是使$|\Delta\Phi|$减小, 而不是阻碍磁通量本身.
    \item 结果如何?——阻碍并不是阻止, 更不是逆转, 只是延缓了磁通量的变化. 
\end{enumerate}

由楞次定律可知, \textbf{当磁通量增加时, 感应电流产生的磁场方向与原磁场方向相反;当磁通量减少时, 感应电流产生的磁场方向与原磁场方向相同.}
这个规律我们可以简单说成\textbf{``增反减同''}.

楞次定律中的``阻碍''作用是把其他形式的能量(或其他电路的电能)转化(或转移)为感应电流所在回路的电能.
在这个过程中, 能量总是守恒的. 楞次定律正是能量守恒定律在电磁感应现象中的体现. 

从本质上看, 楞次定律可广义的表述为:
\textbf{感应电流的效果总是要反抗引起感应电流的原因. }常见的情况有以下几种:
\begin{enumerate}
    \item 当感应电流是由相对运动引起时, 感应电流受到的安培力总是阻碍相对运动. 即\textbf{``来拒去留'' }.
    \item 当感应电流因磁场变化而引起时, 感应电流受到的安培力通常使回路发生运动以阻碍原磁通量的变化.
    \item 感应电流使回路受到的安培力总要引起回路发生形变或具有形变的趋势, 而形变或形变的趋势仍是向阻碍磁通量变化的方向进行.比如\textbf{``增缩减扩''}.
\end{enumerate}

可以用右手的手掌和手指的方向来判断导线切割磁感
线时产生的感应电流的方向,即:\textbf{伸开右手,
使拇指与其余四个手指垂直, 并且都与手掌在同一个平面
内;让磁感线从掌心进入,并使拇指指向导线运动的方向,
这时四指所指的方向就是感应电流的方向.}这就是更便
于判定导线切割磁感线时感应电流方向的\textbf{右手定则}.

需要注意的是, 右手定则只适用于导线切割磁感线的情况, 是
楞次定律的特殊形式.

\section{法拉第电磁感应定律}

\subsection{法拉第电磁感应定律}

在用导线切割磁感线产生感应电流的实验中, 导线切
割磁感线的速度越快, 磁体的磁场越强, 产生的感应电流
就越大.在向线圈中插入条形磁体的实验中, 磁体的磁场
越强, 插入的速度越快, 产生的感应电流就越大.这些现
象表明, 当回路中的电阻一定时, 感应电流的大小
可能与磁通量变化的快慢有关, 也就是说, 感应电流的大小与磁
通量的变化率有关.

电路中有感应电流, 就一定有电动势.如果电路没有闭
合, 这时虽然没有感应电流, 电动势依然存在.
电磁感应现象中产生的电动势称为\textbf{感应电动势},
产生感应电动势的那部分导体就相当于电源.

德国物理学家纽曼和韦伯在严格分析后指出:\textbf{闭合电路中感
应电动势的大小, 跟穿过这一电路的磁通量的变化率成正
比.}后人称之为\textbf{法拉第电磁感应定律}.

如果在极短的时间$\Delta t$内, 磁通量的变化量为$\Delta \Phi$, 产生的感应电动势为$E$,
那么法拉第电磁感应定律就可以表示为$$E = k \frac{\Delta \Phi}{\Delta t},$$
其中$k$是比例系数.如果电动势$E$, 磁通量$\Phi$, 时间$t$的单位分别用伏(V), 
韦伯(Wb), 秒(s), 那么$k = 1$. 于是
\begin{empheq}[box=\fbox]{equation*}
    E = \frac{\Delta \Phi}{\Delta t}.
\end{empheq}
或者利用极限写成
$$E = \lim_{\Delta t\rightarrow 0}\frac{\Delta \Phi}{\Delta t}.$$

\setlength{\abovedisplayskip}{10pt}
\setlength{\belowdisplayskip}{10pt}
闭合电路常常是一个匝数为 $n$ 的线圈, 而且穿过每匝
线圈的磁通量总是相同的. 这样的线圈可以看成是由
$n$ 个单匝线圈串联而成的, 因此整个线圈中的感应电动势是
单匝线圈的 $n$ 倍, 即
\begin{empheq}[box=\fbox]{equation*}
    E = n\frac{\Delta \Phi}{\Delta t}.
\end{empheq}

\setlength{\abovedisplayskip}{0pt}
\setlength{\belowdisplayskip}{0pt}

在中学阶段, 我们只计算感应电动势的大小, 不涉及它的正负, 
因此上式中的$\Delta\Phi$应取绝对值. 至于感应电流的方向, 
我们用楞次定律或右手定则来判断.

不论电路是否闭合, 只要穿过电路的磁通量发生变化, 都会产生
感应电动势; 若电路闭合, 就会产生感应电流.

\subsection{变化的磁场产生的感应电动势}

麦克斯韦认为, 磁场变化时会在空间激发一种电场.
这种电场与静电场不同, 它不是由电荷产生的, 我们把它
叫作\textbf{感生电场}.\footnote{感生电场与静电场的性质有很大不同. 比如, 感生电场的电场线是闭合曲线; 感生电场的电场力也不属于静电力, 而是导致感生电动势的``非静电力''.}

如果将闭合导体放置在感生电场中, 导体中的电荷将会做定向移动,
产生感应电流, 也就是说导体中产生了感应电
动势.
这种由变化的磁场产生感生电场, 从而产生的感应电动势, 
称为\textbf{感生电动势}.由法拉第电磁感应定律可知, 
感生电动势的大小为
$$E = nS\frac{\Delta B}{\Delta t}, $$
其中$S$为置于感生电场中的闭合导体的面积, $\Delta B$为磁感应强度的变化量.

如果导体中的自由电荷是正电荷, 那么它们定向移动的方向就是
感应电流的方向, 也就是感生电场的方向.

\subsection{导线切割磁感线时的感应电动势}
\label{导线切割磁感线一节}

\begin{figure}[htbp]
    \centering
    \includegraphics[width=0.4\textwidth]{pic/3.6-1.pdf}
    \caption{导线切割磁感线}
    \label{3.6-1}
\end{figure}

导体切割磁感线时, 法拉第电磁感应定律可以写成更简单的形式.
下面我们推导这个公式.

如图\ref{3.6-1}, 矩形线框$CDMN$置于磁感应强度为$B$的匀强磁场中, 线框
平面与磁感应强度方向垂直.  设可以移动的导体棒$MN$的长度为$l$, 
它以速度$v$向右运动.

在很短的$\Delta t$时间内, 导体棒从$MN$运动到$M_1N_1$.这个过程中
矩形线框的面积变化量为$$\Delta S = lv\Delta t.$$
穿过线框的磁通量变化量则是$$\Delta \Phi = B\Delta S = Blv\Delta t.$$

根据法拉第电磁感应定律$E = \displaystyle\frac{\Delta\Phi}{\Delta t}$, 
可求得感应电动势
\begin{empheq}[box=\fbox]{equation}
    E = Blv.
    \label{平动切割公式}
\end{empheq}
这就是导体平动切割磁感线产生感应电动势的计算公式, 
简称\textbf{平动切割公式}\footnote{平动与转动相对. 事实上, 我们还有转动切割公式, 但这超出了本书的范围.}.

如果导线的运动方向与导线本身是垂直的, 但与磁感
线方向有一个夹角$\theta$, 那么导线的速度$v$与
磁感应强度方向垂直的分量切割磁感线, 容易知道, 
产生的感应电动势为
$$E = Blv\sin\theta.$$

由导体运动而导致磁通量变化, 从而产生的感应电动势, 
叫作\textbf{动生电动势}, 产生的电流叫作\textbf{动生电流}.

\begin{wrapfigure}{o}{6cm}
    \flushright
    \includegraphics[width=0.24\textwidth]{pic/3.6-2.pdf}
    \caption{运动的孤立导体棒}
    \label{3.6-2}
\end{wrapfigure}

在以上的讨论中, 导体棒连接在线框中, 感应电动势
使线框中的自由电荷做定向移动, 从而产生感应电流.
如果是孤立的直导体棒在匀强磁场中运动, 因为没有闭合回路, 
所以也没有感应电流. 但是导体棒内的自由电荷
会随着导体棒运动, 并因此受到洛伦兹力. 导体棒中自由
电荷相对于纸面的运动大致沿什么方向?导体棒哪端的电势比较高?
导体棒一直运动下去, 自由电荷是否总会沿着导体棒运动? 
为讨论方便, 下面把导体内的自由电荷看作正电荷.

由左手定则可知, 在导体棒向右运动的过程中(图\ref{3.6-2}), 导体棒内的自由电荷
受向上的洛伦兹力, 所以自由电荷相对于导体棒向上运动, 
也就是相对于纸面斜向右上方运动.

另一方面, 虽然右手定则用于判断感应电流的方向, 但是感应电流
本质是自由电荷定向移动产生的, 所以我们也用右手定则判断
运动导体棒的自由电荷运动方向. 根据右手定则, 导体棒内的自由电荷
相对于导体棒向上运动.

一段导线在做切割磁感线的运动时相当于一个电源, 通
过上面的分析可以看到, 这时的非静电力与洛伦兹力有关. 
在电源内, 非静电力做功, 把电荷从低电势点移动到高电势点, 
因此$C$点处电势高, $D$点处电势低; $C$点相当于电源正极, 
$D$点相当于电源负极.

如果导体棒一直运动下去, 那么导体棒两端将聚集异种电荷而形成电场, 
电场力与洛伦兹力反向. 因为$C, D$两端聚集的电荷越来越多, 在$CD$棒
间的电场越来越强. 当自由电荷受到的电场力等于洛伦兹力时, 自由电荷不再
相对于导体棒定向移动.

在图\ref{3.6-1} 中, 由于导体棒运动产生感应电动势, 电路中有
电流通过, 导体棒在运动过程中会受到安培力的作用. 可
以判断, \textbf{安培力的方向与推动导体棒运动的力的方向是相
反的}. 这时即使导体棒做匀速运动, 推力也做功. 如果没
有推力的作用, 由于安培力做负功,导体棒将消耗的自身的机械能.

在\ref{切割磁感线一节} 节中, 我们将会从力学角度更详细地分析导体切割磁感线的运动.

\section{电磁感应的应用}

\subsection{涡流}\label{涡流}
当某线圈中的电流随时间变化时, 由于电磁感应, 附
近的另一个线圈中可能会产生感应电流. 这样的感应电流
仍然是环绕的, 它产生的磁场阻碍通过它的磁通量的变化.如果用图
表示这样的感应电流, 看起来就像水中的漩涡, 所以把它
叫作\textbf{涡电流}, 简称\textbf{涡流}.\footnote{形成涡流的电场
不是恒定电场, 而是感生电场. 所以感生电场也叫作涡旋电场.}

通过金属块的磁通量变化时, 金属块中会感应出涡流,
并且会产生热量.冶炼炉, 电磁炉利用的就是涡流.

真空冶炼炉的炉外有线圈, 线圈中通入迅速变化的电
流, 炉内的金属中产生涡流.涡流产生的热量使金属熔化.这种方式
不需要空气, 可以在真空中进行, 防止空气中的杂质进入金属.

电磁炉加热食物时, 迅速变化的电流通过电磁炉面板下方的
线圈, 线圈周围产生迅速变化的磁场, 变化的磁场使面
板上方的铁锅底部产生涡流. 铁锅迅速发热, 从而达到加
热食物的目的; 而电磁炉的面板却不会发热, 因此电磁炉只能
搭配金属灶具使用.

电动机, 变压器的线圈都绕在铁芯上. 线圈中流过变
化的电流, 在铁芯中产生的涡流使铁芯发热, 浪费了能量,
还可能损坏电器.因此, 我们要想办法减小涡流. 途径之
一是增大铁芯材料的电阻率, 另一个途径就是用互相绝缘的硅钢片叠
成的铁芯来代替整块硅钢铁芯.

机场, 车站的金属探测器, 士兵用的地雷探测器, 利用的也是涡流.
探测器中的线圈有不断变化的电流, 当它靠近金属时, 金属中感应出涡流.
涡流产生的磁场反过来影响探测器中的线圈, 使探测器报警.

\subsection{电磁阻尼\bre 电磁驱动}

上一节中我们分析到, 当导体在磁场中运动时, 感应电流会使导体受到安培力, 而安培力的方向
总是阻碍导体的运动.这种现象称为\textbf{电磁阻尼}.

这一现象可以用楞次定律解释:闭合导体与发生切割磁感线的运动时,
由于通过它的磁通量发生了变化, 闭合导体会产生感应电流.
我们知道, 这称为动生电流. 根据楞次定律, 动生电流产生的磁场
必定阻碍磁通量的变化; 而磁通量变化使由于导体运动引起的, 所以
动生电流受到的安培力会阻碍导体的运动.

如果是磁场相对于导体运动,那么导体中会产生感应电流,
感应电流使导体受到安培力的作用,安培力使导体运动起
来,这种作用称为\textbf{电磁驱动}.
交流感应电动机就是利用电磁驱动的原理工作的.

\section{互感\bre 自感}
\subsection{互感现象}
在 \ref{电磁感应1} 节和 \ref{涡流} 节中我们提到, 一个线圈中的电流变化时, 它
所产生的变化的磁场会在另一个线圈中产生感应电动势. 
这种现象叫作\textbf{互感}, 这种感应电动势叫作\textbf{互感电动势}.
利用互感现象可以把能量由一个线圈传递到另一个线圈, 在电子技术中有广泛应用.

互感现象不仅发生在两个线圈间, 而是可以发生在任何两个靠近的电路间.
有时, 互感现象会影响电路的正常工作, 这时要设法减小电
路间的互感.

\subsection{自感现象}
当一个线圈中的电流变化时, 它所产生的变化的磁场
在线圈本身激发出感应电动势. 这种现象称为\textbf{自感},
由于自感而产生的感应电动势叫作\textbf{自感电动势}.
根据楞次定律, \textbf{自感电动势总是阻碍电流的变化}.

将灯泡与一个线圈串联, 当闭合开关时, 我们发现灯泡慢慢亮起, 这是因为线圈产生了与原电动势方向相反的自感电动势, 减缓了电流的增加.

将灯泡与一个线圈并联, 当断开开关时, 干路电流立刻消失, 但我们发现灯泡慢慢熄灭(或者闪亮一下再熄灭), 这是因为线圈所在的支路上产生了与原电流方向相同的自感电流.
需要注意的是, 断开开关后, 流过灯泡的电流方向与之前相反. 如果把灯泡换成二极管, 则二极管会立即熄灭.

对于第二个例子来说, 事实上, 如果线圈的直流电阻大于灯泡的电阻, 那么电路稳定时线圈所在的支路电流较小, 灯泡会慢慢熄灭;
如果线圈的直流电阻小于灯泡的电阻, 则电路稳定时线圈所在支路得电流更大, 断电一瞬间产生的感应电流经过灯泡所在的支路, 并且
感应电流大于原本流经灯泡的电流, 所以灯泡会闪亮一下再熄灭.

家中灯在开关时, 开关中的金属片容易产生电火花, 这也是自感电动势的存在造成的.
所以说, 燃气泄漏时不能随便开关灯. 
变压器, 电动机等设备中有匝数很多的线圈, 当电路
中的开关断开时会产生更大的自感电动势. 因此, 切断这类电路时, 必须采用特制的安全开关, 
避免出现电火花.

\subsection{自感系数}

根据法拉第电磁感应定律,自感电动势$$E = n\frac {\Delta \Phi}{\Delta t} = nS\frac{\Delta B}{\Delta t}.$$
其中$n$是线圈匝数.

实验表明, 磁场的强弱正比于电流的强弱, 因此
$$E \propto nS\frac{\Delta I}{\Delta t}.$$
写成等式, 就是
$$E = L \frac{\Delta I}{\Delta t}.$$
其中, $L$叫作\textbf{自感系数}, 简称\textbf{电感}.它
与线圈的大小, 形状, 匝数, 以及是否有铁芯等因素有关.
电感的单位是\textbf{亨利},简称\textbf{亨},符号是 H.

\subsection{磁场的能量}

在上一小节的例子中, 开关断开后, 灯泡的发光还能维
持一小段时间, 有时甚至会比开关断开之前更亮. 这时灯
泡的能量是从哪里来的?

开关断开以后, 线圈中的电流并未立即消失, 线圈中
有电流, 有电流就有磁场, 能量储存在磁场中. 当开关闭
合时, 线圈中的电流从无到有, 其中的磁场也是从无到有, 
这可以看作电源把能量输送给磁场, 储存在磁场中. 

当线圈刚刚接通电源的时候, 自感电动势阻碍线圈中
电流的增加; 当电源断开的时候, 自感电动势又阻碍线圈
中电流的减小. 线圈的自感系数越大, 这个现象越明显, 
线圈能够体现电的``惯性''. 

\section{通电导线在磁场中的运动}
\label{切割磁感线一节}
\subsection{通电导线在磁场中的受力}

通电导线在磁场中的运动问题, 是电磁感应, 电路与力学的结合.分析这类问题的思路如下.
\begin{enumerate}
    \item 用法拉第电磁感应定律和楞次定律求感应电动势的大小和方向;
    \item 画出必要的等效电路图根据闭合电路欧姆定律求回路中的电流;
    \item 分析导体的受力情况(包括安培力), 结合力学方程求解.
\end{enumerate}

在与电磁感应结合的动力学问题中,
导体一般不是做匀变速运动,而是经历一个动态变化的过程,再趋于一个稳定状态.
因此,解这类问题时,正确进行动态分析,确定最终的稳定状态是解题的关键.

下面举一个例子.

\begin{example}
    在\pageref{3.6-1} 页的图\ref{3.6-1} 中, 水平放置的光滑的矩形线框 $CDMN$ 置于磁感应强度为 $B$ 的匀强磁场中, 匀强磁场的方向垂直于线框平面, 导体棒 $MN$ 的长度为
$l$. 导体棒的电阻为$R$, 矩形线框的电阻不计. 现给导体棒一个向右的初速度$v_0$, 则导体棒此后的运动状态如何? 
\end{example}

\textbf{分析}\bre 导体棒运动切割磁感线, 根据平动切割公式 \eqref{平动切割公式}, 它产生大小为
$$E = Blv_0$$的感应电动势. 此时, 在闭合电路中, 导体棒就相当于电源. 由右手定则可以判断, 在导体棒上感应电流
的方向向上, 正电荷从$M$端移向$N$端. 我们知道, 在电源内, 非静电力把正电荷从低电势点迁移至
高电势点, 所以$M$端相当于电源负极, $N$端相当于电源正极.

根据闭合电路的欧姆定律, 回路中的电流$$I = \frac{E}{R} = \frac{Blv_0}{R}.$$
由于导体棒中存在电流, 它在磁场中会受到安培力的作用, 大小为$$F = ILB = \frac{Blv_0}{R}lB = \frac{B^2l^2v_0}{R},$$
方向向左.导体棒在安培力的作用下做减速运动, 这称为电磁阻尼. 随着导体棒的速度减小, 电流中的感应电流也减小, 
从而物体受到的安培力也减小, 但安培力仍然阻碍导体棒的运动. 因此, \textbf{导体棒做加速度减小的减速运动}, 直至静止在线框上.

\subsection{电磁感应中的能量问题}

电磁感应过程中往往涉及多种能量之间的相互转化. 
因为感应电流在磁场中必定受到安培力的作用, 安培力做功的过程涉及电能与其他形式能的转化.

要分析电磁感应现象中的能量转化, 首先要确定安培力所做的功是正功还是负功.

如果电路里本来就有电流, 那么安培力可能会做正功, 把电能转化为机械能;
如果电路里原本没有电流, 我们用一根导体棒切割磁感线, 那么感应电流受到的安培力一般做负功,
阻碍感应电流的增加. 下面我们就具体分析安培力做负功的情况.

我们仍沿用上一个例子中的物理量. 不同的是, 为了简单起见, 我们先假设导体棒在外力的作用下做速度为$v$的匀速运动.

根据平动切割公式 \eqref{平动切割公式}, 导体棒产生的感应电动势
$$E = Blv, $$
在$\Delta t$时间内, 电流做的功为
$$W_\text{电} = EI\Delta t = Blv\cdot I\Delta t.$$
因为电路中只有电阻, 所以电流做的功全部转化为焦耳热, 即$$W_\text{电} = Q.$$

导体棒切割磁感线产生了感应电流, 因此会受到安培力作用. 分析可知, 
安培力的方向与导体棒运动方向相反, 大小为
$$F = IlB.$$
于是, 安培力所做的功为$$W_\text{安} = -IlB\cdot v\Delta t.$$

可以看出, 安培力做的功与电流所做的功大小相等, 方向相反, 数值上等于电流产生的热量.
也就是说, \textbf{导体棒克服安培力做的功等于电路产生的焦耳热}, 即
\begin{empheq}[box=\fbox]{equation*}
    - W_\text{安} = Q.
\end{empheq}

上述推导过程中为了简单起见, 我们假定导体棒匀速运动. 事实上, 如果导体棒非匀速运动, 我们只需把上面的
$E, v, I$改为$\overline{E},\overline{v},\overline{I}$, 便可以得到相同的结论.

既然是研究能量问题, 我们很自然的想到能量的守恒与转化. 在解决问题的时候, 使用上面的结论并结合
动能定理列方程, 或者利用能量的守恒与转化列方程, 可以达到相似的效果.下面我们看一个例子, 并用
两种不同的方法解决问题.

\begin{figure}
    \centering
    \includegraphics[width=0.4\textwidth]{pic/4.7-2.pdf}
    \caption{例\ref{切割磁感线的例题} 题图}
    \label{4.7-2}
\end{figure}

\begin{example}
    \label{切割磁感线的例题}
    如图\ref{4.7-2} 所示, $PQ, MN$是两条平行金属轨道, 轨道平面与水平面的夹角为$\theta = 36^{\circ}$, 上端连接电阻$R$. 
    轨道上方放置一根质量为$m = 1\unit{kg}$的金属棒$ab$, 金属棒和轨道其他部分的电阻不计. 
    空间内存在方向垂直于轨道平面向上的匀强磁场, 磁感应强度为$B = 1\unit{T}$. 金属棒$ab$从轨道上方某一位置
    下滑,至位移为$x = 5\unit{m}$时恰好以速度$v = 4\unit{m/s}$做匀速运动. 已知重力加速度$g = 10\unit{m/s^2}$, 
    $\sin 36^{\circ} = 0.6$, $\cos 36^{\circ} = 0.8$, 金属棒与轨道间的动摩擦因数$\mu = 0.5$.
    求此过程中电阻$R$上产生的热量$Q$.
\end{example}
\begin{solution}
    {\fangsong 解法一}\bre 金属棒从静止到匀速运动的过程中, 重力对它做正功, 摩擦力和安培力做负功.
    设安培力做功的大小为$W_\text{克安}$, 对这个过程应用动能定理, 有
    $$mgx\sin \theta - \mu mgx\cos\theta - W_\text{克安} = \frac12mv^2 - 0.$$
    
    而导体棒克服安培力做的功等于电阻上产生的焦耳热, 即
    $$W_\text{克安} = Q.$$

    由此解得$Q = 18\unit{J}.$

    {\fangsong 解法二}\bre 金属棒从静止到匀速运动的过程中, 重力势能减小, 动能和内能增大.
    重力势能转化为动能, 摩擦热和电阻上的焦耳热.
    
    列出能量守恒与转化方程, 有
    $$mgx\sin \theta = \frac12mv^2 + \mu mgx\cos\theta + Q.$$

    由此解得$Q = 18\unit{J}.$
\end{solution}
可以看出, 能量守恒和动能定理实际上是等价的. 

\subsection{电磁感应与电荷量}

由平动切割公式的推导过程可以知道, 它也适用于计算平均感应电动势, 即
$$\overline{E} = Bl\overline{v}.$$
式中$\overline{E}$为时间$\Delta t$内的平均感应电动势, 
$\overline{v}$为时间$\Delta t$内的平均速度.设导体棒
位移为$x$,则$x = \overline{v}\Delta t$.

上面所说的``平均''均指某个物理量关于时间的均值. 类似地, 
时间$\Delta t$内的平均电流
$$\overline{I} = \frac{E}{R} = \frac{Bl\overline{v}}{R} = 
\frac{Bl\overline{x}}{R\Delta t}.$$
其中$R$是电路中的总电阻.
于是, 通过电路中某一导体的电荷量为
$$q = \overline{I}\Delta t = \frac{Bl\overline{x}}{R\Delta t}\Delta t = \frac{Blx}{R}.$$

\setlength{\abovedisplayskip}{10pt}
\setlength{\belowdisplayskip}{10pt}

应当考虑到, $\overline{I}$表示的是电流关于时间的均值(而非关于位移或其他物理量的均值), 所以用它乘以
$\Delta t$来表示电荷量是合理的.这就是导体棒切割磁感线时流经某一导体的电荷量的表达式, 即
\begin{empheq}[box=\fbox]{equation*}
    q = \frac{Blx}{R}.
\end{empheq}
式中$x$是导体棒的位移, $R$是电路中的总电阻.由推导过程可知, 无论导体棒
匀速运动还是非匀速运动, 只要知道导体棒的位移$x$, 上式均适用.

\setlength{\abovedisplayskip}{0pt}
\setlength{\belowdisplayskip}{0pt}

回顾我们学过的知识, 力的冲量$\overline{F}\Delta t$也有类似的形式.

如果放置在水平线框上的导体棒以某一初速度运动, 并置于垂直于线框平面的匀强磁场中, 那么导体棒将在
安培力的作用下减速, 即电磁阻尼.设匀强磁场的磁感应强度为$B$, 则在时间$\Delta t$内
导体棒受到的平均安培力$$\overline{F} = \overline{I}lB, $$
其中$\overline{I}$为$\Delta t$内的平均感应电流.

如果导体棒在时间$\Delta t$内动能的增量为$\Delta p$, 则以导体棒的运动方向为正方向, 由动量定理有
$$\overline{F}\Delta t = \Delta p.$$

把以上两式代入$q = \overline{I}\Delta t$,可以消去$\overline{F}$, 得到
$$q = -\frac{\Delta p}{lB}.$$
当除安培力之外的其他力的合冲量为0时, 通过导体棒的电荷量符合上式.

\begin{figure}
    \centering
    \includegraphics[width=0.5\textwidth]{pic/4.7-3.pdf}
    \caption{例\ref{斜面体题} 题图}
    \label{4.7-3}
\end{figure}

\begin{example}
    \label{斜面体题}
    图\ref{4.7-3} 为横截面为一等腰三角形的斜面体, 斜面倾角均为$\theta = 30^{\circ}$.
    两根足够长的平行金属导轨固定在斜面上, 导轨电阻不计, 导轨间距$l = 0.4\unit{m}$.
    两导轨所处的斜面空间存在磁感应强度大小为$B = 0.5\unit{T}$, 方向分别垂直于两斜面向上的磁场.
    在右边的斜面轨道上放置一根质量$m_1 = 0.1\unit{kg}$, 电阻$R_1 = 0.1\unit{\Upomega}$的金属棒$ab$, 
    $ab$刚好不下滑.然后, 在左边的轨道上放置一根质量$m_2 = 0.4\unit{kg}$, 
    电阻$R_2 = 0.1\unit{\Upomega}$的光滑导体棒$cd$, 并让$cd$棒由静止开始下滑.
    $cd$在滑动过程中始终处于磁场中, $ab$, $cd$ 始终与导轨垂直, 且两端与导轨保持良好接触.
    重力加速度$g = 10\unit{m/s^2}$. 求:

    (1) $ab$棒刚要向上滑动时, $cd$棒的速度$v$;

    (2)若从$cd$棒开始下滑, 到$ab$棒刚要向上滑动的过程中, $cd$棒滑动的距离$x = 3.8\unit{m}$, 求此过程中$cd$棒
    上产生的热量$Q$;

    (3) 在(2)的条件下, $cd$棒从静止开始, 运动到$x = 3.8\unit{m}$所用的时间$t$.
\end{example}
\begin{solution}
    (1) 根据题意, 在$cd$下滑之前, $ab$恰好处于静止状态. 设摩擦力大小为$F_\text{f}$, 则有
    $$m_1g\sin \theta = F_\text{f}.$$

    $ab$棒刚要向上滑动时, 它受到安培力, 摩擦力和重力(要注意摩擦力的方向与之前相反).
    $$IlB = m_1g\sin \theta + F_\text{f}.$$

    $cd$棒切割磁感线, 产生的感应电动势$$E = Blv,$$
    由闭合电路欧姆定律可得$$I = \frac{E}{R_1 + R_2}.$$

    联立以上方程, 代入数据, 解得$v = 5\unit{m/s}.$
    
    (2)设此过程中回路产生的总焦耳热为$Q_\text{总}$, 
    则由焦耳定律容易得到$Q$与$Q_\text{总}$的关系, 即
    $$\frac{Q}{Q_\text{总}} = \frac{R_1}{R_1+R_2}.$$

    根据能量的守恒与转化, 有$$m_2gx\sin\theta = \frac12m_2v^2+Q_\text{总}.$$

    联立解得$Q = 1.3\unit{J}.$

    (3)依题意, 当$cd$棒运动到$x = 3.8\unit{m}$时, $ab$棒刚要向上滑动, 此时
    $cd$棒的速度为$v$.以沿斜面向下为正方向, 对这一过程应用动量定理, 可得
    \begin{equation*}
        t\cdot m_2g\sin\theta - \overline{I}lBt = m_2v - 0, \tag{i}
    \end{equation*}
    其中$\overline{I}$表示时间$t$内回路中的平均电流, 
    \begin{equation*}
        \overline{I} = \frac{Bl\overline{v}}{R_1+R_2}, \tag{ii}
    \end{equation*}
    其中$\overline{v}$表示$t$内金属棒$cd$的平均速率.

    由于$\overline{v}t = x$, 所以把 (ii) 代入 (i) 后可以消去$\overline{v}$, 得到
    \begin{equation*}
        t\cdot m_2g\sin\theta - \frac{B^2l^2x}{R_1+R_2} = m_2v - 0.
    \end{equation*}
    代入数据, 解得$t = 1.38\unit{s}.$
\end{solution}

\chapter{交变电流}

我们已经学过了恒定电流. 在恒定电流的电路中, 电
源的电动势不随时间变化, 电路中的电流, 电压也不随时
间变化. 

在工农业用电, 生活用电的电力系统中, 
发电机产生的电动势是随时间做周期性变化的, 因
而, 很多用电器中的电流, 电压也随时间做周期性变化.
方向随时间变化的电流叫作\textbf{交变电流}, 简称\textbf{交流}. 方向不随时间变化的电流称为\textbf{直流}. 
电池供给的电流方向不随时间变化, 所以属于直流.

\section{正弦式交变电流}

图\ref{4.1-1} 为一种交流发电机的示意图. 装置中两磁极之间产生的磁场
可近似为磁感应强度为$B$的匀强磁场. 为了便于
观察, 图中只画出了其中的一匝线圈. 线圈的 $AB$ 边连在金
属滑环 K 上, $CD$ 边连在滑环 L 上; 导体做的两个电刷 E, 
F 分别压在两个滑环上, 线圈在转动时可以通过滑环和电刷
保持与外电路的连接. 

\begin{figure}[h]
    \centering
    \includegraphics[width=16cm]{pic/4.1-1.png}
    \caption{交流发电机示意图}
    \label{4.1-1}
\end{figure}

\begin{wrapfigure}{o}{7cm}
    \centering
    \includegraphics[width=0.32\textwidth]{pic/4.1-2.pdf}
    \caption{线圈转动时产生的感应电流}
    \label{4.1-2}
\end{wrapfigure}

假定矩形线圈$ABCD$绕$OO^{\prime}$轴沿
逆时针方向匀速转动, 我们来分析电流的变化.

根据法拉第电磁感应定律, 感应电动势的大小与磁通量的变化率
成正比, 也就是与导体切割磁感线的速度成正比.$AB$边和$CD$边
切割磁感线, 而它们垂直于磁感线的速度不断变化, 所以感应电动势
不断变化, 感应电流也同时发生变化.

可以看出, 在图\ref{4.1-1} 的甲, 丙两图中, 线圈垂直于磁感线所在的平面, 
即$AB$边, $CD$边的速度方向平行于磁感线所在的平面.
于是它们垂直于磁感线方向的速度为0, 所以
感应电流也为0.

在乙, 丁两图中, $AB$边, $CD$边的速度方向垂直于磁感线所在的平面, 
此时它们垂直于磁感线方向的速度最大, 感应电流也相应取得最大值.

用右手定则判断乙, 丁两图电流的方向, 可以知道, 线圈从甲转到乙的过程中,
电流从E经过负载流向F; 线圈从丙转到丁的过程中, 电流从F经过负载流向E.
感应电流随时间变化的曲线大致如图\ref{4.1-2}.

可以看出, 交变电流似乎在随着正弦函数的规律变化.

\begin{wrapfigure}{o}{6cm}
    \flushright
    \includegraphics[width=0.3\textwidth]{pic/4.1-3.pdf}
    \caption{线圈转到任意位置}
    \label{4.1-3}
\end{wrapfigure}

对于图\ref{4.1-1} 所示的发电机, 设矩形线圈$AB$边长为$l$, $CD$边长为$d$, 
线圈转动的角速度为$\omega$. 线圈垂直于磁感线时所在的平面(如图\ref{4.1-1} 甲)
叫作\textbf{中性面}.设当线圈经过中性面时$t = 0$,此时感应电动势为0.
经过一段时间$t$后, 线圈转过的角度$\theta = \omega t$,  如图\ref{4.1-3} 所示.

\setlength{\abovedisplayskip}{0pt}
\setlength{\belowdisplayskip}{0pt}

线圈旋转过程中, $AB$边和$CD$边的速度$v = \omega\displaystyle\frac{d}{2}$, 
所以与磁感线垂直的速度为$$v\sin\theta = \frac12 \omega d\sin\omega t.$$
根据平动切割公式 \eqref{平动切割公式}, 它们产生的感应电动势
$$e = 2Blv\sin\theta = \omega Bld\sin\omega t= \omega BS\sin\omega t,$$ 
其中, $S$表示线圈的面积.

设$E_\mathrm{m} = \omega BS$, 就得到
\begin{equation*}
    e = E_\mathrm{m}\sin\omega t.
    \label{交流感应电动势}
\end{equation*}
可以看出, 线圈产生的感应电动势是随时间按正弦函数的规律变化的.
式中$E_\mathrm{m}$为常数, 表示变化过程中感应电动势的最大值, 称为\textbf{峰值}, 
即图\ref{4.1-1} 乙对应的感应电动势$e$的值. 如果线圈的匝数为$N$, 则$E_\mathrm{m} = N\omega BS.$

由于发电机的电动势$e$按正弦规律变化, 因此负载两端的电压$u$, 
流过的电流$i$也按正弦规律变化, 即
$$u = U_\mathrm{m}\sin\omega t,$$
$$i = I_\mathrm{m}\sin\omega t,$$
式中$U_\mathrm{m}$, $I_\mathrm{m}$分别为电压, 电流的最大值, 也叫峰值.
用$R$表示外电路的总电阻, $r$表示线圈的内阻, 根据闭合电路欧姆定律
$$U_\mathrm{m} = \frac{R\omega BS}{R+r},$$
$$I_\mathrm{m} = \frac{\omega BS}{R+r}.$$
这种按正弦规律变化的交变电流叫作\textbf{正弦式交变电流}, 简称\textbf{正弦式电流}.

上面, 我们从导体切割磁感线的角度, 利用平动切割公式
推导出了正弦式电流的表达式. 下面我们
从数学的角度进行分析.

沿用前述的条件. 设当线圈经过中性面时$t = 0$,
若线圈的面积为$S$, 则
经过时间$t$后, 通过线圈的磁通量
$$\Phi(t) = BS\cos \omega t.$$

\setlength{\abovedisplayskip}{10pt}
\setlength{\belowdisplayskip}{10pt}

根据法拉第电磁感应定律, $t$时刻线圈产生的感应电动势
$$e = \lim_{\Delta t \rightarrow 0}\frac{\Phi(t+\Delta t)-\Phi(t)}{\Delta t} = (BS\cos\omega t)^{\prime},$$
\setlength{\abovedisplayskip}{0pt}
\setlength{\belowdisplayskip}{0pt}
其中$(\ )^{\prime}$表示求其导函数. 不考虑方向问题, 则由数学知识可得
$$e = \omega BS\sin\omega t.$$
这与导体切割磁感线的观点结果相同.

前面我们从线圈经过中性面时开始计时, 得到了上面的表达式.
如果$t = 0$时线圈与中性面的夹角为$\theta_0$, 根据三角函数
的知识, 上面的表达式将会写成
$$e = E_\mathrm{m}\sin(\omega t + \theta_0),$$
$$u = U_\mathrm{m}\sin(\omega t + \theta_0),$$
$$i = I_\mathrm{m}\sin(\omega t + \theta_0).$$
特别地, 如果$\theta_0 = 
\displaystyle\frac{\uppi}{2}$, 即线圈的初始位置垂直于中性面, 那么
$$e = E_\mathrm{m}\cos\omega t,$$
$$u = U_\mathrm{m}\cos\omega t,$$
$$i = I_\mathrm{m}\cos\omega t.$$

\section{交变电流的描述}

\subsection{频率\bre 周期}
我们知道, 正弦式交变电流具有周期性. 因此可以用频率和周期
表示其变化的快慢.用$f$表示频率, $T$表示周期, 根据频率与周期的
定义可以知道
$$f = \frac{1}{T}.$$
频率的单位是赫兹(Hz), 周期的单位是秒(s).

根据三角函数的知识可知, 在表达式$$i = I_\mathrm{m}\sin\omega t$$
中, 感应电流$i$的周期$$T = \frac{2\uppi}{\omega}.$$
写成角速度$\omega$与频率$f$的关系就是
\begin{empheq}[box=\fbox]{equation*}
    \omega = 2\uppi f.
\end{empheq}

由正弦式电流的特点可以知道, \textbf{在一个周期内, 线圈转动一周, 电流的方向变化2次.}

\subsection{峰值\bre 有效值}
峰值$I_\mathrm{m}$或$U_\mathrm{m}$表示交变电流能达到的最大值.
在考虑电路安全问题时, 我们要关注峰值.比如, 
把电容器接在交流电路中, 就需要知道电压的峰值.
电容器的击穿电压要高于交流电压的峰
值(而非有效值), 否则电容器就可能被击穿.

让交变电流与恒定电流分别通过大小相同的电阻, 如
果在交变电流的一个周期内它们产生的热量相等, 而这
个恒定电流的电流, 电压分别为 $I$, $U$ ,我们就把 $I$, $U$ 叫作
这一交变电流的\textbf{有效值}. 

类似地, 让内阻相同的交流发电机与直流发电机分别接在相同的
电路上, 如果在交变电流的一个周期内它们所输出的功率相等, 而直流发电机
的电动势为$E$, 那么 $E$ 就叫作交流发电机电动势的有效值.

从定义来看, 所谓``有效''指的是电流热效应的等效.
使用交流的电气设备上, 标出的额定电压和额定电流都是有效值;
交流电表测出的数值也是有效值.以后提到交变电流的数值, 凡没有特别说明的,
都指有效值.例如, 家庭电路的电压为220\ V, 指的就是有效值.

\begin{example}
    一个周期为$T\unit{s}$的交变电流, 在$0\sim \displaystyle\frac12T\unit{s}$内的电流为$3\unit{A}$, 
    在$\displaystyle\frac12T\sim T\unit{s}$内的电流为$-1\unit{A}$ (``$-$''表示
    电流的方向与初始方向相反),求该交变电流的有效值.
\end{example}
\begin{solution}
    在一个周期$T$内, 该交变电流在电阻$R$上产生的热量
    $$Q = (3\unit{A})^2\cdot R\cdot\frac12T + (-1\unit{A})^2\cdot R\cdot\frac12T = \frac{10}{2}\unit{A^2}\cdot RT.$$

    而对于电流为$I$的恒定电流, $T$时间内在电阻$R$上产生的热量$Q = I^2RT.$
    所以$I^2 = \displaystyle\frac{10}{2}\unit{A^2}$, 即该交变电流的有效值$I = \sqrt{5}\unit{A}$.
\end{solution}

\setlength{\abovedisplayskip}{5pt}
\setlength{\belowdisplayskip}{5pt}

理论计算表明, 在正弦式电流中, 有效值$I, U, E$与峰值
$I_\mathrm{m}, U_\mathrm{m}, E_\mathrm{m}$的关系是
$$I = \frac{I_\mathrm{m}}{\sqrt{2}}, \bre\bre
U = \frac{U_\mathrm{m}}{\sqrt{2}}, \bre\bre
E = \frac{U_\mathrm{m}}{\sqrt{2}}. $$
其中, $\displaystyle\frac{1}{\sqrt{2}}\approx 0.707.$

\subsection{平均值}
类似于平均速度与瞬时速度的概念, 感应电动势的平均值的计算方法如下.
如果在时间$\Delta t$内, 通过发电机线圈的磁通量变化了$\Delta\Phi$, 
那么根据法拉第电磁感应定律, 电动势的平均值为
$$\overline{E} = \frac{\Delta\Phi}{\Delta t}.$$

如果把发电机接在电阻为$R$的纯电阻电路中, 忽略发电机内阻, 
那么交变电流的平均值为
$$\overline{I} = \frac{\overline{E}}{R}.$$

交变电流的平均值与所选取的时间段有关, 不同时间内的平均值一般不同.
我们可以利用图像与坐标轴围成的的面积在除以对应时间来计算平均值.
对于正弦式电流来说, 在一个周期内, 交变电流的平均值为0.

\section{电感器和电容器对交流的阻碍}

电阻器, 电感器和电容器对交流都有阻碍作用, 称为\textbf{阻抗}.

\subparagraph{电阻阻碍直流和交流} 电阻器一般是金属导体, 其中的电流是自由电子
定向移动而形成的. 在移动过程中, 自由电子与金属正离子不断碰撞, 阻碍了电流.
这种阻碍作用对直流和交流是一样的, 阻抗的大小就是电阻$R$.

\subparagraph{电感器阻碍交流} 把带铁芯的线圈和小灯泡串联起来, 先把电路接在直流电源上, 
再接在电动势有效值与直流电源相同的交流电源上, 可以发现接交流电源时小灯泡暗一些.

这是由于交流通过线圈时发生了电磁感应, 楞次定律告诉我们, 电磁感应对电流产生了阻碍作用,
我们把这种阻碍作用称为\textbf{感抗},用$X_\text{L}$表示.感抗是阻抗的一种. 如果不计线圈的内阻, 那么线圈对于直流来说相当于
一根无电阻的直导线; 而交流会受到线圈的感抗.

线圈的自感越大,交流的频率越高,线圈的感抗就越大.用$L$表示线圈的自感, $f$表示交流的频率, 
则感抗$$X_\text{L} = 2\uppi fL.$$

\subparagraph{交流可以通过电容器} 

我们知道, 把电容器并联在通有恒定电流的电路中, 电容器会先发生充放电, 此时电容器两端的电压就是
路端电压, 流过电容器的平均电流可由$Q = \overline{I}t,$ $Q = CU$给出. 
当电容器达到稳定状态后, 两极板间不再有电荷通过, 电容器所在支路没有电流, 相当于断路.

如果把电容器并联在交流电路中, 那么电容器两端的电压将不断变化, 电容器会不断的发生充放电.
这样, 电容器所在的支路就一直有充放电的电流, 表现为交流``通过''了电容器.

\subparagraph{电容器阻碍交流} 

把灯泡和电容器串联接在交流电源上, 再把电容器取下, 发现取下电容器后灯泡更亮一些.
这说明电容器对交流也有阻碍作用, 称为\textbf{容抗}, 用$X_\text{C}$表示.
容抗也是阻抗的一种.

容抗产生的原因是电容器极板上所带电荷对定向移动的电荷具有阻碍作用.
电容器的电容越大,交流的频率越高, 容抗越小. 用$C$表示电容器的电容, $f$
表示交流的频率, 则容抗$$X_\text{C} = \frac{1}{2\uppi fC}.$$

\section{变压器}
\subsection{变压器的原理}

\begin{figure}
    \centering
    \includegraphics[width=0.3\textwidth]{pic/5.4-1.pdf}
    \caption{变压器}
    \label{5.4-1}
\end{figure}

变压器是由闭合铁芯和绕在铁芯上的两个线圈组成的. 
一个线圈与交流电源连接, 叫作\textbf{原线圈}, 也叫初级线圈; 另一个线圈与
负载连接, 叫作\textbf{副线圈}, 也叫次级线圈(图\ref{5.4-1}). 
原线圈在其所处回路中充当负载, 副线圈在其所处回路中充当电源.

互感现象是变压器工作的基础. 电流通过原线圈时在
铁芯中激发磁场, 由于电流的大小和方向在不断变化, 铁
芯中的磁场也在不断变化. 变化的磁场在副线圈中产生感
应电动势, 所以尽管两个线圈之间没有导线相连, 副线圈
也能够输出电流. 

变压器只能改变交流的电压, 而恒定电流不能通过变压器.正是因为
交流的电压容易改变, 所以它在生产生活中得到了广泛的应用.

在输入的交流电压一定时, 原线圈, 副线圈取不同
的匝数, 副线圈输出的电压也不一样, 变压器由此得名.

我们把没有能量损耗的变压器叫作理想变压器.理想变压器的特点如下.
\begin{enumerate}
    \item 无磁损, 即变压器铁芯内无漏磁;
    \item 无铜损, 即原副线圈不计内阻, 有电流通过时不产生焦耳热;
    \item 无铁损, 即闭合铁芯内的涡流为零.
\end{enumerate}

下面我们就研究理想变压器原线圈, 副线圈两端的电压与线圈匝数的关系.

\subsection{理想变压器的变压规律}

对于理想变压器, 由于原副线圈缠绕在同一个铁芯上, 并且不计漏磁, 所以两线圈内磁通量时刻相等, 
磁通量的变化率$\displaystyle\frac{\Delta\Phi}{\Delta t}$
也时刻相等.用$E_1, E_2$分别表示原副线圈内的感应电动势, 
根据法拉第电磁感应定律有
$$E_1 = n_1\frac{\Delta\Phi}{\Delta t}, \bre \bre E_2 = n_2\frac{\Delta\Phi}{\Delta t}, $$
所以$$\frac{E_1}{E_2} = \frac{n_1}{n_2}.$$

由于不计原副线圈的电阻, 因此原线圈两端的电压$U_1=E_1$, 
副线圈两端的电压$U_2=E_2$, 那么
\begin{empheq}[box=\fbox]{equation*}
    \frac{U_1}{U_2} = \frac{n_1}{n_2}.
\end{empheq}
这就是说, \textbf{理想变压器原副线圈的电压之比, 等于原副线圈的匝数之比.}

\setlength{\abovedisplayskip}{0pt}
\setlength{\belowdisplayskip}{0pt}

实验数据并没有严格遵循这样的规律. 这是因为, 变压器线圈通过电流时会发热(铜损); 
铁芯在交变磁场的作用下也会发热(铁损); 此外, 交变电流产生的磁场也不可
能完全局限在铁芯内(磁损). 所有这些, 使得变压器工作时有能
量损耗. 但有些变压器的能量损耗很小, 可以忽略. 忽略这些
能量损耗后, 我们认为\textbf{理想变压器的输入功率与输出功率是相等的}, 即
$$P_1 = P_2.$$
由此, 我们立即可以推导理想变压器电流与匝数的关系.

\setlength{\abovedisplayskip}{5pt}
\setlength{\belowdisplayskip}{5pt}

根据理想变压器的输入功率等于输出功率,
即$I_1U_2=I_2U_2$, 以及理想变压器电压与匝数的关系, 
可得$$\frac{I_1}{I_2} = \frac{n_2}{n_1}.$$
即通过原副线圈的电流与原副线圈的匝数成反比.

如果副线圈的电压比原线圈的电压低,这样的变压器
叫作\textbf{降压变压器},反之则叫\textbf{升压变压器}.
变压器高压线圈匝数多而导线细, 低压线圈匝数少而导线粗, 
这是高, 低压线圈最直接的区别方法.

如果副线圈上不接负载, 即变压器空载时, 无电流, 电功率输出, 所以输入功率也为零;
如果副线圈短路, 副线圈中电流$I_2$极大, 则原线圈中电流$I_1$也极大, 
将会把变压器烧坏.

\subsection{有多个副线圈时的变压规律}

当理想变压器上连有一个原线圈, 两个或多个副线圈时, 
各线圈两端的电压之比仍然等于它们的匝数之比.
用$U_1$, $n_1$表示原线圈的电压和匝数, $U_2, U_3,\cdots$, 
$n_2, n_3,\cdots$分别表示各副线圈的电压和匝数, 则有
$$\frac{U_1}{n_1} = \frac{U_2}{n_2} = \frac{U_3}{n_3} = \cdots.$$ 
无论副线圈是两个还是更多个, 空载还是有负载, 均遵循此规律.

\setlength{\abovedisplayskip}{0pt}
\setlength{\belowdisplayskip}{0pt}

同样地, 由于理想变压器没有能量损耗, 所以输入功率$P_1$等于各线圈的输出功率$
P_2, P_3, \cdots$之和.即$$P_1 = P_2 + P_3 + \cdots.$$
由此可以推出
$$I_1n_1 = I_2n_2 + I_3n_3 + \cdots.$$

\subsection{自耦变压器}
有这样一类变压器, 它的铁芯上只绕一个线圈, 
低压线圈是高压线圈的一部分, 因此既可以作为升压变压器使用, 
也可以作为降压变压器使用. 这样的变压器称为\textbf{自耦变压器}.

通过自耦变压器, 可以从零至最大值连续调节所需电压, 
与分压电路的滑动变阻器类似.

\section{电能的输送}

\begin{figure}[h]
    \centering
    \includegraphics[width=18cm]{pic/5.4-3.pdf}
    \caption{高压输电示意图}
    \label{5.4-2}
\end{figure}

我国发电站多建在西部, 而用电量大的城市在东部沿海地区.
在输送电能的过程中, 如何减少能量的损失呢? 

\setlength{\abovedisplayskip}{0pt}
\setlength{\belowdisplayskip}{0pt}

如果导线的电阻为0, 那么电能将无损失地全部输送过去.
然而, 在远距离的输电中, 输电的电阻难以忽略, 电流通过输电线
产生焦耳热, 电能转化为内能. 设输电电流为$I_2$, 输电线的电阻为$r$,
则输电线上的功率损失为$$\Delta P = I_2^2 r = \frac{(U_2-U_3)^2}{r}.$$
由此可知, 减少功率损失最直接的方式是减小输电线的电阻$r$.
由电阻定律 \eqref{电阻定律} 可知, 要选择电阻率较小的材料
作为输电线, 比如铜和铝, 其次要减小输电线的长度,
增加输电线的横截面积. 减小长度显然不太可行; 增加输电线的横截面积
一方面可以减小电阻,另一方面会增大输电线的质量, 使布线难度
和成本增加.

另一个途径是减小输电电流$I_2$. 发电站的输出功率由用户实际使用的功率决定.
为了减小输电电流$I_2$, 同时又要保证向用户提供一定的电功率, 
就要提高输电电压$U_2$. 输电电压提高到原来的$n$倍, 输电电流将变为
原来的$\displaystyle\frac{1}{n}$, 
输电线上的功率损失将降为原来的$\displaystyle\frac{1}{n^2}$.

远距离输电的过程可以简化为图\ref{5.4-2}, 其中$U_1$到$U_2$的变压器为升压变压器,
使输电电流从$I_1$降到$I_2$, $U_3$到$U_4$的变压器为降压变压器, 使电压降为
用户所需要的电压.一般来说, 降压又会分为两次进行, 第一次降压在工厂附近降到$10\unit{kV}$, 
第二次在居民楼或办公楼附近降到$220\unit{V}$或$380\unit{V}$.

根据以上分析可知, 采用高压输电是减少输电导线上电能损失最经济有效的方法.
然而输电电压也不是越高越好, 电压越高, 对输电线路绝
缘性能的要求就越高, 线路修建费用就会增多, 对变压器的要求也相应提
高. 另一方面, 高电压会在空间中产生电场和磁场, 长期的电磁辐射会对人体产生一定危害, 
因此要选择合适的输电电压.

\part{光与热}

\chapter{光}

\section{光的折射}

\begin{wrapfigure}{o}{7cm}
    \centering
    \includegraphics[width=0.35\textwidth]{pic/4.1-1q.pdf}
    \caption{光的反射和折射}
    \label{4.1-1q}
\end{wrapfigure}

阳光照射水面时,我们能够看到水中的鱼和草,同时也
能看到太阳的倒影,这说明: 光从空气射到水面时, 一部分
光射进水中, 另一部分光返回到空气中(图\ref{4.1-1q}). 一般来说, 光从第
1 种介质射到该介质与第 2 种介质的分界面时, 一部分光会
返回到第 1 种介质, 这个现象叫作\textbf{光的反射}; 另一部分光会
进入第 2 种介质, 这个现象叫作\textbf{光的折射}.

初中时我们已经学过光的反射定律, 下面我们研究光的折射.

\subsection{折射定律}

让一束光由一种介质斜着射向另一种介
质, 例如, 从空气射向水中, 入射光线与法线的夹角 $\theta_1$ 称
为入射角, 折射光线与法线的夹角 $\theta_2$ 称为折射角, 如图\ref{4.1-1q} 所示.

\setlength{\abovedisplayskip}{10pt}
\setlength{\belowdisplayskip}{10pt}

1621 年, 荷兰
数学家斯涅耳找到了两者之间
的关系, 并把它总结为光的\textbf{折射定律}:

(1) \textbf{折射光线与入射光线, 法线处在同一平面内;}

(2) \textbf{折射光线与入射光线分别位于法线的两侧; }

(3) \textbf{入射角的正弦与折射角的正
弦成正比, 即}
\begin{empheq}[box=\fbox]{equation}
    \frac{\sin\theta_1}{\sin\theta_2} = n_{12}.
    \label{折射定律}
\end{empheq}
其中$n_{12}$是比例常数, 与两角的大小无关, 只与两种介质的性质有关.
我们在初中学过的透镜就是根据光的折射原理制成的.

事实表明, 与光的反射现象一样, 在
光的折射现象中, 光路也是可逆的.

\subsection{折射率}
下面我们主要讨论光从真空射入介质的情形, 这时, 我们把 \eqref{折射定律}
中的$n_{12}$简单记为$n$.

对于不同的介质来说, $n$的值是不同的. 可见$n$与介质有关系, $n$的值越大,
光从真空斜射入这种介质时, 偏折的角度越大.

光从真空射入某种介质发生折射时, 入射角的正弦与
折射角的正弦之比, 叫作这种介质的绝对折射率, 简称\textbf{折
射率}, 用符号 $n$ 表示.

我们规定真空的折射率为
1, 空气的折射率近似为 1 (为1.00028), 水的折射率为 1.33, 
玻璃的折射率为1.5到1.8.

\setlength{\abovedisplayskip}{0pt}
\setlength{\belowdisplayskip}{0pt}

由折射率的定义以及光的折射定律 \eqref{折射定律} 可以得到, 
当光从折射率为$n_1$的介质射入折射率为$n_2$的介质时, 
折射角$\theta_2$与入射角$\theta_1$的关系为
$$n_1\sin\theta_1 = n_2\sin\theta_2.$$

研究表明, 光在不同介质中的传播速度不同; \textbf{某种介
质的折射率, 等于光在真空中的传播速度 $c$ 与光在这种介
质中的传播速度 $v$ 之比}, 即
\begin{empheq}[box=\fbox]{equation}
    n = \frac{c}{v}.
    \label{速率关系}
\end{empheq}

由于光在真空中的传播速度 $c$ 大于光在任何介质中的
传播速度 $v$, 因而任何介质的折射率 $n$ 都大于 1. 所以, 光
从真空射入任何介质时, $\sin\theta_1$ 都大于 $\sin\theta_2$, 
即入射角总是大于折射角.

因为光在空气中的传播速度与在真空中的传播速度相近, 
所以空气的折射率近似为1. 在实际应用中, 我们遇到最多的情况是光从空气射入某种介质, 或光从某种介质
射入空气.而空气对光的传播影响很小, 可以作为真空处理.

\section{光的全反射}
对于折射率不同的两种介质, 我们把折射率较小的介质叫作\textbf{光疏介质}, 
折射率较大的介质称为\textbf{光密介质}.光密介质与光疏介质是相对的.
光从光疏介质射入光密介质时, 折射角小于入射角;光从光密介质射入光疏介质时, 
折射角大于入射角.

光从光密介质射入光疏介质时, 在界面处有一部分光\textbf{反射回}原介质中, 另一部分光\textbf{折射到}光疏介质中. 实验表明, 随着入射角的增大, 折射角也会逐渐增大, 折射光线与
法线越来越远, \textbf{且折射光线越来越弱, 反射光线越来越强}. 

当入射光增大到某一角度$C$, 使折射角达到$90^{\circ}$时, 折射光线会完全消失, 
只剩下反射光线.这种现象叫作\textbf{全反射}, 这时的入射角叫作\textbf{临界角}.

当光从光密介质射入光疏介质时, 如果入射角等于或
大于临界角, 就会发生全反射现象.需要注意的是, 当光从
光疏介质射入光密介质时, 由于折射角小于入射角, 无论入射角如何增大(直至与界面重合),
折射角也不会到达$90^{\circ}$, 因此不会发生全反射.

由光的折射定律 \eqref{折射定律} 可知, 
光从介质射入空气或真空时, 发生全反射的临界角
$C$ 与介质的折射率 $n$ 的关系是
$$\sin C = \frac{1}{n}.$$
从这个式子可以看出, 介质的折射率越大, 发生全反射的临界角越小.
科研和技术中常常通过测量临界角来测定材料的折
射率. 

全反射是自然界中常见的现象. 例如, 水中或玻璃中
的气泡, 看起来特别明亮, 就是因为光从水或玻璃射向气
泡时, 一部分光在界面上发生了全反射的缘故.

生活中利用光的全反射, 我们制作了全反射棱镜, 可用于改变光的方向; 
光导纤维可用于导光. 当光在有机玻璃棒内传播时, 如果从有机玻璃射向
空气的入射角大于临界角, 光会发生全反射, 于是光在有机玻璃棒内
沿锯齿形路线传播. 这就是光导纤维导光的原理.它由内芯和外套两层组成, \textbf{内
芯的折射率比外套的大}, 光传播时在内芯与外套的界面上
发生全反射.

医学上还用这个原理制作了内窥镜. 如果把许多光导纤维聚集成束, 并使两端的排列顺序相同, 
图像就可以从一端传播到另一端. 实际的内窥镜装有两组光纤, 一
组把光传送到人体内部进行照明,另一组把体内的图像传
出供医生观察.

光是一种电磁波. 像无线电波那样, 
光可以作为载体来传播信息. 光纤传输有传输容量大, 衰减小, 抗干扰性及保密性强等多方面的
优点.

\setlength{\abovedisplayskip}{0pt}
\setlength{\belowdisplayskip}{0pt}

\section{光的色散与光的电磁理论}

初中时我们学过, 当白光通过三棱镜时, 在光屏上会形成一条彩色光带, 这种现象叫作\textbf{光的色散}.光的色散现象说明:
\begin{enumerate}
    \item 白光是复色光, 由红,橙,黄,绿,蓝,靛,紫这几种色光组成;
    \item 同一种介质对不同颜色的光的折射率$n$不同, 对红光的折射率最小, 
    对紫光的折射率最大.由 \eqref{速率关系} 可知, 在同一种介质中, 红光的传播速度最大, 
    紫光的最小.
\end{enumerate}

前面的学习中, 我们讨论了很多光的现象, 那么光的本质是什么呢?
关于这个问题, 历史上存在不同的观点.19世纪60年代, 麦克斯韦预言了电磁波
的存在, 并认为光也是一种电磁波. 此后, 赫兹在实验中证实了这种假说.
事实上, 光除了具有电磁波的性质, 还具有粒子的性质, 我们将在之后的内容中介绍.

麦克斯韦的电磁场理论在 \ref{电磁场理论} 节有具体的介绍, 这里我们不再重复叙述.
下面我们介绍电磁波的特点.

\begin{enumerate}
    \item 电磁波是横波;
    \item 与机械波不同, 电磁波的传播不需要介质, 在真空中也能传播;
    \item 电磁波具有反射, 折射, 干涉, 衍射等一切波的特性;
    \item 电磁波在真空中的传播速度$c = 3\times 10^8\unit{m/s}$, 它与电磁波
    的频率$\nu$, 在真空中的波长$\lambda$的关系是$$c = \lambda\nu.$$其中$\nu$由波源决定. 
\end{enumerate}

按照麦克斯韦的理论, 光也是一种电磁波.光的颜色是由光的频率决定的, 而频率
是由光源决定的, 不受介质影响. \textbf{从红光到紫光, 光的频率依次增大.}

我们知道, 机械波的波速只与介质有关, 而与波的频率无关. 但电磁波并非如此.
实验事实表明, \textbf{光(或电磁波)的频率越大(波长越小), 在同种介质中的传播速率越小, 折射率越大.}

\section{光的干涉}

我们知道, 如果两列机械波的频率相同, 相位差恒定, 
振动方向相同, 就会发生干涉. 光是一种电磁波, 那么光也
应该会发生干涉现象.怎样才能观察到光的干涉现象呢?

\subsection{光的双缝干涉}
使白光源通过红色滤光片, 分离出红色光. 单色光先通过横向的单缝, 
然后照射在有两个纵向狭缝$S_1$, $S_2$的挡板上(图\ref{4.4-1}). 
我们认为从单缝发出的光是彼此平行的激光束.平行激光束垂直于挡板
通过两个狭缝, 我们在挡板后面的光屏上观察现象.
这个实验称为\textbf{杨氏双缝干涉实验}.

我们发现, 光屏上呈现明暗(红黑)相间的条纹.
如何解释这个现象呢? 事实上, 
由于狭缝$S_1$, $S_2$很小, 狭缝就成了两个波源, 
它们的频率, 相位和振动方向总是相同的.
这两个波源发出的光在挡板后面的空间互相
叠加, 发生干涉现象: 来自两个光源的光在一些位置相互
加强, 在另一些位置相互削弱, 因此在挡板后面的屏上得
到明暗相间的条纹.

\begin{wrapfigure}{o}{6cm}
    \centering
    \includegraphics[width=0.27\textwidth]{pic/4.4-1.png}
    \caption{杨氏双缝干涉实验示意图}
    \vspace{-1cm}
    \label{4.4-1}
\end{wrapfigure}

具体来说, 狭缝 $S_1$ 和 $S_2$ 发出的波源, 到屏上 $P_0$ 点的距离相
同. 由于两列波到达 $P_0$ 点的路程一样, 所
以这两列波的波峰或波谷同时到达 $P_0$ 点, 也就是相位仍然
相同. 在这点, 两列波叠加后相互加强, 因此这里出现亮
条纹. 

再考察 $P_0$ 点上方的另外一点 $P_1$ , 它距 $S_2$ 比距 $S_1$ 远一
些, 两列波到达 $P_1$ 点的路程不相同, 两列波的波峰或波谷
不一定同时到达$P_1$. 如果路程差正好是半个波长$\displaystyle\frac12\lambda$, 那
么两列波在这里恰好互相抵消, 于是这里出现暗条纹. 

对于更远一些的 $P_2$ 点, 来自两个狭缝的光波的路程差
更大. 如果路程差正好等于波长 $\lambda$, 那么这里也出现亮条
纹. 距离屏的中心越远, 路程差越大. 每当路程差等于波长的整数倍
$$k\lambda, \ k = 1, 2, 3,\cdots$$
时, 两列光波得到加强, 屏上出现亮条纹; 每当路程差等
于半波长的奇数倍$$\frac{k}{2}\lambda, \ k = 1, 3, 5, \cdots$$时, 
两列光波相互削弱, 屏上出现暗条纹. 

通过几何分析可知, 如果狭缝$S_1$, $S_2$的间距为$d$, 挡板与屏的距离为$l$, 
光的波长为$\lambda$, 那么相邻两条亮条纹或暗条纹的中心间距是
$$\Delta x = \frac{l}{d}\lambda.$$
这就是说, 对于同一套实验装置, 光的波长越大, 干涉条纹的间距越大.例如红光的
干涉条纹间距大于蓝光的干涉条纹间距, 蓝光的干涉条纹更密.

通过这个实验, 我们可以测量光的波长.

\subsection{光的薄膜干涉}

把铁丝圈在肥皂水中蘸一下, 让它挂上一层薄薄的液膜. 把这层液膜当作一
个平面镜, 用它观察酒精灯的火焰, 发现火焰呈现彩色条纹, 这是什么原因?

液膜具有厚度, 火焰的像是由液膜的前后两个表面反射的光共同形成的.
来自两个面的反射光相互叠加, 发生干涉, 也称\textbf{薄膜干涉}.
实际上, 由于重力作用, 竖直放置的液膜上薄下厚.从上到下, 两个表面
的反射光有的相互增强, 有的相互减弱, 形成了明暗相间的条纹.
由于火焰的光是复色光, 薄膜上不同颜色的光的条纹的明暗位置不同, 相互交错, 
所以, 看上去会有彩色条纹.

薄膜干涉在技术中有很多应用.比如, 可以用该原理来检测平面的平整度.
把平整的样板放在待检测平面上, 并在一侧垫上一块薄片, 使之形成一定的角度.
用单色光照射, 入射光从两个平面反射出两束光波, 形成干涉条纹. 如果待检测平面平整, 
那么两平面的间距均匀变化, 产生的干涉条纹互相平行; 如果待检测平面不平整, 那里的干涉
条纹就会发生弯曲.

常见光学镜头呈淡紫色, 是因为镜片上涂了增透膜(氟化镁), 膜的厚度经过特别设计.
人眼对绿光最敏感, 各色光射到镜头上时, 膜前后表面反射的绿色光相消, 使得绿色的透射增强, 
以此提高成像质量.由于反射光中缺少绿色, 所以光学镜头看起来呈淡紫色.这种膜称为\textbf{增透膜}.

为了减少紫外线对眼睛的伤害, 生产厂家在眼睛上涂了一层膜. 它的厚度经特别设计, 
使膜前后表面反射的紫外线增强, 从而减弱了紫外线的透射, 这种膜叫作\textbf{增反膜}.

\section{光的衍射}

在挡板上安装一个宽度可调的狭缝, 缝后放一个光屏. 
用单色平行光照射狭缝, 我们看到, 当缝比较宽时, 光沿
着直线通过狭缝, 在屏上产生一条与缝宽相当的亮条纹. 
但是, 当缝调到很窄时, 尽管亮条纹的亮度有所降低, \textbf{但
是宽度反而增大了}, 而且两侧出现了对称分布且不等距的明暗相间的条纹.

这表明, 光没有沿直线传播, 它绕过了缝的边缘, 传
播到了相当宽的地方. 这就是光的衍射现象.

从缝不同位置射入的光到达光屏侧边的距离不同, 它们互相增强或减弱, 
形成了明暗相间的条纹, 这是光的干涉现象.

如果用白光做上述实验, 得到的条纹是彩色的.
这是因为白光中包含了各种颜色的光, 衍射时不
同色光的亮条纹的位置不同, 于是各种色光就区分开了.

光在没有障碍物的均匀介质中是沿直线传播的. 对衍
射现象的研究表明, 在障碍物或狭缝的尺寸很大时, 衍射
现象不明显, 也可以近似认为光是沿直线传播的. 但是, 
在障碍物或狭缝的尺寸足够小的时候, 衍射现象十分明显, 
这时就不能说光沿直线传播了.

由于单缝衍射的条纹比较宽, 而且距离中央亮条纹较远的
条纹, 亮度也很低, 所以单缝衍射的实用性较低. 
实验表明, 如果增加狭缝的个数, 衍射条纹的宽度将
变窄, 亮度将增加. 光学仪器中用的\textbf{衍射光栅}就是据此制
成的. 它是由许多等宽的狭缝等距离地排列起来形成的光
学元件. 在一块很平的玻璃上刻出一系列等距的平行刻痕, 
刻痕产生漫反射而不太透光, 未刻的部分相当于透光的狭
缝, 这样就做成了透射光栅. 如果在高反射率
的金属上刻痕, 就可以做成反射光栅. 

\section{光的偏振}

\subsection{波的偏振}
波分为纵波和横波. 在纵波中, 各点的振动方向总与
波的传播方向在同一直线上. 在横波中, 各点的振动方向
总与波的传播方向垂直.

不同的横波, 即使传播方向相同, 振动方向也可能是
不同的. 这个现象称为\textbf{偏振}. 横波的振动方
向称为\textbf{偏振方向}. 

想象在一条绳上的横波, 它可以通过与振动方向相同的狭缝; 但如果
狭缝的方向与振动方向垂直, 绳上的横波则无法通过. 如果是在一条弹簧上传播
的纵波, 无论狭缝的取向如何, 波都能通过.

\subsection{光的偏振}
光的干涉和衍射现象说明光具有波动性. 光是横波还是
纵波呢? 研究表明, 光是一种横波. 我们可以用与上述实验
类似的方法来研究光的偏振. 为此, 利用偏振片代替狭缝.

偏振片由特定的材料制成, 每个偏振片都有一个特定的方向, 沿着这个方
向振动的光波能顺利通过偏振片, 偏振方向与这个方向垂
直的光不能通过, 这个方向叫作透振方向. 偏振片对光
波的作用就像狭缝对于机械波的作用一样.

事实上, 只要光的振动方向不与透振方向垂直, 
它都可以不同程度地通过偏振片, 不过强度要比振
动方向与透振方向平行的光弱一些. 

生活中, 太阳光, 日光灯, LED等普通光源发出的光, 包含着在垂直于传播方向上
沿一切方向振动的光, 而且沿着各个方向振动的光波的强度都相同. 
这种光称为\textbf{自然光}.

自然光在通过偏振片时, 只有振动方向与偏振片的
透振方向一致的光波才能顺利通过. 也就是说, 通过偏振
片的光波, 在垂直于传播方向的平面上, 沿着某个特定
的方向振动. 这种光叫作\textbf{偏振光}.

偏振光并不罕见. 除了从太阳, 白炽灯等光源直接发
出的光以外, 我们通常看到的绝大部分光, 都是不同程度
的偏振光. 自然光的反射光和折射光都是偏振光, 入射角变化时偏
振的程度也有变化.

光的偏振现象有很多应用. 例如, 摄影师在拍摄池中
的游鱼, 玻璃橱窗里的陈列物时, 由于水面和玻璃表面的
反射光的干扰, 景象会不清楚. 如果在照相机镜头前装一
片偏振滤光片, 转动滤光片, 让它的透振方向与水面和玻
璃表面的反射光的偏振方向垂直, 就可以减弱反射光而使
水下和玻璃后的景象清晰.

电子表的液晶显示, 3D电影等也都利用了偏振光的原理.

\subsection{激光}
光是从物质的原子中发射出来的. 但是, 
普通的光源, 例如白炽灯, 灯丝中某个原子在什么时刻发
光, 在哪个方向偏振, 完全是随机的, 发出的光传播方向
各异, 频率也不一定相同, 这导致不同原子发出的光没有
确定的相位差, 这导致两个独立的普通光源发出的光不会发生干涉. 

1960 年, 美国物理学家梅曼率先在实验室中制造出了
频率相同,相位差恒定,振动方向一致的光波, 
这就是\textbf{激光}. 由于上述性质, 激光具有高度的相干性, 这是
它的第一个特点. 

我们前面讲过的双缝干涉实验
和衍射实验, 用激光要比用自然光更容易完成. 因此, 激
光被广泛地应用于生产生活和科学研究中. 

激光还能用来传递信息. 光纤
通信就是激光和光导纤维相结合的产物. 

激光的另一个特点是它的平行度非常好, 在传播很远
的距离后仍能保持一定的强度. 激光的这个特点使它可以
用来进行精确的测距. 

激光的亮度很高, 也就是说, 它可以在很小的空间和
很短的时间内集中很大的能量. 因此, 可以利用激光束来切割, 焊接, 
以及在很硬的材料上打孔. 

\end{document}