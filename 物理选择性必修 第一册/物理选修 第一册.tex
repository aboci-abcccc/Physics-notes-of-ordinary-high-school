\documentclass[12pt,a4paper]{ctexart}

\title{物理选修\ \ 第一册}
\author{啊波呲}

\setlength{\parskip}{0em}
\usepackage{amsmath,mathtools,amssymb,geometry,wrapfig,graphicx,empheq,pifont,enumitem,upgreek}
\renewcommand{\baselinestretch}{1.70}
\geometry{left=1.0cm,right=0.9cm,bottom=1.9cm,top=2.0cm}
\setenumerate[1]{itemsep=1pt,partopsep=0pt,parsep=\parskip,topsep=3pt}
\setitemize[1]{itemsep=1pt,partopsep=0pt,parsep=\parskip,topsep=3pt}

\usepackage{tikz}
\usepackage{xcolor}
\newcounter{exam}[section]
\setcounter{exam}{0}
\newcommand{\bre}{\ \ \ }
\newcommand{\examlabel}{\textbf{例\theexam}}
\newcommand{\soln}{\textbf{解}\bre}
\newcommand{\notes}{\textbf{注意}\bre}
\newcommand{\unit}[1]{\ \mathrm{#1}}

\newenvironment{example}{\bigskip\par\refstepcounter{exam}\examlabel\bre}{\par}
\newenvironment{solution}{\par\soln}{\par\bigskip}

\begin{document}
\maketitle
\pagenumbering{roman}
\tableofcontents

\newpage
\pagenumbering{arabic}

\setlength{\abovedisplayskip}{3pt}
\setlength{\belowdisplayskip}{3pt}

\section{动量守恒定律}
\subsection{动量}
\subsubsection{寻求碰撞中的不变量}

用两根长度相同的细线, 分别悬挂两个完全
相同的钢球 A, B, 且两球并排放置.拉起 A 球, 
然后放开, 该球与静止的 B 球发生碰撞.可以看
到, 碰撞后 A 球停止运动而静止, B 球开始运动, 
最终摆到和 A 球被拉起时同样的高度.

碰撞后, A 球的速度大
小不变地传给了 B 球.所有的碰撞都有这样的特点吗?

将上面实验中的 A 球换成大小相同的 C 球, 
使 C 球质量大于 B 球质量, 用手拉起 C 球至某一高度后放
开, 撞击静止的 B 球.我们可以看到, 碰撞后 B 球获得较
大的速度, 摆起的最大高度大于 C 球被拉起时的高度.

可以看出, 质量大的 C 球与质量小的 B 球碰撞
后, B 球得到的速度比 C 球碰撞前的速度大, 两球碰撞前
后的速度之和并不相等.

我们可能会猜想:碰撞前后两球动能之和是不变的;
或者两球质量和速度的乘积之和是不变的......

经过多次实验验证, 我们发现:碰撞前后, 
两个小球的动能之和并不相等, 但是质量与速度的乘积之和却基本不变.

上面的实验提示我们, 质量与速度的乘积
$mv$ 这个物理量具有特别的意义.

\subsubsection{动量}

物理学中把质量和速度的乘积$mv$定义为物体的\textbf{动量}, 用字母$p$表示, 即
\begin{empheq}[box=\fbox]{equation*}
    p = mv.
\end{empheq}
在国际单位制中, 动量的单位是\textbf{千克米每秒}($\mathrm{kg\cdot m/s}$).

动量是矢量,动量的方向与速度的方向相同.

动量是状态量. 我们说物体在某一时刻的动量, 
是指物体的质量$m$与它在这一时刻速度$v$的乘积.
这里$v$是物体在这一时刻的瞬时速度, 而不是某段时间内的平均速度.

物体在一个过程中\textbf{动量的变化量}是
它的末动量与初动量的矢量差.
动量的变化量也称为\textbf{动量的增量}.如果初末动量在一条直线上, 即运动是一维的, 那么选定正方向后,
动量的方向可以用正负号表示, 从而将矢量运算转化为代数运算, 即
$$\Delta p = m\Delta v = mv^{\prime} - mv.$$
此时动量的增量$\Delta p$为负表示动量向负方向增加.

\subsection{动量定理}

\subsubsection{冲量}
假定一个质量为$m$的物体在光滑的水平面上受
到恒力 $F$ 的作用,做匀变速直线运动.在初始时刻,物体
的速度为 $v$,经过一段时间$\Delta t$,它的速度为 $v^{\prime}$,那么,这
个物体在这段时间的加速度
$$a = \frac{v^{\prime}-v}{\Delta t}.$$
根据牛顿第二定律, 有
\begin{equation*}
    F = ma = m\frac{v^{\prime}-v}{\Delta t} = \frac{mv^{\prime}-mv}{\Delta t}.
    \label{动量的变化率}
\end{equation*}
即
\begin{equation}
    F\Delta t = p^{\prime}-p.
    \label{动量定理1}
\end{equation}
这就是说, \textbf{力与力的作用时间的乘积, 等于在这段时间内动量的变化量}.

\eqref{动量定理1} 式的左边是力与力作用时间的乘积, 它反映了力的作用对时间的累积效应. 物理学中把
$F\Delta t$这个量称为力的\textbf{冲量}, 用$I$表示, 即
\begin{empheq}[box=\fbox]{equation*}
    I = F\Delta t.
\end{empheq}
冲量的单位是\textbf{牛秒}($\mathrm{N\cdot s}$).事实上, 它与动量的量纲相同, 并且
$$1\ \mathrm{N\cdot s} = 1\ \mathrm{kg\cdot m/s}.$$

冲量是矢量,冲量的方向与力的方向相同.

冲量是过程量. 我们说力$F$在一段时间$\Delta t$内的冲量为$F\Delta t$, 
是指这个力$F$对物体作用了$\Delta t$时间产生的冲量.

\subsubsection{动量定理}
有了冲量的概念, \eqref{动量定理1} 式就可以写成
\begin{empheq}[box=\fbox]{equation*}
    I = \Delta p.
\end{empheq}
也可以展开写为
\begin{empheq}[box=\fbox]{equation*}
    F\Delta t = mv^{\prime}-mv.
\end{empheq}
它们表示, \textbf{物体在一个过程中所受力的冲量等于它在这个过程始末的动量变化量}, 这就是\textbf{动量定理}.

值得注意的是, 动量定理中所提到的``力的冲量''指的是物体所受合力的冲量.除此之外, 物体受到的任何一个分力
都有冲量, 比如质量为$m$的物体重力的冲量为$$I_\mathrm{G} = mg\Delta t.$$
其中$g$是重力加速度, $\Delta t$是重力的作用时间.然而重力无时无刻不作用在物体上, 所以我们可以
取任意的$\Delta t$, 并说$mg\Delta t$是物体的重力在时间$\Delta t$内的冲量.

事实上, 动量定理不仅适用于单体, 也适用于系统, 即我们有:

\textbf{物体系统在一个过程中所受合外力的冲量等于该系统内各个物体在这个过程始末的动量变化量的矢量和.}

\eqref{动量定理1} 式还可以写成$$F = \frac{\Delta p}{\Delta t},$$
其中, $\Delta p$表示在$\Delta t$时间内物体动量的增量.
由此可知, \textbf{物体动量的变化率等于它所受合外力的大小}.

一定质量的物体, 改变一定的速度, 动量的改变量是一定的. 此时$$F\propto \displaystyle\frac{1}{\Delta t},$$
即力的作用时间越长, 力越小. 物体相互碰撞时出现的力称为\textbf{冲力}. 为了安全, 我们通常延长冲力的作用时间
从而减小冲力的大小, 这就是\textbf{``缓冲''}.易碎物品运输时要用柔软材料包装, 跳高时运动员
要落在软垫上, 就是这个道理.

% \begin{wrapfigure}{r}{4cm}
%     \flushright %右侧对齐
%     \includegraphics[width=0.25\textwidth]{路径}
%     \label{fig}
% \end{wrapfigure}
\setlength{\abovedisplayskip}{3pt}
\setlength{\belowdisplayskip}{3pt}

\begin{example}
    一个质量为$m = 2\ \mathrm{kg}$的物体,在$F_1 = 8\ \rm{N}$的水平拉力作用下,
    从静止开始沿水平面运动了$t_1 = 5\ \rm{s}$,然后拉力减小为$F_2=5\ \rm{N}$,方向不变,物体又运动了
    $t_2=4\ \rm{s}$后撤去外力,物体再经过$t_3=6\ \rm{s}$停下来. 试求物体在水平面上所受的摩擦力.
\end{example} 

\textbf{解}\bre 设拉力的方向为正方向, 则摩擦力沿负方向.
注意到全程中摩擦力的大小$F_\mathrm{f}$是恒定不变的, 对全过程应用动量定理得
$$F_1t_1 + F_2t_2 - F_\mathrm{f}(t_1 + t_2 + t_3) = 0-0.$$
由此解得$F_\mathrm{f} = 4\ \mathrm{N}$.

\textbf{注意}\bre 本题中的拉力都是水平的, 如果拉力是倾斜的, 并且在整个过程中
大小发生了变化, 要注意摩擦力也会随之变化.


\begin{example}
    某人所受重力为$G$, 穿着平底鞋起跳, 竖直着地过程中,
    双脚与地面间的作用时间为$t$, 地面对他的平均冲击力大小为$4G$. 若他穿上带有减震气垫
    的鞋起跳, 以与第一次相同的速度着地时, 双脚与地面间的作用时间变为$2.5t$,
    求地面对他的平均冲击力$\bar{F}$变为多少.
\end{example}

\textbf{解}\bre 由两过程的末速度相同, 知两过程的动量变化量相同. 由动量定理知
两过程的合力冲量相同, 即
$$(4G-G)t = (\bar{F} - G)2.5t.$$
由此解得$\bar{F} = 2.2G.$

\textbf{注意}\bre 动量定理$I = \Delta p$中, $I$为合力的冲量.在本题中, 要注意不要把重力当作合力计算.

\subsubsection{流体的冲力}
\setlength{\abovedisplayskip}{3pt}
\setlength{\belowdisplayskip}{3pt}
根据动量定理, 流体的冲力$F$满足$$Ft = mv.$$
其中$m$为流体质量, $v$为流体速度, $t$为全过程的时间.

因为流体的体积$V = vt\cdot S$($S$为流体截面), 所以
流体的质量$m = \rho V = \rho\cdot vt \cdot S$($\rho$为流体的密度).因此
$$Ft = \rho\cdot vt \cdot S \cdot v.$$

由此得到流体的冲力公式
\begin{empheq}[box=\fbox]{equation*}
    F = \rho Sv^2.
\end{empheq}
式中$F$为流体的冲力, $\rho$为流体密度, $S$为流体截面, $v$为流体速度.
\subsubsection{变力的冲量}

当物体所受的合力变化时, 我们通常用动量定理求解合力的
平均冲量, 进而求解某个分力的冲量.

特别地, 当物体所受的某个力$F$与物体的速度$v$成正比时(比如阻力),
设这个力与速度的关系为$F = kv$, 那么这个力的冲量
$$I_F = \sum F_i\Delta t_i =\sum kv_i\Delta t_i= ks.$$
其中$F_i, v_i$分别是物体在每个极短时间$\Delta t_i$内的力和速度, $s$是物体的位移.

\subsection{动量守恒定律}

前面我们学习了动量定理. 力$F$作用在质量为$m$的物体上, 作用时间为$t$,
物体在这个过程始末的速度分别为$v$, $v^{\prime}$, 则有
$$Ft = mv^{\prime} - mv.$$

当外力$F$为0时, 有$$mv^{\prime} = mv.$$
即物体的动量保持不变, 我们可以说这个物体的动量是守恒的.这个结论我们很好理解, 应用牛顿第一定律就可以解释.

上面的讨论是对单个物体而言的, 那么对多个物体有没有``动量守恒''的规律呢?
在研究这个问题之前, 我们先介绍一个概念.

当我们对两个(或多个)物体进行研究时, 它们组成一个\textbf{力学系统}.
系统内两个(或多个)物体的相互作用力称为\textbf{内力}, 系统以外的物体对系统的作用力称为\textbf{外力}.

下面我们结合一个具体情境来研究, 系统所受合外力为0时, 系统的总动量如何变化.

质量为$m_1$和$m_2$的两个小球A和B, 在光滑水平面上沿同一方向
做匀速直线运动, 速度分别是$v_1,\ v_2$, 且$v_2>v_1$. 经过一段时间后,
B追上了A, 两球发生碰撞, 碰撞后A, B的速度分别变为$v_1^{\prime},\ v_2^{\prime}$.

设碰撞过程中, B对A的作用力为$F_1$, A对B的作用力为$F_2$, 由牛顿第三定律可知,
$F_1$与$F_2$大小相等, 方向相反, 即$F_1 = -F_2$.

对A应用动量定理得$$F_1t = m_1v_1^{\prime} - m_1v_1.$$

对B应用动能定理得$$F_2t = m_2v_2^{\prime} - m_2v_2.$$

联立以上三个式子, 得
\begin{equation}
    \label{动量守恒1}
    m_1v_1^{\prime}-m_1v_1 = -(m_2v_2^{\prime} - m_2v_2).
\end{equation}
整理得
\begin{equation}
    \label{动量守恒2}
    m_1v_1 + m_2v_2 = m_1v_1^{\prime} + m_2v_2^{\prime}.
\end{equation}
我们发现, 两个球碰撞前后, 系统的总动量是不变的, 也可以说系统的动量是守恒的.
由此, 我们得到动量守恒定律.

\subparagraph{动量守恒定律} \textbf{如果一个系统不受外力或所受外力的矢量和为0, 那么这个系统的总动量保持不变,
    即系统的动量守恒.}

\eqref{动量守恒1} 和 \eqref{动量守恒2} 分别是动量守恒定律的两种表达形式, 即
\begin{empheq}[box=\fbox]{equation*}
    \Delta p_1 = -\Delta p_2.
\end{empheq}
其中$\Delta p_1$, $\Delta p_2$分别为两个物体在一个过程中的动量变化量,
这种形式适用于系统中有且只有两个物体时.

或者
\begin{empheq}[box=\fbox]{equation*}
    p = p^{\prime}.
\end{empheq}
其中$p$, $p^{\prime}$分别为系统的初动量, 末动量. 这种形式不限制系统内的物体数量.
当系统中有两个物体时, 上式可以写成
\begin{empheq}[box=\fbox]{equation*}
    m_1v_1 + m_2v_2 = m_1v_1^{\prime} + m_2v_2^{\prime}.
\end{empheq}
这是动量守恒最常见的表达形式.

对于整个系统来说, 因为系统的总动量$p = m_\text{系统}v_\text{质心}$,
所以
$$m_\text{系统}v_\text{质心} = m_\text{系统}v_\text{质心}^{\prime},$$
由此可得$$v_\text{质心} = v_\text{质心}^{\prime},$$ 即\textbf{系统质心速度不变}.
因此系统动量守恒等价于系统质心的速度守恒.

应用动量守恒定律时, 有几点注意事项:
\begin{enumerate}
    \item 动量守恒指的是总动量在相互作用的过程中时刻守恒, 而不是只有始末状态才守恒. 因此在实际使用时, 可以任选两个状态来列方程.
    \item 系统的动量守恒, 不能说明单体的动量守恒. 在应用动量守恒定律时, 一定要明确是哪些物体构成的系统.
    \item 动量守恒定律的表达式是矢量式, 对于一维的问题可以先规定正方向, 再进行计算.
    \item 动量与参考系的选择有关, 一般以地面为参考系.
    \item 动量守恒不能说明机械能守恒, 机械能守恒也不能说明动量守恒, 它们的判定条件不同, 没有必然联系.
    \item 动量守恒定律不但适用于宏观低速运动的物体, 而且还适用于微观高速运动的粒子. 它与牛顿运动定律相比, 适用范围要广泛得多. 并且动量守恒定律不需要考虑物体间的作用细节, 在解决问题上比牛顿运动定律更简捷.
\end{enumerate}

\subsubsection*{动量守恒的条件}

虽然动量守恒定律要求系统所受的合外力为0, 但是在以下两种情况下, 我们也可以用
动量守恒定律解决问题:
\begin{enumerate}
    \item 系统所受合外力不为0, 但在系统各部分相互作用的瞬时过程中, 系统内力
          远远大于外力, 外力相对来说可以忽略不计, 这时系统的动量近似守恒. 例如爆炸, 反冲等过程.
    \item 系统所受合外力不为0, 但系统在某一方向上不受外力或该方向上外力之和为0.
          则系统在该方向上的动量守恒.例如物块在斜面上下滑的问题. 在分析这类问题时, 要注意先把速度分解成沿该方向的速度再代入计算.
\end{enumerate}

\begin{example}
    两个木块A, B置于光滑水平面上, 它们的质量$m_\mathrm{A} = m_\mathrm{B} = 2$\ kg.
B与一轻质弹簧的一段相连,
弹簧的另一端固定在墙上. 当A以$v = 4$\ m/s的速度向B撞击时, 由于有橡皮泥(质量不计)而粘在
一起运动. 求弹簧被压缩到最短时, 弹簧的弹性势能$E_\mathrm{p}$.
\end{example}

\textbf{解}\bre 在弹簧压缩前, 两木块的动量守恒, 并且最终具有相同的速度$v_\text{共}$,
根据动量守恒定律有$$m_\mathrm{A}v + 0 = (m_\mathrm{A}+m_\mathrm{B})v_\text{共}.$$

弹簧压缩的过程中, 根据能量守恒定律, 两物体的动能转化为弹簧的弹性势能, 即
$$\frac12 (m_\mathrm{A}+m_\mathrm{B}) v_\text{共}^2 = E_\mathrm{p}.$$

联立以上两式, 代入数据可得$E_\mathrm{p} = 8$\ J.

\subsection{爆炸\ \ \ 反冲}
系统在内力的相互作用下, 当一部分向某一方向运动时, 剩余部分
将向相反方向运动, 这种现象叫做\textbf{反冲}.

以发生反冲的系统为研究对象, 系统两部分间的相互作用力是内力,
在系统外力(如重力, 空气阻力等)可以忽略的情况下, 我们知道,
系统的动量守恒.

爆炸就是反冲运动的一种. 爆炸过程中,
系统内部在极短时间内释放出大量的能量, 内力远远大于外力,
因此, 对整个系统而言, 我们可以利用动量守恒来解决问题.

\subsubsection{爆炸}

在光滑水平面上, 质量为$m_1$, $m_2$的两个物体紧靠在一起, 它们之间有少许炸药(质量不计), 炸药爆炸后,
两个物体分开而向相反的方向运动, 速度大小分别为$v_1$, $v_2$. 以
$v_1$的方向为正方向, 根据动量守恒, 有
\begin{equation}
    0 = m_1v_1 - m_2v_2.
    \label{爆炸1}
\end{equation}
由此可得$$\frac{v_1}{v_2} = \frac{m_2}{m_1}.$$
这就是说, \textbf{原本静止的系统因爆炸而分成两部分, 这两部分的速度大小之比是它们质量比的反比.}

研究完速度, 我们再来研究动能和动量. 根据 \eqref{爆炸1}, 显然这两个物体的动量相等, 那它们的动能又有什么关系呢?

设这两个物体的动能分别为$E_\mathrm{k1}$, $E_\mathrm{k2}$, 则
$$\frac{E_\mathrm{k1}}{E_\mathrm{k2}} = \frac{\frac12 m_1v_1^2}{\frac12m_2v_2^2} = \frac{m_1v_1^2}{m_2v_2^2} = \frac{v_1}{v_2}.$$
即它们的动能之比等于速度之比.

当爆炸产生的化学能全部转化为动能时, 则
$$E_\text{化学} = \Delta E_\mathrm{k} = \left(\frac12m_1v_1^2 + \frac12m_2v_2^2\right)-0.$$
当转化比为$\eta$时, 则
$$\eta E_\text{化学} = \Delta E_\mathrm{k} = \left(\frac12m_1v_1^2 + \frac12m_2v_2^2\right)-0.$$

\begin{example}
    在粗糙水平面上, 质量为$m_1$, $m_2$的两个物体紧靠在一起,
它们之间有少许炸药(质量不计), 炸药爆炸后,
两个物体分别滑动$x_1$, $x_2$的距离而停止,
两物体与水平面间的动摩擦因数均为$\mu$.
求两物体的质量比, 以及爆炸后瞬间的速度大小之比,动能大小之比.
\end{example}

\textbf{分析}\bre 爆炸过程中产生的内力极大, 时间极短, 摩擦力相对来说可以忽略不计,
因此系统的动量可以看作是守恒的.

\textbf{解}\bre 设爆炸后瞬间两物体的速度大小分别为$v_1$, $v_2$.
炸药爆炸后, 分析两物体滑行的过程, 易知两个物体的
加速度恒定且均为$a = \mu g$. 由匀变速直线运动位移与速度的关系可知
$$-2ax_1 = 0 - v_1^2,$$ $$-2ax_2 = 0 - v_2^2.$$
可得两物体爆炸后瞬间的速度大小之比$v_1:v_2 = \sqrt{x_1}:\sqrt{x_2}$.

炸药爆炸过程中, 取$v_1$的方向为正方向, 应用动量守恒定律得
$$0 = m_1v_1 - m_2v_2.$$
解得$m_1:m_2 = v_2:v_1 = \sqrt{x_2}:\sqrt{x_1}$.

爆炸后两物体的动能之比为
$$\frac{E_\mathrm{k1}}{E_\mathrm{k2}} = \frac{\frac12 m_1v_1^2}{\frac12m_2v_2^2}=\frac{m_1v_1\cdot v_1}{m_2v_2\cdot v_2}.$$
由上可知$m_1v_1 = m_2v_2$, 所以$$\frac{E_\mathrm{k1}}{E_\mathrm{k2}} =\frac{v_1}{v_2} = \frac{\sqrt{x_2}}{\sqrt{x_1}}.$$

\subsubsection{反冲}

\subsection{碰撞}

碰撞是两物体间极短的相互作用. 发生碰撞的系统所受合外力为0, 因此系统的
动量守恒, 也就是\textbf{系统质心的速度守恒}.

\subsubsection{碰撞的类型}

\subparagraph{完全非弹性碰撞} 在非弹性碰撞过程中, 物体往往会发生形变, 还会发声发热.
因此, 在非弹性碰撞过程中会有动能损失, 转化为其它形式的能, 即动能不守恒.

系统碰撞后, 物体结合在一起 (即共速), 此时动能损失最大. 这种碰撞叫做\textbf{完全非弹性碰撞}.

质量为$m_1$, $m_2$, 初速度为$v_1$, $v_2$的两个物体发生完全非弹性碰撞,
系统亏损的动能为$$\Delta E_\mathrm{k} = \left(\frac12 m_1 v_1^{\prime 2} + \frac12 m_2 v_2^{\prime 2}\right) - \left(\frac12 m_1 v_1^2 + \frac12 m_2 v_2^2\right).$$
系统的动量守恒, 有$$m_1v_1 + m_2v_2 = (m_1+m_2)v_\text{共}.$$

由上式可知, 它们的末速度$v_\text{共}$可以用加权平均数表示
\begin{equation}
    v_\text{共} = \frac{m_1v_1 + m_2v_2}{m_1+m_2}.
\end{equation}
事实上, 这也是系统质心速度的公式.

\subparagraph{完全弹性碰撞} 在理想情况下, 物体碰撞后, 形变能够完全恢复, 并且不发声发热.
因此系统没有动能损失, 这种碰撞称为\textbf{完全弹性碰撞}.

质量为$m_1$, $m_2$, 初速度为$v_1$, $v_2$的两个物体发生完全非弹性碰撞,
根据动量守恒, 动能守恒分别有
$$m_1v_1 + m_2v_2 = m_1v_1^{\prime} + m_2v_2^{\prime}, $$
$$\displaystyle\frac12 m_1 v_1^2 + \displaystyle\frac12 m_2 v_2^2 = \displaystyle\frac12 m_1 v_1^{\prime 2} + \displaystyle\frac12 m_2 v_2^{\prime 2}.$$


特别地, 当两物体质量相同时, 即$m_1 = m_2$时, 容易得到$$v_1 + v_2 = v_1^{\prime}+ v_2^{\prime},$$
$$v_1^2 + v_2^2 = v_1^{\prime 2}+ v_2^{\prime 2}.$$
解得$v_1^{\prime} = v_2$, $v_2^{\prime} = v_1$, 即\textbf{两物体的速度互换}.

\subsubsection{碰撞的合理性}


\section{简谐运动}

\subsection{机械振动}
我们把物体或物体的一部分在一个位置附近的往复运动, 
称为\textbf{机械振动}, 简称\textbf{震动}. 

把一个有孔的小球连接在弹簧的一端, 弹簧的另一端固定, 小球和弹簧套在光滑的杆上, 能够自由滑动. 弹簧的质量可以忽略, 小球运动时的空气阻力也可以忽略. 


小球静止时所受合力为0, 处于\textbf{平衡位置}. 向一侧拉动小球, 然后放开, 它就在平衡位置附近运动起来. 这样的理想化模型称为\textbf{弹簧振子}. 

由于没有阻力, 小球在运动过程中没有机械能的损失, 因而能够在平衡位置附近做周期性的往复运动, 即周期性的振动. 

\subsubsection{简谐运动}

为了研究弹簧振子的运动规律, 我们利用频闪照相机得到它的位移—时间图像, 即$x-t$图像, 也称为\textbf{振动图像}. 

我们发现, 小球位移与时间的关系可以用正弦函数来表示, 并可通过实验进行证明. 弹簧振子的这种运动是最简单的震动, 我们把这种运动叫做\textbf{简谐运动}. 

\textbf{如果物体的位移与时间的关系遵从正弦函数的规律, 即它的振动图像($x-t$图像)是一条正弦曲线, 这样的振动就是简谐运动. }

因此, 位移$x$关于时间$t$的函数表达式可以写为
\begin{equation}
    x = x(t) = A\sin(\omega t+\varphi_0),
    \label{位移函数}
\end{equation}
其中$A$, $\omega$, $\varphi_0$是参量, 下面我们就来介绍它们的物理意义.

\subsubsection{振幅}

因为$\sin(\omega t+\varphi_0)\in [-1,1]$, 所以$x\in [-A,A]$.
即$$0 \leqslant |x|\leqslant 1.$$
可以看出, $A$是小球离开平衡位置的最大距离, 称为振幅. 振幅是表示物体振动幅度的物理量. 如果记平衡位置为$O$点, $M$和$N$分别是右端和左端的最远位置,  那么$$|OM|=|ON|=\frac12|MN| = A.$$振幅是小球的运动范围的一半. 

\subsubsection{周期\bre 频率}

做简谐运动的小球, 如果在经过$O$点时开始计时, 那么它将向右经过$M$点, 然后向左回到$O$点, 又继续运动到$N$点, 之后又向右回到$O$. 这样一个完整的震动过程称为一次\textbf{全振动}. 事实上, 不管从哪里开始计时, \textbf{做简谐运动的物体完成一次全振动的时间总是相同的}. 

做简谐运动的物体完成一次全振动所需要的时间, 叫做简谐运动的\textbf{周期}. 
在一个周期内, 小球完成一次全震动, \textbf{经过的路程是振幅的四倍}. 

物体完成全振动的次数与所用时间之比, 叫做简谐运动的\textbf{频率}, 数值上等于单位时间内完成全振动的次数. 
用$T$表示周期, $f$表示频率, 则有
$$f = \frac{1}{T}.$$

在国际单位制中, 周期的单位是秒, 频率的单位是\textbf{赫兹}(Hz). $1\ \text{Hz} = 1\ \text{s}^{-1}.$
周期和频率都是描述振动快慢的物理量. 周期越小, 频率越大, 表示震动越快. 

对于函数$x(t)$, 我们知道, $(\omega t+\varphi_0)$在每增加$2\uppi$的过程中, 函数值$x(t)$(即位移$x$)周期性变化一次, 函数的周期$T$就是简谐运动的周期$T$. 于是有
$$\omega(t+T)+\varphi_0 = (\omega t+\varphi_0) +2\uppi.$$
由上式解得$$\omega = \frac{2\uppi}{T}.$$
根据周期与频率的关系, 上式还可以写为
\begin{empheq}[box=\fbox]{equation*}
    \omega = 2\uppi f.
\end{empheq}

可见, $\omega$是一个与频率成正比, 与周期成反比的物理量, 它也可以描述
简谐运动的快慢, 叫做\textbf{圆频率}. 

实验证明, \textbf{简谐运动的周期与其振幅无关}. 

\subsubsection{相位}

对于函数$x(t)$, 在一个周期内, 每一个$(\omega t+\varphi_0)$对应着一个函数值. 也就是说, 
当$(\omega t+\varphi_0)$确定时, 位移$x$也就确定了. 因此, $(\omega t+\varphi_0)$可以描述
物体正处于运动周期中的哪个状态, 物理学中把它叫做\textbf{相位}.

$\varphi_0$是$t = 0$时的相位, 称为\textbf{初相位}, 简称\textbf{初相},用$\varphi$表示.

两个具有相同频率的简谐运动, 我们通常研究它们的\textbf{相位差}. 
如果两个简谐运动的频率相同, 它们的相位差就是初相之差. 即
$$\Delta\varphi = \varphi_1 - \varphi_2.$$
此时, 我们说1的相位比2超前$\Delta\varphi$.

我们把两个相同的弹簧振子并列悬挂. 把它们拉到同一位置, 然后放开.
可以发现, 两个小球同时释放时, 除了振
幅和周期都相同外, 还总是向同一方向运动, 同时经过平
衡位置, 并同时到达同一侧的最大位移处.
在一个周期内, 如果不同时释放小球,它们的步调就不一致.

同时放开的两个小球振动步调总是一致, 我们
说它们的相位是相同的; 而对于不同时放开的两个小球, 
我们说第二个小球的相位落后于第一个小球的相位.

\subsubsection{位移}
做简谐运动的物体, 偏离平衡位置的距离, 叫做这个物体做简谐运动的位移.

通过前面的分析知道, 根据一个简谐运动的振幅$A$, 周期$T$, 初相位$\varphi_0$, 可以
确定物体在任意时间$t$的位移
\begin{empheq}[box=\fbox]{equation}
    x = A\sin\left(\frac{2\uppi}{T}t+\varphi_0\right).
    \label{位移}
\end{empheq}

特别地, 如果规定物体处于平衡位置时$t = 0$, 即初相位$\varphi_0 = 0$, 那么
物体在时间$t$的位移$$x = A\sin \frac{2\uppi}{T}t.$$
如果规定物体处于正方向最大位移处时$t = 0$, 则有$\sin\varphi_0 = 1, $
取$\varphi_0 = \displaystyle\frac{\uppi}{2}$, 由三角函数的知识可知, 物体在时间$t$的位移
$$x = A\cos \frac{2\uppi}{T}t.$$

\subsubsection{速度}

我们知道, 速度是衡量位移变化快慢的物理量.
做机械运动的物体, 其在某一时刻的速度就是它在这一时刻位移的瞬时变化率.
由数学知识可知, 速度就是位移关于时间的导函数.

如果物体做简谐运动, \eqref{位移函数} 式是物体的位移关于时间的函数, 那么该物体在时间$t$的速度
$$v = x^{\prime}(t) = A\omega\cos(\omega t+\varphi_0).$$

\subsubsection{回复力\bre 加速度}
根据牛顿运动定律, 在简谐运动中, 一定存在一个力, 在物体远离平衡位置时, 它迫使物
体的运动速度逐渐减小直到减为0, 然后, 物体在这个力
的作用下, 运动速度又由 0 逐渐增大并回到平衡位置; 物
体由于具有惯性, 到达平衡位置后会继续向另一侧运动, 
这个力使它再一次回到平衡位置.

正是在这个力的作用下, 物体才能在平衡位置附近做往复运动. 我们把这样的力叫做\textbf{回复力}.

对于水平方向的弹簧振子来说, 小球做简谐运动的回复力是弹簧对小球的弹力.大小是$F = kx$, 其中
$k$是弹簧的劲度系数.

理论上可以证明, 做简谐运动的物体都受到这样的回复力的作用, 它的大小与物体相对平衡位置的
位移成正比, 方向与位移方向相反, 这个力可以用$$F = -kx$$表示.其中符号表示$F$与$x$反向.
反过来说, \textbf{如果物体在运动方
向上所受的力与它偏离平衡位置位移的大小成正比,并且
总是指向平衡位置,物体的运动就是简谐运动.}

应当指出的是, 类似于向心力的概念, 回复力是一种效果力, 而不是拉力, 弹力这些性质力.
在分析物体所受的力时, 我们只需考虑物体实际所受的性质力, 不用再外加一个``回复力''.
这一点我们应该在学习向心力的时候就已经清楚了.

有了回复力的概念, 我们重新来定义简谐运动中的平衡位置. 做简谐运动的物体, 所受回复力为0时
所在的位置, 叫做简谐运动的\textbf{平衡位置}.
事实上, 当物体处于平衡位置时, 所受的回复力一定为0, 但合力却不一定为0. 接下来要学习的
单摆就给出了一个例子.

由牛顿第二定律可知, 做简谐运动的物体, 其加速度大小也与位移成正比,
方向与位移方向相反. 另一方面, 加速度是速度的瞬时变化率, 也就是位移关于时间的二阶导数. 
如果物体做简谐运动, \eqref{位移函数} 式是物体的位移关于时间的函数, 那么
该物体在时间$t$的加速度
$$a = x^{\prime}(t) = -A\omega^2\sin(\omega t+\varphi_0).$$

 

\end{document}