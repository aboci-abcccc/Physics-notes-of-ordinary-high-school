\documentclass[12pt,a4paper]{ctexart}

\title{导数例题}
\author{啊波呲}

\setlength{\parskip}{0em}
\usepackage{amsmath,mathtools,amssymb,geometry,wrapfig,graphicx,empheq,pifont,enumitem,upgreek}
\renewcommand{\baselinestretch}{1.77}
\geometry{left=1.5cm,right=1.5cm,bottom=2.1cm,top=2.5cm}
\setenumerate[1]{itemsep=1pt,partopsep=0pt,parsep=\parskip,topsep=3pt}
\setitemize[1]{itemsep=1pt,partopsep=0pt,parsep=\parskip,topsep=3pt}

\usepackage{tikz}
\usepackage{xcolor}
\newcounter{exam}[section]
\setcounter{exam}{0}
\newcommand{\bre}{\ \ \ }
\newcommand{\examlabel}{\textbf{例\theexam}}
\newcommand{\soln}{\textbf{解}\bre}
\newcommand{\prf}{\textbf{证明}\bre}
\newcommand{\notes}{\textbf{注意}\bre}
\newcommand{\unit}[1]{\ \mathrm{#1}}
\renewcommand{\ne}{\mathrm{e}}

\newenvironment{example}{\bigskip\par\refstepcounter{exam}\examlabel\bre}{\par}
\newenvironment{solution}{\par\soln}{\par\bigskip}
\newenvironment{proof}{\par\prf}{\par\bigskip}

\begin{document}
\maketitle
\pagenumbering{roman}
\tableofcontents

\newpage
\pagenumbering{arabic}
\setlength{\abovedisplayskip}{0pt}
\setlength{\belowdisplayskip}{0pt}
\section{隐零点问题}
\begin{example}
    若存在$k\in\mathbb{Z}$, 使得不等式$$x(\ln x+1)>k(x-2)$$在$x\in (2, +\infty)$恒成立, 求$k$的最大值.
    \vspace{10pt}
\end{example}
\begin{solution}
    \setlength{\abovedisplayskip}{8pt}
    \setlength{\belowdisplayskip}{8pt}
题设等价于$k<\displaystyle\frac{x(\ln x+1)}{x-2}$对于任意的$x\in (2, +\infty)$恒成立。

令$g(x)=\displaystyle\frac{x(\ln x+1)}{x-2},x\in(2, +\infty)$,则
$$g^{\prime}(x) = \frac{x-2\ln x-4}{(x-2)^2}.$$

令$\varphi(x) = x- 2\ln x-4, x\in(2, +\infty)$, 则恒有
$$\varphi^{\prime}(x) = 1-\frac{2}{x}>0.$$
所以$\varphi(x)$在$(2, +\infty)$递增. 

注意到$\varphi(8)<0$, $\varphi(10)>0$, 根据零点定理, 
存在唯一的$x_0\in(8,10)$, 使得$\varphi(x) = 0$, 即
\begin{equation}
    \setlength{\abovedisplayskip}{0pt}
    \setlength{\belowdisplayskip}{0pt}
    x_0-2\ln x_0-4 = 0.
    \label{1}
\end{equation}

于是, 当$2<x<x_0$时, $g^{\prime}(x)<0$, $g(x)$递减; 当$x>x_0$时, $g^{\prime}(x)>0$, $g(x)$递增.
所以$$g(x)\leqslant g(x_0) = \frac{x_0(\ln x_0+1)}{x_0-2}.$$代入 \eqref{1} 式可得
$$g(x)\leqslant \frac{x_0\left(\displaystyle\frac{x_0}{2}-1\right)}{x_0-2} = \frac{x_0}{2}.$$
由此可知$k<\displaystyle\frac{x_0}{2}$.由$x\in(8,10)$得$\displaystyle\frac{x_0}{2}\in(4,5)$.
因此,使$k<\displaystyle\frac{x_0}{2}$恒成立的最大整数$k$是4.\vspace{5pt}

即所求$k$的最大值为4.
\end{solution}
\newpage
\setlength{\abovedisplayskip}{6pt}
    \setlength{\belowdisplayskip}{6pt}
\begin{example}
    已知函数$f(x) = \displaystyle\frac{\ln ax}{x}-\mathrm{e}\ln x,$ 其中$a>\ne$.

    (1)证明: $f(x)$存在唯一极值点;
    
    (2)证明: $f(x)<\ne(a-1)$.
\end{example}
\begin{proof}
(1)函数$f(x)$的定义域为$(0,+\infty)$. 对其求导:
$$f^{\prime}(x) = \frac{1-\ln ax-\ne x}{x^2}.$$

令$\varphi(x) = 1-\ln ax-\ne x$, 容易看出$\varphi(x)$单调递减.注意到
\begin{equation*}
    \setlength{\abovedisplayskip}{10pt}
    \setlength{\belowdisplayskip}{10pt}
\varphi\left(\frac{1}{a}\right) = 1-\frac{\ne}{a}>0,\bre\bre \varphi\left(\frac{1}{\ne}\right) =  1-\ln a<0.
\end{equation*}
又由$a>\ne$知$\displaystyle\frac{1}{a}<\displaystyle\frac{1}{\ne}$.
因此$\varphi(x)$在$\left(\displaystyle\frac{1}{a},\displaystyle\frac{1}{e}\right)$上存在唯一零点$x_0$.\vspace{7pt}

于是, 当$0<x<x_0$时, $\varphi(x)>0$, 即$f^{\prime}(x)>0$, $f(x)$递增; 
当$x>x_0$时, $\varphi(x)<0$, 即$f^{\prime}(x)<0$, $f(x)$递减.
$x = x_0$是$f(x)$的唯一极值点.

\setlength{\abovedisplayskip}{7pt}
\setlength{\belowdisplayskip}{7pt}
(2)由(1)可得$1 - \ln ax_0-\ne x_0 = 0$, 即$\ln ax_0 =1-\ne x_0$.从而
$$f(x_0) = \frac{1}{x_0}-\ne \ln x_0-\ne, \ x_0\in \left(\frac{1}{a},\frac{1}{e}\right).$$
因为$f(x_0)$是$f(x)$的唯一极大值, 所以
$$f(x)\leqslant f(x_0) = \frac{1}{x_0}-\ne \ln x_0-\ne.$$

令$g(x) = \displaystyle\frac{1}{x}-\ne \ln x-\ne, x\in\left(\displaystyle\frac{1}{a},\displaystyle\frac{1}{e}\right)$.
易知$g(x)$是减函数, 所以$$g(x)<g\left(\frac{1}{a}\right) = a+\ne\ln a-\ne.$$即
$f(x) < a+\ne\ln a-\ne.$因此, 要证$f(x)<\ne(a-1)$, 只需证明$a+\ne\ln a-\ne>\ne(a-1)$, 即证
\begin{equation}
    \ne\ln a-\ne a+a<0.
    \setlength{\abovedisplayskip}{0pt}
    \setlength{\belowdisplayskip}{0pt}
    \label{2}
\end{equation}

令$h(a) = \ne \ln a-\ne a+a, \ a\in (e,+\infty).$ 由
$$h^{\prime}(a) = \frac{\ne}{a}-\ne+1<1-e+1<0$$
知$h(a)$单调递减. 于是$$h(a)<h(e) = 2\ne - \ne^2<0.$$
这就证明了不等式 \eqref{2}. 以上过程均可逆, 于是原不等式得证.

\end{proof}

\section{对数均值不等式}
熟知均值不等式\[\frac{x_1+x_2}{2}\geqslant\sqrt{x_1x_2},\]
其中\(x_1, x_2\in [0, +\infty)\), 当且仅当\(x_1 = x_2\)时取等号.

对\(x_1, x_2\in (0, +\infty)\), 恒有
\[\frac{x_1+x_2}{2}\geqslant\frac{x_1-x_2}{\ln x_1 - \ln x_2}\geqslant\sqrt{x_1x_2},\]
当且仅当\(x_1 = x_2\)时取等号.
这个不等式称为\textbf{对数均值不等式}.

\begin{proof}
    不妨设\(x_1>x_2>0\). 令\(t = x_1/x_2>1\), 则\(x_1 = tx_2\), 要证上式, 等价于证明
    \[\frac{x_2(t+1)}{2}>\frac{x_2(t - 1)}{\ln t}>x_2\sqrt{t}.\]
    由于\(x_2>0\), 所以等价于证明
    \[\frac{t+1}{2}>\frac{t - 1}{\ln t}>\sqrt{t}.\]

    先证明左侧的不等式\[\frac{t+1}{2}>\frac{t - 1}{\ln t},\] 即证
    \[\ln t-\frac{2(t-1)}{t+1}>0.\]构造函数\[F(t) = \ln t - \frac{2(t-1)}{t+1},t\in (1, +\infty). \]
    求导可得\(F(x)\)在\(x>1\)时单调递增, 则\[F(x)>F(1) = 0.\]
    
    从而左侧得证.右侧可作类似证明, 于是原不等式成立.
\end{proof}

\end{document}