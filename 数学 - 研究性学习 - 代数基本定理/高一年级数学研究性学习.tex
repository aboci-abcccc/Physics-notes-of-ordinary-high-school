\documentclass[12pt,a4paper]{ctexart}

\title{\zihao{-2}关于代数基本定理的猜想与研究}
\author{高一(13)班\ \ \ 郑庭昊}

\setlength{\parskip}{0em}
\usepackage{xeCJK,amsmath,mathtools,amssymb,geometry,wrapfig,graphicx,empheq,pifont,enumitem}
\renewcommand{\baselinestretch}{1.73}
\geometry{left=1.5cm,right=1.5cm,bottom=2.1cm,top=2.5cm}
\setenumerate[1]{itemsep=1pt,partopsep=0pt,parsep=\parskip,topsep=3pt}
\setitemize[1]{itemsep=1pt,partopsep=0pt,parsep=\parskip,topsep=3pt}

\usepackage{tikz}
\usepackage{xcolor}
\newcounter{exam}[section]
\setcounter{exam}{0}
\renewcommand{\thefootnote}{*}

\begin{document}
\maketitle
\pagenumbering{roman}
\tableofcontents
\newpage
\pagenumbering{arabic}


\setlength{\abovedisplayskip}{3pt}
\setlength{\belowdisplayskip}{3pt}

\section{复数的表示\ \ \ 复变函数}

一般的一元多项式方程有没有根? 如果有根,根的个数是多少? 是否存在求根公式? 早在 16 世纪以前,
数学家们就得到了一元二次方程, 一元三次方程, 一元四次方程的求根公式, 但五次及以上的方程就没有
求根公式了.那我们如何保证根的存在性呢?

代数基本定理正好回答了这个问题.

\subparagraph{代数基本定理} \textbf{任何一元$n (n \in \mathbb{N^*} )$次复系数多项式方程$f(x) = 0$
    在复数域上至少有一根.}

这个定理的证明在一般的高中教材, 甚至在大学的高等代数教材里, 都只是一带而过.
那我们是否在高中阶段就没办法理解和掌握代数基本定理的证明了呢?

在证明这个定理之前, 我们先要了解一些预备知识. 第一个是复数的三角表示和指数表示.
《数学必修 第二册》教材第7.3节对此进行了一些介绍,
不过是选学的内容.

\subsection{复数的三角表示与指数表示}

\begin{wrapfigure}{r}{5cm}
    \flushright
    \includegraphics[width=0.3\textwidth]{Untitled.pdf}
    \label{fig5}
\end{wrapfigure}

把一个复数$z = a+b\mathrm{i}$在复平面内表示为
向量$\overrightarrow{OZ}$, 如图所示. 并记
$$r = \left\lvert z \right\rvert = \left\lvert a + b\mathrm{i} \right\rvert .$$
由此可知
\begin{equation*}
    \begin{cases*}
        a = r\cos \theta, \\
        b = r\sin \theta.
    \end{cases*}
\end{equation*}
因此
\begin{align*}
    z = a + b\mathrm{i} & = r\cos \theta + \mathrm{i}r \cos \theta  \\
                        & = r(\cos \theta + \mathrm{i}\sin \theta).
\end{align*}

我们把$r(\cos \theta + i\sin \theta)$叫做\textbf{复数$z$的三角表示}.其中
向量$\overrightarrow{OZ}$与实轴的夹角$\theta$叫做复数$z$的\textbf{辐角}.由任意角的知识
可知, 一个复数的辐角有无数多个值, 我们规定在$[0, 2\pi]$范围内的辐角$\theta$的值为
\textbf{辐角的主值}, 记作$\arg z$.

除此之外, 我们再引出一个著名的公式.
$$\mathrm{e}^{\mathrm{i}\theta} = \cos \theta + i\sin \theta.$$ 这个式子称为欧拉公式,
其中$\mathrm{i}$是虚数单位, $\mathrm{e}$是自然对数的底数. 因此
$$z = r(\cos \theta + \mathrm{i}\sin \theta) = r\mathrm{e}^{\mathrm{i}\theta},$$称为复数$z$的\textbf{指数表示}.由定义可知,
$$|r\mathrm{e}^{\mathrm{i}\theta}| = r.$$

欧拉公式的推导超出了本篇研究的范围, 下面我们直接使用.

\subsection{复变函数}

了解完复数的表示, 我们再来了解一下复变函数.

设$A$是一个复数集, 如果对$A$中的任一复数$z$,
通过一个确定的规则有一个或若干个复数$w$与之对应,就说在复数集$A$上定义了一个\textbf{复变函数},
记作$$w = f(z), z \in A.$$

复变函数$w = f(z)$的零点就是方程$f(z) = 0$的复数根.

\section{代数基本定理的简单证明}

要想证明代数基本定理, 我们还需要引出两个命题.

\begin{enumerate}
    \item 多项式复变函数的模函数在复平面上连续, 而连续函数在给定范围内(含端点)上一定有最大和最小值.\label{1}
    \item 多项式复变函数当自变量趋于无穷大时, 函数值的模也趋于无穷大.\label{2}
\end{enumerate}

\ref{1} 的证明需要用到高等数学知识,故略去; 对于 \ref{2}, 我们用极限的知识就可以理解.

由 \ref{1} 和 \ref{2} 可得多项式复变函数在整个复平面上一定有最小值.
即使没有学过高等数学知识, 只要承认 \ref{1} 的正确性,
也可以理解证明.
\setlength{\abovedisplayskip}{3pt}
\setlength{\belowdisplayskip}{3pt}

有了复变函数的概念, 我们可以把代数基本定理转化为符号语言.

\setlength{\abovedisplayskip}{10pt}
\setlength{\belowdisplayskip}{10pt}

\subparagraph{代数基本定理} \textbf{\heiti 对任意的$n$次多项式复变函数
    $$f(z) = \sum_{i = 0}^{n} c_i z^i = c_0 + c_1z + c_2z^2 +\cdots+ c_nz^n, c_n \neq 0$$
    都存在$z_0 \in \mathbb{C}$, 使$f(z_0) = 0$.}

\textbf{证明}\ \ \  用反证法. 假设$f(z)$没有零点. 根据 \ref{1}, $|f(z)|$在复平面上连续, 并且
$|f(z)|$一定有最小值$r_0 > 0$.

不妨设在$z_0$处取得最小值$|f(z_0)| = r_0$, 而$f(z_0) = r_0 \mathrm{e}^{\mathrm{i}\theta_0}$
(其中$\theta_0 = \arg(f(z_0))$). 把$z$替换为$z + z_0$, 可得
$$
    f(z+z_0)          = r_0 \mathrm{e}^{\mathrm{i}\theta_0} + c_k^{\prime} z^k + c_{k+1}^{\prime} z^{k+1}\cdots + c_n^{\prime}z^n $$ $$
    \Rightarrow f(z)  = r_0 \mathrm{e}^{\mathrm{i}\theta_0} + c_k^{\prime} (z-z_0)^{k} + c_{k+1}^{\prime} (z-z_0)^{k+1} + \cdots + c_n^{\prime} (z-z_0)^n.
$$
其中$r_0 \mathrm{e}^{\mathrm{i}\theta_0}$
是常数项, $c_k^{\prime} \neq 0\ (k\in \mathbb{N}^*, k<n)$是除常数项外最低次非零项的系数\footnote{由泰勒展开可知, 变量替换后的多项式系数除最高次系数与原系数相同, 其它项系数与原系数不一定相同. 事实上, $c^{\prime}_n = c_n$, 为了便于理解, 这里不写为$c_n$.}.

在$z = z_0$附近, $f(z)$和$g(z) = r_0 \mathrm{e}^{\mathrm{i}\theta_0} + c_k^{\prime} (z-z_0)^k$
的值很接近.事实上, 当$0 < |z-z_0| \leqslant 1$时, $0 < |z-z_0|^i \leqslant 1$ $(i = k, k+1,\cdots,n-k+1)$,
因此
\setlength{\abovedisplayskip}{15pt}
\setlength{\belowdisplayskip}{5pt}
\begin{align*}
    \left\lvert \frac{f(z)-g(z)}{(z-z_0)^{k+1}} \right\rvert & = \left\lvert \frac{c_{k+1}^{\prime} (z-z_0)^{k+1} + c_{k+2}^{\prime} (z-z_0)^{k+2} + \cdots +c_{n}^{\prime} (z-z_0)^{n}}{(z-z_0)^{k+1}}\right\rvert \\
                                                             & \leqslant \left\lvert c^{\prime}_{k+1} + c^{\prime}_{k+2} + \cdots + c_n^{\prime} \right\rvert                                                       \\
                                                             & \leqslant \left\lvert c_{k+1}^{\prime} \right\rvert + \left\lvert c_{k+2}^{\prime} \right\rvert + \cdots + \left\lvert c_{n}^{\prime} \right\rvert.
\end{align*}

\setlength{\abovedisplayskip}{3pt}
\setlength{\belowdisplayskip}{3pt}

我们记$M = \left\lvert c_{k+1}^{\prime} \right\rvert + \left\lvert c_{k+2}^{\prime} \right\rvert + \cdots + \left\lvert c_{n}^{\prime} \right\rvert$
,可以得到
\begin{align*}
    \left\lvert f(z)-g(z) \right\rvert \leqslant M |z-z_0|^{k+1}.
\end{align*}
由不等式的知识可得$|f(z)|-|g(z)|\leqslant |f(z)-g(z)|$, 从而
$$\left\lvert f(z) \right\rvert \leqslant |g(z)| + M|z-z_0|^{k+1}.$$

设$r\mathrm{e}^{\mathrm{i}\theta} = z-z_0$,
其中$0 < r \leqslant 1$, 由此可得
\begin{align*}
    f(z) & \leqslant |r_0 \mathrm{e}^{\mathrm{i}\theta_0} + c_k^{\prime} (r\mathrm{e}^{\mathrm{i}\theta})^k| + M|r\mathrm{e}^{\mathrm{i}\theta}|^{k+1} \\
         & = |r_0 \mathrm{e}^{\mathrm{i}\theta_0} + c_k^{\prime} r^{k}\mathrm{e}^{\mathrm{i} k\theta}| + Mr^{k+1}.
\end{align*}
为了使$|r_0 \mathrm{e}^{\mathrm{i}\theta_0} + c_k^{\prime} r^{k}\mathrm{e}^{\mathrm{i} k\theta}|$
最小, 我们取$\theta$使得$\mathrm{e}^{\mathrm{i}k\theta} = -1$, 于是
\setlength{\abovedisplayskip}{3pt}
\setlength{\belowdisplayskip}{10pt}
\begin{align*}
    f(z) & \leqslant |r_0\mathrm{e}^{\mathrm{i}\theta_0} - c_k^{\prime}r^k| + Mr^{k+1}   \\
         & \leqslant |r_0\mathrm{e}^{\mathrm{i}\theta_0}| - |c_k^{\prime}r^k| + Mr^{k+1} \\
         & = r_0 - |c_k^{\prime}|r^k + Mr^{k+1}                                          \\
         & = r_0 - r_k(|c^{\prime}_k| - Mr).
\end{align*}

当$|c^{\prime}_k| - Mr > 0$时, 即$0 < r < \displaystyle\frac{|c^{\prime}_k|}{M}$时,
$|f(z)| < r_0$, 这
与$r_0$的定义矛盾, 故假设不成立.即$|f(z)|$的最小值一定是0, 为$f(z)$的零点.
$\hfill\square $

\section{代数基本定理的推论}

由代数基本定理知, 任意一个一元一次复系数多项式
$$w_i(z) = a_iz + b_i, a, b\in \mathbb{C}, a_i \neq 0 $$一定有1个根,
我们把$n$个这样的多项式相乘, 并记
\setlength{\abovedisplayskip}{10pt}
\setlength{\belowdisplayskip}{15pt}
$$g(z) = c\prod_{i=1}^{n}w_i(z) = cw_1(z)w_2(z)\cdots w_n(z), c\neq 0.\ \footnote{$\prod$是求积符号.\ $\displaystyle\prod_{i=1}^{n}x_i = x_1x_2\cdots x_n.$}$$


\setlength{\abovedisplayskip}{3pt}
\setlength{\belowdisplayskip}{3pt}
显然, 每一个$z_i$使$w_i(z_i) = 0$, 都是$g(z) = 0$的一个根. 所以$n$次复系数多项式方程
$g(z) = 0$有$n$个根(重根按重数计).因此, 我们有如下推论.

\subparagraph{推论} \textbf{\heiti 一元$n (n\in \mathbb{N}^* )$次多项式方程有$n$个复数根(重根按重数计).}\\

我们举几个例子.

一元一次方程$x-1=0$有1个根$x = 1$.

一元二次方程$x^2-1=0$有2个实数根$x_1 = 1, x_2 = -1$;
一元二次方程$x^2+1=0$有2个虚数根 $x_1 = \mathrm{i}, x_2 = -\mathrm{i}.$
它们都有2个复数根.

同理, 一元三次方程$x^3-1 = 0$应该有3个复数根. 我们显然能看出来其中一个根是$x = 1$,
那另外两个根呢?

利用立方差公式, 原方程可化为$$(x-1)(x^2 + x + 1) = 0,$$
于是$$x - 1 = 0, \text{\ 或\ } x^2 + x + 1 = 0.$$

$x^2 + x + 1 = 0$这个方程我们会解, 它的两个根分别是
$\displaystyle\frac{-1+\sqrt{3}\mathrm{i}}{2}$和
$\displaystyle\frac{-1-\sqrt{3}\mathrm{i}}{2}$. 因此,
$x^3-1 = 0$的3个复数根分别为
\setlength{\abovedisplayskip}{10pt}
\setlength{\belowdisplayskip}{10pt}
$$x_1 = 1, x_2 = \frac{-1+\sqrt{3}\mathrm{i}}{2}, x_3 = \frac{-1-\sqrt{3}\mathrm{i}}{2}.$$

我们再来看一元四次方程$x^4-1 = 0$. 我们能直接看出它的4个根
\setlength{\abovedisplayskip}{3pt}
\setlength{\belowdisplayskip}{3pt}
$$x_1 = 1, x_2 = -1, x_3 = \mathrm{i}, x_4 = -\mathrm{i}.$$

通过以上四个例子可以看出, 一元多项式方程复数根的个数确实和它的次数相等.

\section{一元多项式方程根与系数的关系}

从代数基本定理的推论出发, 我们来研究一元多项式方程的根与系数之间的关系.

设一元二次方程$$a_2x^2 + a_1x + a_0 = 0\ (a_2 \neq 0).$$
根据代数基本定理的推论, 它在复数集$\mathbb{C}$内应该有2个根, 设它们分别为
$x_1, x_2$, 容易得到
\setlength{\abovedisplayskip}{10pt}
\setlength{\belowdisplayskip}{10pt}
$$\begin{cases*}
        x_1 + x_2 = -\displaystyle\frac{a_1}{a_2}, \\
        x_1x_2 = \displaystyle\frac{a_0}{a_2}.
    \end{cases*}$$
\setlength{\abovedisplayskip}{3pt}
\setlength{\belowdisplayskip}{3pt}

同理, 设一元三次方程$$a_3x^3 + a_2x^2 + a_1x + a_0 = 0\ (a_3 \neq 0)$$
在复数集$\mathbb{C}$内的3个根为$x_1, x_2, x_3$, 教材已经告诉我们
\setlength{\abovedisplayskip}{10pt}
\setlength{\belowdisplayskip}{10pt}
$$\begin{cases*}
        x_1 + x_2 + x_3 = -\displaystyle\frac{a_2}{a_3},         \\
        x_1x_2 + x_1x_3 + x_2x_3 = \displaystyle\frac{a_1}{a_3}, \\
        x_1x_2x_3 = -\displaystyle\frac{a_0}{a_3}.
    \end{cases*}$$
下面我们来看一元四次方程.
\setlength{\abovedisplayskip}{3pt}
\setlength{\belowdisplayskip}{3pt}

设一元四次方程
\begin{equation}
    a_4x^4 + a_3x^3 + a_2x^2 + a_1x + a_0\ (a_4 \neq 0)
    \label{5}
\end{equation}
在复数集$\mathbb{C}$内的4个根为$x_1, x_2, x_3, x_4$, 则方程 \eqref{5}
可变形为
$$a_4(x - x_1)(x - x_2)(x - x_3)(x - x_4) = 0,$$
展开得
\begin{equation}
    \begin{split}
        a_4x^4 - a_4(x_1 + x_2 + x_3 + x_4)x^3 + a_4(x_1x_2 + x_1x_3 + x_1x_4 + x_2x_3 + x_2x_4 + x_3x_4)x^2 \\- a_4(x_1x_2x_3 + x_1x_2x_4 + x_1x_3x_4 + x_2x_3x_4)x + a_4x_1x_2x_3x_4.
        \label{6}
    \end{split}
\end{equation}
比较 \eqref{5}\eqref{6} 可以得到
\setlength{\abovedisplayskip}{10pt}
\setlength{\belowdisplayskip}{10pt}
$$\begin{cases*}
        x_1 + x_2 + x_3 + x_4 = -\displaystyle\frac{a_3}{a_4},                              \\
        x_1x_2 + x_1x_3 + x_1x_4 + x_2x_3 + x_2x_4 + x_3x_4 = \displaystyle\frac{a_2}{a_4}, \\
        x_1x_2x_3 + x_1x_2x_4 + x_1x_3x_4 + x_2x_3x_4 = -\displaystyle\frac{a_1}{a_4},      \\
        x_1x_2x_3x_4 = \displaystyle\frac{a_0}{a_4}.
    \end{cases*}$$

类似地, 设一元五次方程
\setlength{\abovedisplayskip}{3pt}
\setlength{\belowdisplayskip}{3pt}
$$a_5x^5 + a_4x^4 + a_3x^3 + a_2x^2 + a_1x + a_0 = 0\ (a_5 \neq 0)$$
在复数集$\mathbb{C}$内的5个根为$x_1, x_2, x_3, x_4, x_5$, 可以得到
\setlength{\abovedisplayskip}{10pt}
\setlength{\belowdisplayskip}{10pt}
$$\begin{cases*}
        x_1 + x_2 + x_3 + x_4 + x_5 = -\displaystyle\frac{a_4}{a_5},                                                                                   \\
        x_1x_2 + x_1x_3 + x_1x_4 + x_1x_5 + x_2x_3 + x_2x_4 + x_2x_5 + x_3x_4 + x_3x_5 + x_4x_5 = \displaystyle\frac{a_3}{a_5},                        \\
        x_1x_2x_3 + x_1x_2x_4 + x_1x_2x_5 + x_1x_3x_4 + x_1x_3x_5 + x_1x_4x_5 + x_2x_3x_4                                                              \\
        \ \ \ \ \ \ \ \ \ \ \ \ \ \ \ \ \ \ \ \ \ \ \ \ \ \ \ \ \ \ \ \ \ \ \ \ \ + x_2x_3x_5 + x_2x_4x_5 + x_3x_4x_5 = -\displaystyle\frac{a_2}{a_5}, \\
        x_1x_2x_3x_4 + x_1x_2x_3x_5 + x_1x_2x_4x_5 + x_1x_3x_4x_5 + x_2x_3x_4x_5 = \displaystyle\frac{a_1}{a_5},                                       \\
        x_1x_2x_3x_4x_5 = -\displaystyle\frac{a_0}{a_5}.
    \end{cases*}$$
至此, 我们应该已经找到规律了.
\setlength{\abovedisplayskip}{3pt}
\setlength{\belowdisplayskip}{3pt}

一般地, 设一元$n(n \in \mathbb{N}^*)$次多项式方程
$$a_0 + a_1x + a_2x^2 + \cdots + a_nx^n = 0\ (a_n \neq 0)$$
的$n$个根分别为$x_1, x_2, \cdots, x_n$,则有
$$x_1 + x_2 + \cdots + x_n = -\frac{a_{n-1}}{a_n}.$$


从这$n$个根中取出$m(m\in \mathbb{N}^*, m\leqslant n)$个求积, 最多能得到$\mathrm{C}_n^m$种因式.把这$\mathrm{C}_n^m$个因式求和,
得到
\begin{equation*}
    \sum^{n-m+1}_{i_1 = 1}\sum^{n-m+2}_{i_2 = i_1 + 1} \cdots \sum^{n}_{i_m = i_{m-1} + 1}\prod_{j = 1}^{m}x_{i_{j}}:=S.\ \footnote{$f(x):=M$表示把式子$f(x)$记为$M$.}
\end{equation*}
则有
$$
    S = \begin{cases*}
        -\displaystyle\frac{a_{n-m}}{a_n},  & $m$\text{\ 为奇数}, \\
        \ \displaystyle\frac{a_{n-m}}{a_n}, & $m$\text{\ 为偶数}.
    \end{cases*}
$$

特别地, 当$m = n$时, $$S^{\prime} = x_1x_2\cdots x_n.$$
于是有
$$
    x_1x_2\cdots x_n = \begin{cases*}
        -\displaystyle\frac{a_{0}}{a_n},  & $m$\text{\ 为奇数}, \\
        \ \displaystyle\frac{a_{0}}{a_n}, & $m$\text{\ 为偶数}.
    \end{cases*}
$$\\

这就是一元$n$次多项式方程的根与系数之间的关系, 它对实系数和复系数都适用, 即前述的$a_i \in \mathbb{C}\ (i = 1, 2, \cdots, n).$

以上就是本次研究性学习的全部内容.

\end{document}