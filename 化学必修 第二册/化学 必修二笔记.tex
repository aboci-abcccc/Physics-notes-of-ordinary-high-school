\documentclass[12pt,a4paper]{ctexbook}

\title{化学\ 必修二笔记}
\author{啊波呲}

\setlength{\parskip}{0em}
\usepackage{amsmath,mathtools,embrac,amssymb,geometry,wrapfig,graphicx,empheq,pifont,extarrows,mhchem,enumerate,chemfig}
\renewcommand{\baselinestretch}{1.8}
\geometry{left=1.5cm,right=1.5cm,bottom=2.1cm,top=2.5cm}
\usepackage{tikz}
\CTEXsetup{chapter}
\setcounter{chapter}{+4}
\usepackage{xcolor}

\begin{document}
\maketitle
\pagenumbering{roman}
\tableofcontents

\newpage
\pagenumbering{arabic}
\chapter{化工生产中的重要非金属元素}
\section{硫及其化合物}
\subsection{硫}
\subsubsection{1.存在形式}

硫在自然界中以游离态和化合态两种形式存在。

$$
	\text{硫}
	\begin{cases}
		\text{游离态}                \\
		\text{化合态}
		\begin{cases}
			\text{硫化物}
			\begin{cases}
				\text{黄铁矿($\ce{FeS_2}$)}   \\
				\text{黄铜矿($\ce{CuFeS_2}$)} \\
			\end{cases} \\
			\text{硫酸盐}
			\begin{cases}
				\text{重晶石($\ce{BaSO_4}$)}       \\
				\text{生石膏($\ce{CaSO_4*2H_2O}$)} \\
				\text{熟石膏($\ce{2CaSO4*H2O}$)}   \\
				\text{芒硝($\ce{Na2SO4*10H2O}$)}  \\
				\text{胆矾($\ce{CuSO4*5H2O}$)}    \\
				\text{绿矾($\ce{FeSO4*7H2O}$)}    \\
				\text{明矾($\ce{KAlSO4*12H2O}$)}  \\
				\text{皓矾($\ce{ZnSO4*7H2O}$)}
			\end{cases}
		\end{cases} \\
	\end{cases}
$$

\subsubsection{2. 物理性质}
硫是黄色晶体,质脆,易研成粉末;难溶于水,微溶于酒精,易溶于$\ce{CS2}$。
\subsubsection{3. 化学性质}

\paragraph{(1)与非金属反应}

硫可以在氧气中燃烧,生成二氧化硫。在空气中燃烧时为淡蓝色火焰,在纯氧中燃烧是蓝紫色火焰。在空气中反应时放热更多:
\setlength{\abovedisplayskip}{2pt}
\setlength{\belowdisplayskip}{2pt}
$$\ce{S + O2 \xlongequal{\text{点燃}} SO2}$$
注意,硫和氧气反应,\textbf{不会生成三氧化硫}。

硫在氢气中加热,生成硫化氢气体:
$$\ce{H2 + S \xlongequal{\triangle} H2S}$$

\paragraph{(2)与金属反应}
硫属于弱氧化剂。弱氧化剂与变价金属反应,生成低价态产物:
$$\ce{2Na + S \xlongequal{研磨} Na2S}$$
$$\ce{2Fe + S \xlongequal{\triangle} FeS}$$
$$\ce{2Cu + S \xlongequal{\triangle} Cu2S}$$
$$\ce{Hg + S \xlongequal{\triangle} HgS}$$
其中,\textbf{FeS是不溶于水但溶于酸的黑色沉淀;CuS是既不溶于水也不溶于酸的黑色沉淀}。

\paragraph{(3)与碱反应}
与NaOH反应,生成两种不同的钠盐,其中的S分别为-2、+4价:
$$\ce{3S + 6NaOH \xlongequal{\triangle} 2Na2S + Na2SO3 + 3H2O}$$
这个反应可以用于清洗含硫的试管。

\subsection{硫化氢}
\subsubsection{1. 物理性质}
硫化氢是无色、有刺激性气味的气体,有毒;密度大于空气;易溶于水(1:40)。

\subsubsection{2. 化学性质}
\paragraph{(1)酸性}
$\ce{H2S}$溶于水成氢硫酸,是二元弱酸。它的电离:
$$\ce{H2S <=> H^+ + HS^-}$$

少量$\ce{H2S}$通入NaOH:$$\ce{2OH^- + H2S = S^2- + 2H2O}$$

过量$\ce{H2S}$通入NaOH:$$\ce{OH^- + H2S = HS^- + H2O}$$

$\ce{H2S}$和$\ce{CuSO4}$、$\ce{(CH3COO)2Pb}$等盐溶液也发生反应。

因为$\ce{CuS}$在酸中溶解性极低,所以会发生类似“弱酸制强酸”的反应:
$$\ce{CuSO4 + H2S = CuS v + H2SO4}$$
\textbf{这是复分解反应中唯一的弱酸制强酸,可以用于检验$\ce{H2S}$或除杂。}

\paragraph{(2)还原性}
$\ce{H2S}$与$\ce{HS^-}$、$\ce{S^2-}$的还原性相同。还原性顺序为:
$$\mathrm{S^{2-} > SO_{3}^{2-} > I^{-} > Fe^{2+} > Br^{-} > Cl^{-}}$$

\begin{itemize}
	\setlength{\itemsep}{0pt}
	\setlength{\parsep}{0pt}
	\setlength{\parskip}{0pt}
	\item $\ce{H2S}$在氧气中燃烧,氧气不足时生成硫单质,氧气充足时生成二氧化硫:
	      $$\ce{2H2S + O2 \xlongequal{\text{点燃}} 2S + 2H2O}$$
	      $$\ce{H2S + 3O2 \xlongequal{\text{点燃}} 2SO2 + 2H2O}$$
	      $\ce{H2S}$溶液放置在空气中变质:
	      $$\ce{2H2S + O2 = S v + 2H2O}$$
	\item $\ce{H2S}$与氯气反应,因为非金属性$\ce{Cl} > \ce{S}$,氯气把硫化氢氧化为硫单质:$$\ce{Cl2 + H2S = S v + 2HCl}$$
	\item $\ce{H2S}$与$\ce{FeCl3}$反应,生成硫沉淀:$$\ce{H2S + 2Fe^3+ = 2Fe^2+ + 2H+ + S v}$$
	      \textbf{补充}\ \ \ \textit{$\ce{FeS}$不溶于水,但溶于酸。所以和过量的$\ce{Na2S}$反应时,会生成沉淀:}
	      $$\ce{3S^- + 2Fe^3+ = 2FeS v + S v}$$
	      \textit{少量的则不会:}
	      $$\ce{S^2- + Fe^3+ = 2Fe^2+ + S v}$$
	\item $\ce{H2S}$与浓硫酸反应,化合价靠拢而不交叉:
	      $$\ce{H2SO4_{\text{(浓)}} + H2S = S v + SO2 ^ + 2H2O}$$
\end{itemize}

\subsubsection{3. 实验室制法}

\noindent
\textbf{(1)原理} \ \ \ $\ce{FeS}$或ZnS与$\ce{H2SO4}$或$\ce{HCl}$反应生成$\ce{H2S}$,沉淀溶解。\\
\textbf{(2)装置} \ \ \ 启普发生器。特点是:反应物为块状固体,无需加热。可用于制备$\ce{H2}$、$\ce{CO2}$、$\ce{H2S}$。\\
\textbf{(3)除杂} \ \ \ 除HCl用$\ce{NaHS}$溶液,除水用浓硫酸。

\textbf{补充}\ \ \ \textit{在弱酸气体中除去强酸用弱酸的酸式盐溶液。}\\
\textbf{(4)收集} \ \ \ 排饱和$\ce{NaHS}$溶液,或向上排空气。\\
\textbf{(5)验满} \ \ \ 湿润的醋酸铅试纸。生成黑色沉淀说明已集满。\\
\textbf{(6)尾气处理} \ \ \ $\ce{NaOH}$。$\ce{H2S + 2NaOH = Na2S + 2H2O}$。

\subsection{二氧化硫}

\subsubsection{1. 物理性质}

二氧化硫是无色、有刺激性气味的气体,有毒;密度大于空气;易溶于水(1:40),易液化。

\subsubsection{2. 化学性质}
\paragraph{(1)酸性氧化物}
$\ce{SO2}$溶于水成亚硫酸,是二元弱酸:
$$\ce{SO2 + H2O <=> H2SO3}$$

它水溶液中不完全电离,电离方程式为:
$$\ce{H2SO3 <=> HSO3^- + H^+}$$
$$\ce{HSO3^- <=> SO3^2- + H^+}$$

\textbf{补充\ \ 可逆反应}\ \ \ \textit{在同一条件下,既能正向进行,
	又能逆向进行的反应。比如氯气与水的反应、二氧化碳与水的反应、
	铁离子与硫氰根离子的反应、氢气与碘单质的反应等。}

\begin{itemize}
	\item $\ce{SO2}$与碱反应:

	      \subitem 少量$\ce{SO2}$与NaOH反应:$$\ce{SO2 + 2OH^- = SO3^2- + H2O}$$
	      这个反应在实验室中用于二氧化硫的尾气处理。
	      \subitem 过量$\ce{SO2}$与NaOH反应:$$\ce{SO2 + OH^- = HSO3^-}$$
	      \subitem 少量$\ce{SO2}$与$\ce{Ca(OH)2}$反应:$$\ce{SO2 + Ca^2+ + 2OH^- = CaSO3 v + H2O}$$
	      这说明二氧化硫也能使澄清石灰水变浑浊。
	      \subitem 过量$\ce{SO2}$与$\ce{Ca(OH)2}$反应:$$\ce{SO2 + OH^- = HSO3^-}$$
	      上面两个方程式说明存在下面的反应:
	      \begin{equation}
		      \ce{CaSO3 + H2O + SO2 = Ca(HSO3)2}
		      \label{亚硫酸钙溶解}
	      \end{equation}
	      \subitem 少量$\ce{SO2}$与$\ce{NH3*H2O}$反应:$$\ce{SO2 + 2NH3*H2O = 2NH3^+ + SO3^2- + H2O}$$
	      这个反应在工业上用于二氧化硫的尾气处理。
	      \subitem 过量$\ce{SO2}$与$\ce{NH3*H2O}$反应:$$\ce{SO2 + NH3*H2O = NH3^+ + HSO3^-}$$


	\item $\ce{SO2}$与盐溶液反应,发生“强酸制弱酸”,因为酸性的顺序:
	      $$\ce{H2SO3} > \ce{H2CO3} > \ce{HSO3^-} > \ce{HCO3^-}$$
	      所以会有下面的反应:

	      \setlength{\itemsep}{0pt}
	      \setlength{\parsep}{0pt}
	      \setlength{\parskip}{0pt}
	      \subitem $\ce{SO2}$与$\ce{Na2SO3}$反应:$$\ce{SO2 + SO3^2- + H2O = 2HSO3^-}$$
	      这实际上是亚硫酸制出了硫酸氢根,这是\ref{亚硫酸钙溶解}的实质。
	      \subitem $\ce{SO2}$与$\ce{NaHCO3}$反应:$$\ce{SO2 + HCO3^- = CO2 + HSO3^-}$$
	      这实际上是亚硫酸制出了碳酸。
	      \subitem $\ce{SO2}$与$\ce{Na2CO3}$反应:
	      $$\ce{SO2_{(\text{少量})} + 2CO3^2- + H2O = SO3^2- + 2HCO3^-}$$
	      $$\ce{2SO2_{(\text{过量})} + CO3^2- + H2O = 2HSO3^- + CO2}$$
	      这分别是亚硫酸制出碳酸氢根、亚硫酸制出亚硫酸氢根和碳酸。


	\item $\ce{SO2}$与碱性氧化物反应:
	      $$\ce{SO2 + Na2O = Na2SO3}$$
	      $$\ce{SO2 + CaO = CaSO3}$$

\end{itemize}

上述反应均体现了酸性氧化物的通性。

近年来为防治酸雨,工业上常把含硫煤矿和石灰石混合后燃烧,目的是让石灰石高温生成的
氧化钙吸收含硫煤矿燃烧生成的
$\ce{SO2}$,这个过程称为\textbf{钙基固硫}。一般以石灰石为原料,方程式为:

$$
	\begin{cases}
		\ce{CaCO3 \xlongequal{高温} CaO + CO2 ^} \\
		\ce{CaO + SO2 = CaSO3}                 \\
		\ce{CaSO3 + O2 = 2CaSO4}
	\end{cases}
$$
总反应方程式为:
$$\ce{2CaCO3 + 2SO2 + O2 \xlongequal{高温} 2CaSO4 + 2CO2}$$

除此之外,还可以用碳酸镁固定二氧化硫:
$$\ce{2MgCO3 + 2SO2 + O2 \xlongequal{高温} 2MgSO4 + 2CO2}$$

\paragraph{(2)氧化性} $\ce{SO2}$通入水中,发生归中反应:
$$\ce{2H2S_{(aq)} + SO2 = 3S v + 2H2O}$$

\paragraph{(3)还原性} $\ce{SO2}$与$\ce{KMnO4(H^+)}$、$\ce{K2Cr2O7(H^+)}$、
$\ce{X2}$(除$\ce{F2}$外)、$\ce{O2}$、$\ce{H2O2}$、$\ce{Na2O2}$、$\ce{Fe^3+}$、
硝酸等氧化剂反应,生成$\ce{SO4^2-}$或$\ce{SO3}$。

\begin{itemize}
	\setlength{\itemsep}{0pt}
	\setlength{\parsep}{0pt}
	\setlength{\parskip}{0pt}
	\item $\ce{SO2}$与$\ce{KMnO4(H^+)}$反应,紫色溶液褪色:$$\ce{2MnO4^- + 5SO2 + 2H2O = 2Mn^2+ + 5SO4^2- + 4H^+}$$
	\item $\ce{SO2}$与$\ce{K2Cr2O7(H^+)}$反应,橙色溶液变为绿色:$$\ce{Cr2O7^2- + 2SO2 + 2H^+ = 2Cr^3+ + 3SO4^2- + H2O}$$
	\item $\ce{SO2}$与氯水反应,黄绿色溶液褪色:$$\ce{2H2O + Cl2 + SO2 = 2Cl^- + SO4^2- + 4H^+}$$
	      和溴单质、碘单质反应的方程式与之类似。
	\item $\ce{SO2}$与$\ce{FeCl3}$溶液反应,黄色溶液变为浅绿色:$$\ce{2Fe^3+ + SO2 + 2H2O = 2Fe^2+ + SO4^2- + 4H^+}$$
	\item $\ce{SO2}$与氧气反应,需要催化剂,称为二氧化硫的\textbf{催化氧化}:$$\ce{2SO2 + O2 <=>[\text{催化剂}][\triangle] 2SO3}$$
	\item $\ce{SO2}$与$\ce{Na2O2}$反应:$$\ce{Na2O2 + SO2 = Na2SO4}$$
	\item $\ce{SO2}$与$\ce{H2O2}$反应:$$\ce{H2O2 + SO2 = H2SO4}$$
	\item 少量$\ce{SO2}$与$\ce{Ca(ClO)2}$反应:$$\ce{Ca^2+ + 3ClO^- + SO2 + H2O = CaSO4 v + Cl^- + 2HClO}$$
	      按照一般的氧化还原反应的离子方程式书写步骤,生成物会有两个氢离子,而$\ce{SO2}$少量就是说$\ce{ClO^-}$是过量的,
	      过量的$\ce{ClO^-}$会与这两个氢离子结合生成弱电解质$\ce{HClO}$,于是得到了上面的方程式。
	\item 过量$\ce{SO2}$与$\ce{Ca(ClO)2}$反应:$$\ce{Ca^2+ + 2ClO^- + 2SO2 + H2O = CaSO4 v + 2Cl^- + SO4^2- + 4H^+}$$
	      这里不再生成HClO,是因为$\ce{SO2}$是过量的,它与HClO不共存。
	\item 少量$\ce{SO2}$与$\ce{Ba(NO3)2}$反应:$$\ce{3SO2 + 2NO3^- + 3Ba^2+ + 2H2O = 3BaSO4 v + 2NO + 4H^+}$$
	\item 过量$\ce{SO2}$与$\ce{Ba(NO3)2}$反应:$$\ce{3SO2 + 2NO3^- + Ba^2+ + 2H2O = BaSO4 v + 2NO + 2SO4^2- + 4H^+}$$
\end{itemize}

上述反应均体现了二氧化硫的还原性。

\paragraph{(4)漂白性} $\ce{SO2}$能使品红溶液褪色,并且在加热后颜色恢复。这是因为
$\ce{SO2}$能与某些有色物质结合生成某些无色物质,该无色物质受热会分解。所以,\textbf{
	$\ce{SO2}$的漂白不是永久性的}。常用于漂白纸浆、毛丝等物质。

与$\ce{SO2}$的漂白不同,氯水、次氯酸、过氧根离子的漂白是应用氧化性的永久性漂白。

$\ce{SO2}$使溶液褪色,有时不是应用漂白性:

\textbullet \ $\ce{SO2}$使滴有酚酞的NaOH溶液褪色,体现酸性;

\textbullet \ $\ce{SO2}$使有颜色的氧化剂褪色,体现还原性;

\textbullet \ $\ce{SO2}$使品红溶液褪色,体现氧化性。

需要注意的是,虽然氯水和$\ce{SO2}$都具有漂白性,但\textbf{当氯水和二氧化硫以1:1的物质的量
	通入品红溶液中时,品红溶液不褪色},这是因为:$$\ce{SO2 + Cl2 + 2H2O = H2SO4 + 2HCl}$$

\textbf{补充}\ \ \ \textit{如何鉴别$\ce{SO2}$和$\ce{CO2}$?}

\textbullet \textit{能使品红溶液褪色的是$\ce{SO2}$;}

\textbullet \textit{能使有色氧化剂褪色或变色的是$\ce{SO2}$;}

\textbullet \textit{通入$\ce{H2S}$溶液,出现浑浊的是$\ce{SO2}$;}

\textbullet \textit{通入$\ce{Ba(NO)2}$溶液,出现沉淀的是$\ce{SO2}$。}

\textbf{补充}\ \ \ \textit{若$\ce{SO2}$、$\ce{CO2}$共存,如何一一检验他们?}

\textbullet \textit{方法一\ \ \ 通入品红溶液(褪色),然后通入$\ce{KMnO4}$溶液(褪色),
	然后再通入品红溶液(不褪色),最后通入澄清石灰水(变浑浊);}

\textbullet \textit{方法二\ \ \ 通入品红溶液(褪色),然后通入过量的$\ce{KMnO4}$溶液(\textbf{颜色变淡但不褪色}),
	最后通入澄清石灰水(变浑浊)。}

\textbf{补充}\ \ \ \textit{将$\ce{SO2}$通入$\ce{BaCl2}$溶液,无明显现象,再通入什么气体会产生沉淀?}

\textbullet \textit{通入氯气、氧气、二氧化氮等氧化剂,生成$\ce{BaSO4}$沉淀;}

\textbullet \textit{通入氨气,生成$\ce{BaSO3}$沉淀,原因是氨气与弱电解质亚硫酸反应,会生成亚硫酸根;}

\textbullet \textit{通入$\ce{H2S}$气体,与$\ce{SO2}$反应生成硫沉淀。}

\subsubsection{3. 用途和危害}
\paragraph{(1)用途} 可用于杀菌消毒,或作抗氧化剂;是一种食品添加剂,添加于葡萄酒、干果、蜜饯、糖果中,但不能超标。

\paragraph{(2)危害} 会导致酸雨。酸雨可以用“钙基固硫法”、浓氨水、“双碱”治理。

\textbf{补充}\ \ \ \textit{硫酸型酸雨的形成}
$$\ce{SO2 + 2H2O <=> H2SO3}$$
$$\ce{2H2SO3 + O2 =2H2SO4}$$

\textit{或者}
$$\ce{2SO2 + O2 <=>[\text{粉尘}] 2SO3}$$
$$\ce{SO3 + H2O = H2SO4}$$

\subsubsection{4. 制备}
\noindent
\textbf{(1)实验室制法}

使用分液漏斗、圆底烧瓶。原理是:
$$\ce{Na2SO3_{(s)} + H2SO4_{(70\%)} = Na2SO4 + H2O + SO2 ^}$$
或者
$$\ce{Cu + 2H2SO4_{(\text{浓})} \xlongequal{\triangle} CuSO4 + 2H2O + SO2 ^}$$
其中第二个反应氧化剂与还原剂的物质的量之比为1:1。

使用排饱和$\ce{NaHSO3}$溶液的方法或向上排空气法收集;使用浓硫酸或$\ce{P2O5}$或硅胶($\ce{SiO2*nH2O}$)干燥;
使用湿润的品红试纸或红色石蕊试纸验满;\textbf{使用过量NaOH溶液处理尾气}:$$\ce{SO2 + 2NaOH = Na2SO3 + H2O}$$

此外,为了防止发生倒吸,通常加入安全瓶或使用倒置的漏斗或使用球形干燥管。

\noindent
\textbf{(2)工业制法}
$$\ce{S + O2 \xlongequal{\text{点燃}} SO2}$$
或者
$$\ce{4FeS2 + 11O2 \xlongequal{\text{高温}} 2Fe2O3 + 8SO2}$$

\subsection{三氧化硫}

\subsubsection{1. 物理性质}
常温下是液态,\textbf{标准状况下是固态}。

\subsubsection{2. 化学性质}
$\ce{SO3}$是酸性氧化物。

\textbullet $\ce{SO3}$与水反应:
$$\ce{SO3 + H2O = H2SO4}$$

\textbullet $\ce{SO3}$与碱性氧化物反应:
$$\ce{SO3 + CaO = CaSO4}$$

\textbullet $\ce{SO3}$与碱反应:
$$\ce{SO3 + Ca(OH)2 = CaSO4 + H2O}$$

\textbullet $\ce{SO3}$通入$\ce{BaCl2}$溶液,出现白色沉淀:
$$\ce{Ba^2+ + SO3 + H2O = BaSO4 v + 2H^+}$$

工业上常用硫酸吸收三氧化硫。

\subsection{硫酸}

\subsubsection{1. 物理性质}
纯净的硫酸是无色粘稠液体,98.3\%的硫酸溶液,密度是$1.84\ g/cm^3$;能与水以任意比例互溶;沸点高,难挥发。

高沸点的酸可以制出低沸点的酸,难挥发的酸可以制出易挥发的酸。

\paragraph{硫酸制$\ce{HCl}$} 使用氯化钠和浓硫酸制HCl,装置是分液漏斗、圆底烧瓶;使用向上排空气法收集;需要防倒吸装置和尾气处理装置。

微热时:$$\ce{NaCl_{(s)} + H2SO4_{(\text{浓})} \xlongequal{\text{微热}} NaHSO4 + HCl ^}$$

加热时:$$\ce{2NaCl_{(s)} + H2SO4_{(\text{浓})} \xlongequal{\triangle} Na2SO4 + 2HCl ^}$$

还可以向浓盐酸中加入浓硫酸:浓硫酸吸水放热,氯化氢气体挥发出来。

除此之外,还可以用硫酸制HF、HBr、HI、$\ce{HNO3}$等。

\subsubsection{2. 化学性质}

浓硫酸有吸水性、脱水性、氧化性。

\paragraph{(1)吸水性} 浓硫酸可以使结晶水合物失水,比如:
$$\ce{CuSO4*5H2O \xlongequal{\text{浓}\ce{H2SO4}} CuSO4 + 5H2O}$$

除此之外,利用吸水性,浓硫酸还可以作干燥剂,但不能干燥碱性气体、还原性气体;\textbf{也不能用于干燥三氧化硫,
	因为三氧化硫能被浓硫酸吸收而形成发烟硫酸}。

\paragraph{(2)脱水性} 把有机物中的水以2:1的比例脱去,比如:
$$\ce{C + 2H2SO4 \xlongequal{\triangle} CO2 ^ + 2SO2 ^ + 2H2O}$$
$$\ce{C12H22O11 ->[\text{浓}\ce{H2SO4}] 12C + 11H2O}$$

\paragraph{(3)强氧化性} 硫酸是氧化性酸。
铁、铝遇冷的浓硫酸、浓硝酸会发生钝化,其表面生成一层致密的氧化膜。

浓硫酸在加热的条件下可以溶解铁、铝以及活动性较弱的金属,发生氧化还原反应。

\begin{itemize}
	\item 浓硫酸与铜共热,铜逐渐溶解,产生有刺激性气味的气体:
	      $$\ce{Cu + 2H2SO4_{(\text{浓})} \xlongequal{\triangle} CuSO4 + SO2 ^ + 2H2O}$$
	      冷却后,将反应后的溶液倒入盛有水的烧杯中,观察到溶液呈淡蓝色,这证明浓硫酸与铜反应的生成物有硫酸铜。\\
	      \textbf{补充}\ \ \ \textit{氧气和$\ce{H2O2}$在酸性条件下也可以把铜氧化:$$\ce{2Cu + 4H^+ + O2 = 2Cu^2+ + 2H2O}$$ $$\ce{Cu + 2H^+ + H2O2 = Cu^2+ + 2H2O}$$}
	\item 浓硫酸与锌共热:$$\ce{Zn + 2H2SO4_{(\text{浓})} \xlongequal{\triangle} ZnSO4 + SO2 ^ + 2H2O}$$
	      一段时间后,酸变稀:$$\ce{Zn + H2SO4_{(\text{稀})} = ZnSO4 + H2 ^}$$
	\item 浓硫酸与铁共热,\textbf{铁被氧化到+3价}:$$\ce{2Fe + 6H2SO4_{(\text{稀})} \xlongequal{\triangle} Fe2(SO4)3 + 3SO2 ^ + 6H2O}$$
	      一段时间后,酸变稀:$$\ce{Fe + H2SO4_{(\text{稀})} = FeSO4 + H2 ^}$$
	      上面的两个反应中,$\ce{SO2}$和$\ce{H2}$转移的电子数是相同的,在进行相关计算时可以利用这一点。
\end{itemize}

浓硫酸在加热的条件下也可以溶解一些非金属。

\begin{itemize}
	\item 浓硫酸与碳共热,生成二氧化碳、二氧化硫和水:$$\ce{C + 2H2SO4_{(\text{浓})} \xlongequal{\triangle} CO2 ^ + 2SO2 ^ + 2H2O}$$
	\item 浓硫酸与硫共热,发生归中反应,生成二氧化硫和水:$$\ce{S + 2H2SO4_{(\text{浓})} \xlongequal{\triangle} 3SO2 ^ + 2H2O}$$
\end{itemize}

\subsubsection{3. 浓硫酸、稀硫酸的鉴别}

\textbf{方法一}\ \ \ 看外观,浓稠的是浓硫酸;

\textbf{方法二}\ \ \ 滴在纸上,使纸变黑的使浓硫酸;

\textbf{方法三}\ \ \ 滴入水中,放热的是浓硫酸;

\textbf{方法四}\ \ \ 放入铁片,在常温下有明显现象的是稀硫酸。

\subsection{硫酸根离子与亚硫酸根例子}

\subsubsection{1. 硫酸根离子的鉴别}

为了除去$\ce{CO3^2-}$、$\ce{SO3^2-}$、$Ag^+$等离子的干扰,先加稀盐酸酸化,无明显现象;再加$\ce{BaCl2}$溶液,产生白色沉淀,证明有$\ce{SO4^2-}$存在。

\subsubsection{2. 亚硫酸根离子的还原性}

中性或碱性条件下的$\ce{SO4^2-}$离子也有一定的还原性。

\begin{itemize}
	\item 少量氯气通入$\ce{Na2SO3}$溶液:$$\ce{Cl2 + 3SO3^2- + H2O = 2Cl^- + SO4^2- + 2HSO3^-}$$
	\item 过量氯气通入$\ce{Na2SO3}$溶液:$$\ce{Cl2 + SO3^2- + H2O = SO4^2- + 2Cl- + 2H^+}$$
	\item $\ce{FeCl3}$溶液与$\ce{Na2SO3}$溶液反应:$$\ce{2Fe^3+ + SO3^2- + H2O = 2Fe^2+ + SO4^2- + 2H+}$$
\end{itemize}

\subsubsection{3. 除浓硫酸外,硫酸根无氧化性}

\subsection{硫元素的不同价态}

\subsubsection{1. 硫元素的四种价态}

\ \textbf{-2}\ \ \ $\ce{H2S}$、$\ce{Na2S}$、$\ce{NaHS}$\ \ \ \textbf{只有还原性}

\ \textbf{0}\ \ \ \ \ $\ce{S}$\ \ \ \ \ \textbf{既有氧化性,又有还原性}

\textbf{+4}\ \ \ $\ce{SO2}$、$\ce{H2SO3}$、$\ce{Na2SO3}$、$\ce{NaHSO3}$\ \ \ \textbf{既有氧化性,又有还原性}

\textbf{+6}\ \ \ $\ce{SO3}$、$\ce{H2SO4}$、$\ce{Na2SO4}$、$\ce{NaHSO4}$\ \ \ \textbf{仅浓硫酸有强氧化性,其他条件下的硫酸根无氧化性}

\subsubsection{2. 硫元素不同价态间的转化}

$-2 \rightarrow 0$\ \ \ $\ce{SO2}$、$\ce{O2}$、$\ce{Cl2}$、$\ce{H2O2}$、$\ce{Br2}$、$\ce{I2}$、$\ce{Fe^3+}$

$0 \rightarrow -2$\ \ \ $\ce{H2}$、$\ce{Fe}$、$\ce{Cu}$、$\ce{Hg}$、$\ce{NaOH}$

$0 \rightarrow +4$\ \ \ $\ce{O2}$、$\ce{NaOH}$

$0 \rightarrow +4$\ \ \ $\ce{H2S}$、$\ce{Na2S}$

$+4 \rightarrow +6$\ \ \ $\ce{O2}$、$\ce{Cl2}$、$\ce{Br2}$、$\ce{I2}$、$\ce{Fe^3+}$、$\ce{KMnO4}$、$\ce{H2O2}$

$+6 \rightarrow +4$\ \ \ $\ce{Fe}$、$\ce{Cu}$、$\ce{Zn}$、$\ce{C}$、$\ce{S}$

\subsubsection{3. 硫代硫酸钠}

硫代硫酸钠($\ce{Na2S2O3}$)是硫酸钠中的一个氧原子被-2价的硫离子取代后得到的物质。
硫代硫酸钠俗称大苏打,其水合物$\ce{Na2S2O3*5H2O}$俗称海波。

$\ce{Na2S2O3}$可以与$\ce{H2SO4}$反应,实质是硫代硫酸根与氢离子反应:
$$\ce{S2O3^2- + 2H^+ = SO2 ^ + S v + H2O}$$

硫代硫酸钠有还原性,它与氯气反应的离子方程式为:
$$\ce{S2O3^2- + 4Cl2 + 5H2O = 2SO4^2- + 8Cl^- + 10H^+}$$


\section{有机}

$$\chemfig{CH_2-C-([:90]=O)-*6(-=-=-=)}$$
通过酯化(一般为六元环):

\end{document}